\documentclass[a4paper, 10pt]{article}
\usepackage[margin=1in]{geometry}
\usepackage{amsfonts, amsmath, amssymb, amsthm}
%\usepackage[none]{hyphenat}
\usepackage{fancyhdr} %create a custom header and footer
\usepackage[utf8]{inputenc}
\usepackage[english, main=ukrainian]{babel}
\usepackage{pgfplots}
\usepgfplotslibrary{fillbetween}
\usepackage{tikz}
\usepackage{graphicx}
\usepackage{caption}
\usepackage{float}
\usepackage{physics}
\usepackage[unicode]{hyperref}
\usepgfplotslibrary{polar}
\usepackage{ifthen}
\usepackage{enumitem}
\usetikzlibrary{spy}
\usepackage{bbm}
\usepackage{centernot}

\fancyhead{}
\fancyfoot{}
\parindent 0ex
\def\rightproof{$\boxed{\Rightarrow}$ }
\def\leftproof{$\boxed{\Leftarrow}$ }

\usepackage{pdfpages}

\newtheoremstyle{theoremdd}% name of the style to be used
  {\topsep}% measure of space to leave above the theorem. E.g.: 3pt
  {\topsep}% measure of space to leave below the theorem. E.g.: 3pt
  {\normalfont}% name of font to use in the body of the theorem
  {0pt}% measure of space to indent
  {\bfseries}% name of head font
  {}% punctuation between head and body
  { }% space after theorem head; " " = normal interword space
  {\thmname{#1}\thmnumber{ #2}\textnormal{\thmnote{ \textbf{#3}\\}}}

\theoremstyle{theoremdd}
\newtheorem{theorem}{Theorem}[subsection]
  
\theoremstyle{theoremdd}
\newtheorem{definition}[theorem]{Definition}

\theoremstyle{theoremdd}
\newtheorem{samedef}[theorem]{Definition}

\theoremstyle{theoremdd}
\newtheorem{example}[theorem]{Example}

\theoremstyle{theoremdd}
\newtheorem{proposition}[theorem]{Proposition}

\theoremstyle{theoremdd}
\newtheorem{remark}[theorem]{Remark}

\theoremstyle{theoremdd}
\newtheorem{lemma}[theorem]{Lemma}

\theoremstyle{theoremdd}
\newtheorem{corollary}[theorem]{Corollary}

\renewcommand{\qedsymbol}{$\blacksquare$}
\newcommand{\toweak}{\overset{w}{\to}}
\newcommand{\toweakstar}{\overset{w^*}{\to}}

\makeatletter
\renewenvironment{proof}[1][Proof.\\]{\par
\pushQED{\hfill \qed}%
\normalfont \topsep6\p@\@plus6\p@\relax
\trivlist
\item\relax
{\bfseries
#1\@addpunct{.}}\hspace\labelsep\ignorespaces
}{%
\popQED\endtrivlist\@endpefalse
}
\makeatother

\newenvironment{pf}{\vspace*{-3mm} \textbf{Proof. \\}}{$\blacksquare$}
\newenvironment{pfMI}{\vspace*{-3mm} \textbf{Proof MI. \\}}{$\blacksquare$}
\newenvironment{pfNoTh}{\textbf{Proof. \\}}{$\blacksquare$}

\DeclareMathOperator{\Cl}{Cl}
\DeclareMathOperator{\linspan}{span}

%delete

\begin{document}
\tableofcontents
\newpage

\section{Метричні простори та інше}
\subsection{Означення метричних просторів}
\begin{definition}
Задано $X$ -- множина та $\rho \colon X \to X \to \mathbb{R}$ -- функція.\\
Функція $\rho$ називається \textbf{метрикою}, якщо вона задовольняє таким властивостям:
\begin{align*}
\text{1) } \forall x,y \in X: \rho(x,y) \geq 0 ,\qquad \rho(x,y) = 0 \iff x = y \\
\text{2) } \forall x,y \in X: \rho(x,y) = \rho(y,x) \\
\text{3) } \forall x,y,z \in X: \rho(x,z) \leq \rho(x,y) + \rho(y,z)
\end{align*}
Метрика описує \textbf{відстань} між елементами $x,y$. \\
Пара $(X,\rho)$ з метрикою називається \textbf{метричним простором}.
\end{definition}


\begin{example}
Розглянемо декілька прикладів:
\begin{enumerate}[nosep,wide=0pt,label={\arabic*)}]
\item $X = \mathbb{R}$, \qquad $\rho(x,y) = |x-y|$;
\item $X = \mathbb{R}^n$, можна задати дві метрики: \\
$\rho_1(\vec{x}, \vec{y}) = \sqrt{(x_1-y_1)^2 + \dots + (x_n-y_n)^2}$, \qquad $\rho_2(\vec{x}, \vec{y}) = |x_1-y_1|+\dots+|x_n-y_n|$;
\item $X = C([a,b])$, \qquad $\displaystyle \rho(f,g) = \max_{t \in [a,b]} |f(t)-g(t)|$.
\end{enumerate}
\end{example}

\begin{example}
Окремо розгляну даний приклад. Нехай $X$ -- будь-яка множина, ми визначимо так звану \textbf{дискретну метрику} $d(x,y) = \begin{cases} 1, & x \neq y \\ 0, & x = y \end{cases}$. Тоді $(X,d)$ задає \textbf{дикретний} метричний простір.
\end{example}

\begin{definition}
Задано $(X,\rho)$ -- метричний простір.\\
Пару $(Y,\tilde{\rho})$, де $Y \subset X$, назвемо \textbf{метричним підпростором} $(X,\rho)$, якщо
\begin{align*}
\forall x,y \in Y: \tilde{\rho}(x,y) = \rho(x,y).
\end{align*}
При цьому метрика $\tilde{\rho}$, кажуть, \textbf{індукована в} $Y$ \textbf{метрикою} $\rho$.
\end{definition}

\begin{proposition}
Задано $(X,\rho)$ -- метричний простір та $(Y,\tilde{\rho})$ -- підпростір. Для функції $\tilde{\rho}$ всі три аксіоми зберігаються. Тобто $(Y,\tilde{\rho})$ залишається метричним простором.\\
\textit{Вправа: довести.}
\end{proposition}

\begin{example}
Маємо $X = F([a,b])$ -- множину обмежених функцій та $\rho(f,g) = \displaystyle \sup_{t \in [a,b]} |f(t)-g(t)|$. Тоді в $Y = C([a,b])$ маємо метрику $\tilde{\rho}(f,g) = \displaystyle \max_{t \in [a,b]} |f(t)-g(t)| = \displaystyle \sup_{t \in [a,b]} |f(t)-g(t)|$. Отже, $C([a,b])$ -- метричний підпростір простору $F([a,b])$.
\end{example}

\begin{definition}
Задано $L$ -- лінійний простір над $\mathbb{R}$ або $\mathbb{C}$.\\
Задамо функцію $\| \cdot \|: L \to \mathbb{R}$, що називається \textbf{нормою}, якщо виконуються умови:
\begin{align*}
\text{1) } \forall x \in L: ||x\| \geq 0 \\
\text{2) } \forall x \in L: \forall \alpha \in \mathbb{R} \text{ або } \mathbb{C}: \|\alpha x\| = |\alpha| \|x\| \\
\text{3) } \forall x,y \in L: \|x+y\| \leq \|x\| + \|y\|
\end{align*}
Тоді пару $(L, \|\cdot \|)$ назвемо \textbf{нормованим простором}.
\end{definition}

\begin{corollary}
Задано $(L, \| \cdot \|)$ -- нормований простір. Тоді $\forall x,y \in L: \|x-y\| \geq \left| \|x\| - \|y\| \right|$.\\
\textit{Вказівка: $\|x\| = \|x + y - x\|$ та $\|y\| = \|y + x -y \|$.}
\end{corollary}

\begin{proposition}
Задано $(L, \| \cdot \|)$ -- нормований простір. Тоді $L$ з метрикою $\rho(x,y) = \|x-y\|$ утворює метричний простір.\\
\textit{Вправа: перевірити три аксіоми.}
\end{proposition}

\begin{corollary}
У такому разі справедливі додаткові властивості для заданої метрики:
\begin{enumerate}[nosep,wide=0pt,label={\arabic*)}]
\item $\forall x,y,z \in L: \rho(x+z,y+z) = \rho(x,y)$ (інваріантність по відношенню до зсуву);
\item $\forall x,y \in L, \forall \alpha \in \mathbb{R} \text{ або } \mathbb{C}: \rho(\alpha x, \alpha y) = |\alpha| \rho(x,y)$ (однорідність).
\end{enumerate}
\end{corollary}

\begin{example}
Зокрема дані простори будуть нормованими:
\begin{enumerate}[nosep,wide=0pt,label={\arabic*)}]
\item $\mathbb{R}$, \qquad $\| x\| = |x|$;
\item $\mathbb{R}^n$, \qquad $\| \vec{x} \| = \sqrt{x_1^2 + \dots + x_n^2}$ або навіть $\| \vec{x} \| = |x_1| + \dots + |x_n|$;
\item $\mathbb{C}([a,b])$, \qquad $\displaystyle \| f\| = \max_{t \in [a,b]} |f(t)|$;
\item $L_p(X,\lambda)$, \qquad $\|f\|_p = \displaystyle\left(\int_X |f|^p\,d\lambda\right)^{\frac{1}{p}}$.
\end{enumerate}
Тому всі вони будуть автоматично метричними просторами із метрикою, що вище задана.
\end{example}

\begin{example}
Дискретний простір $(X,d)$ -- метричний, але не нормований.\\

\end{example}

\begin{example}
Задано $(E, (\cdot, \cdot))$ -- евклідів простір. Ми можемо евклідів простір $E$ перетворити в нормований простір $(E, \| \cdot \|)$ функцією $\|x\| = \sqrt{(x,x)}$.\\
Як наслідок, за твердженням вище, $(E, \rho)$ - метричний простір з $\rho(x,y) = \|x-y\|$.
\end{example}

\begin{example} Нехай $\vec{a} = (a_1,a_2,\dots)$ -- дійсна числова послідовність. Задамо простір $l_1 = \left\{ \vec{a} \mid \displaystyle\sum_{n=1}^\infty |a_n| < \infty \right\}$. Задаються такі операції:\\
$\vec{a} + \vec{b} = (a_1,a_2,\dots) + (b_1,b_2,\dots) = (a_1+b_1,a_2+b_2,\dots)$\\
$\alpha \vec{a} = (\alpha a_1, \alpha a_2, \dots)$\\
Неважко переконатися, що $l_1$ -- лінійний простір над $\mathbb{R}$.\\
Важливе зауваження: $\vec{a}+\vec{b}, \alpha \vec{a} \in l_1$, тому що маємо $\displaystyle\sum_{n=1}^\infty a_n$, $\displaystyle\sum_{n=1}^\infty b_n$ -- збіжні, а тому збіжним буде $\displaystyle \sum_{n=1}^\infty (a_n+b_n), \sum_{n=1}^\infty \alpha a_n$. Тобто операції замкнені.\\
Можна задати нормований простір функцією $\| \vec{a} \| = \displaystyle \sum_{n=1}^\infty |a_n|$. А тому це -- метричний простір з $\rho(\vec{a}, \vec{b}) = \|\vec{a} - \vec{b}\|$.
\bigskip \\
Узагальнення: $l_p = \left\{ \vec{a} \text{ } | \displaystyle\sum_{n=1}^\infty |a_n|^p < \infty \right\}$, тут задається норма $\|\vec{a}\| = \left( \displaystyle\sum_{n=1}^\infty |a_n|^p \right)^{\frac{1}{p}}$.
\end{example}

\begin{example}
Тут ще є така множина: $l_{\infty} = \{ \vec{a} \mid \vec{a} \text{ -- обмежені} \}$. Задані такі самі операції, що вище. Задається норма $\|\vec{a}\| = \displaystyle \sup_{n \in \mathbb{N}} |a_n|$. Отже, $l_{\infty}$ -- метричний простір.
\end{example}

\subsection{Відкриті та замкнені множини. Збіжні послідовності}
\begin{definition}
Задано $(X,\rho)$ -- метричний простір та $a \in X$.\\
\textbf{Відкритою кулею радіусом $r$ з центром $a$} називають таку множину:
\begin{align*}
B(a;r) = \{x \in X \mid \rho(a,x) < r\}
\end{align*}
Її ще називають $r$\textbf{-околом точки $a$}.\\
\textbf{Замкненою кулею радіусом $r$ з центром $a$} називають таку множину:
\begin{align*}
B[a;r] = \{x \in X \mid \rho(a,x) \leq r \}
\end{align*}
\end{definition}

\begin{example}
Розглянемо декілька прикладів:
\begin{enumerate}[nosep,wide=0pt,label={\arabic*)}]
\item $X = \mathbb{R}$, $\rho(x,y) = |x-y|$, \qquad $B(a;r) = \{x \in \mathbb{R} \mid |x-a| < r\} = (a-r,a+r)$;
\item $X = \mathbb{R}^2$, $\rho(\vec{x}, \vec{y}) = \|\vec{x}-\vec{y}\|$, \qquad $B(0;1) = \{(x,y) \in \mathbb{R}^2 \mid x^2+y^2 <1 \}$.
\end{enumerate}
\end{example}

\begin{definition}
Задано $A \subset X$ та $a \in A$.\\
Точка $a$ називається \textbf{внутрішньою} для $A$, якщо
\begin{align*}
\exists \varepsilon > 0: B(a; \varepsilon) \subset A.
\end{align*}
\end{definition}

\begin{definition}
Множина $A$ називається \textbf{відкритою}, якщо кожна точка множини $A$ -- внутрішня.
\end{definition}

\begin{example} 
Розглянемо такі приклади:
\begin{enumerate}[wide=0pt,label={\arabic*)}]
\item Маємо $X = \mathbb{R}, \rho(x,y) = |x-y|$ та множину $A = [0,1)$. Точка $a = \dfrac{1}{2}$ -- внутрішня, оскільки $\exists \varepsilon = \dfrac{1}{4}: B\left(\dfrac{1}{2}; \dfrac{1}{4} \right) \subset A$, тобто $\left( \dfrac{1}{4}, \dfrac{3}{4} \right) \subset [0,1)$. Водночас точка $a = 0$ -- не внутрішня. Отже, $A$ -- не відркита, бо знайшли не внутрішню точку.

\item Маємо $X = [0,1], \rho(x,y) = |x-y|$ та множину $A = [0,1)$. У цьому випадку точка $a = 0$ уже внутрішня (в попередньому прикладі ми могли $\varepsilon$-околом вийти за межі нуля ліворуч, а тут вже ні). Тут $A$ тепер відкрита.

\item Маємо $X = \{0,1,2\}$ -- підпростір метричного простору $(\mathbb{R}, \rho(x,y) = |x-y|)$. Задамо множину $A = \{0,1\}$. Тут кожна точка -- внутрішня. Отже, $A$ -- відкрита.
\end{enumerate}
\end{example}

\begin{definition}
Задано $A \subset X$ та $x_0 \in X$.\\
Точка $x_0$ називається \textbf{граничною} для $A$, якщо
\begin{align*}
\forall \varepsilon > 0: (B(x_0;\varepsilon) \setminus \{x_0\}) \cap A \neq \emptyset
\end{align*}
Інколи ще множину $B(x_0;\varepsilon) \setminus \{x_0\} = \mathring{B}(x_0;\varepsilon)$ називають \textbf{проколеним околом точки $x_0$}.
\end{definition}

\begin{definition}
Множина $A$ називається \textbf{замкненою}, якщо вона містить всі свої граничні точки.
\end{definition}

\begin{example}
Розглянемо такі приклади:
\begin{enumerate}[wide=0pt,label={\arabic*)}]
\item Маємо $X = \mathbb{R}, \rho(x,y) = |x-y|$ та множину $A = (0,1)$. Точки $x_0 \in \left\{\dfrac{1}{2}, 0, 1\right\}$ -- граничні. Водночас точка $x_0 = \dfrac{3}{2}$ -- не гранична. Отже, $A$ -- не замкнена, бо $x_0 = 1$ хоча й гранична для $A$, але $x_0 \notin A$.
\item Маємо $X = \mathbb{R}, \rho(x,y) = |x-y|$. Задамо множину $A = \{0,1 \}$. Тут жодна точка -- не гранична. Тим не менш, $A$ -- замкнена. Бо нема жодної граничної точки в $X$ для $A$, щоб порушити означення.
\item $X, \emptyset$ -- замкнені в будь-якому метричному просторі.
\end{enumerate}
\end{example}

\begin{theorem}
Задано $(X,\rho)$, $A \subset X$.\\
Множина $A$ -- відкрита $\iff$ множина $A^c$ -- замкнена
\end{theorem}

\begin{proof}
\rightproof Дано: $A$ -- відкрита.\\
!Припустимо, що $A^c$ -- не замкнена, тобто $\exists x_0 \in A: x_0$ -- гранична для $A^c$, але $x_0 \notin A^c$. За умовою, оскільки $x_0 \in A$, то $x_0$ - внутрішня, тобто $\exists \varepsilon > 0: B(x_0;\varepsilon) \subset A$. Отже, $B(x_0;\varepsilon) \cap A^c = \emptyset$ -- суперечність!
\bigskip \\
\leftproof Дано: $A^c$ -- замкнена. Тоді вона містить всі граничні точки. Тоді $\forall x_0 \in A: x_0$ -- не гранична для $A^c$, тобто $\exists \varepsilon > 0: B(x_0;\varepsilon) \cap A^c = \emptyset \implies B(x_0;\varepsilon) \subset A$. Отже, $x_0$ -- внутрішня для $A$, а тому $A$ -- відкрита.
\end{proof}

\begin{theorem} 
Задано $(X,\rho)$ -- метричний простір. Тоді справедливо наступне:
\begin{enumerate}[nosep,wide=0pt,label={\arabic*)}]
\item Нехай $U_{\alpha} \subset X$, $\alpha \in I$ -- сім'я відкритих множин. Тоді $\displaystyle \bigcup_{\alpha \in I} U_{\alpha}$ -- відкрита множина;
\item Нехай $U_k \subset X, k = \overline{1,n}$ -- сім'я відкритих множин. Тоді $\displaystyle \bigcap_{k=1}^n U_k$ -- відкрита множина;
\item $\emptyset, X$ -- відкриті множини.
\end{enumerate}
\end{theorem}

\begin{proof}
Доведемо кожний пункт окремо:
\begin{enumerate}[wide=0pt,label={\arabic*)}]
\item Задано множину $U = \displaystyle \bigcup_{\alpha \in I} U_{\alpha}$. Зафіксуємо $a \in U$. Тоді $\exists \alpha_0: a \in U_{\alpha_0} \implies a$ -- внутрішня для $U_{\alpha_0} \\ \Rightarrow \exists \varepsilon > 0: B(a;\varepsilon) \subset U_{\alpha_0} \subset U$. Отже, $U$ -- відкрита.

\item Задано множину $U = \displaystyle \bigcap_{k=1}^n U_k$. Зафіксуємо $a \in U$. Тоді $\forall k = \overline{1,n}: a \in U_k \Rightarrow a$ -- внутрішня для $U_k \implies \exists \varepsilon_k > 0: B(a;\varepsilon_k) \subset U_k$. Задамо $\varepsilon = \displaystyle\min_{1 \leq k \leq n} \varepsilon_k \implies B(a;\varepsilon) \subset U$. Отже, $U$ -- відкрита.

\item $\emptyset$ -- відкрита, бо нема внутрішніх точок, тому що там порожньо. Також $X$ -- відкрита, оскільки для $a \in X$, який б $\varepsilon > 0$ не обрав, $B(a;\varepsilon) \subset X$.
\end{enumerate}
Всі твердження доведені.
\end{proof}

\begin{remark}
Відповідь на питання, чому в другому лише скінченна кількість відкритих множин. Розглянемо $X = \mathbb{R}$ із метрикою $\rho(x,y) = |x-y|$. Задана сім'я відкритих множин $U_n = \left( -\dfrac{1}{n}, \dfrac{1}{n} \right)$, причому $\forall n \geq 1$. Тоді зауважимо, що $\displaystyle\bigcap_{n=1}^\infty U_n = \{0\}$, але така множина вже не є відкритою.
\end{remark}

\begin{corollary}
Задано $(X,\rho)$ -- метричний простір. Тоді справедливо наступне:
\begin{enumerate}[nosep,wide=0pt,label={\arabic*)}]
\item Нехай $U_{\alpha} \subset X$, $\alpha \in I$ -- сім'я замкнених множин. Тоді $\displaystyle \bigcap_{\alpha \in I} U_{\alpha}$ -- замкнена множина;
\item Нехай $U_k \subset X, k = \overline{1,n}$ -- сім'я замкнених множин. Тоді $\displaystyle \bigcup_{k=1}^n U_k$ -- замкнена множина;
\item $\emptyset, X$ -- замкнені множини.
\end{enumerate}
\textit{Вказівка: скористатися де Морганом та TODO: вставити референс.}
\end{corollary}

\begin{remark} Такі твердження не є правдивими:\\
$A$ -- не відкрита, а тому $A$ -- замкнена (наприклад, $[0,1)$ в $\mathbb{R}$);\\
$A$ -- відкрита, а тому $A$ -- не замкнена (наприклад, $\emptyset$ в $\mathbb{R}$).
\end{remark}

\begin{proposition} Задано $(X,\rho)$ -- метричний простір, $a \in X, r > 0$. Тоді відкритий окіл $B(a;r)$ -- справді відкритий; замкнений окіл $B[a;r]$ -- справді замкнений.
\end{proposition}

\begin{proof}
(про $B(a;r)$). Задамо точку $b \in B(a;r)$. Нехай $\varepsilon = r - \rho(a,b) > 0$. Тоді якщо $x \in B(b; \varepsilon)$, то тоді $\rho(x, a) \leq \rho(x, b) + \rho(b, a) < \varepsilon + \rho(b,a) = r$. Отже, $B(a;r)$ -- відкрита.
\bigskip \\
(про $B[a;r]$). Для цього достатньо довести, що $B^c[a;r] = \{x | \rho(a,x) > r\}$ -- відкрита. Якщо задати $\varepsilon = \rho(a,b) - r$ для точки $b \in B(a;r)$, то аналогічними міркуваннями отримаємо, що $B^c[a;r]$ -- відкрита. Отже, $B[a;r]$ - замкнена.
\end{proof}

\begin{definition}
Задано $(X,\rho)$ -- метричний простір, послідовність $\{x_n, n \geq 1\} \subset X$ та $x_0 \in X$.\\
Дана послідовність називається \textbf{збіжною} до $x_0$, якщо
\begin{align*}
\rho(x_n, x_0) \to 0, n \to \infty
\end{align*}
Позначення: $\displaystyle\lim_{n \to \infty} x_n = x_0$.
\end{definition}

\begin{theorem}
Задано $(X,\rho)$ -- метричний простір, $A \subset X$ та $x_0 \in X$. Наступні твердження еквівалентні:
\begin{enumerate}[nosep,wide=0pt,label={\arabic*)}]
\item $x_0$ -- гранична точка для $A$;
\item $\forall \varepsilon > 0: B(x_0;\varepsilon) \cap A$ - нескінченна множина;
\item $\exists \{x_n, n \geq 1\} \subset A: \forall n \geq 1: x_n \neq x_0: x_n \to x_0$.
\end{enumerate}
\end{theorem}

\begin{proof}
$\boxed{1) \Rightarrow 2)}$ Дано: $x_0$ - гранична для $A$.\\
!Припустимо, що $\exists \varepsilon^* > 0: B(x_0;\varepsilon) \cap A$ -- скінченна множина, тобто маємо  $x_1,\dots,x_n \in B(x_0;\varepsilon^*)$. Тоді $\rho(x_0,x_1) < \varepsilon^*, \dots, \rho(x_0,x_n)^* < \varepsilon$. Оберемо найменшу відстань та задамо $\varepsilon^*_{new} = \displaystyle \min_{1\leq i \leq n} \rho(x_0,x_i)$. Створимо $B(x_0;\varepsilon^*_{new}) \subset B(x_0; \varepsilon)$. У новому шару жодна точка $x_1,\dots,x_n \in A$ більше сюди не потрапляє. Тоді $B((x_0;\varepsilon^*_{new}) \setminus \{x_0\}) \cap A = \emptyset$ - таке неможливо через те, що $x_0$ -- гранична точка. Суперечність!
\bigskip \\
$\boxed{2) \Rightarrow 3)}$ Дано: $\forall \varepsilon > 0: B(x_0;\varepsilon) \cap A$ - нескінченна множина. Встановимо $\varepsilon = \dfrac{1}{n}$. Тоді оскільки $\forall n \geq 1: B \left(x_0;\dfrac{1}{n} \right) \cap A$ -- нескінченна, то $\forall n \geq 1: \exists x_n \in A: \rho(x_0,x_n) < \dfrac{1}{n}$. Якщо далі $n \to \infty$, тоді $\rho(x_0,x_n) \to 0$. Остаточно, $\exists \{x_n, n \geq 1\} \subset A: x_n \neq x_0: x_n \to x_0$.
\bigskip \\
$\boxed{3) \Rightarrow 1)}$ Дано: $\exists \{x_n, n \geq 1\} \subset A: x_n \neq x_0: x_n \to x_0$. Тобто $\forall \varepsilon > 0: \exists N: \forall n \geq N: \rho(x_0,x_n) < \varepsilon$. Або, інакше кажучи, $\forall \varepsilon > 0: x_N \in B(x_0;\varepsilon) \cap A$. Тоді $\forall \varepsilon > 0: (B(x_0;\varepsilon) \setminus \{x_0\}) \cap A \neq \emptyset$.
\end{proof}

\subsection{Замикання множин}
\begin{definition}
Задано $(X,\rho)$ -- метричний простір, $A \subset X$ та $A'$ - множина граничних точок $A$.\\
\textbf{Замиканням} множини $A$ називають таку множину
\begin{align*}
\bar{A} = A \cup A'
\end{align*}
Часто ще позначають замикання за $\Cl(A)$.
\end{definition}

\begin{example}
Маємо $X = \mathbb{R}$, $\rho(x,y) = |x-y|$ та множину $A = (0,1)$. Тоді множина $A' = [0,1]$. Замикання $\bar{A} = A \cup A' = [0,1]$.
\end{example}

\begin{remark}
Розглянемо зараз сукупність замкнених множин $A \subset A_{\alpha} \subset X$. Перетин $B = \displaystyle\bigcap_{\alpha} A_{\alpha}$ -- також замкнена, водночас $A\alpha \supset B \supset A$. Отже, $B$ -- найменша замкнена множина, що містить $A$.
\end{remark}

\begin{proposition}
Задано $\bar{A}$ -- замикання. Тоді спредливе наступне:
\begin{enumerate}[nosep,wide=0pt,label={\arabic*)}]
\item $\bar{A}$ -- найменша замкнена множина, що містить $A$;
\item $\overline{A \cup B} = \bar{A} \cup \bar{B}$ \qquad $\overline{A \cap B} \subset \bar{A} \cap \bar{B}$;
\item $A$ -- замкнена $\iff A = \bar{A}$.
\end{enumerate}
\end{proposition}

\begin{proof}
Доведемо кожне твердження окремо.
\begin{enumerate}[wide=0pt, label={\arabic*)}]
\item !Припустимо, що $\bar{A}$ не є найменшою замненою, що містить $A$, тобто $\exists B \subset \bar{A}: B \supset A$ -- замкнена. Зафіксуємо точку $x_0 \in \bar{A}$ -- гранична, тоді $x_0 \in A' \cup A$.\\
Якщо $x_0 \in A'$, то тоді $x_0 \in B$, тому що $B$ містить всі граничні т. $A$\\
Якщо $x_0 \in A$, то тоді $x_0 \in B$.\\
В обох випадках $\bar{A} \subset B$. Отже, $\bar{A} = B$. Суперечність!

\item $\overline{A \cup B} = (A \cup B)' \cup (A \cup B) \boxed{=}$\\
$x_0 \in (A \cup B)' \iff x_0$ -- гранична точка $A \cup B \iff$ $\forall \varepsilon > 0: \\ B(x_0;\varepsilon) \cap (A \cup B) = (B(x_0;\varepsilon) \cap A) \cup (B(x_0; \varepsilon) \cap B) \neq \emptyset \text{(без т. } x_0) \iff$\\
$\left[ \begin{gathered} x_0 - \text{гранична для } A \\ x_0 - \text{гранична для } B \end{gathered} \right. \iff \left[ \begin{gathered} x_0 \in A' \\ x_0 \in B' \end{gathered} \right. \iff x_0 \in A' \cup B'$\\
Отже, $(A \cup B)' = A' \cup B'$.\\
$\boxed{=} A' \cup B' \cup A \cup B = \bar{A} \cup \bar{B}$.
\\
$\overline{A \cap B} = (A \cap B)' \cup (A \cap B) \boxed{\subset}$\\
$x_0 \in (A \cap B)' \iff x_0$ - гранична точка $A \cap B \iff$ $\forall \varepsilon > 0: \\ B(x_0;\varepsilon) \cap (A \cap B) = (B(x_0;\varepsilon) \cap A) \cap (B(x_0; \varepsilon) \cap B) \neq \emptyset \text{(без т. } x_0) \implies$\\
$\begin{cases} x_0 - \text{гранична для } A \\ x_0 - \text{гранична для } B \end{cases} \iff \begin{cases} x_0 \in A' \\ x_0 \in B' \end{cases} \iff x_0 \in A' \cap B'$\\
Отже, $(A \cap B)' \subset A' \cap B'$.\\
$\boxed{\subset} (A' \cap B') \cup (A \cap B) = $ (TODO: обміркувати)

\item Доведення в обидва боки.\\
\rightproof Дано: $A$ - замкнена. Тоді $A$ містить всі свої граничні точки. Так само $A'$ містить граничні точки $A$. Тому $A = \bar{A}$.\\
\leftproof Дано: $A = \bar{A}$. Тобто $A$ містить всі свої граничні точки. Отже, $A$ - замкнена.
\end{enumerate}
Всі твердження доведені.
\end{proof}

\begin{definition} Задано $(X, \rho)$ -- метричний простір та $A \subset X$.\\
Множина $A$ називається \textbf{щільною} в $X$, якщо
\begin{align*}
\bar{A} = X
\end{align*}
\end{definition}

\begin{definition}
Задано $(X, \rho)$ -- метричний простір.\\
Метричний простір називається \textbf{сепарабельним}, якщо
\begin{align*}
\text{існує в даному просторі скінченна чи зліченна щільна підмножина.}
\end{align*}
\end{definition}

\begin{example} Розглянемо такі приклади:
\begin{enumerate}[wide=0pt,label={\arabic*)}]
\item $(\mathbb{R}, \rho(x,y) = |x-y|)$ -- сепарабельний, тому що $\mathbb{Q}$ -- зліченна та щільна підмножина в $\mathbb{R}$.
\item Маємо простір $l_2 = \left\{ \vec{a} \mid \displaystyle\sum_{n=1}^\infty a_n^2 < \infty \right\}$ -- нормований простір. Розглянемо множину \\
$l_2O = \left\{ \vec{a} \in l_2 \mid \text{скінченна кількість членів не нуль} \right\}$. Розглянемо $\vec{a} = \{a_1,a_2,\dots\} \in l_2$. Доведемо, що вона -- гранична для $l_2O$.\\
Задамо послідовність $\{\vec{a}_n, n \geq 1\} \subset l_2O$, де кожний елемент задається таким чином: \\ $\vec{a}_n = \{a_1,\dots,a_n,0,\dots\} \implies \rho(\vec{a}, \vec{a}_n) = \|\vec{a} - \vec{a}_n\| = \displaystyle\sum_{n=k+1}^\infty a_n^2 \to 0$, оскільки ряд збіжний, а тому хвіст ряду прямує до нуля. Отже, $\vec{a}_n \to \vec{a}$, тож $\vec{a}_n$ -- гранична точка.\\
Тоді можна ствердити, що $l_2O$ -- щільна в $l_2$, або інакше $\overline{l_2O} = l_2$. А оскільки $l_2O \subset l_2$ та ще й нескінченна, то тоді $l_2$ -- сепарабельний.
\item Простір $C([a,b])$ -- сепарабельний.\\
\textit{Доведу пізніше, коли дізнаюсь про теорему Вейєрштрасса про наближення неперервної на відрізку функції многочленами}.
\item А ось простір $l_{\infty}$ - не сепарабельний.\\
\textit{Доведу пізніше}.
\item Підпростір сепарабельного метричного простору - сепарабельний\\
\textit{Доведу пізніше}.
\end{enumerate}
\end{example}

\subsection{Повнота}
\begin{definition}
Задано $(X,\rho)$ - метричний простір.\\
Послідовність $\{x_n, n \geq 1\}$ називається \textbf{фундаментальною}, якщо
\begin{align*}
\forall \varepsilon > 0: \exists N: \forall m,n \geq N: \rho(x_n,x_m) < \varepsilon
\end{align*}
\end{definition}

\begin{remark}
Це означення можна інакше переписати, більш коротким чином:
\begin{align*}
\rho(x_n,x_m) \overset{m,n \to \infty}{\longrightarrow} 0
\end{align*}
\end{remark}

\begin{proposition}
Будь-яка збіжна послідовність є фундаментальною.
\end{proposition}

\begin{proof}
Маємо $\{x_n, n \geq 1\}$ -- збіжна, тобто $\rho(x_n,x) \overset{n \to \infty}{\longrightarrow} 0$. За нерівністю трикутника, маємо $\rho(x_n,x_m) \leq \rho(x_n,x) + \rho(x,x_m)$. Якщо спрямувати одночасно $m.n \to \infty$, то тоді $\rho(x_n,x_m) \to 0$. Отже, $\{x_n, n \geq 1\}$ - фундаментальна.
\end{proof}

\begin{remark}
Проте не кожна фундаментальна послідовність -- збіжна.
\end{remark}

\begin{example}
Маємо $X = (0,1]$ -- підпростір $\mathbb{R}$. Розглянемо послідовність $\left\{ x_n = \dfrac{1}{n}, n \geq 1 \right\}$, де $x_n \to 0$ при $n \to \infty$ -- збіжна, проте $0 \not\in X$. Тому така послідовність не має границі в $X$, але вона -- фундаментальна за твердженням.
\end{example}

\begin{definition}
Метричний простір $(X, \rho)$ називається \textbf{повним}, якщо 
\begin{align*}
\text{будь-яка фундаментальна послідовність має границю.}
\end{align*}
\end{definition}

\begin{example} 
Зокрема маємо наступне:
\begin{enumerate}[nosep,wide=0pt,label={\arabic*)}]
\item $X = \mathbb{R}$ -- повний за критерієм Коші із матану;\\
\item $X = (0,1]$ -- не повний, бо принаймні $\left\{x_n = \dfrac{1}{n}, n \geq 1 \right\}$ -- фундаментальна, проте не має границі.
\end{enumerate}
\end{example}

\begin{proposition}
Задано $(X,\rho)$ -- повний метричний простір та $(Y,\rho)$ -- підпрострір.\\
$(Y,\rho)$ -- повний $\iff Y$ -- замкнена в $X$.
\end{proposition}

\begin{proof}
\rightproof Дано: $(Y,\rho)$ -- повний.\\
!Припустимо, що $Y$ -- не замкнена, тобто існує $x_0 \in X \setminus Y$ -- гранична точка для $Y$. Тоді існує послідовність $\{y_n\} \subset Y$, для якої $y_n \to x_0$ та $y_n \neq x_0$. Зауважимо, що $\{y_n\} \subset X$ збіжна саме в просторі $X$, тому саме в просторі $X$ послідовність $\{y_n\} \subset X$ -- фундаментальна. Проте зрозумло цілком, що $\{y_n\} \subset Y$ буде фундаментальною в просторі $Y$, проте в силу повноти $(Y,\rho)$, матимемо збіжність саме в $Y$. Таким чином, $x_0 \in Y$ -- суперечність!
\bigskip \\
\leftproof Дано: $Y$ -- замкнена в $X$. Візьмемо $\{y_n\} \subset Y \subset X$ -- фундаментальна. Тоді в силу повноти $X$, вона -- збіжна в просторі $X$. Скажімо, $y_n \to x_0$. Якщо точка $x_0 \in Y$, то тоді послідовність $\{y_n\}$ збіжна в $Y$. Інакше при $x_0 \in X \setminus Y$ зауважимо, що $y_n \neq x_0$, тому $x_0$ -- гранична точка $Y$. У силу замкненості ми отримаємо $x_0 \in Y$ -- послідовність $\{y_n\}$ знову збіжна в $Y$.
\end{proof}

\begin{definition}
Повний нормований простір називається \textbf{банаховим}. Повний евклідів простір (відносно метрики, що породжена скалярним добутков) називається \textbf{гільбертовим}.
\end{definition}

\begin{proposition}
Простір $C([a,b])$ зі стандартною нормою $\|x\| = \displaystyle\max_{t \in [a,b]} |x(t)|$ -- банахів.
\end{proposition}

\begin{proof}
Задамо фундаментальну послідовність $\{x_n, n \geq 1\}$ на множині $C([a,b])$. Тоді \\ $\forall t_0 \in [a,b]: |x_n(t_0)-x_m(t_0)| \leq \|x_n-x_m\| = \displaystyle\max_{t \in [a,b]} |x_n(t)-x_m(t)|$. Із цієї нерівності випливає, що $\forall t_0 \in [a.b]: \{x_n(t_0), n \geq 1\}$ -- фундаментальна.\\
За критерієм Коші (із матану), вона -- збіжна, тобто $x_n(t_0) \overset{n \to \infty}{\to} y(t_0)$. Щойно показали поточкову збіжність $\{x_n, n \geq 1\}$ до функції $y$. Доведемо, що вона збігається рівномірно (тобто за нормою).\\
$\{x_n, n \geq 1\}$ -- фундаментальна, тобто $\forall \varepsilon > 0: \exists N: \forall m,n \geq N: \|x_n(t) - x_m(t)\| < \varepsilon$. Або $\forall t \in [a,b]: |x_n(t) - x_m(t)| < \varepsilon$. Зафіксуємо деякі $t \in [a,b]$ та $n \geq N$. А потім спрямуємо $m \to \infty$. Тоді $|x_n(t)-y(t)| < \varepsilon$. Це виконується $\forall t \in [a,b]$ та $n \geq N$, або це записується інакше:\\
$\forall n \geq N: \|x_n - y\| < \varepsilon$. Отже, $x_n \to y$.
\end{proof}

\begin{example}
Задамо підпростір $C([0,1])$ із нормою із $L_2([0,1],\lambda_1)$, де $\lambda_1$ -- міра Лебега. Доведемо, що в такому разі $C([0,1])$ уже не буде банаховим.\\
Розглянемо таку функціональну послідовність $\{x_n,n \geq 1\} \subset C([0,1])$, що задається таким чином:\\
$x_n(t) \begin{cases}
0, & 0 \leq x \leq \dfrac{1}{2} - \dfrac{1}{n} \\
\dfrac{nx}{2} - \dfrac{n}{4} + \dfrac{1}{2}, & \dfrac{1}{2} - \dfrac{1}{n} \leq x \leq \dfrac{1}{2} + \dfrac{1}{n} \\
1, & \dfrac{1}{2} + \dfrac{1}{n} \leq x \leq 1
\end{cases}$.\\
Це набір функцій, де похила частина зі збільшенням $n$ перетворюється в вертикальну лінію. Зауважимо, що якщо вязти поточкову границю, то отримаємо $x(t) = \begin{cases} 0, & 0 \leq x \leq \dfrac{1}{2} \\1, & \dfrac{1}{2} < x \leq 1 \end{cases}$. При цьому\\
$\|x_n - x\|_2^2 = \displaystyle\int_{[0,1]} |x_n-x|^2\,d\lambda_1 = \int_0^1 |x_n(t)-x(t)|^2\,dt = \dots = \dfrac{1}{6n} \to 0$ при $n \to \infty$.\\
Отже, $\{x_n\}$ в просторі $C([0,1])$ із нормою $L_2$ збігається до точки $x \notin C([0,1])$, але при цьому буде граничною для $C([0,1])$. Тобто $C([0,1])$ не буде замкненим, тож $C([0,1])$ -- не повний, або не банахів.
\end{example}

\begin{proposition}
Евклідів простір $l_2$ -- гільбертів.
\end{proposition}

\begin{proof}
Задамо фундаментальну послідовність $\{\vec{x}_n, n \geq 1\}$ на множині $l_2$\\
Тобто $\forall \varepsilon > 0: \exists N: \forall n, m \geq N: \|\vec{x}_n - \vec{x}_m\| < \varepsilon$\\
$\Rightarrow \|\vec{x}_n - \vec{x}_m\|^2 = \displaystyle\sum_{k=1}^\infty (x_n^k - x_m^k)^2 < \varepsilon^2 \Rightarrow \forall k \geq 1: |x_n^k - x_m^k| < \varepsilon$\\
Тоді послідовність $\{x_n^k, n \geq 1\}$ - фундаментальна - тому (за матаном) збіжна, $x_n^k \to y^k$\\
Доведемо, що $\vec{x}$ збігається до $\vec{y}$ за нормою\\
Маємо $\displaystyle\sum_{k=1}^\infty (x_n^k - x_m^k)^2 < \varepsilon^2 \Rightarrow \forall K \geq 1: \displaystyle\sum_{k=1}^K (x_n^k - x_m^k)^2 < \varepsilon^2$\\
Спрямуємо $m \to \infty$, тоді $\displaystyle\sum_{k=1}^K (x_n^k - y^k)^2 < \varepsilon^2$\\
Звідки випливає збіжність ряду $\displaystyle\sum_{k=1}^\infty (x_n^k - y^k)^2$ та його оцінка\\
$\displaystyle\sum_{k=1}^\infty (x_n^k - y^k)^2 < \varepsilon^2 \Rightarrow \|\vec{x}_n - \vec{y}\| < \varepsilon$\\
Отже, $\vec{x}_n \to \vec{y}$
\end{proof}

\begin{lemma}
Задано $\{x_n, n \geq 1\}$ - фундаментальна та $\{x_{n_k}, k \geq 1\}$ - збіжна. Тоді $\{x_n, n \geq 1\}$ - збіжна
\end{lemma}
\begin{pf}
Маємо $a_{n_k} \to a$, $k \to \infty$\\
$\Rightarrow \forall \varepsilon > 0: \exists K: \forall k \geq K: \rho(a_{n_k}, a) < \varepsilon$\\
Також відомо, що  $\forall n,m \geq N: \rho(a_n,a_m) < \varepsilon$\\
Треба ще $n_k \geq N$. Тоді для $n \geq n_K$\\
$\rho(a_n,a) \leq \rho(a_n,a_{n_K}) + \rho(a_{n_K},a) < 2\varepsilon$\\
Отже, $a_n \to a_0$
\end{pf}

\begin{theorem}[Критерій Кантора]
Умова Кантора: для кожної послдовності $\{B[a_n;r_n], n \geq 1\}$ такої, що $B[a_1;r_1] \supset B[a_2;r_2] \supset \dots$ та $r_n \to 0$, перетин $\displaystyle\bigcap_{n=1}^\infty B[a_n;r_n] \neq \emptyset$ (це послідовність куль, що стягується).\\
$(X,\rho)$ -- повний $\iff$ виконується умова Кантора.
\end{theorem}

Перед доведенням треба зробити кілька зауважень.\\
I. Точка, що належить перетину, буде в цьому випадку єдиною.\\
!Припустимо, що це не так, тобто $\exists b^*, b^{**} \in \displaystyle \bigcap_{n=1}^\infty B[a_n;r_n]$. Тоді $\forall n \geq 1: \begin{cases} \rho(a_n, b^*) \leq r_n \\ \rho(a_n, b^{**}) \leq r_n \end{cases}$.\\
$\implies \rho(b^*, b^{**}) \leq \rho(b^*,a_n) + \rho(a_n, b^{**}) \leq r_n + r_n = 2r_n$.\\
Спрямуємо $n \to \infty$, тоді $\rho(b^*,b^{**}) \leq 0 \implies \rho(b^*,b^{**}) = 0 \implies b^{*} = b^{**}$. Суперечність!
\bigskip \\
II. Покажемо, що $\{a_n, n \geq 1\}$ -- послідовність центрів -- фундаментальна.\\
За умовою, $r_n \to 0 \implies \forall \varepsilon > 0: \exists N: \forall n \geq N: r_n < \varepsilon$. Достатньо взяти лише $r_N < \varepsilon$. Тоді $\forall n,m \geq N: a_m,a_n \in B[a_N,r_N] \implies \rho(a_m,a_N) < r_N$ та $\rho(a_n,a_N) < r_N$.\\
$\implies \rho(a_n,a_m) \leq \rho(a_n,a_N) + \rho(a_N,a_m) < 2r_N < 2 \varepsilon$. Отже, $\{a_n, n \geq 1\}$ - фундаментальна.

\begin{proof}
\rightproof Дано: $(X,\rho)$ - повний. Задамо послідовність куль $\{B[a_n;r_n], n \geq 1\}$, що стягується. Тоді послідовність $\{a_n, n \geq 1\}$ -- фундаментальна. Оскільки $X$ - повний, то тоді $\{a_n, n \geq 1\}$ -- збіжна, тобто $a_n \to a_0$. Оскільки $B[a_n;r_n]$ -- замкнені, то маємо, що $a_0 \in B_n$. Звідси $a_0 \displaystyle \in \bigcap_{n=1}^\infty B_n$.
\bigskip \\
\leftproof Дано: умова Кантора. Нехай $\{a_n, n \geq 1\}$ -- фундаментальна послідовність. Нам достатньо буде у неї взяти збіжну підпослідовність. Нехай маємо $n_1 \in \mathbb{N}$, щоб $\forall n \geq n_1: \rho(a_n,a_{n_1}) < \dfrac{1}{2}$.\\
Тоді $\exists n_2 > n_1: \forall n \geq n_2: \rho(a_n,a_{n_2}) < \dfrac{1}{4}$\\
\vdots \\
Тоді маємо послідовність $n_1 < n_2 < n_3 < \dots$ із властивістю $\forall n \geq n_k: \rho(a_n,a_{n_k}) < \dfrac{1}{2^k}$.\\
Маємо тоді кулі $B\left[a_{n_k}; \dfrac{1}{2^{k-1}} \right]$, що вкладені. Дійсно, $x \in B\left[a_{n_{k+1}}; \dfrac{1}{2^{k}} \right] \implies$\\
$\rho(a_{n_k}, x) \leq \rho(a_{n_k}, a_{n_{k+1}}) + \rho(a_{n_{k+1}},x) \leq \dfrac{1}{2^{k-1}} \implies x \in B\left[a_{n_k}; \dfrac{1}{2^{k-1}} \right]$\\
Якщо $a$ -- спільна точка куль, то $a_{n_k} \to a$.
\end{proof}

\begin{definition}
Задано $(X,\rho)$ та $(Y, \tilde{\rho})$ -- два різних метричних простори.\\
Відображення $f \colon X \to Y$ називається \textbf{ізометрією}, якщо
\begin{align*}
\forall x_1,x_2 \in X: \tilde{\rho}(f(x_1),f(x_2)) = \rho(x_1,x_2)
\end{align*}
Тобто суть ізометрії -- це збереження відстаней.
\end{definition}

\begin{remark}
Кожна ізометрія $f$ -- уже автоматично ін'єктивна.\\
Дійсно, припустимо, що $f(x_1) = f(x_2)$. За визначенням ізометрії, $\tilde{\rho}(f(x_1),f(x_2)) = \rho(x_1,x_2)$. Отримаємо $\rho(x_1,x_2) = 0$, тобто $x_1 = x_2$.
\end{remark}

\begin{definition}
Метричні простори $(X,\rho), (Y,\tilde{\rho})$ називаються \textbf{ізометричними}, якщо
\begin{align*}
\exists f \colon X \to Y \text{ -- бієктивна ізометрія}
\end{align*}
\end{definition}

\begin{example}
Розглянемо $(\mathbb{R}, \tilde{\rho})$ та $\left( \left( -\dfrac{\pi}{2}, \dfrac{\pi}{2} \right), \rho \right)$ -- два метричних простори. У цьому випадку $\rho$ -- стандартна метрика та $\tilde{\rho}(x,y) = |\arctg x - \arctg y|$. Ці два простори -- ізометричні.\\
Дійсно, між ними існує ізометрія $\arctg \colon \mathbb{R} \to \left( -\dfrac{\pi}{2}, \dfrac{\pi}{2} \right)$, що є бієктивною.
\end{example}

\begin{proposition}
Задані $(X,\rho), (Y, \tilde{\rho})$ -- два ізоморфні метричні простори.\\
$(X,\rho)$ -- повний $\iff (Y,\tilde{\rho})$ -- повний.
\end{proposition}

\begin{proof}
\rightproof Дано: $(X,\rho)$ -- повний. Нехай $\{y_n, n \geq 1\}$ -- фундаментальна послідовність. Оскільки $X,Y$ ізометричні, то існує бієкція $f \colon X \to Y$, що є ізометрією. Тож звідси $\exists !x_n \in X: f(x_n) = y_n$. Розглянемо послідовність $\{x_n, n \geq 1\}$ та зауважимо, що $\rho(x_n,x_m) = \tilde{\rho}(y_n,y_m) \to 0$ в силу фундаментальності. Отже, $\{x_n, n \geq 1\}$ -- фундаментальна, тож збіжна за повнотою. Тобто $\rho(x_n,x) \to 0$. Позначимо $f(x) = y$. Звідси випливає, що $\tilde{\rho}(y_n,y) = \rho(x_n,x) \to 0$. Тобто $\{y_n, n \geq 1\}$ -- збіжна.
\bigskip \\
\leftproof \textit{зеркальне доведення.}
\end{proof}

\begin{definition}
Задані $(X,\| \cdot \|_1)$ та $(X,\| \cdot \|_2)$ -- два нормовані простори.
Ці два простори називаються \textbf{ізометричними}, якщо
\begin{align*}
\exists L \colon X \to Y \text{ -- ізоморфізм між просторами}: \|Lx\|_2 = \|x\|_1
\end{align*}
\end{definition}

\begin{remark}
Ізоморфізм $L$ -- автоматично ізометрія, це випливає зі збереження норми. Саме тому слово 'ізометричні' в означенні вище виправдане.
\end{remark}

\begin{definition}
Задано $Y$ -- повний метричний простір.\\
Він буде називатися \textbf{поповненням (completion)} метричного простору $X$, якщо
\begin{align*}
X \text{ -- ізометричний підпростір } Y; \\
X \text{ -- щільна в } Y.
\end{align*}
\end{definition}

\begin{theorem}
Для кожного метричного простору $(X,\rho)$ існує поповнення. Причому це поповнення єдине з точністю до ізометрії.
\end{theorem}

\begin{proof}
І. \textit{Існування.}\\
Позначимо $F$ за множина фундаментальних послідовностей $\{x_n\}$ в $X$. Стаціонарні послідовності є фундаментальними, тож звідси $X$ можна сприймати як підмножину $F$.\\
Розглянемо функцію $d(\{x_n\},\{y_n\}) = \displaystyle\lim_{n \to \infty} \rho(x_n,y_n)$, яка визначена на $F \times F$. Для коректності треба довести існування даної границі. Ми доведемо, що $\{ \rho(x_n,y_n), n \geq 1\}$ -- фундаментальна (це числова послідовність, тому цього буде достатньо).\\
Нам відомо, що $\{x_n\}, \{y_n\}$ фундаментальні, тобто $\exists N_1, N_2$, для яких $\rho(x_n,x_m) < \varepsilon, \rho(y_n,y_m) < \varepsilon$ для всіх $n,m \geq N_1, m,n \geq N_2$. Тоді при $N = \max \{N_1,N_2\}$ справедлива оцінка:\\
$|\rho(x_n,y_n) - \rho(x_m,y_m)| \leq \rho(x_n,y_n) + \rho(x_m,y_m) \leq (\rho(x_n,x_m) + \rho(x_m,y_m) + \rho(y_m,y_n)) - \rho(x_m,y_m) < 2 \varepsilon$.\\
Отже, функція $d$ визначена коректно. Вона майже метрика, оскільки (легко перевірити) виконуються всі властивості. На жаль, $d(\{x_n\},\{y_n\}) = 0 \centernot\implies \{x_n\} = \{y_n\}$ (приклад буде нижче).\\
Створимо відношення еквівалентності $\{x_n\} \sim \{y_n\} \iff d(\{x_n\},\{y_n\}) = 0$. Утвориться фактормножина $F/_{\sim} = \hat{F}$. Елементи з $\hat{F}$ позначатимемо за $\overline{\{x_n\}}$. Наша мета буде довести, що саме $\hat{F}$ буде поповненням $X$. \\
На фактормножині покладемо $\tilde{\rho}\left(\overline{\{x_n\}},\overline{\{y_n\}}\right) = d(\{x_n\},\{y_n\})$. Варто пересвідчитися, що воно визначено коректно.\\
Нехай $\{x_n\} \sim \{x_n'\}$ та $\{y_n\} \sim \{y_n'\}$. Тобто $d(\{x_n\},\{x_n'\}) = 0$ та $d(\{y_n\},\{y_n'\}) = 0$. Тоді\\
$d(\{x_n\},\{y_n\}) = \displaystyle\lim_{n \to \infty} \rho(x_n,y_n) \leq \lim_{n \to \infty} \rho(x_n,x_n') + \lim_{n \to \infty} \rho(x_n',y_n') + \lim_{n \to \infty} \rho(y_n',y_n) = d(\{x_n'\},\{y_n'\})$.\\
Аналогічно отримаємо $d(\{x_n'\},\{y_n'\}) \leq d(\{x_n\},\{y_n\})$. Отже, $d(\{x_n'\},\{y_n'\}) = d(\{x_n\},\{y_n\})$, тобто $\tilde{\rho}$ визначилося коректним чином.\\
Поставимо відображення $f \colon X \to \hat{F}$ таким чином: $f(x) = \overline{\{x\}}$. Це буде ізометрією, тому що\\
$\tilde{\rho}(f(x_1),f(x_2)) = \tilde{\rho}( \overline{\{x_1\}},\overline{\{x_2\}} ) = d( \{x_1\}, \{x_2\} ) = \displaystyle \lim_{n \to \infty} \rho(x_1,x_2) = \rho(x_1,x_2)$. Відображення $f$ зобов'язане бути сюр'єктивним, оскільки повертається клас еквівалентності. Тобто $f$ -- бієктивна ізометрія, а тому $(X,\rho), (\hat{F},\tilde{\rho})$ -- ізометричні.\\
Покажемо, що $(\hat{F},\tilde{\rho})$ -- повний метричний простір. (TODO: обміркувати).
\bigskip \\
II. \textit{Єдиність.}\\
Розглянемо два поповнення $(Y_1,\tilde{\rho}_1), (Y_2,\tilde{\rho}_2)$ простору $(X,\rho)$. Тобто, за означенням, маємо $Y_1 \supset X_1 \sim X \sim X_2 \subset Y_2$, а також $\overline{X_1} = Y_1, \overline{X_2} = Y_2$. Під $\sim$ мається на увазі ізометричність. Із цього $X_1$ ізометричний до $X_2$, нехай $g$ -- відповідна ізометрія.\\
Побудуємо $f \colon Y_1 \to Y_2$ за таким правилом: для кожного $y \in Y_1$ беремо таку послідовність $\{x_n\} \subset X_1$, щоб $x_n \to y$ -- тоді $f(y) = \displaystyle\lim_{n \to \infty} g(x_n)$. Треба пересвідчитися, що визначення коректне. Дійсно, нехай $\{x_n\}, \{x_n'\}$ -- такі дві послідовності, що $x_n \to y, x_n' \to y$. Тоді звідси вилпиває наступне:\\
$\tilde{\rho}_2 (g(x_n),g(x_n')) \overset{\text{ізометричність}}{=} \tilde{\rho}_1(x_n,x_n') \leq \tilde{\rho}_1(x_n,y) + \tilde{\rho}_2(y,x_n') \to 0$ при $n \to \infty$.\\
Таким чином, $\displaystyle\lim_{n \to \infty} g(x_n) = \lim_{n \to \infty} g(x_n')$, а тому значення функцій коректно визначено. (TODO: подумати над тим, чи правильно я все це розписав).
\end{proof}

\begin{example}
Беремо стандартний метричний простір $\mathbb{R}$, послідовності $\{x_n\} = \{0.9, 0.99, 0.999, \dots\}$ та $\{y_n\} = \{1,1,1,\dots\}$. Зауважимо, що $d(\{x_n\},\{y_n\}) = \displaystyle\lim_{n \to \infty} \rho(x_n,y_n) = \lim_{n \to \infty} 0.\underset{n \text{ цифр}}{00\dots 01} = 0$. При цьому зрозуміло, що $\{x_n\} \neq \{y_n\}$.
\end{example}

\subsection{Неперервні відображення}
\begin{definition}
Задані $(X,\rho), (Y,\tilde{\rho})$ -- два метричних простори.\\
Відображення $f \colon X \to Y$ називається \textbf{неперервним у точці $x_0$}, якщо
\begin{align*}
\forall \varepsilon > 0: \exists \delta > 0: \forall x \in X: \rho(x,x_0) < \delta \implies \tilde{\rho}(f(x),f(y)) < \varepsilon
\end{align*}
\end{definition}

\begin{remark}
Дане означення можна записати більш компактним чином. Маємо $f \colon X \to Y$.\\
$f$ -- неперервне в точці $x_0 \in X \iff \forall \varepsilon > 0: \exists \delta > 0: f(B(x_0;\delta)) \subset B(f(x_0);\varepsilon)$.
\end{remark}

\begin{proposition}
Задані $(X,\rho), (Y,\tilde{\rho})$ -- два метричних простори та $f \colon X \to Y$.\\
$f$ -- неперервне в точці $x_0 \in X \iff \forall \{x_n\} \subset X: x_n \to x_0 \text{ в } X \implies f(x_n) \to f(x_0) \text{ в } Y$.\\
\textit{Вправа: довести.}
\end{proposition}

\begin{theorem}
Задані $(X,\rho), (Y,\tilde{\rho})$ -- два метричних простори та $f \colon X \to Y$.\\
$f$ -- неперервне (на множині $X$) $\iff \forall V$ -- замкнена в $Y: f^{-1}(U)$ -- замкнена в $X$.
\end{theorem}

\begin{proof}
\rightproof Дано: $f$ -- неперервне. Нехай $V$ -- замкнена в $Y$. Зафіксуємо $x_n \in f^{-1}(V)$ таким чином, що $x_n \to x_0$. Але за неперервністю, $f(x_n) \to f(x_0)$, та додатково $f(x_n) \in V$. Значить, за замкненістю $V$, точка $f(x_0) \in V \implies x_0 \in f^{-1}(V)$. Отже, $f^{-1}(V)$ -- замкнена.
\bigskip \\
\leftproof Дано: $\forall V$ -- замкнена в $Y: f^{-1}(U)$ -- замкнена в $X$. Оберемо $x_n \to x_0$.\\
!Припустимо, що $f(x_n) \not\to f(x_0)$, тобто існує шар $B(f(x_0);\varepsilon)$, поза яким знаходиться підпослідовність $\{f(x_{n_k})\}$. Якщо $V$ -- замикання множини $\{f(x_{n_k})\}$, то звідси $x_{n_k} \in f^{-1}(V)$; $f(x_0) \notin V$. Тоді звідси $x_0 \notin f^{-1}(V)$, проте $x_{n_k} \to x_0$ та $x_0$ є граничною точкою для $f^{-1}(A)$. Суперечність!
\end{proof}

\begin{corollary}
$f$ -- неперервне $\iff \forall U$ -- відкрита в $Y: f^{-1}(U)$ -- відкрита в $X$.\\
\textit{Вказівка: застосувати попередню теорему та рівність $f^{-1}(A^c) = (f^{-1}(A))^c$.}
\end{corollary}

\begin{proposition}
Задані $X,Y,Z$ -- метричні простори та $f \colon X \to Y, g \colon Y \to Z$. Нехай $f$ -- неперервне в точці $x_0 \in X$ та $g$ -- неперервне в точці $f(x_0) \in Y$. Тоді $g \circ f$ -- неперервне в точці $x_0 \in X$.\\
\textit{Вправа: довести.}
\end{proposition}

\begin{proposition}
Задано $(X,\rho)$ -- метричний простір та зафіксуємо $x_0 \in X$. Тоді функція $f(x) = \rho(x,x_0)$, де $f \colon X \to \mathbb{R}$, -- неперервна на $X$.
\end{proposition}

\begin{proof}
Дійсно, нехай $y_0 \in X$. Припустимо, що $\{y_n\}$ така, що  $y_n \to y_0$. Хочемо $f(y_n) \to f(y_0)$. Справді,\\
$|f(y_n) - f(y_0)| = |\rho(y_n,x_0) - \rho(y_0,x_0)| \leq |\rho(y_n,y_0)| \to 0$.\\
\textit{Для $\mathbb{R}$ береться стандартна метрика, якщо нічого іншого не вказується зазвичай.}
\end{proof}

\begin{corollary}
Задано $(L, \|\cdot \|)$ -- нормований простір. Тоді норма $\| \cdot \| \colon L \to \mathbb{R}$ -- неперервна.\\
\textit{Вказівка: оскільки $\rho(x,y) = \|x-y\|$, то звідси $\|x\| = \rho(x,0)$.}
\end{corollary}

\begin{corollary}
Задано $(E, (\cdot,\cdot))$ -- евклідів простір. Тоді при фіксованому $x_0 \in E$ маємо $(x,x_0)$ -- неперервне відображення.
\end{corollary}

\begin{proof}
Нехай $\{y_n\}$ задана так, що $y_n \to y_0$. Хочемо довести, що $(y_n,x_0) \to (y_0,x_0)$.\\
$|(y_n,x_0) - (y_0,x_0)| = |(y-y_0,x_0)| \leq \sqrt{\|y-y_0\|} \sqrt{\|x_0\|} \to 0$, оскільки $\| \cdot \|$ -- неперервне.
\end{proof}

\begin{definition}
Задано $(X,\rho)$ -- метричний простір та $f \colon X \to X$.\\
Дане відображення називається \textbf{стиском}, якщо
\begin{align*}
\exists q \in (0,1): \forall x,y \in X: \rho(f(x),f(y)) \leq q \cdot \rho(x,y)
\end{align*}
\end{definition}

\begin{remark}
Стискаючі відображення -- неперервні.\\
\textit{Вказівка: обрати $\delta = \dfrac{q}{\varepsilon}$ при всіх $\varepsilon > 0$.}
\end{remark}

\begin{theorem}[Теорема Банаха]
Задано $(X,\rho)$ -- повний метричний просторі та $f \colon X \to X$ -- стискаюче відображення. Тоді існує єдина точка нерухома точка, тобто $\exists ! x \in X: f(x) = x$.
\end{theorem}

\begin{proof}
І. \textit{Існування}.\\
Нехай $x_0 \in X$ -- довільна точка. Зробимо позначення: $x_1 = f(x_0),\ x_2 = f(x_1),\ \dots, x_n = f(x_{n-1}), \dots$ Покажемо, що послідовність $\{x_n, n \geq 0\}$ -- фундаментальна. Дійсно, для $m \leq n$ маємо:\\
$\rho(x_m, x_n) = \rho(f(x_{m-1}), f(x_{n-1})) \leq q \cdot \rho(x_{m-1},x_{n-1}) \leq \dots \leq q^m \rho(x_0,x_{n-m})$.\\
$\rho(x_0,x_{n-m}) \leq \rho(x_0,x_1) + \rho(x_1,x_2) + \dots + \rho(x_{n-m-1}, x_{n-m}) \leq \rho(x_0,x_1)(1+q+\dots+q^{n-m-1}) \leq \\
\leq \rho (x_0,x_1) \dfrac{1}{1-q}$.\\
Разом отримаємо $\rho(x_m,x_n) \leq \dfrac{q^m}{1-q} \rho(x_0,x_1) \to 0, n,m \to \infty$.\\
Оскільки $(X,\rho)$ -- повний, то $\{x_n\}$ -- збіжна, позначимо $a = \displaystyle\lim_{n \to \infty} x_n$. Зважаючи на неперервність стиска, отримаємо $\displaystyle f(a) = f\left( \lim_{n \to \infty} x_n \right) = \lim_{n \to \infty} f(x_n) = \lim_{n \to \infty} x_{n+1} = a$. Тобто $a$ -- це наша шукана нерухома точка.
\bigskip \\
II. \textit{Єдиність}.\\
!Припустимо, що $f$ має дві різні нерухомі точки $a,b$. Буде суперечність! Дійсно,\\
$0 < \rho(a,b) = \rho(f(a),f(b)) \leq q \cdot \rho(a,b) < \rho(a,b)$.
\end{proof}

\begin{remark}
Насправді, в теоремі Банаха достатньо вимагати, щоб саме $f^n \overset{\text{def.}}{=} \underset{n\text{ разів}}{f \circ \dots \circ f}$ було стиском, а не відображення $f$.\\
Дійсно, за теоремою Банаха, $f^n$ матиме єдину нерухому точку $a$, тобто $f^n(a) = a$. Тоді точка $f(a)$ буде теж нерухомою для $f^n$, оскільки $f^n(f(a)) = f(f^n(a)) = f(a)$. Але за єдиністю, $f(a) = a$ -- дві нерухомі мають збігатися. Єдиність нерухомої точки для $f$ доводиться неважко.
\end{remark}

\subsection{Компактність}
\begin{definition}
Задано $(X,\rho)$ -- метричний простір та $A \subset X$.\\
Множина $A$ називається \textbf{компактом}, якщо
\begin{align*}
\forall \{x_n, n \geq 1\} \subset A: \exists \{x_{n_k}, k \geq 1\}: x_{n_k} \to x_0, k \to \infty, \text{ причому } x_0 \in A
\end{align*}
Якщо прибрати умову $x_0 \in A$, то тоді $A$ називається \textbf{передкомпактом}.
\end{definition}

\begin{proposition}
Задано $(X,\rho)$ -- метричний простір та $A \subset X$.\\
$A$ -- компакт $\iff $ $\forall B \subset A$, де $B$ -- нескінченна множина, існує $x_0 \in A$ -- гранична точка $B$.\\
Якщо прибрати умову $x_0 \in A$, то вже мова буде йти про передкомпакт.
\end{proposition}

\begin{proof}
\rightproof Дано: $A$ -- компакт. Нехай $B \subset A$ -- нескінченна множина. Оберемо послідовність $\{x_n, n \geq 1\} \subset B \subset A$, де всі вони між собою різні. Тоді за умовою компактності, існує підпослідовність $x_{n_k} \to x_0$, причому $x_0 \in A$. Зауважимо, що всі $x_{n_k} \neq x_0$, тож $x_0$ -- гранична точка $A$.\\
Якби існували $k \in \mathbb{N}$, для яких $x_{n_k} = x_0$, то тоді ми би сформували підпослідовність $\{x_{n_{k_m}}\}$ без цих елементів, причому $x_{n_{k_m}} \to x_0$, а тепер $x_{n_{k_m}} \neq x_0$. Тож все одно $x_0$ залишається граничною точкою $A$.
\bigskip \\
\leftproof Дано: $\forall B \subset A$, де $B$ -- нескінченна множина, існує $x_0 \in A$ -- гранична точка $B$. Отже, нехай $\{x_n, n \geq 1\} \subset A$ -- довільна послідовність. У нас є два варіанти:\\
I. Множина значень $\{x_n\}$ -- скінченна. Тоді можна відокремити стаціонарну підпослідовність.\\
II. Множина значень $\{x_n\}$ -- нескінченна, всі ці значення покладемо в множину $B \subset A$. Тоді за умовою, існує $x_0 \in A$ -- гранична точка $B$. Отже, $B \cap B(x_0;\varepsilon)$ містить нескінченне число точок для всіх $\varepsilon > 0$. Зокрема:\\
$\varepsilon = 1 \implies B \cap B(x_0;1)$ має нескінченну множину. Там існує елемент $y_1 \in B \cap B(x_0; 1)$, тобто це одне зі значень послідовності. Тобто $y_1 = x_{n_1}$.\\
$\varepsilon = \dfrac{1}{2} \implies B \cap B\left( x_0;\dfrac{1}{2} \right)$ має нескінченну множину. Там існує елемент $y_2 \in B \cap B\left( x_0;\dfrac{1}{2} \right)$, тобто це одне зі значень послідовності. Тобто $y_2 = x_{n_2}$. Причому можна обрати $x_{n_2} > x_{n_1}$. Якби так не було можливо, то $B \cap B\left(x_0; \dfrac{1}{2}\right)$ була б скінченною множиною, що не наше випадок.\\
\vdots \\
Побудували підпослідовність $\{x_{n_k}, k \geq 1\}$, причому $\rho(x_0,x_k) < \dfrac{1}{k}$. Тож при $k \to \infty$ матимемо $x_{n_k} \to x_0 \in A$. Отже, $A$ -- компакт.
\bigskip \\
\textit{Випадок передкомпакту повторюється майже все слово в слово.}
\end{proof}

\begin{proposition}
Задано $(X,\rho)$ -- компактний метричний простір. Тоді $(X,\rho)$ -- повний.
\end{proposition}

\begin{proof}
Дійсно, нехай $\{x_n\} \subset X$ -- фундаментальна. Оскільки $X$ -- компакт, то існує збіжна підпослідовність $\{x_{n_k}\}$, де $x_{n_k} \to x, x \in X$. Ми вже знаємо, що тоді й сама послідовність $\{x_n\} \to x$ буде збіжною. Отже, $(X,\rho)$ -- повний метричний простір.
\end{proof}

\begin{definition}
Задано $(X,\rho)$ -- метричний простір та $A \subset X$.\\
Множина $A$ називається \textbf{обмеженою}, якщо
\begin{align*}
\exists R > 0: A \subset B(a;R)
\end{align*}
\end{definition}

\begin{definition}
Задано $(X,\rho)$ -- метричний простір та $A \subset X$.\\
Множина $A$ називається \textbf{цілком обмеженою}, якщо
\begin{align*}
\forall \varepsilon > 0: \exists C_\varepsilon = \{x_1,x_2,\dots,x_n\}: A \subset \displaystyle\bigcup_{x \in C_\varepsilon} B(x;\varepsilon)
\end{align*}
До речі, $C_\varepsilon$, для якої виконана $A \subset \displaystyle\bigcup_{x \in C_\varepsilon} B(x;\varepsilon)$, називається \textbf{скінченною $\varepsilon$-сіткою}.\\
Тобто $A$ -- цілком обмежена, коли вона має скінченну $\varepsilon$-сітку для всіх $\varepsilon > 0$.
\end{definition}

\begin{proposition}
Задано $(X,\rho)$ -- метричний простір та $A$ -- цілком обмежена множина. Тоді $A$ -- обмежена.
\end{proposition}

\begin{proof}
Для множини $A$ існує $1$-сітка, тобто $C_1 = \{x_1,\dots,x_n\}$, для якої $A \subset \displaystyle\bigcup_{x \in C_1} B(x;1)$.\\
Зафіксуємо $y \in X$ та оберемо $R = 1 +\displaystyle\max_{x \in C_1} \rho(x,y)$. Тоді хочемо довести, що $A \subset B(y;R)$.\\
Нехай $a \in A$, тоді вже $a \in B(x;1)$ при деякому $x \in C_1$, а також $\rho(a;x) < 1$. Звідси\\
$\rho(a;y) \leq \rho(a;x) + \rho(x;y) < 1 + \displaystyle\max_{x \in C_1} \rho(x;y) = R$.\\
Отже, $A$ -- обмежена.
\end{proof}

\begin{remark}
Не обов'язково вимагати, щоб $A$ була цілком обмежена. Подивившись на це доведення, ми можемо лише вимагати, щоб $A$ мала хоча б одну $\varepsilon$-сітку -- тоді буде обмеженість $A$.
\end{remark}

\begin{theorem}[Критерій Фреше-Хаусдорфа]
Нехай $(X,\rho)$ -- повний метричний простір та $A \subset X$.\\
$A$ -- цілком обмежена $\iff A$ -- передкомпакт.
\end{theorem}

\begin{remark}
Під час доведення \leftproof нам не потрібна буде умова повноти метричного простору.
\end{remark}

\begin{proof}
\rightproof Дано: $A$ -- цілком обмежена. Нехай $\{a_n, n \geq 1\} \subset A$ -- довільна послідовність. \\
Оберемо $1$-сітку $C_1$, де $A \subset \displaystyle\bigcup_{x \in C_1} B(x;1)$. В одному з цих шарів нескінченне число членів послідовності, той шар позначу за $B(y_1;1)$; маємо підпослідовність $\{a_{n_k}, k \geq 1\} \subset B(y_1;1)$.\\
Оберемо $\dfrac{1}{2}$-сітку $C_{\frac{1}{2}}$, де $A \subset \displaystyle\bigcup_{x \in C_{\frac{1}{2}}} B\left(x;\dfrac{1}{2}\right)$. В одному з цих шарів нескінченне число членів підпослідовності, той шар позначу за $B\left(y_2;\dfrac{1}{2}\right)$; маємо підпідпослідовність $\{a_{n_{k_m}}, k \geq 1\} \subset B\left(y_2;\dfrac{1}{2}\right)$.\\
\vdots \\
Отримали послідовність центрів $\{y_n, n \geq 1\}$, доведемо її фундаментальність.\\
$\rho(y_n,y_m) \leq \rho(y_n,a_*) + \rho(a_*,y_m) < \dfrac{1}{n} + \dfrac{1}{m} \to 0$ при $n,m \to \infty$. У даному випадку ми підібрали елемент $a_* \in B\left( \dfrac{1}{n};y_n \right) \cap B\left( \dfrac{1}{m};y_m \right)$.\\
Тепер розглянемо підпослідовність $\{a_{n_p}, p \geq 1\}$, яка будується таким чином: беремо перший елемент з $\{a_{n_k}\}$ (це наше $a_{n_1}$), потім перший елемент з $\{a_{n_{k_m}}\}$ (це наше $a_{n_2}$), \dots Доведемо, що $\{a_{n_p}, p \geq 1\}$ -- фундаментальна. Дійсно,\\
$\rho(a_{n_p}, a_{n_t}) \leq \rho(a_{n_p}, y_p) + \rho(y_p,y_t) + \rho(y_t,a_{n_t}) < \dfrac{1}{p} + \dfrac{1}{t} + \rho(y_p,y_t) \to 0, t,p \to \infty$\\
Оскільки $(X,\rho)$ -- повний, то звідси $\{a_{n_p}, n \geq 1\}$ -- збіжна підпослідовність. Довели, що $A$ -- передкомпакт.
\bigskip \\
\leftproof Дано: $A$ -- передкомпакт.\\
!Припустимо, що $A$ -- це є цілком обмеженою. Тобто для деякого $\varepsilon > 0$ не існує $\varepsilon$-сітки. Нехай $x_1 \in A$. Тоді існує $x_2 \in A$, для якої $\rho(x_1,x_2) \geq \varepsilon$ (інакше якби для кожної $x_2 \in A$ була б $\rho(x_1,x_2) < \varepsilon$, то ми си знайшли $\varepsilon$-сітку $\{x_1\}$, що суперечить умові).\\
Далі існує $x_3 \in A$, для якої $\rho(x_1,x_3) \geq \varepsilon$ та $\rho(x_2,x_3) \geq \varepsilon$ (аналогічно якби для кожної $x_3 \in A$ ці два нерівності не виконувалися би, то ми би знайшли один з трьох $\varepsilon$-сіток: $\{x_1\}$ або $\{x_2\}$ або $\{x_1,x_2\}$).\\
\vdots \\
Побудували послідовність $\{x_n, n \geq 1\} \subset A$, для якої справедлива $\rho(x_n,x_m) \geq \varepsilon$ при всіх $n \neq m$. За умовою передкомпактності, існує $\{x_{n_k}, n \geq 1\}$, для якої $x_{n_k} \to x_0$. Водночас звідси ми отримаємо, що існують номери $K_1,K_2$, для яких $\rho(x_{n_{K_1}}, x_{n_{K_2}}) \leq \rho(x_{n_{K_1}}, x_0) + \rho(x_0, x_{n_{K_2}}) < \varepsilon$. Суперечність!\\
Отже, $A$ все ж таки має бути цілком обмеженою.
\end{proof}

\begin{theorem}
Задано $(X,\rho)$ -- метричний простір та $A \subset X$.\\
$A$ -- компакт $\iff$ для кожного відкритого покриття $A$ можна виділити скінченне підпокриття.
\end{theorem}

\begin{proof}
\rightproof Дано: $A$ -- компакт.\\
!Припустимо, що існує відрките покриття $\{U_\alpha\}$ множини $A$, від якої не можна відокремити скінченне підпокриття. Оскільки $A$ -- компакт, то $A$ -- цілком обмежена. Значить, існує $1$-сітка $C_1$ (причому можна підібрати так, щоб $C_1 \subset A$), для якої $A \subset \displaystyle\bigcup_{x \in C_1} B(x;1)$, або можна переписати як $A \subset \displaystyle\bigcup_{x \in C_1} A \cap B(x;1)$. Серед множин $A \cap B(x;1)$ існує одна з них, яка не покривається скінченним чином множинами $\{U_\alpha\}$. Дану множину позначу за $A'$.\\
Сама множина $A'$ -- також цілком обмежена, тож існує $\dfrac{1}{2}$-сітка $C_{\frac{1}{2}}$ (знову підберемо так, щоб $C_{\frac{1}{2}} \subset A'$), для якої виконано $A' \subset \displaystyle\bigcup_{x \in C_{\frac{1}{2}}} A' \cap B\left(x; \dfrac{1}{2}\right)$. Знову ж таки, серед $A' \cap B\left(x; \dfrac{1}{2}\right)$ існує одна з них, що не покривається скінченним чином множинами $\{U_\alpha\}$. Дану множину позначу за $A''$. \\
\vdots \\
Продовжуючи процедуру, отримаємо набір куль $B_n = B\left(x_n;\dfrac{1}{n}\right)$, де центр $x_n \in B_{n-1} \cap A$. Позначимо $\overline{B_n \cap A} = K_n$ та зауважимо, що $K_n$ -- це замкнена куля в метричному підпросторі $A$, де $R = \dfrac{1}{2^n}$ та центр $y_n \in K_{n-1}$.\\
Подвоїмо радіуси кожної з цих куль. Тоді отримаємо послідовність вкладених куль, які стягуються. Оскільки $A$ -- компакт, то $(A,\rho_A)$ -- повний метричний простір, тож за теоремою Кантора, існує $a \in A$ -- спільна точка цих куль. Зважаючи на покриття множини $A$, отримаємо $a \in U_{\alpha_0}$ при деякому $\alpha_0$. Оскільки $U_{\alpha_0}$ -- відкрита, то існує куля $B(z,\delta) \subset U_{\alpha_0}$. Ми можемо підібрати завжди такий $N \in \mathbb{N}$, щоб було виконано $\dfrac{1}{N} < \dfrac{\delta}{2}$, тоді звідси $K_n \subset B(z;\delta) \subset U_{\alpha_0}$. Таким чином, $K_n$ була покрита лише однією множиною із $\{U_\alpha\}$, проте ми обирали такі кулі (на початку), які не допускали скінченне підпокриття. Суперечність!
\bigskip \\
\leftproof Дано: кожне покриття $A$ має скінченне підпокриття.\\
!Припустимо, що $A$ -- не компакт, тобто існує послідовність $\{x_n, n \geq 1\} \subset A$, що не має часткових границь. Тоді кожний відкритий окіл $U_a, a \in A$, містить скінченну кількість членів послідовності $\{x_n\}$ (якби існував окіл $U_a$ із нескінченним числом членів послідовності, то $a$ стала би граничною точкою, що неможливо). Набір $\{U_a, a \in A\}$ -- відкрите покриття множини $A$. За умовою, існує скінченне підпокриття $\{U_{a_1},\dots,U_{a_n}\}$ множини $A$, але тоді $A \subset \displaystyle\bigcup_{i=1}^n U_{a_i}$, де праворуч -- скінченна множина; ліворуч -- нескінченна в силу нескінченності послідовності $\{x_n\}$ -- суперечність!
\end{proof}

\begin{corollary}
Задано $(X,\rho), (Y,\tilde{\rho})$ -- два метричних простори та $f \colon X \to Y$ -- неперервне відображення. Відомо, що $X$ -- компакт. Тоді $f(X)$ -- компакт.
\end{corollary}

\begin{proof}
Маємо $\{U_\alpha\}$ -- відкрите покриття $f(X)$. Тоді $\{f^{-1}(U_\alpha)\}$ -- відкрите покриття $X$, але за компактністю, можна виділити скінченне підпокриття $\{f^{-1}(U_1),\dots,f^{-1}(U_m)\}$, тоді звідси $\{U_1,\dots,U_m\}$ буде скінченним підпокриттям $f(X)$.
\end{proof}

\begin{corollary}
Задано $(X,\rho)$ -- метричний простір та $f \colon X \to \mathbb{R}$ -- числова неперервна функція. Відомо, що $X$ -- компакт. Тоді $f$ -- обмежена та досягає найбільшого та найменшого значень.
\end{corollary}

\begin{theorem}
Задано $(X,\rho), (Y,\tilde{\rho})$ -- два метричних простори та $f \colon X \to Y$ -- неперервне, причому $X$ -- компакт. Тоді $f$ -- рівномірно неперервне.
\end{theorem}

\begin{proof}
!Припустимо, що $\exists \varepsilon > 0: \forall \delta > 0: \exists x,y \in X: \rho(x,y) < \delta$, але $\tilde{\rho}(f(x),f(y)) \geq \varepsilon$.\\
Оберемо $\delta = \dfrac{1}{n}, n \in \mathbb{N}$, тоді утвориться послідовність $\{x_n\}, \{y_n\} \subset X$. Оскільки $X$ -- компакт, то відокремимо збіжні підпослідовності $\{x_{n_k}\}, \{y_{n_k}\}$. Але оскільки $\rho(x_{n_k}, y_{n_k}) < \dfrac{1}{n_k}$, то звідси випливає $\displaystyle\lim_{k \to \infty} x_{n_k} = \lim_{k \to \infty} x_{n_k}$. Із іншого боку, $\displaystyle\lim_{k \to \infty} f(x_{n_k}) \neq \lim_{k \to \infty} f(y_{n_k})$, оскільки виконана нерівність $\tilde{\rho}(f(x_{n_k}), f(y_{n_k}) \geq \varepsilon$. Суперечність!
\end{proof}

\subsection{Теорема Стоуна-Ваєрштраса}
Надалі будемо розглядати компактний метричний простір $(X,\rho)$ та метричний простір $(C(X), \sigma)$ -- простір неперервних функцій із метрикою $\sigma(f,g) = \displaystyle\max_{x \in X}\| f(x) - g(x)\|$. Причому даний метричний простір теж повний (це аналогічно доводиться).

\begin{definition}
Множина $A \subset C(X)$ називається \textbf{алгеброю}, якщо $\forall f,g \in A, \forall \alpha \in \mathbb{R}$:
\begin{align*}
\alpha f,\ f+g,\ f \cdot g \in A
\end{align*}
\end{definition}

\begin{definition}
Нехай $A \subset C(X)$ -- алгебра.\\
Алгебра $A$ \textbf{відділяє точки} множини $X$, якщо
\begin{align*}
\forall x,y \in X: x \neq y: \exists f \in A: f(x) \neq f(y)
\end{align*}
\end{definition}

\begin{theorem}[Теорема Стоуна-Ваєрштраса]
Задано $(X,\rho)$ -- компактний метричний простір та $(C(X),\sigma)$ -- простір неперервних дійсних функцій, заданий вище. Маємо $A \subset C(X)$. Про неї відомо, що
\begin{enumerate}[nosep,wide=0pt,label={\arabic*)}]
\item $A$ -- алгебра, яка віддаляє точки множини $X$;
\item функція $f$, яка визначена як $f(x) = 1, \forall x \in X$, належить $A$.
\end{enumerate}
Тоді множина $A$ скрізь щільна в $(C(X),\sigma)$.
\end{theorem}

\begin{proof}
Ми хочемо довести, що $\bar{A} = C(X)$.\\
Нехай $f \in A$. Хочемо довести, що $|f| \in \bar{A}$. У курсі мат.\ аналізу ми доводили теорему Ваєрштраса про наближення функції многочленом. Зокрема для функції $g(t) = \sqrt{t}, t \in [0,1]$ маємо, що $\forall \varepsilon > 0: \exists P_\varepsilon$ -- многочлен від $t: |\sqrt{t} - P_\varepsilon(t)| < \varepsilon$. Тоді $\forall x \in X:$\\
$\left| \dfrac{|f(x)|}{\|f\|} - P_\varepsilon\left( \dfrac{f^2(x)}{\|f\|^2} \right) \right| = \left| \sqrt{\dfrac{|f(x)|^2}{\|f\|^2}} - P_\varepsilon\left( \dfrac{f^2(x)}{\|f\|^2} \right) \right| < \varepsilon$.\\
Оскільки $f \in A$, то в силу алгебри $\dfrac{f^2}{\|f\|} \in A$. Оскільки $P_\varepsilon$ -- многочлен, то $P_\varepsilon \circ \dfrac{f^2}{\|f\|} \in A$. Ми знайшли $P_\varepsilon \circ \dfrac{f^2}{\|f\|^2} \in A$, для якої $\left\| \dfrac{|f|}{\|f\|} - P_\varepsilon \circ \dfrac{f^2}{\|f\|^2} \right\| < \varepsilon$. Отже, $\dfrac{|f|}{\|f\|}$ -- гранична точка, тобто $\dfrac{|f|}{\|f\|} \in \bar{A}$.\\
Відомо знову з мат.\ аналізу, що для всіх $a,b \in \mathbb{R}$ ми маємо такі рівності:\\
$\max\{a,b\} = \dfrac{1}{2}\left(a+b+|a-b|\right) \qquad \min\{a,b\} = \dfrac{1}{2}\left(a+b-|a-b|\right)$.\\
Значить, маючи $f,g \in A$ та маючи результат вище, отримаємо $\max \{f,g\}, \min\{f,g\} \in \bar{A}$.\\
Оберемо $x,y \in X$ так, що $x \neq y$. Тоді існує функція $g \in A$, для якої $g(x) \neq g(y)$. Далі покладемо нову функцію $f(z) = \alpha + \dfrac{\beta-\alpha}{g(y)-g(x)}(g(z)-g(x)), z \in X,\ \alpha,\beta \in \mathbb{R}$. Тоді звідси $f \in A$ (ми тут користуємося пунктом 2), щоб це показати), причому $f(x) = \alpha,\ f(y) = \beta$.\\
Отже, що ми довели щойно: $\forall x,y \in X: x \neq y, \forall \alpha,\beta \in \mathbb{R}: \exists f \in A: f(x) = \alpha,\ f(y) = \beta$.
\bigskip \\
Нехай $f \in C(X)$ та $\varepsilon > 0$. Зафіксуємо $x \in X$, для $z \in X$ покладемо $\alpha = f(x), \beta = f(z)$. Тоді за щойно доведеним, існує $h_z \in A$, для якої $h_z(x) = \alpha = f(x)$ та $h_z(z) = \beta = f(z)$.\\
Оскільки $h_z - f \in C(X)$, то за означенням, $\exists \delta_z > 0: \forall y \in B(z,\delta_z): h_z(y)-f(y) < \varepsilon$. Сім'я множин $\{B(z,\delta_z) \mid z \in X\}$ -- відкрите покриття компактної множини $X$. Отже, ми можемо взяти скінченне підпокриття $\{B(z_k,\delta_{z_k}) \mid k = \overline{1,n}\}$.\\
Визначимо функцію $g_x(y) = \displaystyle\min_{1 \leq k \leq n} \{h_{z_k}(y))\}, y \in X$. Зауважимо, що по-перше, $g_x \in \bar{A}$; по-друге, $g_x(x) = f(x)$; по-третє, $\forall y \in X: g_x(y) -f(y) < \varepsilon$.\\
Оскільки $g_x - f \in C(X)$, то за означенням, $\exists \delta_x > 0: \forall y \in B(x,\delta_x): g_x(y) - f(y) > - \varepsilon$. Сім'я множин $\{B(x,\delta_x) \mid x \in X \}$ -- відкрите покриття компактної множини $X$. Отже ми можемо взяти скінченне підпокриття $\{B(x_k,\delta_{x_k} \mid k = \overline{1,m}\}$.\\
Визначимо функцію $h(y) = \displaystyle\max_{1 \leq k \leq m} g_{x_k}(y), y \in X$. Тоді $h \in \bar{A}$, причому також $\forall y \in X: \\ f(y) - \varepsilon \leq h(y) \leq f(y) + \varepsilon$. Для будь-якої функції $f \in C(X)$ ми знайшли $h \in A$, для якої $\|h - f\| < \varepsilon$. Отже, $\bar{A} = C(X)$.
\end{proof}
\newpage

\section{Початок функціонального аналізу}
\subsection{Обмежені та неперервні лінійні оператори}
\begin{definition}
Задано $(X, \|\cdot \|_X),(Y, \| \cdot \|_Y)$ -- нормовані простори.\\
Лінійний оператор $A \colon X \to Y$ називають \textbf{обмеженим}, якщо
\begin{align*}
\exists C > 0: \forall x \in X: \| Ax \|_Y \leq C \| x \|_X
\end{align*}
Надалі ми ці норми розрізняти не будемо, бо буде з контексту зрозуміло.
\end{definition}

\begin{remark}
Маємо обмежений оператор $A$. Зауважимо, що множина всіх констант, які обмежують оператор, тобто множина $\{C > 0 \mid \forall x \in X: \| Ax\| \leq C \|x\|\}$, буде непорожньою (бо оператор обмежений) та обмеженою знизу числом $0$. Значить, існує $\displaystyle\inf\{C > 0 \mid \forall x \in X: \| Ax\| \leq C \|x\|\}$.
\end{remark}

\begin{definition}
Задано $X,Y$ -- нормовані простори.\\operators
\textbf{Нормою} лінійного оператора $A$ називається величина
\begin{align*}
 \| A\| = \inf \{C > 0 \mid \forall x \in X: \| Ax \| \leq C \|x\|\}
\end{align*}
\end{definition}

\begin{remark}
Зауважимо, що для всіх $x \in X$ виконується $\|Ax\| \leq \|A\| \cdot \|x\|$.\\
Дійсно, для кожного $\varepsilon > 0$ існус стала $C_\varepsilon > 0$, для якої $C_\varepsilon < \|A\| + \varepsilon$. Тож для всіх $x \in X$ справедлива нерівність $\|A x\| \leq C_\varepsilon \|x\| < (\| A \| + \varepsilon) \|x\|$. Тому ця нерівність виконуватиметься також при $\varepsilon \to 0+0$. Таким чином, $\|A\| \in \{C > 0 \mid \forall x \in X: \|Ax\| \leq C \|x\|\}$, тобто інфімум досягається. \\
Отже, норма $\|A\|$ -- це найменше число, що обмежує лінійний оператор $A$.
\end{remark}

\begin{theorem}
Задано $X,Y$ -- нормовані простори та $A \colon X \to Y$ -- обмежений оператор. Тоді $\displaystyle\|A\| = \sup_{x \in X \setminus \{0\}} \dfrac{\| Ax\|}{\|x\|}$.
\end{theorem}

\begin{proof}
Спочатку доведемо, що $\|A\| = \displaystyle\sup_{x \in X \setminus \{0\}} \dfrac{\| Ax\|}{\|x\|}$. Уже відомо, що $\forall x \in X: \|Ax \| \leq \|A \| \|x\|$, тоді звідси $\forall x \in X \setminus \{0\}: \dfrac{\|Ax\|}{\|x\|} \leq \|A\|$, таким чином $\displaystyle\sup_{x \in X \setminus \{0\}} \dfrac{\|Ax\|}{\|x\|} \leq \|A\|$. Залишилося довести, що строга нерівність не допускається.\\
!Припустимо, що $\displaystyle\sup_{x \in X \setminus \{0\}} \dfrac{\|Ax\|}{\|x\|} < \|A\|$, тобто існує $\varepsilon > 0$, для якого $\displaystyle\sup_{x \in X \setminus \{0\}} \dfrac{\|Ax\|}{\|x\|} = \|A\|- \varepsilon$. Тоді звідси випливає, що $\forall x \in X \setminus \{0\}: \dfrac{\|Ax\| }{\|x\|} \leq \|A\| - \varepsilon \implies \forall x \in X: \|Ax\| \leq (\|A\|-\varepsilon) \|x\|$. Таким чином, $\|A\|-\varepsilon$ -- це константа, яка обмежує оператор, тоді за означенням норми, $\|A\| - \varepsilon \geq \|A\|$ -- суперечність!\\
Отже, ми довели рівність, тобто $\|A\| = \displaystyle\sup_{x \in X \setminus \{0\}} \dfrac{\| Ax\|}{\|x\|}$.
\end{proof}

\begin{theorem}
Задано $X,Y$ -- нормовані простори та $A \colon X \to Y$ -- обмежений оператор. Тоді $\displaystyle\|A\| = \sup_{\|x\| \leq 1} \|Ax\| = \sup_{\|x\| = 1} \|Ax\|$.
\end{theorem}

\begin{proof}
Ми доведемо ось такий ланцюг нерівностей: $\|A\| = \displaystyle\sup_{x \neq 0} \dfrac{\|Ax\|}{\|x\|} \geq \sup_{\|x\|\leq 1} \|Ax\| \geq \sup_{\|x\| = 1} \|Ax\| \geq \sup_{x \neq 0} \dfrac{\|Ax\|}{\|x\|}$.\\
Оберемо такий $x \neq 0$, щоб $\|x\| \leq 1$. Тоді виконується нерівність $\dfrac{\|Ax\|}{\|x\|} \geq \|A x\|$. Таким чином, $\displaystyle\sup_{\|x\| \leq 1} \|Ax\| \leq \sup_{\substack{\|x\| \leq 1 \\ x \neq 0 }} \dfrac{\|Ax\|}{\|x\|} \leq \sup_{x \neq 0} \dfrac{\|Ax\|}{\|x\|} = \|A\|$.\\
Зрозуміло, що виконується нерівність $\displaystyle\sup_{\|x\| = 1} \|Ax\| \leq \sup_{\|x\| \leq 1} \|Ax\|$.\\
Залишилося довести, що $\displaystyle\sup_{x \neq 0} \dfrac{\|Ax\|}{\|x\|} \leq \sup_{\|x\|=  1} \|Ax\|$. Дана нерівність є наслідком того, що для кожного $x \neq 0$ число $\dfrac{\|Ax\|}{\|x\|} = \left\| A \left( \dfrac{x}{\|x\|} \right) \right\|$ належить множині $\{ \|Ax\| \mid \|x\| = 1\}$.
\end{proof}

\begin{example}
Задано лінійний оператор $A \colon l_2 \to l_2$ таким чином: $A(x_1,x_2,\dots) = (x_2,x_3,\dots)$. Довести, що $A$ -- обмежений оператор та знайду норму.\\
Згадаємо, що норма $\|(x_1,x_2,\dots\| = \sqrt{|x_1|^2 + |x_2|^2 + \dots}$. Оцінимо оператор:\\
$\|A (x_1,x_2,\dots)\| = \|(x_2,x_3,\dots)\| = \sqrt{|x_2|^2 + |x_3|^2 + \dots} \leq \sqrt{|x_1|^2 + |x_2|^2 + |x_3|^2 + \dots} = 1 \cdot \|(x_1,x_2,\dots)\|$.\\
Отже, $A$ -- обмежений оператор, бо знайшли константу $C = 1$, що обмежує.\\
$\|A\| = \displaystyle\sup_{\|(x_1,x_2,\dots)\| = 1} \|A(x_1,x_2,\dots)\| = \displaystyle\sup_{\|(x_1,x_2,\dots)\| = 1} \|(x_2,x_3,\dots)\| = \sup_{\|(x_1,x_2,\dots)\| = 1} \sqrt{|x_2|^2 + |x_3|^2 + \dots} = \\
= \sup_{\|(x_1,x_2,\dots)\| = 1} \sqrt{1 - \|x_1\|^2} = 1$.
\end{example}

\begin{example}
Задано лінійний оператор $A \colon C([0,1]) \to C([0,1])$, таким чином: $(Ax)(t) = \displaystyle\int_0^t \tau x(\tau)\,d\tau$. Довести, що $A$ -- обмежений оператор та знайти норму.\\
Конкретно в цьому випадку розглядатиметься норма $\|f\| = \displaystyle\max_{t \in [0,1]} |f(t)|$.\\
$\displaystyle\|Ax\| = \max_{t \in [0,1]} \left| \int_0^t \tau x(\tau)\,d\tau \right| \leq \max_{t \in [0,1]} \int_0^t |\tau| |x(\tau)|\,d\tau = \int_0^1 |\tau| |x(\tau)|\,d\tau \leq \int_0^1 |\tau| \max_{\tau \in [0,1]} |x(\tau)|\,d\tau = \\
= \int_0^1 \tau \|x \|\,d\tau = \|x\| \dfrac{\tau^2}{2} \Big|_0^1 = \dfrac{1}{2} \|x\|$.\\
Отже, $A$ -- обмежений оператор. Залишилося знайти норму.\\
Оскільки $\|Ax\| \leq \dfrac{1}{2} \|x\|$, то звідси випливає $\|A\| = \displaystyle\sup_{\|x\| = 1} \|Ax\| \leq \dfrac{1}{2}$. Із іншого боку, оберемо функцію $x(t) = 1$, для якої $\|x\| = 1$. Тоді отримаємо, що $\|Ax\| = \displaystyle\max_{t \in [0,1]} \left|\int_0^\tau \tau\,d\tau\right| = \max_{t \in [0,1]} \dfrac{t^2}{2} = \dfrac{1}{2}$.\\
Таким чином, отримаємо $\|A\| = \dfrac{1}{2}$.
\end{example}

\begin{proposition}
Задано $X,Y$ -- нормовані простори та $\dim X < \infty$ та $A \colon X \to Y$ -- лінійний оператор. Тоді $A$ -- обмежений.\\
Внаслідок цього, всі оператори між скінченновимірними векторними просторами -- обмежені.
\end{proposition}

\begin{proof}
Дійсно, нехай $\{e_1,\dots,e_n\}$ -- базис $X$, нехай на неї стоїть норма $\|x\|_2$, тоді маємо наступне:\\
$\|Ax\| = \|A (x_1 e_1 + \dots + x_n e_n) \| = \| x_1 A e_1 + \dots + x_n A e_n \| \leq |x_1| \|A e_1 \| + \dots + |x_n| \|A e_n \| \leq \\
\leq \sqrt{|x_1|^2 + \dots + |x_n|^2} \sqrt{\|Ae_1\|^2 + \dots + \|Ae_n\|^2} = C \|x\|_2$.\\
Якби була би інша норма $\| \cdot \|$, то вона еквівалентна $\| \cdot \|_2$, а тому обмеженість зберігається.
\end{proof}

\begin{theorem}
Задано $X,Y$ -- нормовані простори та $A \colon X \to Y$ -- лінійний оператор.\\
$A$ -- обмежений $\iff A $ -- неперервний в точці $0$.
\end{theorem}

\begin{proof}
\rightproof Дано: $A$ -- обмежений. Оберемо послідовність $\{x_n\} \subset X$ так, щоб $x_n \to 0$. Звідси отримаємо $\|Ax_n - A0\| = \|Ax_n\| \leq \|A\| \|x_n\| \to 0$. Отже, $Ax_n \to A0$ при $n \to \infty$, що підтверджує неперервність. 
\bigskip \\
\leftproof Дано: $A$ -- неперервний в точці $0$.\\
!Припустимо, що $A$ -- необмежений оператор. Тоді для кожного $n \in \mathbb{N}$ існує точка $x_n \in X$, для якої $\|Ax_n\| > n \|x_n\|$ (ясно, що $x_n \neq 0$).  Таким чином, $\dfrac{\|Ax_n\|}{\|x_n\|} = \left\| A\left( \dfrac{x_n}{\|x_n\|} \right) \right\| > n$. Для зручності позначу $w_n = \dfrac{x_n}{\|x_n\|} \in X$, тобто ми вже маємо $\| Aw_n\| > n$. Оскільки відображення $A$ -- неперервне в нулі, то для послідовності $\left\{ \dfrac{1}{n}w_n, n \geq 1 \right\}$, для якої $\dfrac{1}{n} w_n \to 0$ виконується $A \dfrac{w_n}{n} \to A0 = 0$ -- суперечність в силу нерівності! Бо в нас $\left\| A \dfrac{w_n}{n} \right\| > 1$.
\end{proof}

\begin{remark}
Насправді, $A$ -- неперервний в точці $0 \iff A$ -- неперервний на $X$.\\
Сторона \leftproof зрозуміла. По стороні \rightproof маємо $x_0 \in X$ та припустимо, що $\{x_n\}$ -- довільна послідовність, де $x_n \to x_0$. Тоді цілком зрозуміло, що $x_n - x_0 \to 0$, але за неперервністю в нулі, маємо $A(x_n - x_0) = A x_n - A x_0 \to A0 = 0$. Таким чином, $A x_n \to A x_0$.
\end{remark}

\begin{theorem}
Множина $\mathcal{B}(X,Y)$ -- множина всіх обмежених лінійних операторів -- буде підпростором $\mathcal{L}(X,Y)$, а також буде нормованим простором із заданою нормою за означенням вище.
\end{theorem}

\begin{proof}
Дійсно, нехай $A,B \in \mathcal{B}(X,Y)$, тобто вони обмежені. Хочемо довести, що $A+B, \alpha A \in \mathcal{B}(X,Y)$, тобто вони теж обмежені. Дійсно, справедливі наступні оцінки:\\
$\| (A+B) x \| = \| Ax + Bx \| \leq \|Ax \| + \|Bx\| \leq \|A\| \|x\| + \|B\| \|x \| = (\|A\| + \|B\|) \|x\|$.\\
$\| (\alpha A) x \| = |\alpha| \|Ax\| \leq |\alpha| \|A\| \|x\|$.\\
Отже, дійсно $A+B, \alpha A \in \mathcal{B}(X,Y)$. Тепер доведемо, що вищезгадана норма лінійного обмеженого оператора -- дійсно норма.\\
$\|A\| \geq 0$ -- зрозуміло. Також якщо $\|A\| = 0$, то звідси $\|Ax\| \leq \|A\| \|x\| = 0$, тобто $Ax = 0$, причому для всіх $x \in X$; або $A = O$. Навпаки, якщо $A = O$, тобто $\|A\| = \displaystyle\sup_{\|x\| = 1} \|Ax\| = \sup_{\|x\|= 1} \{0\} = 0$.\\
Ми вже маємо оцінку $\| \alpha Ax\| \leq |\alpha| \|A\| \|x\|$ при всіх $x \in X$, тому й при всіх $x$ з умовою $\|x\| = 1$. Таким чином, $\|\alpha A\| = \displaystyle\sup_{\|x\|= 1} \|\alpha Ax\| \leq |\alpha|\|A\|$. Із цієї оцінки випливає, що $\|A\| = \|\alpha^{-1} \alpha A\| \leq |\alpha^{-1}| \|\alpha A\| \implies \|\alpha A\| \geq |\alpha| \|A\|$. Таким чином, $\|\alpha A\| = |\alpha| \|A\|$ (у тому числі при $\alpha = 0$).\\
Ми вже маємо оцінку $\| (A+B) x\| \leq (\|A\| + \|B\|) \|x\|$ при всіх $x \in X$, тому й при всіх $x$ з умовою $\|x\| = 1$. Таким чином, $\|A+B\| = \displaystyle\sup_{\|x\| = 1} \|(A+B)x\| \leq \|A\| + \|B\|$ -- третя властивість норми.
\end{proof}

\begin{theorem}
Простір $\mathcal{B}(X,Y)$ буде повним, якщо $Y$ -- повний.
\end{theorem}

\begin{proof}
Нехай $\{A_n, n \geq 1\} \subset \mathcal{B}(X,Y)$ -- фундаментальна послідовність. Зауважимо, що $\{A_nx, n \geq 1\} \subset Y$ -- фундаментальна також при всіх $x \in X$. Із фундаментальності $\{A_n\}$ маємо, що $\forall \varepsilon > 0: \exists N: \forall n,m \geq N: \|A_n-A_m\| < \varepsilon$, але тоді $\forall x \in X: \|(A_n-A_m)x\| \leq \|A_n-A_m\| \|x\| < \varepsilon \|x\|$, звідси й випливає фундаментальність.\\
Тоді при кожному $x \in X$ існує $\displaystyle\lim_{n \to \infty} A_n x = z_x$. Ми можемо визначити як раз новий оператор $A \colon X \to Y$, де $x \mapsto z_x$ (границя єдина, тому визначення адекватне). Залишилися три етапи доведення.\\
I. \textit{Лінійність}. \quad Дійсно, нехай $x,y \in X$ та $\alpha,\beta \in \mathbb{R}$, тоді маємо\\
$A(\alpha x + \beta y) = \displaystyle\lim_{n \to \infty} A_n(\alpha x + \beta y) = \lim_{n \to \infty} (\alpha A_nx + \beta A_n y) = \alpha \lim_{n \to \infty} A_n x + \beta \lim_{n \to \infty} A_n y = \alpha Ax + \beta Ay$.\\
II. \textit{Обмеженість}. \quad Оскільки $\{A_n\}$ -- фундаментальна, то $\{A_n\}$ -- обмежена: $\exists C > 0: \forall n \geq 1: \|A_n\| \leq C$. Тоді в силу неперервності норми матимемо $\|Ax\| = \displaystyle\lim_{n \to \infty} \|A_nx\| \leq C \|x\|$.\\
III. \textit{$A_n \to A$}. \quad Згадаємо нерівність $\|(A_n-A_m)x\| < \varepsilon \|x\|$ при всіх $x \in X$, при всіх $\varepsilon > 0$ та $n,m \geq N$. Спрямуємо $m \to \infty$, тоді отримаємо $\|(A_n-A)x\| \leq \varepsilon \|x\|$, тому й $\|A_n-A\| \leq \varepsilon < 2\varepsilon$.
\end{proof}

\subsection{Продовження неперервних операторів}
Задані $X,Y$ -- нормовані простори, $X_0 \subset X$ та $A \colon X_0 \to Y$ -- обмежений оператор. Питання полягає в тому, чи існує розширення $\tilde{A} \colon X \to Y$ таким чином, що $\tilde{A}|_{X_0} = A$. Причому нас буде цікавити таке розширення, що $\|\tilde{A}\| = \|A\|$.

\begin{remark}
Просто якщо таке розширення допустиме, то звідси $\|\tilde{A}\| \geq \|A\|$. Дійсно,\\
$\|\tilde{A}\| = \displaystyle\sup_{x \in X \setminus \{0\}} \dfrac{\|\tilde{A}x\|}{\|x\|} \geq \sup_{x \in X_0 \setminus \{0\}} \dfrac{\|\tilde{A}x\|}{\|x\|} = \sup_{x \in X_0 \setminus \{0\}} \dfrac{\|Ax\|}{\|x\|} = \|A\|$.
\end{remark}

\begin{proposition}
Задані $X,Y$ -- відповідно нормований та банахів простори та $X_0 \subset X$ -- щільний підпростір. Тоді для кожного обмеженого оператору $A \colon X_0 \to Y$ існує єдиний розширений обмежений оператор $\tilde{A} \colon X \to Y$, для якого $\tilde{A}|_{X_0} = A$ та при цьому $\|\tilde{A}\| = \|A\|$.
\end{proposition}

\begin{proof}
Нехай є послідовність $\{x_n\} \subset X_0$, де $x_n \to x \in X$. Зауважимо, що тоді в цьому випадку $\{Ax_n\}$ -- фундаментальна. У силу банаховості $\{Ax_n\}$ буде збіжним. Тож визначимо оператор $\tilde{A}x = \displaystyle\lim_{n \to \infty} Ax_n$.\\
I. \textit{$\tilde{A}$ визначений коректно}.\\
Нехай є дві послідовності $\{x_n\},\{y_n\}$, для яких $x_n \to x,\ y_n \to x$. Значить, тоді\\
$\|Ax_n - Ay_n\| = \|A(x_n-y_n)\| \leq \|A\| \|x_n-y_n\| \to 0 \implies \displaystyle\lim_{n \to \infty} Ax_n = \lim_{n \to \infty} Ay_n$.
\bigskip \\
II. \textit{$\tilde{A}$ розширює оператор $A$}.\\
Справді, нехай $x \in X_0$. Оберемо стаціонарну послідовність $\{x\} \subset X_0$, де $x \to x$. Тоді $\tilde{A}x = \displaystyle\lim_{n \to \infty} Ax = Ax$. Отже, звідси $\tilde{A}|_{X_0} = A$.
\bigskip \\
III. \textit{$\tilde{A}$ лінійний оператор}.\\
Нехай $x,y \in E$ та $\alpha,\beta \in \mathbb{R}$. Тоді звідси\\
$A(\alpha x + \beta y) = \displaystyle\lim_{n \to \infty} A(\alpha x_n + \beta y_n) = \alpha \lim_{n \to \infty} Ax_n + \beta \lim_{n \to \infty} Ay_n = \alpha Ax + \beta Ay$.
\bigskip \\
\iffalse
IV. \textit{$\tilde{A}$ -- обмежений оператор}.\\
Оберемо послідовність $\{x_n\} \subset X$ так, що $x_n \to 0$. Тоді\\
$\|\tilde{A}x_n\| = \displaystyle \| \lim_{m \to \infty} A x_n^{(m)} \| = \lim_{m \to \infty} \|A x_n^{(m)}\| \leq \lim_{m \to \infty} \|A\| \|x_n^{(m)}\| = \|A\| \| \lim_{m \to \infty} x_n^{(m)} \| = \|A\| \|x_n\| \to 0$ при $n \to \infty \implies \tilde{A}x_n \to A0$.
\bigskip \\
\fi
IV. $\|\tilde{A}\| = \|A\|$.\\
Оберемо $X_0 \ni x_n \to x \in X$. Оскільки $A$ -- обмежений, то $\|Ax_n\| \leq \|A\| \|x_n\|$. Спрямовуючи $n \to \infty$, ми отримаємо $\|\tilde{A}x\| \le \|A\| \|x\|$. Автоматично довели, що $\tilde{A}$ -- обмежений оператор. Раз це виконується для всіх $x \in E$, то отримаємо $\|\tilde{A}\|= \displaystyle\sup_{\|x\| = 1} \|\tilde{A}x\| \leq \sup_{\|x\| = 1} \|A\| \|x\| = \|A\|$. Тобто звідси $\|\tilde{A}\| \leq \|A\|$. Зважаючи на зауваження вище, маємо $\|\tilde{A}\| = \|A\|$.
\bigskip \\
V. \textit{$\tilde{A}$ -- єдине розширення}.\\
!Припустимо, що існує інший оператор $\tilde{\tilde{A}}$, яке також є розширенням $A$ з усіма умовами, що задані в твердженні. Маємо $x \in X$, тож існує послідовність $\{x_n\} \subset X_0, x_n \to x$. Тоді\\
$\tilde{\tilde{A}}x \overset{\tilde{\tilde{A}} \text{ -- обмежений}}{=} \displaystyle\lim_{n \to \infty} \tilde{\tilde{A}}x_n = \lim_{n \to \infty} Ax_n \overset{\text{def. } \tilde{A}}{=} \tilde{A}x$. Суперечність!
\end{proof}

%спочатку запишу теорему для окремого випадку
\begin{theorem}[Теорема Гана-Банаха]
Задано $E$ -- нормований простір та $G \subset E$ -- підпростір. Тоді для кожного обмеженого функціонала $l \colon G \to \mathbb{R}$ існує продовження $\tilde{l} \colon E \to \mathbb{R}$ так, що $\tilde{l}|_G = l,\ \| \tilde{l} \| = \|l\|$.
\end{theorem}

\begin{proof}
1. Обмежимось випадком, коли $E$ -- дісний та сепарабельний простір.\\
I. \textit{Доведемо, що $l$ можна продовжити на деякий підпростір $E \supset F \supsetneq G$.}\\
Нехай $G$ -- підпростір $E$ та $G \neq E$. Зафіксуємо $y \notin G$ та розглянемо підпростір $F = \linspan\{G \cup \{y\}\}$. Тобто кожний елемент $x \in F$ записується як $x = g + \lambda y$ при $g \in G, \lambda \in \mathbb{R}$. Визначимо оператор $\tilde{l}(x) = l(g) + \lambda c$, де $c = \tilde{l}(y)$. За побудовою, такий оператор -- лінійний.\\
Тепер залишилося підібрати таке $c \in \mathbb{R}$, щоб виконувалося $\|\tilde{l}\| = \|l\|$ -- тим самим ми й обмеженість доведемо автоматично. Але згідно зі зауваження, нам треба підібрати $c \in \mathbb{R}$, щоб $\|\tilde{l}\| \leq \|l\|$.\\
Обмежимося поки що $\lambda > 0$. Нехай зафіксовано $h_1,h_2 \in G$ та зауважимо, що справедлива нерівність:\\
$l(h_2) - l(h_1) = l(h_2-h_1) \leq |l(h_2-h_1)| \leq \|l\| \|h_2 - h_1\| = \|h \| \|(h_2+y) - (y+h_1)\| \leq \|l\| \|h_1+y\| + \|l\| \|h_2+y\|$.\\
Звідси випливає, що $-\|l\| \|h_1+y\| - l(h_1) \leq \|l\| \|h_2+y\| - l(h_2)$.\\
Оскільки це $\forall h_1,h_2 \in G$, то тоді $\displaystyle\sup_{h_1 \in G} (-\|l\| \|h_1+y\| - l(h_1)) \leq \inf_{h_2 \in G} ( \|l\| \|h_2+y\| - l(h_2))$.\\
Для зручності супремум позначу за $a_1$ та інфімум за $a_2$. Оберемо число $c \in \mathbb{R}$ так, щоб $a_1 \leq c \leq a_2$. Звідси справедлива така нерівність:\\
$\forall h \in G: -\|l\| \|h+y\| - l(h) \leq c \leq \|l\| \|h+y\| - l(h)$.\\
Тепер покладемо елемент $h = \lambda^{-1}g$ та домножимо обидві частини нерівності на $\lambda$. Оскільки ми домовилися $\lambda > 0$, то знаки нерівностей зберігаються. Коротше, отримаємо:\\
$-\|l\| \|g+\lambda y \| - l(g) \leq \lambda c \leq \|l\| \|g + \lambda y \| - l(g)$.\\
$-\|l\| \|g+\lambda y\| \leq l(g) + \lambda c \leq \|l\| \|g+\lambda y\|$.\\
$|\tilde{l}(x)| = |l(g) + \lambda c| \leq \|l\| \|g + \lambda y\| = \|l\| \|x\|$.\\
Власне, далі аналогічними міркуваннями (як в попередньому твердженні) отримаємо $\|\tilde{l}\| \leq \|l\|$.\\
Тепер що робити при $\lambda < 0$. Перепишемо $x = -(-g + (-\lambda)y)$. У нас тепер $-\lambda > 0$ та $-x = t = -g + (-\lambda)y$, звідси отримаємо\\
$|\tilde{l}(t)| \leq \|l\| \|t\| \implies |\tilde{l}(x)| \leq \|l\| \|x\|$.
\bigskip \\
II. \textit{Тепер доведемо, що продовежння на нашому конкретному $E$ існує}.\\
Оскільки $E$ -- сепарабельний, то існує (ми оберемо зліченну) множина $A = \{x_1,x_2,\dots\}$, яка є щільною підмножиною $E$. Також ми досі маємо $G \subset E$ -- підпростір.\\
Позначимо $x_{n_1} \in A$ -- перший з елементів, де $x_{n_1} \notin G$. За кроком І, існує $l_1$ -- продовження $l$ на $G_1 = \linspan\{G \cup \{x_{n_1}\}\}$.\\
Позначимо $x_{n_2} \in A$ -- перший з елементів, де $x_{n_2} \notin G_1$. За кроком І, існує $l_2$ -- продовження $l_1$ на $G_2 = \linspan\{G_1 \cup \{x_{n_2}\}\}$.\\
\vdots \\
Отримаємо ланцюг підпросторів $G \subset G_1 \subset G_2 \subset \dots$ та набір функціоналів $l_1,l_2,\dots$, для яких:\\
$\forall n \geq 1:$ \qquad $l_n \colon G_n \to \mathbb{R}$ -- обмежена; \qquad $l_n|_G = l$; \qquad $\|l_n\| = \|l\|$.\\
Покладемо множину $M = \displaystyle\bigcup_{n=1}^\infty G_n$, яка є лінійною. Визначимо функціонал $L_0 \colon M \to \mathbb{R}$ таким чином: $x \in M \implies x \in G_N \implies L_0(x) = l_N(x)$. Зрозуміло цілком, що $L_0$ -- лінійний, а також $\|L_0\| = \|l\|$.\\
Оскільки $M \supset A$ та $A$ всюди щільна, то $M$ -- всюди щільна. Отже, за попереднім твердженням, існує продовження $L \colon E \to \mathbb{R}$, для якого $\|L\| = \|L_0\| = \|l\|$.\\
Висновок: ми довели теорему Гана-Банаха для випадку, коли $E$ -- дійсний сепарабельний.
\bigskip \\
2. Тепер будемо доводити теорему для $E$ -- довільний дійсний нормований простір. Все ще $G \subset E$. Позначимо за $l_p$ -- продовження $l$  зі збереженням норми на множині $P \supset G$. Таке продовження існує див (1. та I.). Позначимо $X$ -- множина всіх таких продовжень. На ній введемо відношення $\preceq$ таким чином:\\
$l_p \preceq l_q \iff P \subset Q$ та $l_Q(x) = l_P(x), \forall x \in P$.\\
Зрозуміло, що $\preceq$ задає відношення порядку, внаслідок чого $X$ -- частково впорядкована. Зафіксуємо $Y = \{l_{P_\alpha} \mid \alpha \in A\}$ -- будь-яку лінійно впорядкувану підмножину $X$. Знайдемо верхню грань.\\
Для цього покладемо $P_* = \displaystyle\bigcup_{\alpha \in A} P_\alpha$ та на множині $P_*$ задамо функціонал $l_*$ таким чином:\\
$x \in P_* \implies x \in P_{\alpha_0} \implies l_*(x) = l_{\alpha_0}(x)$.\\
Зрозуміло, що $l_*$ -- лінійний, причому $\|l_*\| = \|l\|$. На множині $\bar{P_*}$ продовжимо функціонал, як було в твердженні -- отримаємо функціонал $l_{\bar{P_*}}$, причому $\|l_{\bar{P_*}}\| = \|l_*\| = \|l\|$. Даний функціонал $l_{\bar{P}_*}$ на $\bar{P}_*$ буде верхньою гранню $Y$. Отже, за лемою Цорна, існує максимальний елемент $X$. Це буде функціонал $L$, який визначений на $E$ (у протилежному випадку його можна було би ще продовжити та він не був би максимальним елементом).\\
Висновок: ми довели теорему Гана-Банаха для випадку, коли $E$ -- дійсний (не обов'язково сепарабельний) нормований простір.
\end{proof}
\noindent
Насправді, на цьому теорема Гана-Банаха ще не закінчена. Ми можемо її довести на випадок, коли нормований простір $E$ -- комплексний. Спершу кілька деталей.\\
Нехайй $E$ -- комплексний лінійний нормований простір. Розглянемо одночасно $E_{\mathbb{R}}$ -- асоційований з $E$ дійсний нормований простір; тобто під час множення на скаляр ми допускаємо лише дійсні коефіцієнти. Зауважимо, що $E_{\mathbb{R}} = E$ як множини, утім не як простори.\\
Розглянемо довільний функціонал $l \colon E \to \mathbb{C}$. Раз $l(x) \in \mathbb{C}$, то для кожного $x \in E$ можна записати функціонал як $l(x) = m(x) + i n(x)$. У цьому випадку $m(x) = \Re l(x),\ n(x) = \Im l(x)$.

\begin{proposition}
Нехай $l \colon E \to \mathbb{C}$ -- лінійний та обмежений функціонал. Тоді $m,n \colon E_{\mathbb{R}} \to \mathbb{R}$ задають лінійний обмежений функціонал.
\end{proposition}

\begin{proof}
Нехай $\alpha,\beta \in \mathbb{R}$ та $x,y \in E$. Тоді ми отримаємо наступне:\\
$l(\alpha x + \beta y) = m(\alpha x + \beta y) + i n(\alpha x + \beta y)$ (з одного боку)\\
$l(\alpha x + \beta y) = \alpha l(x) + \beta l(y) = \alpha (m(x) + i n(x)) + \beta (m(y) + i n(y)) = (\alpha m(x) + \beta m(y)) + i(\alpha n(x) + \beta n(y))$ (з іншого боку).\\
Знаючи, що комплексне число рівне тоді й лише тоді, коли дійсні та уявні частини збігаються, отримаємо\\
$m(\alpha x + \beta y) = \alpha m(x) + \beta m(y)$ \qquad $n(\alpha x + \beta y) = \alpha n(x) + \beta n(y)$.\\
Отже, $m,n$ -- лінійний функціонали.\\
Обмеженість $m$ (аналогічно з $n$) випливає з такго ланцюга нерівностей:\\
$|m(x)| \leq |m(x) + in(x)| = |l(x)| \leq \|l\| \|x\|$.
\end{proof}

\begin{proposition}
$n(x) = -m(ix)$.\\
Іншими словами, ми можемо функціонал $l$ відновити повністю, знаючи функціонал $m$.
\end{proposition}

\begin{proof}
$m(ix) + in(ix) = l(ix) = i l(x) = i(m(x) + in(x)) = -n(x) + im(x)$.\\
$\implies n(x) = -m(ix)$.\\
$l(x) = m(x) -im(ix)$.
\end{proof}

Повернімось назад до теореми Гана-Банаха. Доб'ємо її на випадок, коли $E$ -- комплексний нормований простір.
\begin{proof}
Маємо $E \supset G$ -- два комплексних простори та $E_{\mathbb{R}}, G_{\mathbb{R}}$ -- асоційовані простори. Маємо функціонал $l \colon G \to \mathbb{C}$, який визначається дійсним функціоналом $m \colon G_{\mathbb{R}} \to \mathbb{R}$. Оскільки це дійсний функціонал, ми можемо продовжити до $M \colon E_{\mathbb{R}} \to \mathbb{R}$ зі збереженням норми.\\
Покладемо $L(x) = M(x) - i M(ix)$. Неважко буде довести, що $L$ -- комплексний лінійний функціонал. Залишилося довести, що $\|L\| = \|l\|$. Знову ж таки, достатньо довести $\|L\| \leq \|l\|$. Запишемо $L(x) = |L(x)|e^{i \varphi}$, де $\varphi = \arg L(x)$. Тоді\\
$|L(x)| = e^{-i \varphi} L(x) = L(e^{-i \varphi} x) = M(e^{-i \varphi} x) = |M(e^{-i \varphi}x)| \leq \|M\| \|e^{-i \varphi} x\| = \|m\| \|x\| \leq \|l\| \|x\|$.\\
Отже, $\|L\|$. Ми тут юзали той факт, що $L(y) = M(y)$ при $L(y) \in \mathbb{R}$.
\end{proof}

\begin{remark}
Зауважимо, що якщо $G$ -- лінійна множина (але не підпростір), то теорема Гана-Банаха все одно виконується.\\
У цьому випадку $\bar{G}$ буде підпростором $E$. Функціонал $l$ продовжується неперервним чином на $\tilde{G}$, а далі застосовується доведена теорема.
\end{remark}


\subsection{Деякі наслідки з теореми Гана-Банаха}
\begin{theorem}
Нехай $E$ -- лінійний нормований простір та $G \subset E$ -- підпростір. Тоді для будь-якого вектора $y \notin G$ існує функціонал $l$ на $E$, для якого $\|l\| = 1,\ l(y) = \rho(y,G),\ l|_G = 0$.
\end{theorem}

\begin{proof}
На підпросторі $F = \linspan\{G \cup \{y\}\}$ визначимо функціонал $l_0$ таким чином:\\
$l_0(g + \lambda y) = \lambda \rho(y,G)$.\\
Цілком зрозуміло, що $l_0$ -- лінійний неперервний функціонал на $F$, також $l_0(y) = \rho(y,G)$, нарешті $l_0(g) = l_0(g+0y) = 0$. Обчислимо $\|l_0\|$.\\
$\|l_0\| = \displaystyle\sup \left\{ \dfrac{|l(g+\lambda y)|}{\|g+\lambda y\|} \mid g + \lambda y \in F \right\} = \sup\left\{ \dfrac{|\lambda| \rho(y,G)}{|\lambda| \cdot \|\lambda^{-1}g + y \|} \mid g+\lambda y \in F \right\} = \\ = \rho(y,G) \sup \{ \|g'-y\|^{-1} \mid g' \in G \} = \rho(y,G) \inf_{g' \in G} \|g'-y\| = 1$, де елемент $g' = \lambda^{-1}g \in G$.\\
За теоремою Банаха, існує продовження $l$ до $E$, причому $\|l\| = \|l_0\| = 1$.
\end{proof}

\begin{corollary}
Для кожного $y \in E \setminus \{0\}$ існує функціонал на $E$, що $\|l\| = 1,\ l(y) = \|y\|$.\\
\textit{Вказівка: $G = \{0\}$}.
\end{corollary}

\begin{corollary}
Лінійні непреревні функціонали розділяють точки нормованого простора $E$.\\
Іншими словами, $\forall x_1,y_2 \in E: x_1 \neq x_2: \exists l$ -- функціонал на $E: l(x_1) \neq l(x_2)$.\\
\textit{Вказівка: попередній наслідок, $y = x_1 - x_2 \neq 0$.}
\end{corollary}

\begin{definition}
Задано $E$ -- нормований простір.\\
Підмножина $M \subset E$ називається \textbf{тотальною}, якщо
\begin{align*}
\overline{\linspan{M}} = E
\end{align*}
\end{definition}

\begin{theorem}
Нехай $E$ -- нормований простір та $M \subset E$.\\
$M$ -- тотально в $E \iff \forall x \in M: l(x) = 0 \implies \forall x \in E: l(x) = 0$.
\end{theorem}

\begin{proof}
\rightproof Дано: $M$ -- тотальна множина. Нехай $l$ -- неперервний лінійний функціонал такий, що $\forall x \in M: l(x) = 0$. Оскільки функціонал лінійний, то $\forall x \in \linspan{M}: l(x) = 0$. Оскільки $\overline{\linspan{M}} = E$, то ми можемо неперервно продовжити $l$ до $E$. Отримаємо $l(x) = 0, \forall x \in E$.
\bigskip \\
\leftproof Дано: будь-який лінійно неперервний функціонал $l $ на $E$ такий, що $l(x) = 0, x \in M$ випливає $l(x) = 0, x \in E$.\\
!Припустимо, що $M$ не є тотальною. Тобто $\overline{\linspan{M}} = G \neq E$, тобто існує вектор $y \in E \setminus G$. Внаслідок першої теореми даного підрозділу, існує функціонал $l$ на $E$ такий, що $\|l\| = 1,\ l|_G = 0$. Але з того, що $l|_G = 0$ випливає $l = 0$. Суперечність!
\end{proof}

\begin{proposition}
Нехай $E$ -- нормований простір та $l$ -- лінійний неперервний функціонал з $E$. Тоді $\ker l$ -- підпростір $E$. Більш того, $\ker l$ буде гіперпідпростором, тобто це означає, що $E = \linspan\{\ker l \cup \{y\}\}$ при $y \notin \ker l$.
\end{proposition}

\begin{proof}
Те, що $\ker l$ підпростір, тут все зрозуміло.\\
Нехай $y \notin \ker l$. Тоді доведемо, що кожний елемент $x \in E$ записується як $x = g + \lambda y$, де $g \in \ker l, \lambda \in \mathbb{K}$. Покладемо $\lambda = \dfrac{l(x)}{l(y)}$ та розглянемо вектор $g = x - \lambda y$. Оскільки $l(g) = l(x) - \lambda l(y) = 0$, то звідси $g \in \ker l$. Отже, $x = g + \lambda y$ -- шукане представлення.
\end{proof}

\begin{proposition}
Зафіксуємо лінійний функціонал $l$ на $E$. Покладемо множину $\Gamma_c = \{x \in E \mid l(x) = c\}$, що називається \textbf{гіперплощиною}. Позначимо $\Gamma_0 = \ker l$. Тоді існує такий вектор $z \in E$, що $\Gamma_c = \Gamma_0 + z \equiv \{g + z \mid g \in \Gamma_0 \}$.
\end{proposition}

\begin{proof}
Дійсно, зафіксуємо $z \in \Gamma_c$. Тоді для кожного $x \in \Gamma_c$ маємо $l(x-z) = l(x) - l(z) = 0$, тобто $g = x-z \in \Gamma_0 \implies x = g +z$.
\end{proof}

\begin{definition}
Нехай $E$ -- дійсний нормований простір та $A \subset E$, точка $x_0 \in \partial A$. Також нехай $l$ -- лінійний неперервний функціонал на $E$.\\
Гіперплощина $\Gamma_c$ називається \textbf{опорною гіперплощиною} множини $A$, що проходить через точку $x_0$, якщо
\begin{align*}
x_0 \in \Gamma_c \\
A \text{ лежить по одну сторону від гіперплощини } \Gamma_c \text{ (тобто $l(x) - c$ не міняє знак на $A$)}
\end{align*}
\end{definition}

\begin{theorem}
Зокрема маємо $A = B[r;0]$ -- замкнуту кулю, границя $\partial A = S_r(0)$.\\
Через будь-яку точку $x \in S_r(0)$ проходить опорна гіперплощина шара $B[r;0]$.
\end{theorem}

\begin{proof}
Для кожної точки $x_0 \in S_r(0)$ існує лінійний неперервний функціонал $l$ на $E$, де $\|l\| = 1,\ l(x_0) = \|x_0\| = r$. Тоді гіперплощина $\Gamma_r$ -- наша шукана. Дійсно, $x_0 \in \Gamma_r$, бо $l(x_0) = r$.\\
$\forall x \in B[r;0]: l(x) \leq |l(x)| \leq \|x\| \leq r$, тобто весь шар лежить по одну сторону від $\Gamma_r$.
\end{proof}

\subsection{Загальний вигляд лінійних неперервних функціоналів у деяких банахових просторах}
\subsubsection{Базис Шаудера}
\begin{definition}
Нехай $E$ -- банахів простір.\\
Послідовність $\{e_1,e_2,\dots\} \subset E$ називається \textbf{базисом Шаудера} простора $E$, якщо
\begin{align*}
\forall x \in E: \exists ! x_k \in \mathbb{K}: x = \displaystyle\sum_{k=1}^\infty x_k e_k
\end{align*}
\end{definition}

\begin{proposition}
Нехай $E$ -- банахів простір, що містить базис Шаудера. Тоді $E$ -- сепарабельний.
\end{proposition}

\begin{proof}
\textit{Випадок дійсного нормованого простору}.\\
Оберемо множину $A = \left\{ \displaystyle\sum_{k=1}^\infty x_k e_k \mid x_k \in \mathbb{Q} \right\}$.\\
Нехай $x \in E$, тоді за умовою, $x = \displaystyle\sum_{k=1}^\infty x_k e_k$ єдиним чином. Нехай задане $\varepsilon > 0$. Тоді на кожному з $\left(x_k-\dfrac{\varepsilon}{\|e_k\| 2^k},x_k + \dfrac{\varepsilon}{\|e_k\| 2^k} \right)$ існує раціональне число $y_k \in \mathbb{Q}$. Оберемо $y \in A$ так, що $y = \displaystyle\sum_{k=1}^\infty y_k e_k$. Позначимо $x^{(n)},y^{(n)}$ за часткову суму ряда (перші $n$ додаються). Тоді\\
$\|x^{(n)} - y^{(n)}\| = \displaystyle\left\| \sum_{k=1}^n (x_k-y_k) e_k \right\| \leq \sum_{k=1}^n \| (x_k-y_k) e_k \| = \sum_{k=1}^n |x_k-y_k| \|e_k\| \leq \sum_{k=1}^n \dfrac{\varepsilon}{2^k}$.\\
Далі спрямовуємо $n \to \infty$. Тоді $x^{(n)} \to x, y^{(n)} \to y$. Після чого отримаємо $\|x-y\| \displaystyle \leq \sum_{k=1}^\infty \dfrac{\varepsilon}{2^k} = \varepsilon$. Отже, $A$ скрізь щільна множина, ну тобто $\bar{A} = E$.
\bigskip \\
\textit{Випадок комплексного нормованого простору}.\\
Оберемо множину $A = \left\{ \displaystyle\sum_{k=1}^\infty x_k e_k \mid x_k = \alpha_k + i \beta_k, \alpha_k, \beta_k \in \mathbb{Q} \right\}$. Далі плюс-мінус аналогічно.
\end{proof}

\begin{remark}
Якщо зробити \textcolor{red}{\href{https://projecteuclid.org/download/pdf_1/euclid.acta/1485889774}{*клік*}} сюди, то тут буде стаття про приклад сепарабельного банахового простору, який не містить базис Шаудера. Доведено П.\ Енфло. Власне, це означає, що зворотне твердження не працює.
\end{remark}

\begin{theorem}
Простір $l_p$ містить базис Шаудера. Причому цей базис матиме вигляд $\{e_1,e_2,e_3,\dots\}$, де кожний $e_i = (0,\dots,0,\underbrace{1}_{\text{на $i$-ій позиції}},0,\dots)$.
\end{theorem}

\begin{proof}
І. \textit{Існування}.\\
Фіксуємо елемент $x \in l_p$, де $x = (x_1,x_2,\dots)$ та $\displaystyle\sum_{k=1}^\infty |x_k|^p < +\infty$. Покладемо елемент (що є частковою сумою) $s_n = x_1 e_1 + \dots + x_n e_n$ та доведемо, що послідовність $\{s_n, n \geq 1\}$ -- фундаментальна. При $n > m$\\
$\|s_n - s_m \|_p = \left\| (0,\dots,0,x_{m+1},\dots,x_n,0,\dots ) \right\|_p = \displaystyle \left( \sum_{k=m+1}^n |x_k|^p \right)^{\frac{1}{p}}$.\\
Фундаментальність $\{s_n\}$ випливає зі збіжності числового ряда $\displaystyle\sum_{k=1}^\infty |x_k|^p$. Оскільки $l_p$ -- повний простір, то $\{s_n\}$ -- збіжний, тобто $\displaystyle\sum_{k=1}^\infty x_k e_k$ збігається до деякого елемента. Зокрема доведемо, що $\displaystyle\sum_{k=1}^\infty x_k e_k = x$. Дійсно,\\
$\|x-s_n\|_p = \displaystyle \left\| x - \sum_{k=1}^n x_k e_k \right\|_p = \| (0,\dots,0,x_{n+1},x_{n+2},\dots)\|_p = \left( \sum_{k=n+1}^\infty |x_k| \right)^{\frac{1}{p}}$.\\
Знову зі збіжності числового ряда $\displaystyle\sum_{k=1}^\infty |x_k|^p$ випливає бажане.
\bigskip \\
II. \textit{Єдиність}.\\
!Припустимо, що $x = \displaystyle\sum_{k=1}^\infty y_k e_k$ -- друге представлення. Тоді отримаємо $\displaystyle\lim_{n \to \infty} \sum_{k=1}^n (x_k-y_k)e_k = 0$. Звідси отримаємо $\displaystyle\lim_{n \to \infty}\left\| \sum_{k=1}^n (x_k-y_k) e_k \right\| = \lim_{n \to \infty} \left(\sum_{k=1}^n |x_k-y_k|^p\right)^{\frac{1}{p}} = 0$. Єдина можливість тут -- це $x_k = y_k$ при всіх $k \in \mathbb{N}$ -- суперечність!
\end{proof}

\subsubsection{Простір, що спряжений до $l_p$}
\begin{theorem}
Нехай $p,p' > 1$ таким чином, що $\dfrac{1}{p} + \dfrac{1}{p'} = 1$. Тоді для будь-якого лінійного неперервного функціонала $f$ на $l_p$ існує елемент $(f_k)_{k=1}^\infty \in l_{p'}$, такий, що $f(x) = \displaystyle\sum_{k=1}^\infty f_k x_k$ для всіх $x \in l_p$.
\end{theorem}

\begin{proof}
Нехай $f \in (l_p)'$ (тобто лінійний неперервний функціонал). Тоді звідси отримаємо:\\
$f(x) = \displaystyle f\left( \sum_{k=1}^\infty x_k e_k \right) = f\left( \lim_{n \to \infty} \sum_{k=1}^n x_k e_k \right) = \lim_{n \to \infty} \sum_{k=1}^n x_k f(e_k) = \sum_{k=1}^\infty x_k f(e_k) \overset{f(e_k) \overset{\text{покл.}}{=} f_k}{=} \sum_{k=1}^\infty f_k x_k$.\\
Доведемо, що $(f_k)_{k=1}^\infty \in l_{p'}$. Для цього підберемо елемент $y \in l_p$ ось таким чином, щоб\\
$f(y) = \displaystyle\sum_{k=1}^\infty y_k f_k \overset{\text{був рівний}}{=} \sum_{k=1}^n |f_k|^{p'}$.\\
Можна для цього взяти елемент $y = \left( |f_1|^{p'-1}e^{-i \arg f_1}, \dots, |f_n|^{p'-1}e^{-i \arg f_n}, 0, 0, \dots \right)$. Оскільки $f$ обмежений, то звідси $\displaystyle |f(y)| \leq \|f\| \|y\| = \|f\| \left( \sum_{k=1}^n  ||f_k|^{p'-1} \cdot e^{-i \arg f_k}|^p \right)^{\frac{1}{p}} = \|f\| \left( \sum_{k=1}^n |f_k|^{p'} \right)^{\frac{1}{p}}$.\\
Маючи щойно отриману нерівність та рівність трошки вище, отримаємо\\
$\displaystyle\sum_{k=1}^n |f_k|^{p'} \leq \| f\| \left( \sum_{k=1}^n |f_k|^{p'} \right)^{\frac{1}{p}} \implies \left( \sum_{k=1}^n |f_k|^{p'} \right)^{\frac{1}{p'}} \leq \|f\|, \forall n \in \mathbb{N}$.\\
Остання оцінка стверджує, що ряд збіжний, внаслідок чого $(f_k)_{k=1}^{\infty} \in l_{p'}$.
\end{proof}

\begin{theorem}
І навпаки: для кожного $(f_k)_{k=1}^\infty \in l_{p'}$ рівність $f(x) = \displaystyle\sum_{k=1}^\infty f_k x_k$ визначає лінійний та неперервний функціонал на $l_p$.
\end{theorem}

\begin{proof}
Нехай $(f_k)_{k=1}^\infty \in l_{p'}$. Завдяки нерівності Гьольдера, отримаємо:\\
$\displaystyle |f(x)| \leq \left( \sum_{k=1}^\infty |f_k|^{p'} \right)^{\frac{1}{p'}} \left( \sum_{k=1}^\infty |x_k|^p \right)^{\frac{1}{p}} = c \| x \|_p < +\infty$.\\
Отже, $f$ -- обмежений та $\|f\| \leq c$. Як доводиться лінійність, цілком зрозуміло.
\bigskip \\
До речі, під час минулого доведення ми довели нерівність $c \leq \|f\|$. Маючи ще тут нерівність $c \leq \|f\|$, звідси випливатиме, що $\|f\| = c$, тобто $\|f\| =  \displaystyle\left( \sum_{k=1}^\infty |f_k|^{p'} \right)^{\frac{1}{p'}}$.
\end{proof}

Множину всіх лінійних та обмежених функціоналів на $l_p$ позначатимемо за $(l_p)'$. Ці дві теореми стверджують, що $(l_p)' \cong l_{p'}$ ізометричним чином. Адже ми маємо $f(x) = \displaystyle\sum_{k=1}^\infty f_kx_k$, який задає ізоморфізм. (TODO: додумати).

\subsubsection{Простір, що спряжений до $l_1$}
\begin{theorem}
Простір $(l_1)' \cong l_\infty$ ізометричним чином. Ізоморфізм між ними встановлюється формулою $f(x) = \displaystyle\sum_{k=1}^\infty f_k x_k$.
\end{theorem}

\begin{proof}
Нехай $(f_k)^{\infty}_{k=1} \in l_{\infty}$. Визначимо функціонал $f(x) = \displaystyle\sum_{k=1}^\infty f_k x_k$, який вже ясно, що лінійний. Залишилося довести обмеженість.\\
$|f(x)| = \displaystyle \left| \sum_{k=1}^\infty f_k x_k \right| \leq \sup_{k \in \mathbb{N}} |f_k| \sum_{k=1}^\infty |x_k| = \|f\|_\infty \|x\|_1$.\\
Із цього всього ми встановили $l_\infty \subset (l_1)'$.\\
Нехай $f \in (l_1)'$, тобто лінійний та обмежений функціонал. Аналогічним чином отримаємо, що $f(x) = \displaystyle\sum_{k=1}^\infty f_k x_k$, де $f_k = f(e_k)$. Тепер хочемо $(f_k)^{\infty}_{k=1} \in l_\infty$. Дійсно це спрацює, бо\\
$\|f\|_\infty = \displaystyle\sup_{k \in \mathbb{N}} |f_k| = \sup_{k \in \mathbb{N}} |f(e_k)| \leq \sup_{k \in \mathbb{N}} \|f\| \|e_k\|_1 = \|f\| \sup \{1,1,\dots\} = \|f\| < \infty$.\\
Причому ми також довели, що $\|f\| = \|f\|_\infty$.
\end{proof}

\subsubsection{Простори, що спряжені до $l_\infty$}
\begin{remark}
$(l_\infty)' \supsetneq l_1$.\\
Дійсно, нехай $(f_k)_{k=1}^\infty \in l_1$, тоді функціонал $f(x) = \displaystyle\sum_{k=1}^\infty f_k x_k,\ x \in l_\infty$ все одно лінійний, а обмеженість доводиться, завдяки оцінки\\
$|f(x)| \leq \displaystyle\sum_{k=1}^\infty |f_k x_k| \leq \sup_{k \in \mathbb{N}} |x_k| \sum_{k=1}^\infty |f_k| = \|x\|_\infty \sum_{k=1}^\infty |f_k|$.\\
Отже, довели вкладення, при цьому ми ще довели $\|f\| \leq \displaystyle\sum_{k=1}^\infty |f_k|$.\\
Якщо покласти такий $x \in l_\infty$, де $x_k = e^{-i \arg f_k}$, то взагалі отримаємо $f(x) = \displaystyle\sum_{k=1}^\infty |f_k| = \|f\|_1$.
\end{remark}

\begin{remark}
Тепер чому це вкладення лише в одну сторону.\\
Розглянемо лінійну множину $C \subset l_\infty$, яка містить збіжні послідовності комплексних чисел. Визначимо $f(x) = \displaystyle\lim_{k \to \infty} x_k$ для кожного $x = (x_1,x_2,\dots) \in C$. Цілком ясно, що це лінійний функціонал. Обмеженість вилпиває з оцінки $|f(x)| = \displaystyle \left| \lim_{k \to \infty} x_k \right| \leq \sup_{k \in \mathbb{N}} |x_k| = \|x\|$.\\
Отже, $f \in C'$ (лінійний та неперервний функціонал), причому $\|f\| \leq 1$. Ми можемо продовжити функціонал $f$ до функціонала $F \in (l_\infty)'$ зі збереженням норми, за теоремою Гана-Банаха. Функціонал $F$ не можна записати як $F(x) = \displaystyle\sum_{k=1}^\infty x_k f_k$. Представимо, що можна. Маємо послідовність $x \in C$, ліміт не зміниться при зміннні скінченного числа членів, тобто $F(x) = f(x)$ залишиться таким самим. Проте із іншого боку, зміниться $F(x) = \displaystyle\sum_{k=1}^\infty x_k f_k$.
\end{remark}

\subsubsection{Простір, що спряжений до $L_p, 1 < p < \infty$.}
\begin{theorem}
Нехайй $1 < p < \infty$ та $p' > 1$, причому $\dfrac{1}{p} + \dfrac{1}{p'} = 1$. Також задано $(X,\lambda,\mathcal{F})$ -- вимірний простір, де $\lambda$ -- $\sigma$-скінченна міра. Простір $(L_p)' \cong L_{p'}$ ізометричним чином. Ізоморфізм $l \colon (L_p)' \to L_{p'}$ задається наступним чином:\\
$l(x) = \displaystyle\int_X h(q) x(q)\,d\lambda(q)$.\\
\textit{Доведення див.\ в pdf теорії міри.}
\end{theorem}

\subsubsection{Простір, що спряжений до $C(K)$}
Припустимо, що $K$ -- метричний компакт та $\mathfrak{B}(K)$ -- борельова $\sigma$-алгебра.

\begin{definition}
Заряд $\omega$ на вимірній множині $(K, \mathfrak{B}(K))$ назвемо \textbf{регулярним}, якщо
\begin{align*}
\omega_+, \omega_- \text{ -- обидва регулярні}
\end{align*}
Позначення: $W(K)$ -- множина регулярних зарядів.
\end{definition}

\begin{remark}
$W(K)$ буде векторним простором. Також якщо покласти $\|\omega\| = |\omega|(K)$, де $|\omega|$ -- повна варіація заряда, то тоді ми отримаємо нормований простір. Причому $W(K)$ -- банахів додатково.
\end{remark}

\begin{theorem}[Теорема Маркова]
$(C(K))' \cong W(K)$ ізометричним чином. Ізоморфізм $l \colon (C(K))' \to W(K)$ задається таким чином:\\
$l(x) = \displaystyle\int_K x(q)\,d\omega(q)$.\\
\textit{Без доведення. Наведу частинний випадок даної теореми.}
\end{theorem}

\begin{theorem}[Теорема Ріса]
Для кожного функціонала $l \in (C([0,1]))'$ існує фукнція $g$ обмеженої варіації, для якої $l$ можна представити через інтеграл Рімана-Стілт'єса таким чином:\\
$l(X) = \displaystyle\int_0^1 x(t)\,dg(t)$, причому $V(g;[0,1]) = \|l\|$.\\
\textit{Без доведення}.
\end{theorem}

\subsection{Вкладення нормованих просторів}
\begin{theorem}
Нехай $E$ -- лінійний нормований простір. Тоді $E \subset E''$, під другою множиною мається на увазі друге спряження, тобто $E'' = (E')'$. При цьому $\|x\|_E = \|x\|_{E''}$.
\end{theorem}

\begin{proof}
Для зручності елементи простору $E$ позначимо через $x,y,\dots$; елементи простору $E'$ -- через $l,m,\dots$; елементри простору $E''$ -- через $L,M,\dots$\\
Визначимо відображення $\varphi$ ось так: кожному $x \in E$ поставимо в відповідність $\varphi(x) = L_x \in E''$. При цьому ми покладемо $L_x(l) = l(x)$ при всіх $l \in E'$. \\
Доведемо, що $L_x$ -- лінійний та неперервний функціонал. Нехай $l,m \in E', \lambda,\mu \in \mathbb{K}$, тоді звідси $L_x(\lambda l + \mu m) = (\lambda l + \mu m)(x) = \lambda l(x) + \mu m(x) = \lambda L_x(l) + \mu L_x(m)$. Далі маємо\\
$|L_x(l)| = |l(x)| \leq \|l\| \cdot \|x\|$.\\
Отже, довели бажане, причому ми отримали оцінку $\|L_x\| \leq \|x\|$.\\
Тепер доведемо, що саме $\varphi$ -- лінійне відображення. Нехай $x,y \in E, \lambda, \mu \in \mathbb{K}$, тоді ми хочемо довести рівність $\varphi(\lambda x + \mu y) = \lambda \varphi(x) + \mu \varphi(y)$, або що теж саме $L_{\lambda x + \mu y} = \lambda L_x + \mu L_y$. Така рівність має виконуватися для кожного функціонала $l \in E'$. Дійсно,\\
$L_{\lambda x + \mu y}(l) = l(\lambda x + \mu y) = \lambda l(x) + \mu l(y) = \lambda L_x(l) + \mu L_y(l) = (\lambda L_x + \mu L_y)(l)$.\\
Доведемо, що $\varphi$ -- ін'єктивне відображення. !Припустимо, що $x \in \ker \varphi$ та $x \neq 0$. Тоді за наслідком теореми Гана-Банаха, існує функціонал $l \in E'$, для якого $\|l\| = 1,\ l(x) = \|x\|$. Звідси $L_x(l) = l(x) = \|x\| \neq 0$, тобто $L_x \neq 0$. Це означає лише, що $x \notin \ker \varphi$ -- суперечність!\\
Залишилося довести, що $\|x\| = \|L_x\|$. Точніше, залишилося $\|x\| \leq \|L_x\|$. При $x = 0$ все ясно. При $x \neq 0$, знову за наслідком Гана-Банаха, існує функціонал $l \in E'$, для якого $\|l\| = 1,\ l(x) = \|x\|$. Тоді\\
$\|x\| = l(x) = L_x(l) \leq \|L_x\| \|l\| = \|L_x\|$.\\
Отже, $\varphi \colon E \to E''$ -- лінійне та ін'єктивне відображення, що зберігає норму. Значить, $E$ ізометрично ізоморфний $\Im E \subset E''$. Отже, кожний елемент $x \in E$ можемо ототожнити з його елементом $L_x \in E''$. Звідси отримаємо вкладення $E \subset E''$ та рівність $\|x\|_E = \|x\|_{E''}$.
\end{proof}

\begin{definition}
Задано $E$ -- банахів простір.\\
Простір $E$ називають \textbf{рефлексивним}, якщо
\begin{align*}
E'' = \varphi(E),
\end{align*}
де $\varphi \colon E \to E''$, який задавали під час доведення теореми.
\end{definition}

\begin{example}
Зокрема рефлексивними будуть такі простори: $l_p$ та $L_p$ при $1 < p < \infty$.\\
Також скінченновимірний простір $E$ буде рефлексивним.
\end{example}

\begin{example}
Водночас нерефлексивними будуть такі простори: $l_1,\ l_\infty,\ L_1,\ L_\infty$ (останні два нерефлексивні при $\dim L_1 = \infty$, $\dim L_\infty = \infty$; $C(K)$ (буде нерефелексивним, якщо $K$ нескінченна множина).
\end{example}

\begin{theorem}[Теорема Банаха-Штайнгауза]
Задано $E$ -- банахів простір та $(l_n)_{n=1}^\infty$ -- послідовність функціоналів з $E'$. Припустимо, що $\forall x \in E: (l_n(x))_{n=1}^\infty$ -- обмежена послідовність. Тоді $(\|l_n\|)_{n=1}^\infty$ (послідовність норм) -- обмежена.\\
\textit{Дана теорема носить назву 'принцип рівномірної обмеженості'.}
\end{theorem}

\begin{proof}
Нехай $\forall x \in E: (l_n(x))_{n=1}^\infty$ -- обмежена послідовність. Доведемо, що існує замкнений шар $B[a;r]$, де множина $\{l_n(x), x \in B[a;r]\}_{n=1}^\infty$ обмежена.\\ 
!Припустимо навпаки, що множина $\{l_n(x)\}_{n=1}^\infty$ не обмежена в жодному замкненому кулі (як наслідок, в жодному відкритому кулі). \\
Візьмемо довільну відкриту кулю $B(x_0;r_0)$, де ось ця множина $\{l_n(x), x \in B(x_0;r_0)\}_{n=1}^\infty$ не обмежена. Це, що знайдуться $x_1 \in B(x_0;r_0)$ та $n_1 \in \mathbb{N}$, для яких $|l_{n_1}(x_1)| > 1$. Оскільки $l_{n_1}$ неперервний, то нерівність $|l_{n_1}(x)| > 1$ виконується в деякому околі $B(x_1;r_1)$ (?). За необхідністю, зменшимо радіус $r_1$ таким чином, щоб $B[x_1;r_1] \subset B(x_0;r_0)$, причому сам радіус $r_1 \overset{\text{зобов'язаний}}{\leq} \dfrac{r_0}{2}$ (дійсно, можна підібрати $r_1 = \dfrac{r_0 - \rho(x_0,x_1)}{2}$). \\
Ця множина $\{l_n(x), x \in B(x_1;x_1)\}$ теж не обмежена. Тоді знайдуться $x_2 \in B(x_1;x_1)$ та $n_2 > n_1$, для яких $|l_{n_2}(x_2)| > 2$. Аналогічно нерівність $|l_{n_2}(x)| > 2$ виконуватиметься в деякому замкненому шарі $B[x_2;r_2] \subset B(x_1;r_1)$, причому $r_2 \overset{\text{зобов'язаний}}{\leq} \dfrac{r_0}{2^2}$.\\
\vdots \\
Продовжуючи процес, отримаємо послідовність замкнених шарів $B[x_0;r_0] \supset B[x_1;r_1] \supset \dots$, причому $r_k \to 0$, числа $n_1 < n_2 < \dots$ такі, що $|l_{n_k}(x)| > k$ при $x \in B[x_k;r_k]$. За теоремою Кантора, існує точка $x^* = \displaystyle\lim_{k \to \infty} x_k$. Звідси випливає, що $|l_{n_k}(x^*)| > k$ при всіх $k$ -- суперечність! Бо послідовність $(l_{n_k}(x^*))_{k=1}^\infty$ мала б бути обмеженою за початковими умовами.\\
Висновок: існує шар $B[a;r]$, де множина $\{l_n(x), x \in B[a;r]\}$ обмежена. Тобто $\exists c' > 0: \forall x \in B[a;r], \forall n \in \mathbb{N}: |l_n(x)| \leq c'$. Досить буде довести, що множина $\{l_n(x), x \in B[0;1]\}$ обмежена. Для кожного $x \in B[0;1]$ покладемо $x' = rx + a$, тоді $x = \dfrac{1}{r}(x'-a)$. Оскільки $x' \in B[a;r]$, то $|l_n(x')| < c'$. Звідси\\
$|l_n(x)| = \left| l_n \left( \dfrac{1}{r}(x'-a) \right) \right| = \dfrac{1}{r} |l_n(x') - l_n(a)| \leq \dfrac{1}{r} (|l_n(x')| + |l_n(a)|) \leq \dfrac{c'+c_a}{r} = c$.\\
Висновок: $\exists c > 0: \forall x \in B[0;1], \forall n \geq 1: |l_n(x)| \leq c$. Проте умова $x \in B[0;1]$ означає, що $\|x\| \leq 1$. Тобто нерівність $|l_n(x)| \leq c$ для всіх $\|x\| \leq 1$. Зокрема звідси $\displaystyle\sup_{\|x\| \leq 1} |l_n(x)| = \|l_n\| \leq c$.
\end{proof}

\begin{remark}
Пояснення (?). Якби для кожного околу $B(x_1,r)$ (зокрема при $r = \dfrac{1}{n}$) існувала точка, де нерівність порушується, то ми би побудували послідовність, що прямує до $x_1$, при цьому ми би отримали $|l_{n_1}(x)| \leq 1$.
\end{remark}

\begin{remark}
У теоремі Банаха-Штайнгауза умова того, що $E$ -- банахів, -- суттєва.\\
Зокрема розглянемо простір $c_0$ -- послідовності, що збігаються до нуля. Далі розглянемо підпростір $c_{00} \subset c_0$ -- послідовності, де всі члени нулі, починаючи з деякого номера.
\end{remark}

\subsection{Про види збіжностей}
Ми вже знаємо один тип збіжностей. Переформулюю ще раз означення, але доповню це одним словом в дужках.
\begin{definition}
Задано $E$ -- лінійний нормований простір.\\
Послідовність $(x_n)_{n=1}^\infty$ називається \textbf{(сильно) збіжною} до $x \in E$, якщо
\begin{align*}
\lim_{n \to \infty} \|x - x_n\| = 0
\end{align*}
Позначення: $x_n \to x$.\\
Тобто сильна збіжність -- це збіжність за нормою.
\end{definition}

\begin{definition}
Нехай $E$ -- лінійний нормований простір.\\
Послідовність $(x_n)_{n=1}^\infty$ називається \textbf{слабко збіжною} до $x \in E$, якщо
\begin{align*}
\forall l \in E': l(x_n) \to l(x)
\end{align*}
Позначення: $x_n \toweak x$.
\end{definition}

\begin{proposition}
Задано $E$ -- лінійний нормований простір та послідовність $(x_n)_{n=1}^\infty$. Тоді:\\
$x_n \to x \implies x_n \toweak x$.
\end{proposition}

\begin{proof}
Дійсно, нехай $x_n \to x$, тобто звідси $\|x-x_n\| \to 0$ при $n \to \infty$. Маючи це, отримаємо $\forall l \in E'$:\\
$|l(x_n)-l(x)| = |l(x_n-x)| \leq \|l\| \|x_n - x\| \to 0$. Таким чином, $x_n \toweak x$.
\end{proof}

Якщо розглядати спряжений простір $E'$, то крім сильної та слабкої збіжності існує ще один тип.

\begin{definition}
Нехай $E$ -- лінійний нормований простір.\\
Послідовність функціоналів $(l_n)_{n=1}^\infty \subset E'$ називається \textbf{слабко* збіжною} до $l \in E'$, якщо
\begin{align*}
\forall x \in E: l_n(x) \to l(x)
\end{align*}
Позначення: $l_n \toweakstar l$.
\end{definition}

\begin{proposition}
Задано $E$ -- лінійний нормований простір та послідовність $(l_n)_{n=1}^\infty \subset E'$. Тоді:\\
$l_n \to l \implies l_n \toweak l \implies l_n \toweakstar l$.
\end{proposition}

\begin{proof}
Імплікація $l_n \to l \implies l_n \toweak l$ була доведена вище. Залишилося $l_n \toweak l \implies l_n \toweakstar l$.\\
Нехай $l_n \toweak l$, тобто $\forall L \in E'': L(l_n) \to L(l)$. Зафіксуємо елемент $x \in E$. Ми вже доводили, що $E \subset E''$, тобто $x \in E''$, де в цьому випадку $x = L_x$ такий, що $L_x(l) = l(x)$. Звідси\\
$l_n(x) = L_x(l_n) \to L_x(l) = l(x)$. Звідси випливає, що $l_n \toweakstar l$.
\end{proof}

\begin{example}
Зараз покажемо, чому в зворотний бік не працює.
\bigskip \\
$x_n \toweak x \centernot\implies x_n \to x$.\\
Розглянемо простір $l_p$ та зафіксуємо послідовність $(e_n)_{n=1}^\infty$, де кожний $e_j$ -- елемент базиса Шаудера. Спочатку покажемо, що $(e_n)_{n=1}^\infty$ слабко збігається. Зафіксуємо довільний функціонал $l \in (l_p)' = l_{p'}$, тобто $l = (l_1,l_2,\dots)$. Це означає, що $\displaystyle\sum_{j=1}^\infty |l_j|^{p'} < + \infty$, а тому за необіхдною умовою, $|l_j|^{p'} \to 0 \implies l_j \to 0$. Із іншого боку, ми вже знаємо, що $l_j = l(e_j) \to 0 = l(0)$ при $j \to \infty$. Це як раз свідчить про те, що $e_j \toweak 0$.\\
Проте зауважимо, що $\|e_j - 0 \| = \|e_j\| = 1 \centernot\to 0$. Це як раз означає, що $e_j \centernot\to 0$.
\bigskip \\
$f_n \toweakstar f \centernot\implies f_n \toweak f$.\\
Розглянемо простір $l_1$ та зафіксуємо послідовність $(f_n)_{n=1}^\infty$, $f_n = ?$. Спочатку покажемо, що $(f_n)_{n=1}^\infty$ слабко* збігається. Зафіксуємо довільний елемент $x \in c_0$. Значить, $x_j \to 0$ при $j \to \infty$. Оберемо $f_n = (-1)^n$. Тоді звідси $f_n(x) = \displaystyle\sum_{k=1}^\infty x_k f_n^k(e_k)$. (TODO: не можу добити)
\end{example}

\begin{proposition}
Утім якщо $E$ -- рефлексивний лінійний нормований простір та $(l_n)_{n=1}^\infty \subset E'$, тоді\\
$l_n \toweak l \iff l_n \toweakstar l$.
\end{proposition}

\begin{remark}
Границя єдина за слабкою* збіжністю, слабкою збіжністю та сильною збіжністю.
\end{remark}

\begin{proposition}
Задано $E$ -- банахів та послідовність $(l_n)_{n=1}^\infty$, яка слабко* збігається. Тоді $(l_n)_{n=1}^\infty$ -- обмежена.
\end{proposition}

\begin{proof}
Дійсно, маємо $\forall x \in E: l_n(x) \to l(x)$, тобто числова послідовність $(l_n(x))_{n=1}^\infty$ збігається, тоді обмежена. Значить, за теоремою Банаха-Штайнгауза, послідовність $(\|l_n\|)_{n=1}^\infty$ обмежена.
\end{proof}


\begin{theorem}
Задано $E$ -- банахів простір та $(l_n)_{n=1}^\infty \subset E'$ -- така послідовність, що $\forall x \in E: (l_n(x))_{n=1}^\infty$ -- фундаментальна. Тоді $\exists l \in E': l_n \toweakstar l$.
\end{theorem}

\begin{proof}
Оскільки $\forall x \in E: (l_n(x))_{n=1}^\infty$ фундаментальна, то (як числова послідовність) вона збіжна. Визначимо функціонал $l(x) = \displaystyle\lim_{n \to \infty} l_n(x)$. Зважаючи на той факт, що $l_n$ -- лінійний, то $l$ -- лінійний в силу граничного переходу. Залишилося довести обмеженість.\\
При кожному $x \in E$ послідовність $(l_n(x))_{n=1}^\infty$ (вже з'ясували) збіжна, тож обмежена. Але за теоремою Банаха-Штайнгауза, $\exists c > 0: \forall n \geq 1: \|l_n\| \leq c$. Значить, $\forall n \geq 1, \forall x \in E: |l_n(x)| \leq \|l\| \|x\| \leq c \|x\|$. Знову переходячи до границі, отримаємо $|l(x)| \leq c \|x\|$.\\
Отже, $\forall x \in E: l_n(x) \to l(x) \implies l_n \toweakstar l$. 
\end{proof}

\begin{theorem}[Критерій слабкої* збіжності]
Задано $E$ -- банахів та множина $M$ -- скрізь щільна в $E$. Нехай $(l_n)_{n=1}^\infty \subset E'$.\\
$l_n \toweakstar l \iff \begin{cases} \forall x \in M: l_n(x) \to l(x) \\ \exists c > 0: \forall n \geq 1: \|l_n\| \leq c \end{cases}$.
\end{theorem}

\begin{proof}
\rightproof Дано: $l_n \toweakstar l$. Тобто $\forall x \in E: l_n(x) \to l(x)$, зокрема $\forall x \in M$. Обмеженість норм $\|l_n\|$ автоматично виконується.
\bigskip \\
\leftproof Дано: ці дві умови. Ми хочемо $\forall y \in E: l_n(y) \to l(y)$.\\
При $x \in M$ маємо наступне:\\
$|l_n(y) - l(y)| \leq |l_n(y) - l_n(x)| + |l_n(x) - l(x)| + |l(x) - l(y)| \leq \| l_n \| \| y - x \| + |l_n(x) - l(x)| + \|l\| \|x-y\| \leq \\
\leq (c + \|l\|) \|x-y\| + |l_n(x) - l(x)|$.\\
Проте $\Cl(M) = E$, тож звідси $\forall y \in E: \forall \varepsilon > 0: \exists x \in M: \|x-y\| < \dfrac{\varepsilon}{2(c+ \|l\|)}$. В силу першої умови, $\exists N: \forall n > N: |l_n(x) - l(x)| < \dfrac{\varepsilon}{2}$.\\
Значить, $|l_n(y) - l(y)| < \varepsilon$.
\end{proof}

\begin{remark}
Судячи з доведення, в \leftproof не обов'язково вимагати бути $E$ повним. Також в формулюванні теореми досить вимагати, щоб $M$ була тотальною в $E$.
\end{remark}

\newpage

\section*{Твердження, які потім вставлю в необхідне місце}
\begin{proof}
Достатньо довести, що всі норми еквівалентні до $\| \cdot \|_2$.\\
Нехай $\{\vec{e}_1,\dots,\vec{e}_d\}$ -- стандартний базис $\mathbb{R}^d$, тоді звідси $\vec{x} = \displaystyle\sum_{i=1}^d x_i \vec{e}_i$.\\
$\displaystyle\left\| \sum_{i=1}^d x_i \vec{e}_i
 \right\| \leq \sum_{i=1}^d \| x_i e_i \| = \sum_{i=1}^d |x_i| \|e_i\| = \sqrt{\left( \sum_{i=1}^d |x_i| \|\vec{e}_i\| \right)^2} \overset{\text{К-Б}}{\leq} \sqrt{\sum_{i=1}^d \|e_i\|^2} \sqrt{\sum_{j=1}^d |x_j|^2} \leq \sqrt{\sum_{i=1}^d \|e_i\|^2} \sqrt{\sum_{j=1}^\infty |x_j|^2} = \sqrt{\sum_{i=1}^d \|e_i\|^2} \|\vec{x}\|_2 = M \|\vec{x}\|_2$.\\
Зауважимо, що $M \in \mathbb{R}_{\geq 0}$ та не залежить від $\vec{x}$. Отже, $\|\vec{x}\| \leq M \|\vec{x}\|_2$.
\bigskip \\
Розглянемо тепер $S$ -- одинична сфера на $(\mathbb{R}^d, \|\cdot \|_2)$. Відомо, що $S$ -- замкнена множина та обмежена. Тож за лемою Гейне-Бореля, $S$ -- компактна множина. Відомо, що відображення $\| \cdot \| \colon S \to \mathbb{R}_{\geq 0}$ -- неперервне відображення, тож вона досягає найменшого значення $m$ для деякого $\vec{y} \in S$.\\
Припустимо $m = 0$, тоді звідси $\|\vec{y}\| = 0 \implies \vec{y} = \vec{0} \implies \vec{y} \notin S$ -- неможливо. Отже, $m > 0$.\\
Значить, $\forall \vec{y} \in \mathbb{R}^d: \|\vec{y}\|_2 = 1: \|y\| \geq m$. Треба довести те саме для інших векторів.\\
Якщо $\vec{x} = \vec{0}$, то це виконано. Тому $\vec{x} \neq \vec{0}$. Покладемо вектор $\vec{y} = \dfrac{\vec{x}}{\| \vec{x} \|_2}$, причому $\|\vec{y}\|_2 = 1$. Із цього випливає, що $\| \vec{y} \|_2 \leq m \implies m \|\vec{x}\|_2 \leq \|\vec{x}\|$.
\bigskip \\
Всі інші норми будуть еквівалентними в силу транзитивності.
\end{proof}

\begin{definition}
Задано $X,Y$ -- нормовані простори.\\
Вони називаються \textbf{ізоморфними}, якщо існує бієктивний лінійний оператор $A \colon X \to Y$, для якого
\begin{align*}
\forall x \in X: \|Ax\|_Y = \|x\|_X
\end{align*}
Водночас такий оператор $A$ називають \textbf{ізоморфізмом}.\\
Позначення: $X \cong Y$
\end{definition}

\end{document}
