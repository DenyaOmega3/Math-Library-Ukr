\documentclass[a4paper, 10pt]{article}
\usepackage[margin=1in]{geometry}
\usepackage{amsfonts, amsmath, amssymb, amsthm}
%\usepackage[none]{hyphenat}
\usepackage{fancyhdr} %create a custom header and footer
\usepackage[utf8]{inputenc}
\usepackage[english, main=ukrainian]{babel}
\usepackage{pgfplots}
\usepgfplotslibrary{fillbetween}
\usepackage{tikz}
\usepackage{graphicx}
\usepackage{caption}
\usepackage{float}
\usepackage{physics}
\usepackage[unicode]{hyperref}
\usepgfplotslibrary{polar}
\usepackage{xifthen}
\usepackage{enumitem}
\usetikzlibrary{spy}
\usepackage{bbm}
\usepackage{centernot}

\fancyhead{}
\fancyfoot{}
\parindent 0ex
\def\rightproof{$\boxed{\Rightarrow}$ }
\def\leftproof{$\boxed{\Leftarrow}$ }

\usepackage{pdfpages}

\newtheoremstyle{theoremdd}% name of the style to be used
  {\topsep}% measure of space to leave above the theorem. E.g.: 3pt
  {\topsep}% measure of space to leave below the theorem. E.g.: 3pt
  {\normalfont}% name of font to use in the body of the theorem
  {0pt}% measure of space to indent
  {\bfseries}% name of head font
  {}% punctuation between head and body
  { }% space after theorem head; " " = normal interword space
  {\thmname{#1}\thmnumber{ #2}\textnormal{\thmnote{ \textbf{#3}\\}}}

\theoremstyle{theoremdd}
\newtheorem{theorem}{Theorem}[subsection]
  
\theoremstyle{theoremdd}
\newtheorem{definition}[theorem]{Definition}

\theoremstyle{theoremdd}
\newtheorem{samedef}[theorem]{Definition}

\theoremstyle{theoremdd}
\newtheorem{example}[theorem]{Example}

\theoremstyle{theoremdd}
\newtheorem{proposition}[theorem]{Proposition}

\theoremstyle{theoremdd}
\newtheorem{remark}[theorem]{Remark}

\theoremstyle{theoremdd}
\newtheorem{lemma}[theorem]{Lemma}

\theoremstyle{theoremdd}
\newtheorem{corollary}[theorem]{Corollary}

\newcommand\thref[1]{\textbf{Th.~\ref{#1}}}
\newcommand\defref[1]{\textbf{Def.~\ref{#1}}}
\newcommand\exref[1]{\textbf{Ex.~\ref{#1}}}
\newcommand\prpref[1]{\textbf{Prp.~\ref{#1}}}
\newcommand\rmref[1]{\textbf{Rm.~\ref{#1}}}
\newcommand\lmref[1]{\textbf{Lm.~\ref{#1}}}
\newcommand\crlref[1]{\textbf{Crl.~\ref{#1}}}

\renewcommand{\qedsymbol}{$\blacksquare$}
\newcommand{\toweak}{\overset{w}{\to}}
\newcommand{\tostrong}{\overset{s}{\to}}
\newcommand{\toweakstar}{\overset{w^*}{\to}}
\newcommand{\sequence}[2][]{\left({#2}\right)_{n=\ifthenelse{\isempty{#1}}{1}{#1}}^{\infty}}
%first argument is optional: where to start sequence
%second argument is mandatory: which sequence

\DeclareMathOperator{\ort}{ort}
\DeclareMathOperator{\pr}{pr}

\makeatletter
\renewenvironment{proof}[1][Proof.\\]{\par
\pushQED{\hfill \qed}%
\normalfont \topsep6\p@\@plus6\p@\relax
\trivlist
\item\relax
{\bfseries
#1\@addpunct{.}}\hspace\labelsep\ignorespaces
}{%
\popQED\endtrivlist\@endpefalse
}
\makeatother

\newenvironment{pf}{\vspace*{-3mm} \textbf{Proof. \\}}{$\blacksquare$}
\newenvironment{pfMI}{\vspace*{-3mm} \textbf{Proof MI. \\}}{$\blacksquare$}
\newenvironment{pfNoTh}{\textbf{Proof. \\}}{$\blacksquare$}

\DeclareMathOperator{\Cl}{Cl}
\DeclareMathOperator{\linspan}{span}
\DeclareMathOperator{\sign}{sgn}
\DeclareMathOperator{\diag}{diag}

%delete

\begin{document}
\tableofcontents
\newpage

\section{Метричні простори та інше}
\subsection{Означення метричних просторів}
\begin{definition}
Задано $X$ -- множина та $\rho \colon X \to X \to \mathbb{R}$ -- функція.\\
Функція $\rho$ називається \textbf{метрикою}, якщо вона задовольняє таким властивостям:
\begin{align*}
\text{1) } \forall x,y \in X: \rho(x,y) \geq 0 ,\qquad \rho(x,y) = 0 \iff x = y \\
\text{2) } \forall x,y \in X: \rho(x,y) = \rho(y,x) \\
\text{3) } \forall x,y,z \in X: \rho(x,z) \leq \rho(x,y) + \rho(y,z)
\end{align*}
Метрика описує \textbf{відстань} між елементами $x,y$. \\
Пара $(X,\rho)$ з метрикою називається \textbf{метричним простором}.
\end{definition}


\begin{example}
Розглянемо декілька прикладів:
\begin{enumerate}[nosep,wide=0pt,label={\arabic*)}]
\item $X = \mathbb{R}$, \qquad $\rho(x,y) = |x-y|$;
\item $X = \mathbb{R}^n$, можна задати дві метрики: \\
$\rho_1(\vec{x}, \vec{y}) = \sqrt{(x_1-y_1)^2 + \dots + (x_n-y_n)^2}$, \qquad $\rho_2(\vec{x}, \vec{y}) = |x_1-y_1|+\dots+|x_n-y_n|$;
\item $X = C([a,b])$, \qquad $\displaystyle \rho(f,g) = \max_{t \in [a,b]} |f(t)-g(t)|$.
\end{enumerate}
\end{example}

\begin{example}
Окремо розгляну даний приклад. Нехай $X$ -- будь-яка множина, ми визначимо так звану \textbf{дискретну метрику} $d(x,y) = \begin{cases} 1, & x \neq y \\ 0, & x = y \end{cases}$. Тоді $(X,d)$ задає \textbf{дикретний} метричний простір.
\end{example}

\begin{example}
Розглянемо $X = \mathbb{N}$ та функцію $\rho(m,n) = 1 + \dfrac{1}{m+n}$ при $m \neq n$, інакше $\rho(m,n) = 0$. Доведемо, що $\rho$ задає метрику.
\begin{enumerate}[nosep,wide=0pt,label={\arabic*)}]
\item $\rho(m,n) \geq 0$ -- це зрозуміло, також $\rho(m,n) = 0 \iff m = n$ за визначенням функції;
\item $\rho(n,m) = 1 + \dfrac{1}{n+m} = 1 + \dfrac{1}{m+n} = \rho(m,n)$;
\item Тут ситуація менш приємна, ми хочемо $\rho(m,n) \leq \rho(m,k) + \rho(k,n)$. Спочатку розглянемо випадки, коли $m,n,k$ попарно не рівні. Зауважимо, що справедлива нерівність при $m,n,k \in \mathbb{N}$:\\
$\dfrac{1}{m+n} \leq 1 + \dfrac{1}{m+k} + \dfrac{1}{k+n}$.\\
Якщо додати до обох частей нерівності $1$, то ми отримаємо:\\
$\rho(m,n) = 1 + \dfrac{1}{m+n} \leq 1 + 1 + \dfrac{1}{m+k} + \dfrac{1}{k+n} = 1 + \dfrac{1}{m+k} + 1 + \dfrac{1}{k+n} = \rho(m,k) + \rho(k,n)$.
\end{enumerate}
Отже, $(\mathbb{N}, \rho)$ задає метричний простір.
\end{example}

\begin{definition}
Задано $(X,\rho)$ -- метричний простір.\\
Пару $(Y,\tilde{\rho})$, де $Y \subset X$, назвемо \textbf{метричним підпростором} $(X,\rho)$, якщо
\begin{align*}
\forall x,y \in Y: \tilde{\rho}(x,y) = \rho(x,y).
\end{align*}
При цьому метрика $\tilde{\rho}$, кажуть, \textbf{індукована в} $Y$ \textbf{метрикою} $\rho$.
\end{definition}

\begin{proposition}
Задано $(X,\rho)$ -- метричний простір та $(Y,\tilde{\rho})$ -- підпростір. Для функції $\tilde{\rho}$ всі три аксіоми зберігаються. Тобто $(Y,\tilde{\rho})$ залишається метричним простором.\\
\textit{Вправа: довести.}
\end{proposition}

\begin{example}
Маємо $X = F([a,b])$ -- множину обмежених функцій та $\rho(f,g) = \displaystyle \sup_{t \in [a,b]} |f(t)-g(t)|$. Тоді в $Y = C([a,b])$ маємо метрику $\tilde{\rho}(f,g) = \displaystyle \max_{t \in [a,b]} |f(t)-g(t)| = \displaystyle \sup_{t \in [a,b]} |f(t)-g(t)|$. Отже, $C([a,b])$ -- метричний підпростір простору $F([a,b])$.
\end{example}

\iffalse
(TODO: інформацію про скалярний добуток перемістити в інші місця).
\begin{example}
Задано $(E, (\cdot, \cdot))$ -- евклідів простір. Ми можемо евклідів простір $E$ перетворити в нормований простір $(E, \| \cdot \|)$ функцією $\|x\| = \sqrt{(x,x)}$.\\
Як наслідок, за твердженням вище, $(E, \rho)$ - метричний простір з $\rho(x,y) = \|x-y\|$.
\end{example}
\fi

\iffalse %TODO про простір l_p додати в теорії міри
\begin{example} Нехай $\vec{a} = (a_1,a_2,\dots)$ -- дійсна числова послідовність. Задамо простір $l_1 = \left\{ \vec{a} \mid \displaystyle\sum_{n=1}^\infty |a_n| < \infty \right\}$. Задаються такі операції:\\
$\vec{a} + \vec{b} = (a_1,a_2,\dots) + (b_1,b_2,\dots) = (a_1+b_1,a_2+b_2,\dots)$\\
$\alpha \vec{a} = (\alpha a_1, \alpha a_2, \dots)$\\
Неважко переконатися, що $l_1$ -- лінійний простір над $\mathbb{R}$.\\
Важливе зауваження: $\vec{a}+\vec{b}, \alpha \vec{a} \in l_1$, тому що маємо $\displaystyle\sum_{n=1}^\infty a_n$, $\displaystyle\sum_{n=1}^\infty b_n$ -- збіжні, а тому збіжним буде $\displaystyle \sum_{n=1}^\infty (a_n+b_n), \sum_{n=1}^\infty \alpha a_n$. Тобто операції замкнені.\\
Можна задати нормований простір функцією $\| \vec{a} \| = \displaystyle \sum_{n=1}^\infty |a_n|$. А тому це -- метричний простір з $\rho(\vec{a}, \vec{b}) = \|\vec{a} - \vec{b}\|$.
\bigskip \\
Узагальнення: $l_p = \left\{ \vec{a} \text{ } | \displaystyle\sum_{n=1}^\infty |a_n|^p < \infty \right\}$, тут задається норма $\|\vec{a}\| = \left( \displaystyle\sum_{n=1}^\infty |a_n|^p \right)^{\frac{1}{p}}$.
\end{example}

\begin{example}
Тут ще є така множина: $l_{\infty} = \{ \vec{a} \mid \vec{a} \text{ -- обмежені} \}$. Задані такі самі операції, що вище. Задається норма $\|\vec{a}\| = \displaystyle \sup_{n \in \mathbb{N}} |a_n|$. Отже, $l_{\infty}$ -- метричний простір.
\end{example}
\fi

\subsection{Відкриті та замкнені множини. Збіжні послідовності}
\begin{definition}
Задано $(X,\rho)$ -- метричний простір та $a \in X$.\\
\textbf{Відкритою кулею радіусом $r$ з центром $a$} називають таку множину:
\begin{align*}
B(a;r) = \{x \in X \mid \rho(a,x) < r\}
\end{align*}
Її ще називають $r$\textbf{-околом точки $a$}.\\
\textbf{Замкненою кулею радіусом $r$ з центром $a$} називають таку множину:
\begin{align*}
B[a;r] = \{x \in X \mid \rho(a,x) \leq r \}
\end{align*}
\end{definition}

\begin{example}
Розглянемо декілька прикладів:
\begin{enumerate}[nosep,wide=0pt,label={\arabic*)}]
\item $X = \mathbb{R}$, $\rho(x,y) = |x-y|$, \qquad $B(a;r) = \{x \in \mathbb{R} \mid |x-a| < r\} = (a-r,a+r)$;
\item $X = \mathbb{R}^2$, $\rho(\vec{x}, \vec{y}) = \sqrt{(x_1-y_1)^2 + (x_2-y_2)^2}$, \qquad $B(0;1) = \{(x,y) \in \mathbb{R}^2 \mid x^2+y^2 <1 \}$.
\end{enumerate}
\end{example}

\begin{definition}
Задані $(X,\rho)$ -- метричний простір, $A \subset X$ та $a \in A$.\\
Точка $a$ називається \textbf{внутрішньою} для $A$, якщо
\begin{align*}
\exists \varepsilon > 0: B(a; \varepsilon) \subset A.
\end{align*}
\end{definition}

\begin{definition}
Задані $(X,\rho)$ -- метричний простір та $A \subset X$.\\
Множина $A$ називається \textbf{відкритою}, якщо
\begin{align*}
\text{кожна точка множини $A$ -- внутрішня.}
\end{align*}
\end{definition}

\begin{example} 
Розглянемо такі приклади:
\begin{enumerate}[wide=0pt,label={\arabic*)}]
\item Маємо $X = \mathbb{R}, \rho(x,y) = |x-y|$ та множину $A = [0,1)$. Точка $a = \dfrac{1}{2}$ -- внутрішня, оскільки $\exists \varepsilon = \dfrac{1}{4}: B\left(\dfrac{1}{2}; \dfrac{1}{4} \right) \subset A$, тобто $\left( \dfrac{1}{4}, \dfrac{3}{4} \right) \subset [0,1)$. Водночас точка $a = 0$ -- не внутрішня. Отже, $A$ -- не відркита, бо знайшли не внутрішню точку.

\item Маємо $X = [0,1], \rho(x,y) = |x-y|$ та множину $A = [0,1)$. У цьому випадку точка $a = 0$ уже внутрішня (в попередньому прикладі ми могли $\varepsilon$-околом вийти за межі нуля ліворуч, а тут вже ні). Тут $A$ тепер відкрита.

\item Маємо $X = \{0,1,2\}$ -- підпростір метричного простору $(\mathbb{R}, \rho(x,y) = |x-y|)$. Задамо множину $A = \{0,1\}$. Тут кожна точка -- внутрішня. Отже, $A$ -- відкрита.
\end{enumerate}
\end{example}

\begin{definition}
Задані $(X,\rho)$ -- метричний простір, $A \subset X$ та $x_0 \in X$.\\
Точка $x_0$ називається \textbf{граничною} для $A$, якщо
\begin{align*}
\forall \varepsilon > 0: (B(x_0;\varepsilon) \setminus \{x_0\}) \cap A \neq \emptyset
\end{align*}
Інколи ще множину $B(x_0;\varepsilon) \setminus \{x_0\} \overset{\text{позн.}}{=} \mathring{B}(x_0;\varepsilon)$ називають \textbf{проколеним околом точки $x_0$}.
\end{definition}

\begin{definition}
Задані $(X,\rho)$ -- метричний простір та $A \subset X$.\\
Множина $A$ називається \textbf{замкненою}, якщо
\begin{align*}
\text{вона містить всі свої граничні точки}
\end{align*}
\end{definition}

\begin{example}
Розглянемо такі приклади:
\begin{enumerate}[wide=0pt,label={\arabic*)}]
\item Маємо $X = \mathbb{R}, \rho(x,y) = |x-y|$ та множину $A = (0,1)$. Точки $x_0 \in \left\{\dfrac{1}{2}, 0, 1\right\}$ -- граничні. Водночас точка $x_0 = \dfrac{3}{2}$ -- не гранична. Отже, $A$ -- не замкнена, бо $x_0 = 1$ хоча й гранична для $A$, але $x_0 \notin A$.
\item Маємо $X = \mathbb{R}, \rho(x,y) = |x-y|$. Задамо множину $A = \{0,1 \}$. Тут жодна точка -- не гранична. Тим не менш, $A$ -- замкнена. Бо нема жодної граничної точки в $X$ для $A$, щоб порушити означення.
\item $X, \emptyset$ -- замкнені в будь-якому метричному просторі.
\end{enumerate}
\end{example}

\begin{theorem}
\label{set_open_iff_complement_set_closed}
Задані $(X,\rho)$ -- метричний простір та $A \subset X$.\\
Множина $A$ -- відкрита $\iff$ множина $A^c$ -- замкнена
\end{theorem}

\begin{proof}
\rightproof Дано: $A$ -- відкрита.\\
!Припустимо, що $A^c$ -- не замкнена, тобто $\exists x_0 \in A: x_0$ -- гранична для $A^c$, але $x_0 \notin A^c$. За умовою, оскільки $x_0 \in A$, то $x_0$ - внутрішня, тобто $\exists \varepsilon > 0: B(x_0;\varepsilon) \subset A$. Отже, $B(x_0;\varepsilon) \cap A^c = \emptyset$ -- суперечність!
\bigskip \\
\leftproof Дано: $A^c$ -- замкнена. Тоді вона містить всі граничні точки. Тоді $\forall x_0 \in A: x_0$ -- не гранична для $A^c$, тобто $\exists \varepsilon > 0: B(x_0;\varepsilon) \cap A^c = \emptyset \implies B(x_0;\varepsilon) \subset A$. Отже, $x_0$ -- внутрішня для $A$, а тому $A$ -- відкрита.
\end{proof}

\begin{example}
Розглянемо дискретний метричний простір $(X,d)$. Покажемо, що всі множини -- відкриті.\\
Нехай $A \subset X$, розглянемо $a \in A$. Тоді існує окіл $B\left(a;\dfrac{1}{2}\right) = \left\{ x \in X \mid \rho(x,a) < \dfrac{1}{2} \right\} = \{a\} \subset A$. Це виконується для всіх $a \in A$, тому $A$ -- відкрита.\\
Всі множини відкриті, а тому всі множини також замкнені.
\end{example}

\begin{theorem} 
Задано $(X,\rho)$ -- метричний простір. Тоді справедливе наступне:
\begin{enumerate}[nosep,wide=0pt,label={\arabic*)}]
\item Нехай $\{U_{\alpha} \subset X,\ \alpha \in I\}$ -- (довільна) сім'я відкритих множин. Тоді $\displaystyle \bigcup_{\alpha \in I} U_{\alpha}$ -- відкрита множина;
\item Нехай $\{U_k \subset X, k = \overline{1,n}\}$ -- (скінченна) сім'я відкритих множин. Тоді $\displaystyle \bigcap_{k=1}^n U_k$ -- відкрита множина;
\item $\emptyset, X$ -- відкриті множини.
\end{enumerate}
\end{theorem}

\begin{proof}
Доведемо кожний пункт окремо:
\begin{enumerate}[wide=0pt,label={\arabic*)}]
\item Задано множину $U = \displaystyle \bigcup_{\alpha \in I} U_{\alpha}$. Зафіксуємо $a \in U$. Тоді $\exists \alpha_0: a \in U_{\alpha_0} \implies a$ -- внутрішня для $U_{\alpha_0} \\ \implies \exists \varepsilon > 0: B(a;\varepsilon) \subset U_{\alpha_0} \subset U$. Отже, $U$ -- відкрита.

\item Задано множину $U = \displaystyle \bigcap_{k=1}^n U_k$. Зафіксуємо $a \in U$. Тоді $\forall k = \overline{1,n}: a \in U_k \implies a$ -- внутрішня для $U_k \implies \exists \varepsilon_k > 0: B(a;\varepsilon_k) \subset U_k$. Задамо $\varepsilon = \displaystyle\min_{1 \leq k \leq n} \varepsilon_k \implies B(a;\varepsilon) \subset U$. Отже, $U$ -- відкрита.

\item $\emptyset$ -- відкрита, бо нема внутрішніх точок, тому що там порожньо. Також $X$ -- відкрита, оскільки для $a \in X$, який б $\varepsilon > 0$ не обрав, $B(a;\varepsilon) \subset X$.
\end{enumerate}
Всі твердження доведені.
\end{proof}

\begin{remark}
Нижче буде наданий приклад, чому в другому твердженні лише скінченна кількість відкритих множин.
\end{remark}

\begin{example}
Розглянемо $X = \mathbb{R}$ із метрикою $\rho(x,y) = |x-y|$. Задана сім'я відкритих множин $U_n = \left( -\dfrac{1}{n}, \dfrac{1}{n} \right)$, причому $\forall n \geq 1$. Тоді зауважимо, що $\displaystyle\bigcap_{n=1}^\infty U_n = \{0\}$, але така множина вже не є відкритою.
\end{example}

\begin{corollary}
Задано $(X,\rho)$ -- метричний простір. Тоді справедливо наступне:
\begin{enumerate}[nosep,wide=0pt,label={\arabic*)}]
\item Нехай $U_{\alpha} \subset X$, $\alpha \in I$ -- сім'я замкнених множин. Тоді $\displaystyle \bigcap_{\alpha \in I} U_{\alpha}$ -- замкнена множина;
\item Нехай $U_k \subset X, k = \overline{1,n}$ -- сім'я замкнених множин. Тоді $\displaystyle \bigcup_{k=1}^n U_k$ -- замкнена множина;
\item $\emptyset, X$ -- замкнені множини.
\end{enumerate}
\textit{Вказівка: скористатися де Морганом та \thref{set_open_iff_complement_set_closed}.}
\end{corollary}

\begin{remark} Такі твердження НЕ є правдивими:
\begin{enumerate}[nosep,wide=0pt,label={\arabic*)}]
\item $A$ -- не відкрита, а тому $A$ -- замкнена (наприклад, $[0,1)$ в $\mathbb{R}$);
\item $A$ -- відкрита, а тому $A$ -- не замкнена (наприклад, $\emptyset$ в $\mathbb{R}$).
\end{enumerate}
\end{remark}

\begin{proposition} Задано $(X,\rho)$ -- метричний простір, $a \in X, r > 0$. Тоді відкритий окіл $B(a;r)$ -- справді відкритий; замкнений окіл $B[a;r]$ -- справді замкнений.
\end{proposition}

\begin{proof}
\textit{(про $B(a;r)$)}. Задамо точку $b \in B(a;r)$. Нехай $\varepsilon = r - \rho(a,b) > 0$. Тоді якщо $x \in B(b; \varepsilon)$, то тоді $\rho(x, a) \leq \rho(x, b) + \rho(b, a) < \varepsilon + \rho(b,a) = r$. Отже, $B(a;r)$ -- відкрита.
\bigskip \\
\textit{(про $B[a;r]$)}. Для цього досить довести, що $B^c[a;r] = \{x | \rho(a,x) > r\}$ -- відкрита. Якщо задати $\varepsilon = \rho(a,b) - r$ для точки $b \in B(a;r)$, то аналогічними міркуваннями отримаємо, що $B^c[a;r]$ -- відкрита. Отже, $B[a;r]$ -- замкнена.
\end{proof}

\begin{definition}
Задано $(X,\rho)$ -- метричний простір, послідовність $\{x_n, n \geq 1\} \subset X$ та $x_0 \in X$.\\
Дана послідовність називається \textbf{збіжною} до $x_0$, якщо
\begin{align*}
\rho(x_n, x_0) \to 0, n \to \infty
\end{align*}
Позначення: $\displaystyle\lim_{n \to \infty} x_n = x_0$.
\end{definition}

\begin{theorem}
Задано $(X,\rho)$ -- метричний простір, $A \subset X$ та $x_0 \in X$. Наступні твердження еквівалентні:
\begin{enumerate}[nosep,wide=0pt,label={\arabic*)}]
\item $x_0$ -- гранична точка для $A$;
\item $\forall \varepsilon > 0: B(x_0;\varepsilon) \cap A$ -- нескінченна множина;
\item $\exists \{x_n, n \geq 1\} \subset A: \forall n \geq 1: x_n \neq x_0: x_n \to x_0$.
\end{enumerate}
\end{theorem}

\begin{proof}
$\boxed{1) \Rightarrow 2)}$ Дано: $x_0$ -- гранична для $A$.\\
!Припустимо, що $\exists \varepsilon^* > 0: B(x_0;\varepsilon) \cap A$ -- скінченна множина, тобто маємо  $x_1,\dots,x_n \in B(x_0;\varepsilon^*)$. Тоді $\rho(x_0,x_1) < \varepsilon^*, \dots, \rho(x_0,x_n)^* < \varepsilon$. Оберемо найменшу відстань та задамо $\varepsilon^*_{new} = \displaystyle \min_{1\leq i \leq n} \rho(x_0,x_i)$. Створимо $B(x_0;\varepsilon^*_{new}) \subset B(x_0; \varepsilon)$. У новому шару жодна точка $x_1,\dots,x_n \in A$ більше сюди не потрапляє. Тоді $B((x_0;\varepsilon^*_{new}) \setminus \{x_0\}) \cap A = \emptyset$ -- таке неможливо через те, що $x_0$ -- гранична точка. Суперечність!
\bigskip \\
$\boxed{2) \Rightarrow 3)}$ Дано: $\forall \varepsilon > 0: B(x_0;\varepsilon) \cap A$ -- нескінченна множина. Встановимо $\varepsilon = \dfrac{1}{n}$. Тоді оскільки $\forall n \geq 1: B \left(x_0;\dfrac{1}{n} \right) \cap A$ -- нескінченна, то $\forall n \geq 1: \exists x_n \in A: \rho(x_0,x_n) < \dfrac{1}{n}$. Якщо далі $n \to \infty$, тоді $\rho(x_0,x_n) \to 0$. Остаточно, $\exists \{x_n, n \geq 1\} \subset A: x_n \neq x_0: x_n \to x_0$.
\bigskip \\
$\boxed{3) \Rightarrow 1)}$ Дано: $\exists \{x_n, n \geq 1\} \subset A: x_n \neq x_0: x_n \to x_0$. Тобто $\forall \varepsilon > 0: \exists N: \forall n \geq N: \rho(x_0,x_n) < \varepsilon$. Або, інакше кажучи, $\forall \varepsilon > 0: x_N \in B(x_0;\varepsilon) \cap A$. Тоді $\forall \varepsilon > 0: (B(x_0;\varepsilon) \setminus \{x_0\}) \cap A \neq \emptyset$.
\end{proof}

\begin{proposition}
\label{closed_iff_limit_of_sequence_in_closed_set}
Задано $(X,\rho)$ -- метричний простір та $A \subset X$.\\
$A$ -- замкнена $\iff \forall (x_n)_{n \in \mathbb{N}} \subset A: x_n \to x_0 \implies x_0 \in A$.
\end{proposition}

\begin{proof}
\rightproof Дано: $A$ -- замкнена. Нехай $(x_n)_{n \in \mathbb{N}} \subset A$ така, що $x_n \to x_0$.\\
!Припустимо, що $x_0 \notin A$, тобто $x_0 \in X \setminus A$. Зауважимо, що тоді $x_0$ має бути граничною точкою $A$. Оскільки $A$ -- замкнена, то звідси $x_0 \in A$ -- суперечність!
\bigskip \\
\leftproof Дано: $\forall (x_n)_{n \in \mathbb{N}} \subset A: x_n \to x_0 \implies x_0 \in A$.\\
Нехай $a$ -- гранична точка $A$. Тобто існує послідовність $(x_n)_{n \in \mathbb{N}} \subset A: x_n \neq a: x_n \to a$. Але тоді звідси $a \in A$. Отже, $A$ містить всі граничні точки, тому замкнена.
\end{proof}

\subsection{Замикання множин. Щільність та сепарабельність}
\begin{definition}
Задано $(X,\rho)$ -- метричний простір, $A \subset X$ та $A'$ -- множина граничних точок $A$.\\
\textbf{Замиканням} множини $A$ називають таку множину
\begin{align*}
\bar{A} = A \cup A'
\end{align*}
Часто ще позначають замикання за $\Cl(A)$.
\end{definition}

\begin{example}
Маємо $X = \mathbb{R}$, $\rho(x,y) = |x-y|$ та множину $A = (0,1)$. Тоді множина $A' = [0,1]$. Замикання $\bar{A} = A \cup A' = [0,1]$.
\end{example}

\begin{remark}
Розглянемо зараз сукупність замкнених множин $A \subset A_{\alpha} \subset X$. Перетин $B = \displaystyle\bigcap_{\alpha} A_{\alpha}$ -- також замкнена, водночас $A_\alpha \supset B \supset A$. Отже, $B$ -- найменша замкнена множина, що містить $A$.
\end{remark}

\begin{proposition}
Задано $(X,\rho)$ -- метричний простір, $A,B \subset X$. Тоді справедливе наступне:
\begin{enumerate}[nosep,wide=0pt,label={\arabic*)}]
\item $(A \cup B)' = A' \cup B'$;
\item $(A \cap B)' \subset A' \cap B'$.
\end{enumerate}
\end{proposition}

\begin{proof}
Доведемо кожне твердження окремо.
\begin{enumerate}[wide=0pt, label={\arabic*)}]
\item $x_0 \in (A \cup B)' \iff x_0$ -- гранична точка $A \cup B \iff$ $\forall \varepsilon > 0: \\ \mathring{B}(x_0;\varepsilon) \cap (A \cup B) = (\mathring{B}(x_0;\varepsilon) \cap A) \cup (\mathring{B}(x_0; \varepsilon) \cap B) \neq \emptyset \iff \left[ \begin{gathered} \text{$x_0$ -- гранична для $A$} \\ \text{$x_0$ -- гранична для $B$} \end{gathered} \right. \iff \left[ \begin{gathered} x_0 \in A' \\ x_0 \in B' \end{gathered} \right. \iff x_0 \in A' \cup B'$.\\
Отже, тим довели щойно, що $(A \cup B)' = A' \cup B'$.

\item $x_0 \in (A \cap B)' \iff x_0$ -- гранична точка $A \cap B \iff$ $\forall \varepsilon > 0: \\ \mathring{B}(x_0;\varepsilon) \cap (A \cap B) = (\mathring{B}(x_0;\varepsilon) \cap A) \cap (\mathring{B}(x_0; \varepsilon) \cap B) \neq \emptyset \textcolor{red}{\implies} \begin{cases} \text{$x_0$ -- гранична для $A$} \\ \text{$x_0$ -- гранична для $B$} \end{cases} \iff \begin{cases} x_0 \in A' \\ x_0 \in B' \end{cases} \iff x_0 \in A' \cap B'$.\\
Отже, тим довели щойно, що $(A \cap B)' \subset A' \cap B'$.
\end{enumerate}
Всі твердження доведені.
\end{proof}

\begin{proposition}
Задано $(X,\rho)$ -- метричний простір, $\bar{A}$ -- замикання. Тоді спредливе наступне:
\begin{enumerate}[nosep,wide=0pt,label={\arabic*)}]
\item $\bar{A}$ -- найменша замкнена множина, що містить $A$;
\item $\overline{A \cup B} = \bar{A} \cup \bar{B}$ \qquad $\overline{A \cap B} \subset \bar{A} \cap \bar{B}$;
\item $A$ -- замкнена $\iff A = \bar{A}$.
\end{enumerate}
\end{proposition}

\begin{proof}
Доведемо кожне твердження окремо.
\begin{enumerate}[wide=0pt, label={\arabic*)}]
\item !Припустимо, що $\bar{A}$ не є найменшою замкненою, що містить $A$, тобто $\exists B \subset \bar{A}: B \supset A$ -- замкнена. Зафіксуємо точку $x_0 \in \bar{A}$ -- гранична, тоді $x_0 \in A' \cup A$. Далі маємо два випадки:\\
якщо $x_0 \in A'$, то тоді $x_0 \in B$, тому що $B$ містить всі граничні точци $A$;\\
якщо $x_0 \in A$, то тоді $x_0 \in B$.\\
В обох випадках $\bar{A} \subset B$. Отже, $\bar{A} = B$. Суперечність!

\item Маємо такі ланцюги рівностей та вкладень:\\
$\overline{A \cup B} = (A \cup B)' \cup (A \cup B) = A' \cup B' \cup A \cup B = \bar{A} \cup \bar{B}$.\\
$\overline{A \cap B} = (A \cap B)' \cup (A \cap B) \subset (A' \cap B') \cup (A \cap B) \subset (A \cup A') \cap (B \cup B') = \bar{A} \cap \bar{B}$.

\item Доведення в обидва боки.\\
\rightproof Дано: $A$ -- замкнена. Тоді $A$ містить всі свої граничні точки. Так само $A'$ містить граничні точки $A$. Тому $A = \bar{A}$.\\
\leftproof Дано: $A = \bar{A}$. Тобто $A$ містить всі свої граничні точки. Отже, $A$ -- замкнена.
\end{enumerate}
Всі твердження доведені.
\end{proof}

\begin{example}
У стандартному метричному просторі $R$ Розглянемо множини $A = (0,1),\ B = (1,2)$. Зауважимо, що $A \cap B = \emptyset$, тож звідси випливає $\overline{A \cap B} = \emptyset$. А з іншого боку, $\bar{A} = [0,1],\ \bar{B} = [1,2]$, а звідси $\bar{A} \cap \bar{B} = \{1\}$.\\
Таким чином, $\overline{A \cap B} \subsetneq \bar{A} \cap \bar{B}$.\\
Буквально так само $(A \cap B)' \subsetneq A' \cap B'$.
\end{example}

\begin{remark}
У загальному випадку $\overline{B(x;r)} \neq B[x;r]$.\\
Розглянемо дискретний простір $(X,d)$, де множина $X$ містить не менше двох елементів. Зауважимо, що $B(a;1) = \{a\}$ та $B[a;1] = X$. Ми вже знаємо, що там всі множини -- відкрити (тому відповідно замкнені). Отже, $\overline{B(a;1)} = B(a;1) = \{a\} \neq X = B[a;1]$.
\end{remark}

\begin{definition} Задано $(X, \rho)$ -- метричний простір та $A \subset X$.\\
Множина $A$ називається \textbf{щільною} в $X$, якщо
\begin{align*}
\forall x \in X, \forall \varepsilon > 0: \exists y \in A: \rho(x,y) < \varepsilon
\end{align*}
Інколи ще бачу, щоб називали множину $A$ \textbf{скрізь щільною}.
\end{definition}

\begin{proposition}
Задано $(X, \rho)$ -- метричний простір та $A \subset X$.\\
Множина $A$ -- скрізь щільна $\iff \bar{A} = X$.
\end{proposition}

\begin{proof}
\rightproof Дано: $A$ -- скрізь щільна. Цілком зрозуміло, що $\bar{A} \subset X$, тому залишилося тільки в зворотний бік провести.\\
Нехай $x \in X$. тоді за умовою щільності, $\forall \varepsilon > 0: \exists y \in A: \rho(x,y) < \varepsilon$. Якщо $x \in A$, автоматично $x \in \bar{A}$. Якщо $x \notin A$, то тоді там записано, що $x$ -- гранична точка $A$, тож все одно $x \in \bar{A}$.
\bigskip \\
Дано $\bar{A} = X$. Оберемо $x \in X$ та $\varepsilon > 0$. Якщо $x \in A$, то тоді можна взяти $y = x \in A$ і тоді $\rho(x,y) = 0 < \varepsilon$. Якщо $x \notin A$, то тоді $x$ має бути просто граничною точкою $A$, але тоді $\exists y \in A: y \neq x: \rho(x,y) < \varepsilon$. Таким чином, $A$ -- скрізь щільна.
\end{proof}

\begin{proposition}
Задано $(X, \rho)$ -- метричний простір та $A \subset X$.\\
Множина $A$ -- скрізь щільна $\iff \forall x \in X: \exists \{x_n, n \geq 1\} \subset A: x_n \to x, n \to \infty$.\\
\textit{Вправа: довести.}
\end{proposition}

\begin{definition}
Задано $(X, \rho)$ -- метричний простір.\\
Метричний простір називається \textbf{сепарабельним}, якщо
\begin{align*}
\text{існує в даному просторі скінченна чи зліченна щільна підмножина.}
\end{align*}
\end{definition}

\begin{example}
Зокрема $(\mathbb{R}, \rho)$, де $\rho(x,y) = |x-y|$ -- сепарабельний, оскільки $\mathbb{Q}$ -- зліченна та скрізь щільна підмножина (див.\ курс матаналізу за 1 семестр).
\end{example}

\begin{example}
Простір $C([a,b])$ також сепарабельний.\\
Покладемо $A = \{ Q \in \mathbb{Q}[x] \text{ -- многочлени на $[a,b]$}\}$. Цілком ясно, що $A$ -- зліченна множина. Залишилося показати, що $A$ -- скрізь щільна.\\
Нехай $f \in C([a,b])$ та $\varepsilon > 0$. За теоремою Ваєрштраса про наближення функці, існує многочлен $P_\varepsilon \in \mathbb{R}[x]$, для якого $\displaystyle\sup_{x \in [a,b]} |f(x) - P_\varepsilon(x)| < \varepsilon$. Запишемо $P_\varepsilon(x) = a_0 + a_1 x + \dots + a_k x^k$. Оскільки $\mathbb{Q}$ -- скрізь щільна на $\mathbb{R}$, то ми можемо знайти $b_0,b_1,\dots,b_k \in \mathbb{Q}$ такі, що $|a_i-b_i| < \varepsilon$. Отримаємо многочлен $Q_\varepsilon \in \mathbb{Q}[x]$ вигляду $Q_\varepsilon(x) = b_0 + b_1 x + \dots + b_k x^k$. Тоді $\forall x \in [a,b]$ маємо наступне:\\
$|P_\varepsilon(x) - Q_\varepsilon(x)| \leq |a_0-b_0| + |a_1-b_1| |x| + \dots + |a_k - b_k| |x^k| < \varepsilon M_0 + \varepsilon M_1 + \dots + \varepsilon M_k = M \varepsilon$.\\
У цьому випадку $M_i = \displaystyle\max_{x \in [a,b]} |x^i|$, який існує, оскільки $x^i \in C([a,b])$. Отже, довели\\
$\displaystyle\sup_{x \in [a,b]} |P_\varepsilon(x) - Q_\varepsilon(x)| < \varepsilon$.\\
Використаємо тепер нерівність трикутника -- отримаємо:\\
$\displaystyle\sup_{x \in [a,b]} |f(x) - Q_\varepsilon(x)| \leq \sup_{x \in [a,b]} |f(x) - P_\varepsilon(x)| + \sup_{x \in [a,b]} |P_\varepsilon(x) - Q_\varepsilon(x)| < 2\varepsilon$.
\end{example}

\begin{theorem}
Задано $(X,\rho)$ -- сепарабельний метричний простір та $Y \subset X$ -- підпростір. Тоді $(Y,\rho_Y)$ -- також сепарабельний.
\end{theorem}

\begin{proof}
Ми розглянемо випадок, коли $Y \subsetneq X$. Оберемо елемент $x \in X \setminus Y$. Оскільки $(X,\rho)$ -- сепарабельний, то маємо $Q = \{x_n, n \geq 1\}$ -- зліченна та скрізь щільна в $X$.\\
Розглянемо такий набір елементів $R = \{y_{n,k}, n \geq 1, k \geq 1: y_{n,k} \neq x\}$. Пояснюємо, як ми це сформували. Проходимося по всіх можливих парам натуральних числах $(n,k)$. Якщо $B\left(  x_n, \dfrac{1}{k} \right) \cap Y \neq \emptyset$, то звідти обираємо елемент $y_{n,k}$. Інакше елемент $y_{n,k} = x$.\\
Доведемо, що $R$ -- скрізь щільна множина в $Y$. Єдине варто пересвідчитися, що отримана множина $R \neq \emptyset$. Дійсно, нехай $y \in Y$ та $\varepsilon > 0$, ми оберемо таке $k \geq 1$, щоб $\dfrac{1}{k} < \dfrac{\varepsilon}{2}$. Оскільки $Q$ -- скрізь щільна, то звідси $\exists x_n \in Q: \rho(y,x_n) < \dfrac{1}{k}$. Отже $B\left( \dfrac{1}{k},x_n \right) \cap Y \neq \emptyset$ і там існує точка $y_{n,k}$, тож $R \neq \emptyset$.\\
Тепер ще раз беремо $\varepsilon > 0$ та елемент $y \in Y$. Тоді ми щойно знайшли елемент $y_{n,k}$, для якого\\
$\rho_Y(y,y_{n,k}) \leq \rho(y,x_n) + \rho(x_n,y_{n,k}) < \dfrac{1}{k} + \dfrac{1}{k} < \varepsilon$.\\
Отже, ми довели скрізь щільність. Те, що $R$ зліченна, тут цілком зрозуміло.
\end{proof}

\iffalse
\begin{example} Розглянемо такі приклади:
\begin{enumerate}[wide=0pt,label={\arabic*)}]
\item $(\mathbb{R}, \rho(x,y) = |x-y|)$ -- сепарабельний, тому що $\mathbb{Q}$ -- зліченна та щільна підмножина в $\mathbb{R}$.
\item Маємо простір $l_2 = \left\{ \vec{a} \mid \displaystyle\sum_{n=1}^\infty a_n^2 < \infty \right\}$ -- нормований простір. Розглянемо множину \\
$l_2O = \left\{ \vec{a} \in l_2 \mid \text{скінченна кількість членів не нуль} \right\}$. Розглянемо $\vec{a} = \{a_1,a_2,\dots\} \in l_2$. Доведемо, що вона -- гранична для $l_2O$.\\
Задамо послідовність $\{\vec{a}_n, n \geq 1\} \subset l_2O$, де кожний елемент задається таким чином: \\ $\vec{a}_n = \{a_1,\dots,a_n,0,\dots\} \implies \rho(\vec{a}, \vec{a}_n) = \|\vec{a} - \vec{a}_n\| = \displaystyle\sum_{n=k+1}^\infty a_n^2 \to 0$, оскільки ряд збіжний, а тому хвіст ряду прямує до нуля. Отже, $\vec{a}_n \to \vec{a}$, тож $\vec{a}_n$ -- гранична точка.\\
Тоді можна ствердити, що $l_2O$ -- щільна в $l_2$, або інакше $\overline{l_2O} = l_2$. А оскільки $l_2O \subset l_2$ та ще й нескінченна, то тоді $l_2$ -- сепарабельний.
\item Простір $C([a,b])$ -- сепарабельний.\\
\textit{Доведу пізніше, коли дізнаюсь про теорему Вейєрштрасса про наближення неперервної на відрізку функції многочленами}.
\item А ось простір $l_{\infty}$ - не сепарабельний.\\
\textit{Доведу пізніше}.
\item Підпростір сепарабельного метричного простору - сепарабельний\\
\textit{Доведу пізніше}.
\end{enumerate}
\end{example}
\fi

\subsection{Повнота}
\begin{definition}
Задано $(X,\rho)$ -- метричний простір.\\
Послідовність $(x_n)_{n \in \mathbb{N}}$ називається \textbf{фундаментальною}, якщо
\begin{align*}
\forall \varepsilon > 0: \exists N: \forall m,n \geq N: \rho(x_n,x_m) < \varepsilon.
\end{align*}
\end{definition}

\begin{remark}
Це означення можна інакше переписати, більш компактним чином:
\begin{align*}
\rho(x_n,x_m) \overset{m,n \to \infty}{\longrightarrow} 0
\end{align*}
\end{remark}

\begin{proposition}
Будь-яка збіжна послідовність є фундаментальною.
\end{proposition}

\begin{proof}
Маємо $(x_n)_{n \in \mathbb{N}}$ -- збіжна, тобто $\rho(x_n,x) \overset{n \to \infty}{\longrightarrow} 0$. За нерівністю трикутника, маємо $\rho(x_n,x_m) \leq \rho(x_n,x) + \rho(x,x_m)$. Якщо спрямувати одночасно $m.n \to \infty$, то тоді $\rho(x_n,x_m) \to 0$. Отже, $(x_n)_{n \in \mathbb{N}}$ -- фундаментальна.
\end{proof}

\begin{remark}
Проте не кожна фундаментальна послідовність -- збіжна.
\end{remark}

\begin{example}
Маємо $X = (0,1]$ -- підпростір $\mathbb{R}$. Розглянемо послідовність $\left( x_n = \dfrac{1}{n}, n \geq 1 \right)$, де $x_n \to 0$ при $n \to \infty$ -- збіжна, проте $0 \notin X$. Тому така послідовність не має границі в $X$, але вона -- фундаментальна за твердженням.
\end{example}

\begin{definition}
Метричний простір $(X, \rho)$ називається \textbf{повним}, якщо 
\begin{align*}
\text{будь-яка фундаментальна послідовність має границю.}
\end{align*}
\end{definition}

\begin{example} 
Зокрема маємо наступне:
\begin{enumerate}[nosep,wide=0pt,label={\arabic*)}]
\item $X = \mathbb{R}$ -- повний за критерієм Коші із матану;
\item $X = (0,1]$ -- не повний, бо принаймні $\left(x_n = \dfrac{1}{n}, n \geq 1 \right)$ -- фундаментальна, проте не має границі.
\end{enumerate}
\end{example}

\begin{example}
Покажемо, що $(\mathbb{N},\rho)$ -- повний метричний простір, де $\rho(m,n) = 1 + \dfrac{1}{m+n}, m \neq n$.\\
Нехай $(x_n)_{n \in \mathbb{N}} \subset \mathbb{N}$ -- фундаментальна послідовність. Тоді для $\varepsilon = 1$ маємо, що $\exists N: \forall n,m \geq N: \rho(x_m,x_n) < 1$. Зауважимо, що взагалі $\rho(k,l) \geq 1$ при $k \neq l$, тому для нерівності треба вимагати $x_m = x_n, \forall n,m \geq N$. Отже, ми отримали послідовність $(x_1,x_2,\dots,x_N,x_N,x_N,\dots)$ -- стаціонарна, починаючи з деякого номеру, яка буде збіжною.
\end{example}

\begin{proposition}
Задано $(X,\rho)$ -- повний метричний простір та $(Y,\rho)$ -- підпрострір.\\
$(Y,\rho)$ -- повний $\iff Y$ -- замкнена в $X$.
\end{proposition}

\begin{proof}
\rightproof Дано: $(Y,\rho)$ -- повний.\\
!Припустимо, що $Y$ -- не замкнена, тобто існує $x_0 \in X \setminus Y$ -- гранична точка для $Y$. Тоді існує послідовність $(y_n)_{n \in \mathbb{N}} \subset Y$, для якої $y_n \to x_0$ та $y_n \neq x_0$. Зауважимо, що $(y_n)_{n \in \mathbb{N}} \subset X$ збіжна саме в просторі $X$, тому саме в просторі $X$ послідовність $(y_n)_{n \in \mathbb{N}} \subset X$ -- фундаментальна. Проте зрозумло цілком, що $(y_n)_{n \in \mathbb{N}} \subset Y$ буде фундаментальною в просторі $Y$, проте в силу повноти $(Y,\rho)$, матимемо збіжність саме в $Y$. Таким чином, $x_0 \in Y$ -- суперечність!
\bigskip \\
\leftproof Дано: $Y$ -- замкнена в $X$. Візьмемо $(y_n)_{n \in \mathbb{N}} \subset Y \subset X$ -- фундаментальна. Тоді в силу повноти $X$, вона -- збіжна в просторі $X$. Скажімо, $y_n \to x_0$. Якщо точка $x_0 \in Y$, то тоді послідовність $(y_n)_{n \in \mathbb{N}}$ збіжна в $Y$. Інакше при $x_0 \in X \setminus Y$ зауважимо, що $y_n \neq x_0$, тому $x_0$ -- гранична точка $Y$. У силу замкненості ми отримаємо $x_0 \in Y$ -- послідовність $(y_n)_{n \in \mathbb{N}}$ знову збіжна в $Y$.
\end{proof}

\begin{lemma}
\label{Cauchy_and_convergent_subsequence_implies_convergent_sequence}
Задано $(x_n)_{n \in \mathbb{N}}$ -- фундаментальна та $\left(x_{n_k}\right)_{k \in \mathbb{N}}$ -- збіжна. Тоді $(x_n)_{n \in \mathbb{N}}$ -- збіжна.
\end{lemma}

\begin{proof}
Маємо $x_{n_k} \to x,\ k \to \infty$, тобто це означає $\forall \varepsilon > 0: \exists K: \forall k \geq K: \rho(x_{n_k}, x) < \varepsilon$.\\
Також відомо, що  $\exists N: \forall n,m \geq N: \rho(x_n,x_m) < \varepsilon$. Тоді $\forall n \geq N^* = \max\{N,K\}$ маємо\\
$\rho(x_n,x) \leq \rho(x_n,x_{n_{N^*}}) + \rho(x_{n_{N^*}},x) < 2\varepsilon$.\\
Отже, $x_n \to x$.
\end{proof}

\begin{theorem}[Критерій Кантора]
Умова Кантора звучить так: для кожної послдовності $(B[a_n;r_n], n \geq 1)$ такої, що $B[a_1;r_1] \supset B[a_2;r_2] \supset \dots$ та $r_n \to 0$ (послідовність замкнених куль, що стягується), перетин $\displaystyle\bigcap_{n=1}^\infty B[a_n;r_n] \neq \emptyset$.\\
$(X,\rho)$ -- повний метричний простір $\iff$ виконується умова Кантора.
\end{theorem}

\begin{proof}
\rightproof Дано: $(X,\rho)$ -- повний. Задамо послідовність куль $(B[a_n;r_n], n \geq 1)$, що стягується.\\
\textit{$(a_n)_{n \in \mathbb{N}}$ -- послідовність центрів -- фундаментальна.}\\
За умовою, $r_n \to 0$, тож $\forall \varepsilon > 0: \exists N: \forall n \geq N: r_n < \varepsilon$. Досить взяти лише $r_N < \varepsilon$. Тоді $\forall n,m \geq N: a_m,a_n \in B[a_N,r_N] \implies \rho(a_m,a_N) < r_N$ та $\rho(a_n,a_N) < r_N$.\\
$\implies \rho(a_n,a_m) \leq \rho(a_n,a_N) + \rho(a_N,a_m) < 2r_N < 2 \varepsilon$. Отже, $(a_n)_{n \in \mathbb{N}}$ -- фундаментальна.
\bigskip \\
\textit{Замкнені кулі, що стягуються, мають непорожній перетин.}\\
Оскільки $X$ -- повний, то тоді $(a_n)_{n \in \mathbb{N}}$ -- збіжна, тобто $a_n \to a_0$. Оскільки $B[a_n;r_n]$ -- замкнені, то за \prpref{closed_iff_limit_of_sequence_in_closed_set} маємо, що $a_0 \in B[a_n;r_n]$. Звідси $a_0 \displaystyle \in \bigcap_{n=1}^\infty B_n[a_n;r_n]$.
\bigskip \\
\leftproof Дано: умова Кантора. Нехай $(a_n)_{n \in \mathbb{N}}$ -- фундаментальна послідовність. Тобто $\forall \varepsilon > 0: \exists N: \forall n,m \geq N: \rho(a_n,a_m) < \varepsilon$.\\
При $\varepsilon = \dfrac{1}{2}$ маємо $n_1 \in \mathbb{N}$ таке, що $\forall n \geq n_1: \rho(a_n,a_{n_1}) < \dfrac{1}{2}$.\\
При $\varepsilon = \dfrac{1}{4}$ маємо $n_2 > n_1$ таке, що $\forall n \geq n_2: \rho(a_n,a_{n_2}) < \dfrac{1}{4}$.\\
\vdots \\
Тоді маємо підпослідовність $(a_{n_k})_{k \in \mathbb{N}}$ із властивістю $\forall n \geq n_k: \rho(a_n,a_{n_k}) < \dfrac{1}{2^k}$. Звідси випливає, що замкнені кулі $B\left[a_{n_k}; \dfrac{1}{2^{k-1}} \right]$ будуть вкладеними, тобто $B\left[ a_{n_k}; \dfrac{1}{2^{k-1}} \right] \supset B\left[ a_{n_{k+1}}; \dfrac{1}{2^k}\right], k \geq 1$.\\
Справді, беремо $x \in B\left[a_{n_{k+1}}; \dfrac{1}{2^{k}} \right]$, тобто $\rho\left(x,a_{n_{k+1}}\right) \leq \dfrac{1}{2^k}$. Через нерівність трикутника отримаємо $\rho\left(a_{n_k}, x\right) \leq \rho\left(a_{n_k}, a_{n_{k+1}}\right) + \rho\left(a_{n_{k+1}},x\right) \leq \dfrac{1}{2^{k-1}}$, тому звідси $x \in B\left[a_{n_k}; \dfrac{1}{2^{k-1}} \right]$.\\
Далі всі радіуси $\dfrac{1}{2^{k-1}} \to 0$, тому за умовою Кантора існує точка $a \in B\left[a_{n_k}, \dfrac{1}{2^{k-1}}\right], \forall k \geq 1$. Тобто $\forall k \geq 1$ маємо $\rho\left(a_{n_k},a\right) \leq \dfrac{1}{2^{k-1}}$, після спрямування $k \to \infty$ отримаємо $a_{n_k} \to a$. Значить, за \lmref{Cauchy_and_convergent_subsequence_implies_convergent_sequence}, послідовність $(a_n)_{n \in \mathbb{N}}$ збіжна.\\
Висновок: метричний простір $(X,\rho)$ -- повний.
\end{proof}

\begin{remark}
До речі, точка, що належить перетину замкнених кіль, буде єдиною.\\
!Припустимо, що це не так, тобто $\exists b^*, b^{**} \in \displaystyle \bigcap_{n=1}^\infty B[a_n;r_n]$. Тоді $\forall n \geq 1: \begin{cases} \rho(a_n, b^*) \leq r_n \\ \rho(a_n, b^{**}) \leq r_n \end{cases}$.\\
$\implies \rho(b^*, b^{**}) \leq \rho(b^*,a_n) + \rho(a_n, b^{**}) \leq r_n + r_n = 2r_n$.\\
Спрямуємо $n \to \infty$, тоді $\rho(b^*,b^{**}) \leq 0 \implies \rho(b^*,b^{**}) = 0 \implies b^{*} = b^{**}$. Суперечність!
\end{remark}

\begin{remark}
Умова того, що $r_n \to 0$ в теоремі Кантора, є суттєвою.
\end{remark}

\begin{example}
Розглянемо $(\mathbb{N}, \rho)$ -- повний метричний простір, де $\rho(m,n) = 1 +\dfrac{1}{n+m},m \neq n$. Тепер оберемо ось такі замкнені кулі $B\left[ n, 1 + \dfrac{1}{2n}\right]$. Зауважимо, що\\
$B\left[n, 1 + \dfrac{1}{2n} \right] = \left\{x \in \mathbb{N} \mid \rho(x,n) \leq 1 + \dfrac{1}{2n} \right\} = \left\{ x \in \mathbb{N} \mid \dfrac{1}{x+n} \leq \dfrac{1}{2n} \right\} = \{x \in \mathbb{N} \mid x \geq n\} = \\ = \{n,n+1,\dots\} = \mathbb{N} \setminus \{1,2,\dots,n-1\}$.\\
Аналогічно $B\left[ 1, 1 + \dfrac{1}{2}\right] = \mathbb{N}$.\\
Отже, маємо $B\left[1, 1 + \dfrac{1}{2}\right] \supset B\left[ 2, 1 + \dfrac{1}{4} \right] \supset B\left[3, 1 + \dfrac{1}{6}\right] \supset \dots$, при цьому $\displaystyle\bigcap_{n=1}^\infty B\left[n, 1 + \dfrac{1}{2n}\right] = \emptyset$.\\
У цьому випадку радіуси $1 + \dfrac{1}{2n} \centernot\to 0$, тому точки перетину нема.
\end{example}

\subsection{Поповнення метричного простору та трошки про ізометрію}
\begin{definition}
Задано $(X,\rho)$ та $(Y, \tilde{\rho})$ -- два різних метричних простори.\\
Відображення $f \colon X \to Y$ називається \textbf{ізометрією}, якщо
\begin{align*}
\forall x_1,x_2 \in X: \tilde{\rho}(f(x_1),f(x_2)) = \rho(x_1,x_2)
\end{align*}
Тобто суть ізометрії -- це збереження відстаней.
\end{definition}

\begin{remark}
Кожна ізометрія $f$ -- уже автоматично ін'єктивна.\\
Дійсно, припустимо, що $f(x_1) = f(x_2)$. За визначенням ізометрії, $\tilde{\rho}(f(x_1),f(x_2)) = \rho(x_1,x_2)$. Отримаємо $\rho(x_1,x_2) = 0$, тобто $x_1 = x_2$.
\end{remark}

\begin{definition}
Метричні простори $(X,\rho), (Y,\tilde{\rho})$ називаються \textbf{ізометричними}, якщо
\begin{align*}
\exists f \colon X \to Y \text{ -- бієктивна ізометрія}
\end{align*}
\end{definition}

\begin{example}
Розглянемо $(\mathbb{R}, \tilde{\rho})$ та $\left( \left( -\dfrac{\pi}{2}, \dfrac{\pi}{2} \right), \rho \right)$ -- два метричних простори. У цьому випадку $\rho$ -- стандартна метрика та $\tilde{\rho}(x,y) = |\arctg x - \arctg y|$. Ці два простори -- ізометричні.\\
Дійсно, між ними існує ізометрія $\arctg \colon \mathbb{R} \to \left( -\dfrac{\pi}{2}, \dfrac{\pi}{2} \right)$, що є бієктивною.
\end{example}

\begin{proposition}
Задані $(X,\rho), (Y, \tilde{\rho})$ -- два ізоморфні метричні простори.\\
$(X,\rho)$ -- повний $\iff (Y,\tilde{\rho})$ -- повний.
\end{proposition}

\begin{proof}
\rightproof Дано: $(X,\rho)$ -- повний. Нехай $(y_n)_{n \in \mathbb{N}}$ -- фундаментальна послідовність. Оскільки $X,Y$ ізометричні, то існує бієкція $f \colon X \to Y$, що є ізометрією. Тож звідси $\exists !x_n \in X: f(x_n) = y_n$. Розглянемо послідовність $(x_n)_{n \in \mathbb{N}}$ та зауважимо, що $\rho(x_n,x_m) = \tilde{\rho}(y_n,y_m) \to 0$ в силу фундаментальності. Отже, $(x_n)_{n \in \mathbb{N}}$ -- фундаментальна, тож збіжна за повнотою. Тобто $\rho(x_n,x) \to 0$. Позначимо $f(x) = y$. Звідси випливає, що $\tilde{\rho}(y_n,y) = \rho(x_n,x) \to 0$. Тобто $(y_n)_{n \in \mathbb{N}}$ -- збіжна.
\bigskip \\
\leftproof \textit{дзеркальне доведення.}
\end{proof}

\begin{definition}
Задано $Y$ -- повний метричний простір.\\
Він буде називатися \textbf{поповненням (completion)} метричного простору $X$, якщо
\begin{align*}
X \text{ -- ізометричний підпростір } Y; \\
X \text{ -- щільна в } Y.
\end{align*}
\end{definition}

\begin{theorem}
Для кожного метричного простору $(X,\rho)$ існує поповнення. Причому це поповнення єдине з точністю до ізометрії.
\end{theorem}

\begin{proof}
І. \textit{Існування.}\\
Позначимо $F$ за множина фундаментальних послідовностей $\{x_n\}$ в $X$. Стаціонарні послідовності є фундаментальними, тож звідси $X$ можна сприймати як підмножину $F$.\\
Розглянемо функцію $d(\{x_n\},\{y_n\}) = \displaystyle\lim_{n \to \infty} \rho(x_n,y_n)$, яка визначена на $F \times F$. Для коректності треба довести існування даної границі. Ми доведемо, що $\{ \rho(x_n,y_n), n \geq 1\}$ -- фундаментальна (це числова послідовність, тому цього буде достатньо).\\
Нам відомо, що $\{x_n\}, \{y_n\}$ фундаментальні, тобто $\exists N_1, N_2$, для яких $\rho(x_n,x_m) < \varepsilon, \rho(y_n,y_m) < \varepsilon$ для всіх $n,m \geq N_1, m,n \geq N_2$. Тоді при $N = \max \{N_1,N_2\}$ справедлива оцінка:\\
$|\rho(x_n,y_n) - \rho(x_m,y_m)| \leq \rho(x_n,y_n) + \rho(x_m,y_m) \leq (\rho(x_n,x_m) + \rho(x_m,y_m) + \rho(y_m,y_n)) - \rho(x_m,y_m) < 2 \varepsilon$.\\
Отже, функція $d$ визначена коректно. Вона майже метрика, оскільки (легко перевірити) виконуються всі властивості. На жаль, $d(\{x_n\},\{y_n\}) = 0 \centernot\implies \{x_n\} = \{y_n\}$ (приклад буде нижче).\\
Створимо відношення еквівалентності $\{x_n\} \sim \{y_n\} \iff d(\{x_n\},\{y_n\}) = 0$. Утвориться фактормножина $F/_{\sim} = \hat{F}$. Елементи з $\hat{F}$ позначатимемо за $\overline{\{x_n\}}$. Наша мета буде довести, що саме $\hat{F}$ буде поповненням $X$. \\
На фактормножині покладемо $\tilde{\rho}\left(\overline{\{x_n\}},\overline{\{y_n\}}\right) = d(\{x_n\},\{y_n\})$. Варто пересвідчитися, що воно визначено коректно.\\
Нехай $\{x_n\} \sim \{x_n'\}$ та $\{y_n\} \sim \{y_n'\}$. Тобто $d(\{x_n\},\{x_n'\}) = 0$ та $d(\{y_n\},\{y_n'\}) = 0$. Тоді\\
$d(\{x_n\},\{y_n\}) = \displaystyle\lim_{n \to \infty} \rho(x_n,y_n) \leq \lim_{n \to \infty} \rho(x_n,x_n') + \lim_{n \to \infty} \rho(x_n',y_n') + \lim_{n \to \infty} \rho(y_n',y_n) = d(\{x_n'\},\{y_n'\})$.\\
Аналогічно отримаємо $d(\{x_n'\},\{y_n'\}) \leq d(\{x_n\},\{y_n\})$. Отже, $d(\{x_n'\},\{y_n'\}) = d(\{x_n\},\{y_n\})$, тобто $\tilde{\rho}$ визначилося коректним чином.\\
Поставимо відображення $f \colon X \to \hat{F}$ таким чином: $f(x) = \overline{\{x\}}$. Це буде ізометрією, тому що\\
$\tilde{\rho}(f(x_1),f(x_2)) = \tilde{\rho}( \overline{\{x_1\}},\overline{\{x_2\}} ) = d( \{x_1\}, \{x_2\} ) = \displaystyle \lim_{n \to \infty} \rho(x_1,x_2) = \rho(x_1,x_2)$. Відображення $f$ зобов'язане бути сюр'єктивним, оскільки повертається клас еквівалентності. Тобто $f$ -- бієктивна ізометрія, а тому $(X,\rho), (\hat{F},\tilde{\rho})$ -- ізометричні.\\
Покажемо, що $(\hat{F},\tilde{\rho})$ -- повний метричний простір. (TODO: обміркувати).
\bigskip \\
II. \textit{Єдиність.}\\
Розглянемо два поповнення $(Y_1,\tilde{\rho}_1), (Y_2,\tilde{\rho}_2)$ простору $(X,\rho)$. Тобто, за означенням, маємо $Y_1 \supset X_1 \sim X \sim X_2 \subset Y_2$, а також $\overline{X_1} = Y_1, \overline{X_2} = Y_2$. Під $\sim$ мається на увазі ізометричність. Із цього $X_1$ ізометричний до $X_2$, нехай $g$ -- відповідна ізометрія.\\
Побудуємо $f \colon Y_1 \to Y_2$ за таким правилом: для кожного $y \in Y_1$ беремо таку послідовність $\{x_n\} \subset X_1$, щоб $x_n \to y$ -- тоді $f(y) = \displaystyle\lim_{n \to \infty} g(x_n)$. Треба пересвідчитися, що визначення коректне. Дійсно, нехай $\{x_n\}, \{x_n'\}$ -- такі дві послідовності, що $x_n \to y, x_n' \to y$. Тоді звідси вилпиває наступне:\\
$\tilde{\rho}_2 (g(x_n),g(x_n')) \overset{\text{ізометричність}}{=} \tilde{\rho}_1(x_n,x_n') \leq \tilde{\rho}_1(x_n,y) + \tilde{\rho}_2(y,x_n') \to 0$ при $n \to \infty$.\\
Таким чином, $\displaystyle\lim_{n \to \infty} g(x_n) = \lim_{n \to \infty} g(x_n')$, а тому значення функцій коректно визначено. (TODO: подумати над тим, чи правильно я все це розписав).
\end{proof}

\begin{example}
Беремо стандартний метричний простір $\mathbb{R}$, послідовності $\{x_n\} = \{0.9, 0.99, 0.999, \dots\}$ та $\{y_n\} = \{1,1,1,\dots\}$. Зауважимо, що $d(\{x_n\},\{y_n\}) = \displaystyle\lim_{n \to \infty} \rho(x_n,y_n) = \lim_{n \to \infty} 0.\underset{n \text{ цифр}}{00\dots 01} = 0$. При цьому зрозуміло, що $\{x_n\} \neq \{y_n\}$.
\end{example}

\begin{definition}
Повний нормований простір називається \textbf{банаховим}. Повний евклідів простір (відносно метрики, що породжена скалярним добутков) називається \textbf{гільбертовим}.
\end{definition}

\begin{proposition}
Евклідів простір $l_2$ -- гільбертів.
\end{proposition}

\begin{proof}
Задамо фундаментальну послідовність $\{\vec{x}_n, n \geq 1\}$ на множині $l_2$\\
Тобто $\forall \varepsilon > 0: \exists N: \forall n, m \geq N: \|\vec{x}_n - \vec{x}_m\| < \varepsilon$\\
$\Rightarrow \|\vec{x}_n - \vec{x}_m\|^2 = \displaystyle\sum_{k=1}^\infty (x_n^k - x_m^k)^2 < \varepsilon^2 \Rightarrow \forall k \geq 1: |x_n^k - x_m^k| < \varepsilon$\\
Тоді послідовність $\{x_n^k, n \geq 1\}$ - фундаментальна - тому (за матаном) збіжна, $x_n^k \to y^k$\\
Доведемо, що $\vec{x}$ збігається до $\vec{y}$ за нормою\\
Маємо $\displaystyle\sum_{k=1}^\infty (x_n^k - x_m^k)^2 < \varepsilon^2 \Rightarrow \forall K \geq 1: \displaystyle\sum_{k=1}^K (x_n^k - x_m^k)^2 < \varepsilon^2$\\
Спрямуємо $m \to \infty$, тоді $\displaystyle\sum_{k=1}^K (x_n^k - y^k)^2 < \varepsilon^2$\\
Звідки випливає збіжність ряду $\displaystyle\sum_{k=1}^\infty (x_n^k - y^k)^2$ та його оцінка\\
$\displaystyle\sum_{k=1}^\infty (x_n^k - y^k)^2 < \varepsilon^2 \Rightarrow \|\vec{x}_n - \vec{y}\| < \varepsilon$\\
Отже, $\vec{x}_n \to \vec{y}$
\end{proof}

\subsection{Неперервні відображення}
\begin{definition}
Задані $(X,\rho), (Y,\tilde{\rho})$ -- два метричних простори.\\
Відображення $f \colon X \to Y$ називається \textbf{неперервним у точці $x_0$}, якщо
\begin{align*}
\forall \varepsilon > 0: \exists \delta > 0: \forall x \in X: \rho(x,x_0) < \delta \implies \tilde{\rho}(f(x),f(y)) < \varepsilon
\end{align*}
\end{definition}

\begin{remark}
Дане означення можна записати більш компактним чином. Маємо $f \colon X \to Y$.\\
$f$ -- неперервне в точці $x_0 \in X \iff \forall \varepsilon > 0: \exists \delta > 0: f(B(x_0;\delta)) \subset B(f(x_0);\varepsilon)$.
\end{remark}

\begin{proposition}
Задані $(X,\rho), (Y,\tilde{\rho})$ -- два метричних простори та $f \colon X \to Y$.\\
$f$ -- неперервне в точці $x_0 \in X \iff \forall \{x_n\} \subset X: x_n \to x_0 \text{ в } X \implies f(x_n) \to f(x_0) \text{ в } Y$.\\
\textit{Вправа: довести.}
\end{proposition}

\begin{theorem}
Задані $(X,\rho), (Y,\tilde{\rho})$ -- два метричних простори та $f \colon X \to Y$.\\
$f$ -- неперервне (на множині $X$) $\iff \forall V$ -- замкнена в $Y: f^{-1}(U)$ -- замкнена в $X$.
\end{theorem}

\begin{proof}
\rightproof Дано: $f$ -- неперервне. Нехай $V$ -- замкнена в $Y$. Зафіксуємо $x_n \in f^{-1}(V)$ таким чином, що $x_n \to x_0$. Але за неперервністю, $f(x_n) \to f(x_0)$, та додатково $f(x_n) \in V$. Значить, за замкненістю $V$, точка $f(x_0) \in V \implies x_0 \in f^{-1}(V)$. Отже, $f^{-1}(V)$ -- замкнена.
\bigskip \\
\leftproof Дано: $\forall V$ -- замкнена в $Y: f^{-1}(U)$ -- замкнена в $X$. Оберемо $x_n \to x_0$.\\
!Припустимо, що $f(x_n) \not\to f(x_0)$, тобто існує шар $B(f(x_0);\varepsilon)$, поза яким знаходиться підпослідовність $\{f(x_{n_k})\}$. Якщо $V$ -- замикання множини $\{f(x_{n_k})\}$, то звідси $x_{n_k} \in f^{-1}(V)$; $f(x_0) \notin V$. Тоді звідси $x_0 \notin f^{-1}(V)$, проте $x_{n_k} \to x_0$ та $x_0$ є граничною точкою для $f^{-1}(A)$. Суперечність!
\end{proof}

\begin{corollary}
$f$ -- неперервне $\iff \forall U$ -- відкрита в $Y: f^{-1}(U)$ -- відкрита в $X$.\\
\textit{Вказівка: застосувати попередню теорему та рівність $f^{-1}(A^c) = (f^{-1}(A))^c$.}
\end{corollary}

\begin{proposition}
Задані $X,Y,Z$ -- метричні простори та $f \colon X \to Y, g \colon Y \to Z$. Нехай $f$ -- неперервне в точці $x_0 \in X$ та $g$ -- неперервне в точці $f(x_0) \in Y$. Тоді $g \circ f$ -- неперервне в точці $x_0 \in X$.\\
\textit{Вправа: довести.}
\end{proposition}

\begin{proposition}
\label{continuity_of_metric_function_of_one_argument}
Задано $(X,\rho)$ -- метричний простір та зафіксуємо $x_0 \in X$. Тоді функція $f(x) = \rho(x,x_0)$, де $f \colon X \to \mathbb{R}$, -- неперервна на $X$.
\end{proposition}

\begin{proof}
Дійсно, нехай $y_0 \in X$. Припустимо, що $\{y_n\}$ така, що  $y_n \to y_0$. Хочемо $f(y_n) \to f(y_0)$. Справді,\\
$|f(y_n) - f(y_0)| = |\rho(y_n,x_0) - \rho(y_0,x_0)| \leq |\rho(y_n,y_0)| \to 0$.\\
\textit{Для $\mathbb{R}$ береться стандартна метрика, якщо нічого іншого не вказується зазвичай.}
\end{proof}

\begin{definition}
Задано $(X,\rho)$ -- метричний простір та $f \colon X \to X$.\\
Дане відображення називається \textbf{стиском}, якщо
\begin{align*}
\exists q \in (0,1): \forall x,y \in X: \rho(f(x),f(y)) \leq q \cdot \rho(x,y)
\end{align*}
\end{definition}

\begin{remark}
Стискаючі відображення -- неперервні.\\
\textit{Вказівка: обрати $\delta = \dfrac{q}{\varepsilon}$ при всіх $\varepsilon > 0$.}
\end{remark}

\begin{theorem}[Теорема Банаха]
Задано $(X,\rho)$ -- повний метричний просторі та $f \colon X \to X$ -- стискаюче відображення. Тоді існує єдина точка нерухома точка, тобто $\exists ! x \in X: f(x) = x$.
\end{theorem}

\begin{proof}
І. \textit{Існування}.\\
Нехай $x_0 \in X$ -- довільна точка. Зробимо позначення: $x_1 = f(x_0),\ x_2 = f(x_1),\ \dots, x_n = f(x_{n-1}), \dots$ Покажемо, що послідовність $\{x_n, n \geq 0\}$ -- фундаментальна. Дійсно, для $m \leq n$ маємо:\\
$\rho(x_m, x_n) = \rho(f(x_{m-1}), f(x_{n-1})) \leq q \cdot \rho(x_{m-1},x_{n-1}) \leq \dots \leq q^m \rho(x_0,x_{n-m})$.\\
$\rho(x_0,x_{n-m}) \leq \rho(x_0,x_1) + \rho(x_1,x_2) + \dots + \rho(x_{n-m-1}, x_{n-m}) \leq \rho(x_0,x_1)(1+q+\dots+q^{n-m-1}) \leq \\
\leq \rho (x_0,x_1) \dfrac{1}{1-q}$.\\
Разом отримаємо $\rho(x_m,x_n) \leq \dfrac{q^m}{1-q} \rho(x_0,x_1) \to 0, n,m \to \infty$.\\
Оскільки $(X,\rho)$ -- повний, то $\{x_n\}$ -- збіжна, позначимо $a = \displaystyle\lim_{n \to \infty} x_n$. Зважаючи на неперервність стиска, отримаємо $\displaystyle f(a) = f\left( \lim_{n \to \infty} x_n \right) = \lim_{n \to \infty} f(x_n) = \lim_{n \to \infty} x_{n+1} = a$. Тобто $a$ -- це наша шукана нерухома точка.
\bigskip \\
II. \textit{Єдиність}.\\
!Припустимо, що $f$ має дві різні нерухомі точки $a,b$. Буде суперечність! Дійсно,\\
$0 < \rho(a,b) = \rho(f(a),f(b)) \leq q \cdot \rho(a,b) < \rho(a,b)$.
\end{proof}

\begin{remark}
Насправді, в теоремі Банаха досить вимагати, щоб саме $f^n \overset{\text{def.}}{=} \underset{n\text{ разів}}{f \circ \dots \circ f}$ було стиском, а не відображення $f$.\\
Дійсно, за теоремою Банаха, $f^n$ матиме єдину нерухому точку $a$, тобто $f^n(a) = a$. Тоді точка $f(a)$ буде теж нерухомою для $f^n$, оскільки $f^n(f(a)) = f(f^n(a)) = f(a)$. Але за єдиністю, $f(a) = a$ -- дві нерухомі мають збігатися. Єдиність нерухомої точки для $f$ доводиться неважко.
\end{remark}

\subsection{Компактність}
\begin{definition}
Задано $(X,\rho)$ -- метричний простір та $A \subset X$.\\
Множина $A$ називається \textbf{компактом}, якщо
\begin{align*}
\forall \{x_n, n \geq 1\} \subset A: \exists \{x_{n_k}, k \geq 1\}: x_{n_k} \to x_0, k \to \infty, \text{ причому } x_0 \in A
\end{align*}
Якщо прибрати умову $x_0 \in A$, то тоді $A$ називається \textbf{передкомпактом}.
\end{definition}

\begin{proposition}
Задано $(X,\rho)$ -- метричний простір та $A \subset X$.\\
$A$ -- компакт $\iff $ $\forall B \subset A$, де $B$ -- нескінченна множина, існує $x_0 \in A$ -- гранична точка $B$.\\
Якщо прибрати умову $x_0 \in A$, то вже мова буде йти про передкомпакт.
\end{proposition}

\begin{proof}
\rightproof Дано: $A$ -- компакт. Нехай $B \subset A$ -- нескінченна множина. Оберемо послідовність $\{x_n, n \geq 1\} \subset B \subset A$, де всі вони між собою різні. Тоді за умовою компактності, існує підпослідовність $x_{n_k} \to x_0$, причому $x_0 \in A$. Зауважимо, що всі $x_{n_k} \neq x_0$, тож $x_0$ -- гранична точка $A$.\\
Якби існували $k \in \mathbb{N}$, для яких $x_{n_k} = x_0$, то тоді ми би сформували підпослідовність $\{x_{n_{k_m}}\}$ без цих елементів, причому $x_{n_{k_m}} \to x_0$, а тепер $x_{n_{k_m}} \neq x_0$. Тож все одно $x_0$ залишається граничною точкою $A$.
\bigskip \\
\leftproof Дано: $\forall B \subset A$, де $B$ -- нескінченна множина, існує $x_0 \in A$ -- гранична точка $B$. Отже, нехай $\{x_n, n \geq 1\} \subset A$ -- довільна послідовність. У нас є два варіанти:\\
I. Множина значень $\{x_n\}$ -- скінченна. Тоді можна відокремити стаціонарну підпослідовність.\\
II. Множина значень $\{x_n\}$ -- нескінченна, всі ці значення покладемо в множину $B \subset A$. Тоді за умовою, існує $x_0 \in A$ -- гранична точка $B$. Отже, $B \cap B(x_0;\varepsilon)$ містить нескінченне число точок для всіх $\varepsilon > 0$. Зокрема:\\
$\varepsilon = 1 \implies B \cap B(x_0;1)$ має нескінченну множину. Там існує елемент $y_1 \in B \cap B(x_0; 1)$, тобто це одне зі значень послідовності. Тобто $y_1 = x_{n_1}$.\\
$\varepsilon = \dfrac{1}{2} \implies B \cap B\left( x_0;\dfrac{1}{2} \right)$ має нескінченну множину. Там існує елемент $y_2 \in B \cap B\left( x_0;\dfrac{1}{2} \right)$, тобто це одне зі значень послідовності. Тобто $y_2 = x_{n_2}$. Причому можна обрати $x_{n_2} > x_{n_1}$. Якби так не було можливо, то $B \cap B\left(x_0; \dfrac{1}{2}\right)$ була б скінченною множиною, що не наше випадок.\\
\vdots \\
Побудували підпослідовність $\{x_{n_k}, k \geq 1\}$, причому $\rho(x_0,x_k) < \dfrac{1}{k}$. Тож при $k \to \infty$ матимемо $x_{n_k} \to x_0 \in A$. Отже, $A$ -- компакт.
\bigskip \\
\textit{Випадок передкомпакту повторюється майже все слово в слово.}
\end{proof}

\begin{proposition}
Задано $(X,\rho)$ -- компактний метричний простір. Тоді $(X,\rho)$ -- повний.
\end{proposition}

\begin{proof}
Дійсно, нехай $\{x_n\} \subset X$ -- фундаментальна. Оскільки $X$ -- компакт, то існує збіжна підпослідовність $\{x_{n_k}\}$, де $x_{n_k} \to x, x \in X$. Ми вже знаємо, що тоді й сама послідовність $\{x_n\} \to x$ буде збіжною. Отже, $(X,\rho)$ -- повний метричний простір.
\end{proof}

\begin{definition}
Задано $(X,\rho)$ -- метричний простір та $A \subset X$.\\
Множина $A$ називається \textbf{обмеженою}, якщо
\begin{align*}
\exists R > 0: A \subset B(a;R)
\end{align*}
\end{definition}

\begin{definition}
Задано $(X,\rho)$ -- метричний простір та $A \subset X$.\\
Множина $A$ називається \textbf{цілком обмеженою}, якщо
\begin{align*}
\forall \varepsilon > 0: \exists C_\varepsilon = \{x_1,x_2,\dots,x_n\}: A \subset \displaystyle\bigcup_{x \in C_\varepsilon} B(x;\varepsilon)
\end{align*}
До речі, $C_\varepsilon$, для якої виконана $A \subset \displaystyle\bigcup_{x \in C_\varepsilon} B(x;\varepsilon)$, називається \textbf{скінченною $\varepsilon$-сіткою}.\\
Тобто $A$ -- цілком обмежена, коли вона має скінченну $\varepsilon$-сітку для всіх $\varepsilon > 0$.
\end{definition}

\begin{proposition}
Задано $(X,\rho)$ -- метричний простір та $A$ -- цілком обмежена множина. Тоді $A$ -- обмежена.
\end{proposition}

\begin{proof}
Для множини $A$ існує $1$-сітка, тобто $C_1 = \{x_1,\dots,x_n\}$, для якої $A \subset \displaystyle\bigcup_{x \in C_1} B(x;1)$.\\
Зафіксуємо $y \in X$ та оберемо $R = 1 +\displaystyle\max_{x \in C_1} \rho(x,y)$. Тоді хочемо довести, що $A \subset B(y;R)$.\\
Нехай $a \in A$, тоді вже $a \in B(x;1)$ при деякому $x \in C_1$, а також $\rho(a;x) < 1$. Звідси\\
$\rho(a;y) \leq \rho(a;x) + \rho(x;y) < 1 + \displaystyle\max_{x \in C_1} \rho(x;y) = R$.\\
Отже, $A$ -- обмежена.
\end{proof}

\begin{remark}
Не обов'язково вимагати, щоб $A$ була цілком обмежена. Подивившись на це доведення, ми можемо лише вимагати, щоб $A$ мала хоча б одну $\varepsilon$-сітку -- тоді буде обмеженість $A$.
\end{remark}

\begin{theorem}[Критерій Фреше-Хаусдорфа]
Нехай $(X,\rho)$ -- повний метричний простір та $A \subset X$.\\
$A$ -- цілком обмежена $\iff A$ -- передкомпакт.
\end{theorem}

\begin{remark}
Під час доведення \leftproof нам не потрібна буде умова повноти метричного простору.
\end{remark}

\begin{proof}
\rightproof Дано: $A$ -- цілком обмежена. Нехай $\{a_n, n \geq 1\} \subset A$ -- довільна послідовність. \\
Оберемо $1$-сітку $C_1$, де $A \subset \displaystyle\bigcup_{x \in C_1} B(x;1)$. В одному з цих шарів нескінченне число членів послідовності, той шар позначу за $B(y_1;1)$; маємо підпослідовність $\{a_{n_k}, k \geq 1\} \subset B(y_1;1)$.\\
Оберемо $\dfrac{1}{2}$-сітку $C_{\frac{1}{2}}$, де $A \subset \displaystyle\bigcup_{x \in C_{\frac{1}{2}}} B\left(x;\dfrac{1}{2}\right)$. В одному з цих шарів нескінченне число членів підпослідовності, той шар позначу за $B\left(y_2;\dfrac{1}{2}\right)$; маємо підпідпослідовність $\{a_{n_{k_m}}, k \geq 1\} \subset B\left(y_2;\dfrac{1}{2}\right)$.\\
\vdots \\
Отримали послідовність центрів $\{y_n, n \geq 1\}$, доведемо її фундаментальність.\\
$\rho(y_n,y_m) \leq \rho(y_n,a_*) + \rho(a_*,y_m) < \dfrac{1}{n} + \dfrac{1}{m} \to 0$ при $n,m \to \infty$. У даному випадку ми підібрали елемент $a_* \in B\left( \dfrac{1}{n};y_n \right) \cap B\left( \dfrac{1}{m};y_m \right)$.\\
Тепер розглянемо підпослідовність $\{a_{n_p}, p \geq 1\}$, яка будується таким чином: беремо перший елемент з $\{a_{n_k}\}$ (це наше $a_{n_1}$), потім перший елемент з $\{a_{n_{k_m}}\}$ (це наше $a_{n_2}$), \dots Доведемо, що $\{a_{n_p}, p \geq 1\}$ -- фундаментальна. Дійсно,\\
$\rho(a_{n_p}, a_{n_t}) \leq \rho(a_{n_p}, y_p) + \rho(y_p,y_t) + \rho(y_t,a_{n_t}) < \dfrac{1}{p} + \dfrac{1}{t} + \rho(y_p,y_t) \to 0, t,p \to \infty$\\
Оскільки $(X,\rho)$ -- повний, то звідси $\{a_{n_p}, n \geq 1\}$ -- збіжна підпослідовність. Довели, що $A$ -- передкомпакт.
\bigskip \\
\leftproof Дано: $A$ -- передкомпакт.\\
!Припустимо, що $A$ -- це є цілком обмеженою. Тобто для деякого $\varepsilon > 0$ не існує $\varepsilon$-сітки. Нехай $x_1 \in A$. Тоді існує $x_2 \in A$, для якої $\rho(x_1,x_2) \geq \varepsilon$ (інакше якби для кожної $x_2 \in A$ була б $\rho(x_1,x_2) < \varepsilon$, то ми си знайшли $\varepsilon$-сітку $\{x_1\}$, що суперечить умові).\\
Далі існує $x_3 \in A$, для якої $\rho(x_1,x_3) \geq \varepsilon$ та $\rho(x_2,x_3) \geq \varepsilon$ (аналогічно якби для кожної $x_3 \in A$ ці два нерівності не виконувалися би, то ми би знайшли один з трьох $\varepsilon$-сіток: $\{x_1\}$ або $\{x_2\}$ або $\{x_1,x_2\}$).\\
\vdots \\
Побудували послідовність $\{x_n, n \geq 1\} \subset A$, для якої справедлива $\rho(x_n,x_m) \geq \varepsilon$ при всіх $n \neq m$. За умовою передкомпактності, існує $\{x_{n_k}, n \geq 1\}$, для якої $x_{n_k} \to x_0$. Водночас звідси ми отримаємо, що існують номери $K_1,K_2$, для яких $\rho(x_{n_{K_1}}, x_{n_{K_2}}) \leq \rho(x_{n_{K_1}}, x_0) + \rho(x_0, x_{n_{K_2}}) < \varepsilon$. Суперечність!\\
Отже, $A$ все ж таки має бути цілком обмеженою.
\end{proof}

\begin{theorem}
Задано $(X,\rho)$ -- метричний простір та $A \subset X$.\\
$A$ -- компакт $\iff$ для кожного відкритого покриття $A$ можна виділити скінченне підпокриття.
\end{theorem}

\begin{proof}
\rightproof Дано: $A$ -- компакт.\\
!Припустимо, що існує відрките покриття $\{U_\alpha\}$ множини $A$, від якої не можна відокремити скінченне підпокриття. Оскільки $A$ -- компакт, то $A$ -- цілком обмежена. Значить, існує $1$-сітка $C_1$ (причому можна підібрати так, щоб $C_1 \subset A$), для якої $A \subset \displaystyle\bigcup_{x \in C_1} B(x;1)$, або можна переписати як $A \subset \displaystyle\bigcup_{x \in C_1} A \cap B(x;1)$. Серед множин $A \cap B(x;1)$ існує одна з них, яка не покривається скінченним чином множинами $\{U_\alpha\}$. Дану множину позначу за $A'$.\\
Сама множина $A'$ -- також цілком обмежена, тож існує $\dfrac{1}{2}$-сітка $C_{\frac{1}{2}}$ (знову підберемо так, щоб $C_{\frac{1}{2}} \subset A'$), для якої виконано $A' \subset \displaystyle\bigcup_{x \in C_{\frac{1}{2}}} A' \cap B\left(x; \dfrac{1}{2}\right)$. Знову ж таки, серед $A' \cap B\left(x; \dfrac{1}{2}\right)$ існує одна з них, що не покривається скінченним чином множинами $\{U_\alpha\}$. Дану множину позначу за $A''$. \\
\vdots \\
Продовжуючи процедуру, отримаємо набір куль $B_n = B\left(x_n;\dfrac{1}{n}\right)$, де центр $x_n \in B_{n-1} \cap A$. Позначимо $\overline{B_n \cap A} = K_n$ та зауважимо, що $K_n$ -- це замкнена куля в метричному підпросторі $A$, де $R = \dfrac{1}{2^n}$ та центр $y_n \in K_{n-1}$.\\
Подвоїмо радіуси кожної з цих куль. Тоді отримаємо послідовність вкладених куль, які стягуються. Оскільки $A$ -- компакт, то $(A,\rho_A)$ -- повний метричний простір, тож за теоремою Кантора, існує $a \in A$ -- спільна точка цих куль. Зважаючи на покриття множини $A$, отримаємо $a \in U_{\alpha_0}$ при деякому $\alpha_0$. Оскільки $U_{\alpha_0}$ -- відкрита, то існує куля $B(z,\delta) \subset U_{\alpha_0}$. Ми можемо підібрати завжди такий $N \in \mathbb{N}$, щоб було виконано $\dfrac{1}{N} < \dfrac{\delta}{2}$, тоді звідси $K_n \subset B(z;\delta) \subset U_{\alpha_0}$. Таким чином, $K_n$ була покрита лише однією множиною із $\{U_\alpha\}$, проте ми обирали такі кулі (на початку), які не допускали скінченне підпокриття. Суперечність!
\bigskip \\
\leftproof Дано: кожне покриття $A$ має скінченне підпокриття.\\
!Припустимо, що $A$ -- не компакт, тобто існує послідовність $\{x_n, n \geq 1\} \subset A$, що не має часткових границь. Тоді кожний відкритий окіл $U_a, a \in A$, містить скінченну кількість членів послідовності $\{x_n\}$ (якби існував окіл $U_a$ із нескінченним числом членів послідовності, то $a$ стала би граничною точкою, що неможливо). Набір $\{U_a, a \in A\}$ -- відкрите покриття множини $A$. За умовою, існує скінченне підпокриття $\{U_{a_1},\dots,U_{a_n}\}$ множини $A$, але тоді $A \subset \displaystyle\bigcup_{i=1}^n U_{a_i}$, де праворуч -- скінченна множина; ліворуч -- нескінченна в силу нескінченності послідовності $\{x_n\}$ -- суперечність!
\end{proof}

\begin{corollary}
Задано $(X,\rho), (Y,\tilde{\rho})$ -- два метричних простори та $f \colon X \to Y$ -- неперервне відображення. Відомо, що $X$ -- компакт. Тоді $f(X)$ -- компакт.
\end{corollary}

\begin{proof}
Маємо $\{U_\alpha\}$ -- відкрите покриття $f(X)$. Тоді $\{f^{-1}(U_\alpha)\}$ -- відкрите покриття $X$, але за компактністю, можна виділити скінченне підпокриття $\{f^{-1}(U_1),\dots,f^{-1}(U_m)\}$, тоді звідси $\{U_1,\dots,U_m\}$ буде скінченним підпокриттям $f(X)$.
\end{proof}

\begin{corollary}
Задано $(X,\rho)$ -- метричний простір та $f \colon X \to \mathbb{R}$ -- числова неперервна функція. Відомо, що $X$ -- компакт. Тоді $f$ -- обмежена та досягає найбільшого та найменшого значень.
\end{corollary}

\begin{theorem}
Задано $(X,\rho), (Y,\tilde{\rho})$ -- два метричних простори та $f \colon X \to Y$ -- неперервне, причому $X$ -- компакт. Тоді $f$ -- рівномірно неперервне.
\end{theorem}

\begin{proof}
!Припустимо, що $\exists \varepsilon > 0: \forall \delta > 0: \exists x,y \in X: \rho(x,y) < \delta$, але $\tilde{\rho}(f(x),f(y)) \geq \varepsilon$.\\
Оберемо $\delta = \dfrac{1}{n}, n \in \mathbb{N}$, тоді утвориться послідовність $\{x_n\}, \{y_n\} \subset X$. Оскільки $X$ -- компакт, то відокремимо збіжні підпослідовності $\{x_{n_k}\}, \{y_{n_k}\}$. Але оскільки $\rho(x_{n_k}, y_{n_k}) < \dfrac{1}{n_k}$, то звідси випливає $\displaystyle\lim_{k \to \infty} x_{n_k} = \lim_{k \to \infty} x_{n_k}$. Із іншого боку, $\displaystyle\lim_{k \to \infty} f(x_{n_k}) \neq \lim_{k \to \infty} f(y_{n_k})$, оскільки виконана нерівність $\tilde{\rho}(f(x_{n_k}), f(y_{n_k}) \geq \varepsilon$. Суперечність!
\end{proof}

\subsection{Теорема Стоуна-Ваєрштраса}
Надалі будемо розглядати компактний метричний простір $(X,\rho)$ та метричний простір $(C(X), \sigma)$ -- простір неперервних функцій із метрикою $\sigma(f,g) = \displaystyle\max_{x \in X}\| f(x) - g(x)\|$. Причому даний метричний простір теж повний (це аналогічно доводиться).

\begin{definition}
Множина $A \subset C(X)$ називається \textbf{алгеброю}, якщо $\forall f,g \in A, \forall \alpha \in \mathbb{R}$:
\begin{align*}
\alpha f,\ f+g,\ f \cdot g \in A
\end{align*}
\end{definition}

\begin{definition}
Нехай $A \subset C(X)$ -- алгебра.\\
Алгебра $A$ \textbf{відділяє точки} множини $X$, якщо
\begin{align*}
\forall x,y \in X: x \neq y: \exists f \in A: f(x) \neq f(y)
\end{align*}
\end{definition}

\begin{theorem}[Теорема Стоуна-Ваєрштраса]
Задано $(X,\rho)$ -- компактний метричний простір та $(C(X),\sigma)$ -- простір неперервних дійсних функцій, заданий вище. Маємо $A \subset C(X)$. Про неї відомо, що
\begin{enumerate}[nosep,wide=0pt,label={\arabic*)}]
\item $A$ -- алгебра, яка віддаляє точки множини $X$;
\item функція $f$, яка визначена як $f(x) = 1, \forall x \in X$, належить $A$.
\end{enumerate}
Тоді множина $A$ скрізь щільна в $(C(X),\sigma)$.
\end{theorem}

\begin{proof}
Ми хочемо довести, що $\bar{A} = C(X)$.\\
Нехай $f \in A$. Хочемо довести, що $|f| \in \bar{A}$. У курсі мат.\ аналізу ми доводили теорему Ваєрштраса про наближення функції многочленом. Зокрема для функції $g(t) = \sqrt{t}, t \in [0,1]$ маємо, що $\forall \varepsilon > 0: \exists P_\varepsilon$ -- многочлен від $t: |\sqrt{t} - P_\varepsilon(t)| < \varepsilon$. Тоді $\forall x \in X:$\\
$\left| \dfrac{|f(x)|}{\|f\|} - P_\varepsilon\left( \dfrac{f^2(x)}{\|f\|^2} \right) \right| = \left| \sqrt{\dfrac{|f(x)|^2}{\|f\|^2}} - P_\varepsilon\left( \dfrac{f^2(x)}{\|f\|^2} \right) \right| < \varepsilon$.\\
Оскільки $f \in A$, то в силу алгебри $\dfrac{f^2}{\|f\|} \in A$. Оскільки $P_\varepsilon$ -- многочлен, то $P_\varepsilon \circ \dfrac{f^2}{\|f\|} \in A$. Ми знайшли $P_\varepsilon \circ \dfrac{f^2}{\|f\|^2} \in A$, для якої $\left\| \dfrac{|f|}{\|f\|} - P_\varepsilon \circ \dfrac{f^2}{\|f\|^2} \right\| < \varepsilon$. Отже, $\dfrac{|f|}{\|f\|}$ -- гранична точка, тобто $\dfrac{|f|}{\|f\|} \in \bar{A}$.\\
Відомо знову з мат.\ аналізу, що для всіх $a,b \in \mathbb{R}$ ми маємо такі рівності:\\
$\max\{a,b\} = \dfrac{1}{2}\left(a+b+|a-b|\right) \qquad \min\{a,b\} = \dfrac{1}{2}\left(a+b-|a-b|\right)$.\\
Значить, маючи $f,g \in A$ та маючи результат вище, отримаємо $\max \{f,g\}, \min\{f,g\} \in \bar{A}$.\\
Оберемо $x,y \in X$ так, що $x \neq y$. Тоді існує функція $g \in A$, для якої $g(x) \neq g(y)$. Далі покладемо нову функцію $f(z) = \alpha + \dfrac{\beta-\alpha}{g(y)-g(x)}(g(z)-g(x)), z \in X,\ \alpha,\beta \in \mathbb{R}$. Тоді звідси $f \in A$ (ми тут користуємося пунктом 2), щоб це показати), причому $f(x) = \alpha,\ f(y) = \beta$.\\
Отже, що ми довели щойно: $\forall x,y \in X: x \neq y, \forall \alpha,\beta \in \mathbb{R}: \exists f \in A: f(x) = \alpha,\ f(y) = \beta$.
\bigskip \\
Нехай $f \in C(X)$ та $\varepsilon > 0$. Зафіксуємо $x \in X$, для $z \in X$ покладемо $\alpha = f(x), \beta = f(z)$. Тоді за щойно доведеним, існує $h_z \in A$, для якої $h_z(x) = \alpha = f(x)$ та $h_z(z) = \beta = f(z)$.\\
Оскільки $h_z - f \in C(X)$, то за означенням, $\exists \delta_z > 0: \forall y \in B(z,\delta_z): h_z(y)-f(y) < \varepsilon$. Сім'я множин $\{B(z,\delta_z) \mid z \in X\}$ -- відкрите покриття компактної множини $X$. Отже, ми можемо взяти скінченне підпокриття $\{B(z_k,\delta_{z_k}) \mid k = \overline{1,n}\}$.\\
Визначимо функцію $g_x(y) = \displaystyle\min_{1 \leq k \leq n} \{h_{z_k}(y))\}, y \in X$. Зауважимо, що по-перше, $g_x \in \bar{A}$; по-друге, $g_x(x) = f(x)$; по-третє, $\forall y \in X: g_x(y) -f(y) < \varepsilon$.\\
Оскільки $g_x - f \in C(X)$, то за означенням, $\exists \delta_x > 0: \forall y \in B(x,\delta_x): g_x(y) - f(y) > - \varepsilon$. Сім'я множин $\{B(x,\delta_x) \mid x \in X \}$ -- відкрите покриття компактної множини $X$. Отже ми можемо взяти скінченне підпокриття $\{B(x_k,\delta_{x_k} \mid k = \overline{1,m}\}$.\\
Визначимо функцію $h(y) = \displaystyle\max_{1 \leq k \leq m} g_{x_k}(y), y \in X$. Тоді $h \in \bar{A}$, причому також $\forall y \in X: \\ f(y) - \varepsilon \leq h(y) \leq f(y) + \varepsilon$. Для будь-якої функції $f \in C(X)$ ми знайшли $h \in A$, для якої $\|h - f\| < \varepsilon$. Отже, $\bar{A} = C(X)$.
\end{proof}

\subsection{Теорема Арцела-Асколі}
\begin{definition}
Сім'я функції $\mathcal{F} \subset C(X)$ називається \textbf{рівномірно обмеженою}, якщо
\begin{align*}
\exists M > 0: \forall x \in X, \forall f \in \mathcal{F}: |f(x)| \leq M
\end{align*}
\end{definition}

\begin{definition}
Сім'я функції $\mathcal{F} \subset C(X)$ називається \textbf{одностайно неперервними}, якщо
\begin{align*}
\forall \varepsilon > 0: \exists \delta > 0: \forall f \in \mathcal{F}, \forall x,y \in X: \rho(x,y) < \delta \implies |f(x) - f(y)| < \varepsilon
\end{align*}
\end{definition}

\begin{theorem}[Теорема Арцела-Асколі]
Задано $X$ -- компактний метричний простір. Нехай послідовність $\sequence{f_n} \subset C(X)$ рівномірно обмежена та одностайно неперервна. Тоді $\exists \sequence{f_{n_k}}$ -- рівномірно збіжна підпослідовність.
\end{theorem}

\begin{proof}
Оскільки $X$ -- компакт, то $X$ -- сепарабельний автоматично (TODO: подивитися, чи є таке твердження). Цю скрізь щільну зліченну множину позначу за $S = \{x_1,x_2,\dots\}$.\\
Розглянемо послідовність $\sequence{f_n(x_1)}$, яка обмежена. Тоді за теоремою Бользано-Ваєрштраса, існує збіжна підпослідовність $\sequence{f_{1,n}(x_1)}$.\\
Розглянемо послідовність $\sequence{f_{1,n}(x_2)}$, яка обмежена. Аналогічно існує збіжна підпослідовність $\sequence{f_{2,n}(x_2)}$.\\
\vdots \\
Тепер розглянемо діагональну послідовність $\sequence{f_{n,n}}$. Зауважимо, що вона збігається в кожній точці $x \in S$. Дійсно, $\sequence{f_{n,n}} \subset \sequence{f_{k,n}}$, а остання послідовність збігається в точці $x_k$.
\bigskip \\
Для зручності цю послідовність перезпозначу за $\sequence{f_n}$. Доведемо, що вона рівномірно фундаментальна, внаслідок чого буде рівномірно збіжною. За одностайною неперервністю, $\exists \delta > 0: \forall x,y, \forall n: \rho(x,y) < \delta \implies |g_n(x) - g_n(y)| < \dfrac{\varepsilon}{3}$.\\
Нехай $x \in X$ та $M > \dfrac{1}{\delta}$. Розглянемо $n,m > M$, тоді\\
$|g_n(x) - g_m(x)| \leq |g_n(x) - g_n(s)| + |g_n(s) - g_m(s)| + |g_m(s) - g_m(x)| < 3 \cdot \dfrac{\varepsilon}{3} = \varepsilon$.\\
У цьому випадку $s \in S$ така точка, що $\rho(x,s) < \delta$. Ми можемо це знайти в силу скрізь щільності.
(TODO: добити).
\end{proof}
\newpage

\section{Початок функціонального аналізу}
\subsection{Лінійні нормовані простори}
\begin{definition}
Задано $L$ -- лінійний простір над $\mathbb{R}$ або $\mathbb{C}$.\\
Задамо функцію $\| \cdot \| \colon L \to \mathbb{R}$, що називається \textbf{нормою}, якщо виконуються умови:
\begin{align*}
\text{1) } \forall x \in L: \|x\| \geq 0 \qquad \|x\| = 0 \iff x = 0 \\
\text{2) } \forall x \in L: \forall \alpha \in \mathbb{R} \text{ або } \mathbb{C}: \|\alpha x\| = |\alpha| \|x\| \\
\text{3) } \forall x,y \in L: \|x+y\| \leq \|x\| + \|y\|
\end{align*}
Тоді пару $(L, \|\cdot \|)$ назвемо \textbf{нормованим простором}.\\
Функцію $\| \cdot \| \colon L \to \mathbb{R}$ ще називають \textbf{переднормою}, якщо всі умови виконуються, окрім умови $\|x\| = 0 \iff x = 0$.
\end{definition}

\begin{proposition}
Задано $(L, \| \cdot \|)$ -- нормований простір. Тоді $\forall x,y \in L: \|x-y\| \geq \left| \|x\| - \|y\| \right|$.\\
\textit{Вказівка: $\|x\| = \|x + y - x\|$ та $\|y\| = \|y + x -y \|$.}
\end{proposition}

\begin{proposition}
Задано $(L, \| \cdot \|)$ -- нормований простір. Тоді $L$ з метрикою $\rho(x,y) = \|x-y\|$ утворює метричний простір $(L,\rho)$.\\
\textit{Вправа: перевірити три аксіоми.}
\end{proposition}

\begin{corollary}
У такому разі справедливі додаткові властивості для заданої метрики:
\begin{enumerate}[nosep,wide=0pt,label={\arabic*)}]
\item $\forall x,y,z \in L: \rho(x+z,y+z) = \rho(x,y)$ (інваріантність по відношенню до зсуву);
\item $\forall x,y \in L, \forall \alpha \in \mathbb{R} \text{ або } \mathbb{C}: \rho(\alpha x, \alpha y) = |\alpha| \rho(x,y)$ (однорідність).
\end{enumerate}
\end{corollary}

\begin{example}
Зокрема дані простори будуть нормованими:\\
\begin{tabular}{rll}
1) & $\mathbb{R}$, & $\| x\| = |x|$;\\
2) & $\mathbb{R}^n$, & $\| \vec{x} \| = \sqrt{x_1^2 + \dots + x_n^2}$ або навіть $\| \vec{x} \| = |x_1| + \dots + |x_n|$;\\
3) & $C([a,b])$ & $\displaystyle \| f\| = \max_{t \in [a,b]} |f(t)|$;\\
4) & $L_p(X,\lambda)$, & $\|f\|_p = \displaystyle\left(\int_X |f|^p\,d\lambda\right)^{\frac{1}{p}}$.
\end{tabular}\\
Тому всі вони будуть автоматично метричними просторами із метрикою, що вище задана.
\end{example}

\begin{example}
Дискретний простір $(X,d)$ -- метричний, але не нормований.
\end{example}

\begin{definition}
Задано $(L, \|\cdot\|)$ -- лінійний нормований простір. Оскільки в неї запроваджена метрика, то можна щось казати про присутність чи відсутність повноти метричного простору.\\
Повний нормований простір називається \textbf{банаховим}.
\end{definition}

\begin{example}
Зокрема нормований простір $C([a,b])$ зі стандартною нормою $\|x\| = \displaystyle\max_{t \in [a,b]} |x(t)|$ -- банехів. Це випливає з курсу математичного аналізу 2 семестру.
\end{example}

\begin{example}
Задамо підпростір $C([0,1])$ із нормою із $L_2([0,1],\lambda_1)$, де $\lambda_1$ -- міра Лебега. Доведемо, що в такому разі $C([0,1])$ уже не буде банаховим.\\
Розглянемо таку функціональну послідовність $\{x_n,n \geq 1\} \subset C([0,1])$, що задається таким чином:\\
$x_n(t) \begin{cases}
0, & 0 \leq x \leq \dfrac{1}{2} - \dfrac{1}{n} \\
\dfrac{nx}{2} - \dfrac{n}{4} + \dfrac{1}{2}, & \dfrac{1}{2} - \dfrac{1}{n} \leq x \leq \dfrac{1}{2} + \dfrac{1}{n} \\
1, & \dfrac{1}{2} + \dfrac{1}{n} \leq x \leq 1
\end{cases}$.\\
Це набір функцій, де похила частина зі збільшенням $n$ перетворюється в вертикальну лінію. Зауважимо, що якщо вязти поточкову границю, то отримаємо $x(t) = \begin{cases} 0, & 0 \leq x \leq \dfrac{1}{2} \\1, & \dfrac{1}{2} < x \leq 1 \end{cases}$. При цьому\\
$\|x_n - x\|_2^2 = \displaystyle\int_{[0,1]} |x_n-x|^2\,d\lambda_1 = \int_0^1 |x_n(t)-x(t)|^2\,dt = \dots = \dfrac{1}{6n} \to 0$ при $n \to \infty$.\\
Отже, $\{x_n\}$ в просторі $C([0,1])$ із нормою $L_2$ збігається до точки $x \notin C([0,1])$, але при цьому буде граничною для $C([0,1])$. Тобто $C([0,1])$ не буде замкненим, тож $C([0,1])$ -- не повний, або не банахів.
\end{example}

\begin{proposition}
Задано $(L, \|\cdot \|)$ -- нормований простір. Тоді норма $\| \cdot \| \colon L \to \mathbb{R}$ -- неперервна.\\
\textit{Вказівка: оскільки $\rho(x,y) = \|x-y\|$, то звідси $\|x\| = \rho(x,0)$. Далі \prpref{continuity_of_metric_function_of_one_argument}.}
\end{proposition}

\begin{definition}
Задані $(X,\|\cdot\|_1)$ та $(X,\|\cdot\|_2)$ -- два нормовані простори.\\
Ці два нормовані простори називаються \textbf{ізометричними}, якщо
\begin{align*}
\exists A \colon X \to Y \text{ -- ізоморфізм між просторами}: \|Ax\|_2 = \|x\|_1
\end{align*}
\end{definition}

\begin{remark}
Ізоморфізм $L$ -- автоматично ізометрія, це випливає зі збережння норми. Саме тому слово ''ізометричні'' в означенні вище виправдане.
\end{remark}

\begin{remark}
У метричному просторі був критерій Кантора, який я переформулюю під нормований простір.\\
$(L,\|\cdot \|)$ -- банахів $\iff$ виконується умова Кантора (тобто будь-яка послідовність замкнених куль, що стягується, має непорожній перетин).\\
Так ось, в нормованому просторі не обов'язково вимагати умову $r_n \to 0$.
\end{remark}

\begin{definition}
Задано $(L, \|\cdot\|)$ -- нормований простір та $\sequence{x_n} \subset L$.\\
Вираз $S_k = \displaystyle\sum_{n=1}^k x_n$ називають \textbf{частковою сумою}. Припустимо, що послідовність часткових сум збігається -- тоді границю позначають за ряд $\displaystyle\sum_{n=1}^\infty x_n$, а сам ряд називають \textbf{збіжним}.
\end{definition}

\subsection{Коротко про топологічні векторні простори}
\begin{definition}
Векторний простір $E$ називається \textbf{топологічним}, якщо існує на ній така топологія, що
\begin{align*}
+ \colon E \times E \to E \text{ -- неперервна операція}; \\
\cdot \lambda \colon \mathbb{R} (\mathbb{C}) \times E \to E \text{ -- неперервна операція.}
\end{align*}
\noindent
На множині $\mathbb{R}, \mathbb{C}$ розглядається стандартна топологія.
\end{definition}
\noindent
Тимчасово позначу два відображення по-нормальному, маємо $\text{add} \colon E \times E \to E$ та $\text{scalar} \colon \mathbb{R} \times E \to E$.

\begin{theorem}
Задано $L$ -- нормований простір. Тоді $L$ -- топологічний векторний простір.
\end{theorem}

\begin{proof}
Оскільки $L$ -- нормований простір, то він метричний, а кожний метричний простір індукує топологію. Оберемо самю ту топологію $\tau_{\| \cdot \|}$ та доведемо, що на ній лінійні операції -- неперервні.\\
Оберемо будь-яку точку $(x,y) \in L \times L$ та покажемо неперервність операції $+$ на неї.\\
Нехай $B(x+y;r)$ -- окіл $x+y$. Хочемо довести, що існує окіл $U$ точки $(x,y)$, щоб виконувалось $+(U) \subset B(x+y;r)$. Маємо $z_1+z_2 \in +(U)$, тоді хочемо $z_1+z_2 \in B(x+y;r)$. Тобто потрібно $\|z_1 + z_2 - x - y\| \leq \|z_1 - x\| + \|z_2 - y\| < r$. Якщо розглядати точки $z_1 \in B\left(x; \dfrac{r}{2}\right)$ та $z_2 \in B\left(y; \dfrac{r}{2}\right)$, то отримаємо бажане. Можемо покласти $U = B\left(x; \dfrac{r}{2}\right) \times B\left(y; \dfrac{r}{2}\right)$.\\
Якщо взяти інший окіл $V$ точки $x+y$, то ми можемо охопити кулею $B(x+y;r) \supset V$.\\
Отже, $+$ -- неперервна операція.
\bigskip \\
Оберемо будь-яку точку $(\lambda,x) \in \mathbb{R} \times L$ та покажемо неперервність операції $\cdot \lambda$ на неї.\\
Нехай $B(\lambda x;r)$ -- окіл точки $\lambda x$. Хочемо довести, що існує окіл $U$ точки $(\lambda,x)$, щоб виконувалось $\cdot \lambda(U) \subset B(\lambda x;r)$. Маємо $\mu z \in \cdot \lambda(U)$, тоді хочемо $\mu z \in B(\lambda x;r)$. Тобто потрібно\\
$\| \lambda x - \mu z \| \leq |\lambda - \mu| \|x\| + |\mu| \|x-z\|$\\
(TODO: додумати)
\end{proof}

\subsection{Факторизація напівнорми}
Задано $L$ -- простір із напівнормою $\| \cdot \|$. Позначимо $M = \{x \in L: \|x\| = 0\}$.

\begin{lemma}
$M$ -- підпростір векторного простору $L$.
\end{lemma}

\noindent
Як було в лінійній алгебрі, встановимо відношення еквівалентності $x \sim y \iff x - y \in M$ на векторному просторі $L$. Ми вже знаємо, що $L/_M$ буде векторним простором, де задаються операції так:\\
$(x_1+M) + (x_2+M) = (x_1+x_2) + M$;\\
$\lambda(x+M) = \lambda x + M$.\\
Тепер уведемо функцію $\| \cdot\|_{L/_M}$ ось таким чином: $\| x+M \|_{L/_M} \overset{\text{def.}}{=} \|x\|_L$. Доведемо, що це буде задавати норму на $L/_M$.\\
Спочатку доведемо коректність означення. Дійсно, нехай $x+M = y+M$. Тоді звідси $x-y \in M$. Зауважимо, що\\
$| \|x \|_L - \|y\|_L | \leq \|x - y \|_L = 0 \implies \|x\|_L = \|y\|_L \implies \|x+M\|_{L/_M} = \|y+M\|_{L/_M}$.\\
Щодо властивостей номри. Це вже точно напівнома. Тобто залишилося довести, що $\|x+M\|_{L/_M} = 0 \iff x+M = M$.\\
$\|x+M\|_{L/_M} = 0 \implies \|x\| = 0 \implies x \in M \implies x+M=M$.

\begin{example}
Маємо простір $C^1([a,b])$ із напівнормою $\|x\| = \displaystyle\max_{t \in [a,b]} |x'(t)|$. Зауважимо, що лише функції $x(t) = const$ задовольняють умові $\|x\| = 0$. Тобто в нашому випадку підпростір $M = \{x \in C^1([a,b]) : \|x\| = 0\} = \{c \mid c \in \mathbb{R}\} = \mathbb{R}$. Тоді звідси маємо факторпростір $C^1([a,b])/_{\mathbb{R}}$ -- функції, що рівні з точністю до константи. 
\end{example}

\subsection{Обмежені та неперервні лінійні оператори}
\begin{definition}
Задано $(X, \|\cdot \|_X),(Y, \| \cdot \|_Y)$ -- нормовані простори.\\
Лінійний оператор $A \colon X \to Y$ називають \textbf{обмеженим}, якщо
\begin{align*}
\exists C > 0: \forall x \in X: \| Ax \|_Y \leq C \| x \|_X
\end{align*}
Надалі ми ці норми розрізняти не будемо, бо буде з контексту зрозуміло.
\end{definition}

\begin{remark}
Маємо обмежений оператор $A$. Зауважимо, що множина всіх констант, які обмежують оператор, тобто множина $\{C > 0 \mid \forall x \in X: \| Ax\| \leq C \|x\|\}$, буде непорожньою (бо оператор обмежений) та обмеженою знизу числом $0$. Значить, існує $\displaystyle\inf\{C > 0 \mid \forall x \in X: \| Ax\| \leq C \|x\|\}$.
\end{remark}

\begin{definition}
Задано $X,Y$ -- нормовані простори.\\
\textbf{Нормою} лінійного оператора $A$ називається величина
\begin{align*}
 \| A\| = \inf \{C > 0 \mid \forall x \in X: \| Ax \| \leq C \|x\|\}
\end{align*}
\end{definition}

\begin{remark}
Зауважимо, що для всіх $x \in X$ виконується $\|Ax\| \leq \|A\| \cdot \|x\|$.\\
Дійсно, для кожного $\varepsilon > 0$ існус стала $C_\varepsilon > 0$, для якої $C_\varepsilon < \|A\| + \varepsilon$. Тож для всіх $x \in X$ справедлива нерівність $\|A x\| \leq C_\varepsilon \|x\| < (\| A \| + \varepsilon) \|x\|$. Тому ця нерівність виконуватиметься також при $\varepsilon \to 0+0$. Таким чином, $\|A\| \in \{C > 0 \mid \forall x \in X: \|Ax\| \leq C \|x\|\}$, тобто інфімум досягається. \\
Отже, норма $\|A\|$ -- це найменше число, що обмежує лінійний оператор $A$.
\end{remark}

\begin{theorem}
Задано $X,Y$ -- нормовані простори та $A \colon X \to Y$ -- обмежений оператор. Тоді $\displaystyle\|A\| = \sup_{x \in X \setminus \{0\}} \dfrac{\| Ax\|}{\|x\|}$.
\end{theorem}

\begin{proof}
Спочатку доведемо, що $\|A\| = \displaystyle\sup_{x \in X \setminus \{0\}} \dfrac{\| Ax\|}{\|x\|}$. Уже відомо, що $\forall x \in X: \|Ax \| \leq \|A \| \|x\|$, тоді звідси $\forall x \in X \setminus \{0\}: \dfrac{\|Ax\|}{\|x\|} \leq \|A\|$, таким чином $\displaystyle\sup_{x \in X \setminus \{0\}} \dfrac{\|Ax\|}{\|x\|} \leq \|A\|$. Залишилося довести, що строга нерівність не допускається.\\
!Припустимо, що $\displaystyle\sup_{x \in X \setminus \{0\}} \dfrac{\|Ax\|}{\|x\|} < \|A\|$, тобто існує $\varepsilon > 0$, для якого $\displaystyle\sup_{x \in X \setminus \{0\}} \dfrac{\|Ax\|}{\|x\|} = \|A\|- \varepsilon$. Тоді звідси випливає, що $\forall x \in X \setminus \{0\}: \dfrac{\|Ax\| }{\|x\|} \leq \|A\| - \varepsilon \implies \forall x \in X: \|Ax\| \leq (\|A\|-\varepsilon) \|x\|$. Таким чином, $\|A\|-\varepsilon$ -- це константа, яка обмежує оператор, тоді за означенням норми, $\|A\| - \varepsilon \geq \|A\|$ -- суперечність!\\
Отже, ми довели рівність, тобто $\|A\| = \displaystyle\sup_{x \in X \setminus \{0\}} \dfrac{\| Ax\|}{\|x\|}$.
\end{proof}

\begin{theorem}
Задано $X,Y$ -- нормовані простори та $A \colon X \to Y$ -- обмежений оператор. Тоді $\displaystyle\|A\| = \sup_{\|x\| \leq 1} \|Ax\| = \sup_{\|x\| = 1} \|Ax\|$.
\end{theorem}

\begin{proof}
Ми доведемо ось такий ланцюг нерівностей: $\|A\| = \displaystyle\sup_{x \neq 0} \dfrac{\|Ax\|}{\|x\|} \geq \sup_{\|x\|\leq 1} \|Ax\| \geq \sup_{\|x\| = 1} \|Ax\| \geq \sup_{x \neq 0} \dfrac{\|Ax\|}{\|x\|}$.\\
Оберемо такий $x \neq 0$, щоб $\|x\| \leq 1$. Тоді виконується нерівність $\dfrac{\|Ax\|}{\|x\|} \geq \|A x\|$. Таким чином, $\displaystyle\sup_{\|x\| \leq 1} \|Ax\| \leq \sup_{\substack{\|x\| \leq 1 \\ x \neq 0 }} \dfrac{\|Ax\|}{\|x\|} \leq \sup_{x \neq 0} \dfrac{\|Ax\|}{\|x\|} = \|A\|$.\\
Зрозуміло, що виконується нерівність $\displaystyle\sup_{\|x\| = 1} \|Ax\| \leq \sup_{\|x\| \leq 1} \|Ax\|$.\\
Залишилося довести, що $\displaystyle\sup_{x \neq 0} \dfrac{\|Ax\|}{\|x\|} \leq \sup_{\|x\|=  1} \|Ax\|$. Дана нерівність є наслідком того, що для кожного $x \neq 0$ число $\dfrac{\|Ax\|}{\|x\|} = \left\| A \left( \dfrac{x}{\|x\|} \right) \right\|$ належить множині $\{ \|Ax\| \mid \|x\| = 1\}$.
\end{proof}

\begin{example}
Задано лінійний оператор $A \colon l_2 \to l_2$ таким чином: $A(x_1,x_2,\dots) = (x_2,x_3,\dots)$. Довести, що $A$ -- обмежений оператор та знайду норму.\\
Згадаємо, що норма $\|(x_1,x_2,\dots\| = \sqrt{|x_1|^2 + |x_2|^2 + \dots}$. Оцінимо оператор:\\
$\|A (x_1,x_2,\dots)\| = \|(x_2,x_3,\dots)\| = \sqrt{|x_2|^2 + |x_3|^2 + \dots} \leq \sqrt{|x_1|^2 + |x_2|^2 + |x_3|^2 + \dots} = 1 \cdot \|(x_1,x_2,\dots)\|$.\\
Отже, $A$ -- обмежений оператор, бо знайшли константу $C = 1$, що обмежує.\\
$\|A\| = \displaystyle\sup_{\|(x_1,x_2,\dots)\| = 1} \|A(x_1,x_2,\dots)\| = \displaystyle\sup_{\|(x_1,x_2,\dots)\| = 1} \|(x_2,x_3,\dots)\| = \sup_{\|(x_1,x_2,\dots)\| = 1} \sqrt{|x_2|^2 + |x_3|^2 + \dots} = \\
= \sup_{\|(x_1,x_2,\dots)\| = 1} \sqrt{1 - \|x_1\|^2} = 1$.
\end{example}

\begin{example}
Задано лінійний оператор $A \colon C([0,1]) \to C([0,1])$, таким чином: $(Ax)(t) = \displaystyle\int_0^t \tau x(\tau)\,d\tau$. Довести, що $A$ -- обмежений оператор та знайти норму.\\
Конкретно в цьому випадку розглядатиметься норма $\|f\| = \displaystyle\max_{t \in [0,1]} |f(t)|$.\\
$\displaystyle\|Ax\| = \max_{t \in [0,1]} \left| \int_0^t \tau x(\tau)\,d\tau \right| \leq \max_{t \in [0,1]} \int_0^t |\tau| |x(\tau)|\,d\tau = \int_0^1 |\tau| |x(\tau)|\,d\tau \leq \int_0^1 |\tau| \max_{\tau \in [0,1]} |x(\tau)|\,d\tau = \\
= \int_0^1 \tau \|x \|\,d\tau = \|x\| \dfrac{\tau^2}{2} \Big|_0^1 = \dfrac{1}{2} \|x\|$.\\
Отже, $A$ -- обмежений оператор. Залишилося знайти норму.\\
Оскільки $\|Ax\| \leq \dfrac{1}{2} \|x\|$, то звідси випливає $\|A\| = \displaystyle\sup_{\|x\| = 1} \|Ax\| \leq \dfrac{1}{2}$. Із іншого боку, оберемо функцію $x(t) = 1$, для якої $\|x\| = 1$. Тоді отримаємо, що $\|Ax\| = \displaystyle\max_{t \in [0,1]} \left|\int_0^\tau \tau\,d\tau\right| = \max_{t \in [0,1]} \dfrac{t^2}{2} = \dfrac{1}{2}$.\\
Таким чином, отримаємо $\|A\| = \dfrac{1}{2}$.
\end{example}

\begin{example}
Покажемо, що оператор $A \colon C^1([0,1]) \to C^1([0,1])$, що заданий як $(Af)(t) = f'(t)$, буде необмеженим.\\
Оберемо послідовність $f_n = \sin (2\pi n t)$, причому $\|f_n\| = \displaystyle\max_{t \in [0,1]} |\sin (2\pi n t)| = 1$. Тоді звідси\\
$\|Af_n\| = \displaystyle \| 2\pi n t \cos (2\pi n t)\| = 2 \pi n \| \cos (2\pi n t)\| = 2 \pi n \max_{t \in [0,1]} |\cos (2\pi n t)| = 2 \pi n \to +\infty$.
\end{example}

\begin{proposition}
Задано $X,Y$ -- нормовані простори та $\dim X < \infty$ та $A \colon X \to Y$ -- лінійний оператор. Тоді $A$ -- обмежений.\\
Внаслідок цього, всі оператори між скінченновимірними векторними просторами -- обмежені.
\end{proposition}

\begin{proof}
Дійсно, нехай $\{e_1,\dots,e_n\}$ -- базис $X$, нехай на неї стоїть норма $\|x\|_2$, тоді маємо наступне:\\
$\|Ax\| = \|A (x_1 e_1 + \dots + x_n e_n) \| = \| x_1 A e_1 + \dots + x_n A e_n \| \leq |x_1| \|A e_1 \| + \dots + |x_n| \|A e_n \| \leq \\
\leq \sqrt{|x_1|^2 + \dots + |x_n|^2} \sqrt{\|Ae_1\|^2 + \dots + \|Ae_n\|^2} = C \|x\|_2$.\\
Якби була би інша норма $\| \cdot \|$, то вона еквівалентна $\| \cdot \|_2$, а тому обмеженість зберігається.
\end{proof}

\begin{theorem}
Задано $X,Y$ -- нормовані простори та $A \colon X \to Y$ -- лінійний оператор.\\
$A$ -- обмежений $\iff A $ -- неперервний в точці $0$.
\end{theorem}

\begin{proof}
\rightproof Дано: $A$ -- обмежений. Оберемо послідовність $\{x_n\} \subset X$ так, щоб $x_n \to 0$. Звідси отримаємо $\|Ax_n - A0\| = \|Ax_n\| \leq \|A\| \|x_n\| \to 0$. Отже, $Ax_n \to A0$ при $n \to \infty$, що підтверджує неперервність. 
\bigskip \\
\leftproof Дано: $A$ -- неперервний в точці $0$.\\
!Припустимо, що $A$ -- необмежений оператор. Тоді для кожного $n \in \mathbb{N}$ існує точка $x_n \in X$, для якої $\|Ax_n\| > n \|x_n\|$ (ясно, що $x_n \neq 0$).  Таким чином, $\dfrac{\|Ax_n\|}{\|x_n\|} = \left\| A\left( \dfrac{x_n}{\|x_n\|} \right) \right\| > n$. Для зручності позначу $w_n = \dfrac{x_n}{\|x_n\|} \in X$, тобто ми вже маємо $\| Aw_n\| > n$. Оскільки відображення $A$ -- неперервне в нулі, то для послідовності $\left\{ \dfrac{1}{n}w_n, n \geq 1 \right\}$, для якої $\dfrac{1}{n} w_n \to 0$ виконується $A \dfrac{w_n}{n} \to A0 = 0$ -- суперечність в силу нерівності! Бо в нас $\left\| A \dfrac{w_n}{n} \right\| > 1$.
\end{proof}

\begin{remark}
Насправді, $A$ -- неперервний в точці $0 \iff A$ -- неперервний на $X$.\\
Сторона \leftproof зрозуміла. По стороні \rightproof маємо $x_0 \in X$ та припустимо, що $\{x_n\}$ -- довільна послідовність, де $x_n \to x_0$. Тоді цілком зрозуміло, що $x_n - x_0 \to 0$, але за неперервністю в нулі, маємо $A(x_n - x_0) = A x_n - A x_0 \to A0 = 0$. Таким чином, $A x_n \to A x_0$.
\end{remark}

\begin{theorem}
Множина $\mathcal{B}(X,Y)$ -- множина всіх обмежених лінійних операторів -- буде підпростором $\mathcal{L}(X,Y)$, а також буде нормованим простором із заданою нормою за означенням вище.
\end{theorem}

\begin{proof}
Дійсно, нехай $A,B \in \mathcal{B}(X,Y)$, тобто вони обмежені. Хочемо довести, що $A+B, \alpha A \in \mathcal{B}(X,Y)$, тобто вони теж обмежені. Дійсно, справедливі наступні оцінки:\\
$\| (A+B) x \| = \| Ax + Bx \| \leq \|Ax \| + \|Bx\| \leq \|A\| \|x\| + \|B\| \|x \| = (\|A\| + \|B\|) \|x\|$.\\
$\| (\alpha A) x \| = |\alpha| \|Ax\| \leq |\alpha| \|A\| \|x\|$.\\
Отже, дійсно $A+B, \alpha A \in \mathcal{B}(X,Y)$. Тепер доведемо, що вищезгадана норма лінійного обмеженого оператора -- дійсно норма.\\
$\|A\| \geq 0$ -- зрозуміло. Також якщо $\|A\| = 0$, то звідси $\|Ax\| \leq \|A\| \|x\| = 0$, тобто $Ax = 0$, причому для всіх $x \in X$; або $A = O$. Навпаки, якщо $A = O$, тобто $\|A\| = \displaystyle\sup_{\|x\| = 1} \|Ax\| = \sup_{\|x\|= 1} \{0\} = 0$.\\
Ми вже маємо оцінку $\| \alpha Ax\| \leq |\alpha| \|A\| \|x\|$ при всіх $x \in X$, тому й при всіх $x$ з умовою $\|x\| = 1$. Таким чином, $\|\alpha A\| = \displaystyle\sup_{\|x\|= 1} \|\alpha Ax\| \leq |\alpha|\|A\|$. Із цієї оцінки випливає, що $\|A\| = \|\alpha^{-1} \alpha A\| \leq |\alpha^{-1}| \|\alpha A\| \implies \|\alpha A\| \geq |\alpha| \|A\|$. Таким чином, $\|\alpha A\| = |\alpha| \|A\|$ (у тому числі при $\alpha = 0$).\\
Ми вже маємо оцінку $\| (A+B) x\| \leq (\|A\| + \|B\|) \|x\|$ при всіх $x \in X$, тому й при всіх $x$ з умовою $\|x\| = 1$. Таким чином, $\|A+B\| = \displaystyle\sup_{\|x\| = 1} \|(A+B)x\| \leq \|A\| + \|B\|$ -- третя властивість норми.
\end{proof}

\begin{theorem}
Простір $\mathcal{B}(X,Y)$ буде банаховим, якщо $Y$ -- банахів.
\end{theorem}

\begin{proof}
Нехай $\sequence{A_n} \subset \mathcal{B}(X,Y)$ -- фундаментальна послідовність.\\ Зауважимо, що $\sequence{A_n x} \subset Y$ -- фундаментальна також при всіх $x \in X$. Дійсно, із фундаментальності $\sequence{A_n}$ маємо, що $\forall \varepsilon > 0: \exists N: \forall n,m \geq N: \|A_n-A_m\| < \varepsilon$, але тоді $\forall x \in X: \|(A_n-A_m)x\| \leq \|A_n-A_m\| \|x\| < \varepsilon \|x\|$, звідси й випливає фундаментальність.\\
Тоді при кожному $x \in X$ існує $\displaystyle\lim_{n \to \infty} A_n x = z_x$. Ми можемо визначити як раз новий оператор $A \colon X \to Y$, де $Ax = z_x$ (границя єдина, тому визначення адекватне). Залишилися три етапи.\\
I. \textit{Лінійність}. \quad Дійсно, нехай $x,y \in X$ та $\alpha,\beta \in \mathbb{R}$, тоді маємо\\
$A(\alpha x + \beta y) = \displaystyle\lim_{n \to \infty} A_n(\alpha x + \beta y) = \lim_{n \to \infty} (\alpha A_nx + \beta A_n y) = \alpha \lim_{n \to \infty} A_n x + \beta \lim_{n \to \infty} A_n y = \alpha Ax + \beta Ay$.\\
II. \textit{Обмеженість}. \quad Оскільки $\sequence{A_n}$ -- фундаментальна, то $\sequence{A_n}$ -- обмежена, тобто $\exists C > 0: \forall n \geq 1: \|A_n\| \leq C$. Тоді в силу неперервності норми матимемо $\|Ax\| = \displaystyle\lim_{n \to \infty} \|A_nx\| \leq C \|x\|$.\\
III. \textit{$A_n \to A$}. \quad Згадаємо нерівність $\|(A_n-A_m)x\| < \varepsilon \|x\|$ при всіх $x \in X$, при всіх $\varepsilon > 0$ та $n,m \geq N$. Спрямуємо $m \to \infty$, тоді отримаємо $\|(A_n-A)x\| \leq \varepsilon \|x\|$, тому й $\|A_n-A\| \leq \varepsilon < 2\varepsilon$.
\end{proof}

\begin{proposition}
Нехай $A,B \in \mathcal{B}(X,Y)$. Тоді композиція $AB \in \mathcal{B}(X,Y)$.
\end{proposition}

\begin{proof}
Можна, звісно, було опиратися на композицію неперервних відображень. Але доведемо неперервність інакше.\\
$\|BA x\| \leq \|B\| \|Ax\| \leq \|B\| \|A\| \|x\|$.\\
Звідси, окрім обмеженості, ми довели оцінку $\|BA\| \leq \|B\| \cdot \|A\|$.
\end{proof}

\subsection{Продовження неперервних операторів}
Задані $X,Y$ -- нормовані простори, $X_0 \subset X$ та $A \in \mathcal{B}(X_0,Y)$. Питання полягає в тому, чи існує розширення $\tilde{A} \in \mathcal{B}(X,Y)$ оператора $A$, тобто $\tilde{A}|_{X_0} = A$. Причому нас буде цікавити таке розширення, що $\|\tilde{A}\| = \|A\|$.

\begin{remark}
Якщо таке розширення допустиме, то вже звідси $\|\tilde{A}\| \geq \|A\|$. Дійсно,\\
$\|\tilde{A}\| = \displaystyle\sup_{x \in X \setminus \{0\}} \dfrac{\|\tilde{A}x\|}{\|x\|} \geq \sup_{x \in X_0 \setminus \{0\}} \dfrac{\|\tilde{A}x\|}{\|x\|} = \sup_{x \in X_0 \setminus \{0\}} \dfrac{\|Ax\|}{\|x\|} = \|A\|$.
\end{remark}

\begin{proposition}
Задані $X,Y$ -- відповідно нормований та банахів простори та $X_0 \subset X$ -- щільний підпростір. Тоді для кожного $A \in \mathcal{B}(X_0,Y)$ існує єдиний розширений оператор $\tilde{A} \in \mathcal{B}(X,Y)$, тобто $\tilde{A}|_{X_0} = A$ та при цьому $\|\tilde{A}\| = \|A\|$.\\
\textit{Це твердежння описує так зване неперервне продовження оператора.}
\end{proposition}

\begin{proof}
Нехай є послідовність $\sequence{x_n} \subset X_0$, де $x_n \to x \in X$. Зауважимо, що тоді в цьому випадку $\sequence{Ax_n}$ -- фундаментальна. У силу банаховості $\sequence{Ax_n}$ буде збіжним. Тож визначимо оператор $\tilde{A}x = \displaystyle\lim_{n \to \infty} Ax_n$.\\
I. \textit{$\tilde{A}$ визначений коректно}.\\
Нехай є дві послідовності $\sequence{x_n},\sequence{y_n}$, для яких $x_n \to x,\ y_n \to x$. Значить, тоді\\
$\|Ax_n - Ay_n\| = \|A(x_n-y_n)\| \leq \|A\| \|x_n-y_n\| \to 0 \implies \displaystyle\lim_{n \to \infty} Ax_n = \lim_{n \to \infty} Ay_n$.
\bigskip \\
II. \textit{$\tilde{A}$ розширює оператор $A$}.\\
Справді, нехай $x \in X_0$. Оберемо стаціонарну послідовність $\sequence{x} \subset X_0$, де $x \to x$. Тоді $\tilde{A}x = \displaystyle\lim_{n \to \infty} Ax = Ax$. Отже, звідси $\tilde{A}|_{X_0} = A$.
\bigskip \\
III. \textit{$\tilde{A}$ лінійний оператор}.\\
Нехай $x,y \in E$ та $\alpha,\beta \in \mathbb{R}$. Тоді звідси\\
$A(\alpha x + \beta y) = \displaystyle\lim_{n \to \infty} A(\alpha x_n + \beta y_n) = \alpha \lim_{n \to \infty} Ax_n + \beta \lim_{n \to \infty} Ay_n = \alpha Ax + \beta Ay$.
\bigskip \\
\iffalse
IV. \textit{$\tilde{A}$ -- обмежений оператор}.\\
Оберемо послідовність $\sequence{x_n} \subset X$ так, що $x_n \to 0$. Тоді\\
$\|\tilde{A}x_n\| = \displaystyle \| \lim_{m \to \infty} A x_n^{(m)} \| = \lim_{m \to \infty} \|A x_n^{(m)}\| \leq \lim_{m \to \infty} \|A\| \|x_n^{(m)}\| = \|A\| \| \lim_{m \to \infty} x_n^{(m)} \| = \|A\| \|x_n\| \to 0$ при $n \to \infty \implies \tilde{A}x_n \to A0$.
\bigskip \\
\fi
IV. $\|\tilde{A}\| = \|A\|$.\\
Оберемо $X_0 \ni x_n \to x \in X$. Оскільки $A$ -- обмежений, то $\|Ax_n\| \leq \|A\| \|x_n\|$. Спрямовуючи $n \to \infty$, ми отримаємо $\|\tilde{A}x\| \le \|A\| \|x\|$. Автоматично довели, що $\tilde{A}$ -- обмежений оператор. Раз це виконується для всіх $x \in E$, то отримаємо $\|\tilde{A}\|= \displaystyle\sup_{\|x\| = 1} \|\tilde{A}x\| \leq \sup_{\|x\| = 1} \|A\| \|x\| = \|A\|$. Тобто звідси $\|\tilde{A}\| \leq \|A\|$. Зважаючи на зауваження вище, маємо $\|\tilde{A}\| = \|A\|$.
\bigskip \\
V. \textit{$\tilde{A}$ -- єдине розширення}.\\
!Припустимо, що існує інший оператор $\tilde{\tilde{A}}$, яке також є розширенням $A$ з усіма умовами, що задані в твердженні. Маємо $x \in X$, тож існує послідовність $\sequence{x_n} \subset X_0, x_n \to x$. Тоді\\
$\tilde{\tilde{A}}x \overset{\tilde{\tilde{A}} \text{ -- обмежений}}{=} \displaystyle\lim_{n \to \infty} \tilde{\tilde{A}}x_n = \lim_{n \to \infty} Ax_n \overset{\text{def. } \tilde{A}}{=} \tilde{A}x$. Суперечність!
\end{proof}

%спочатку запишу теорему для окремого випадку
\begin{theorem}[Теорема Гана-Банаха]
Задано $E$ -- нормований простір та $G \subset E$ -- підпростір. Тоді для кожного функціонала $l \in G'$ існує продовження $\tilde{l} \in E'$, тобто $\tilde{l}|_G = l$, причому $\| \tilde{l} \| = \|l\|$.
\end{theorem}

\begin{proof}
1. Обмежимось випадком, коли $E$ -- дісний та сепарабельний простір.\\
I. \textit{Доведемо, що $l$ можна продовжити на деякий підпростір $E \supset F \supsetneq G$.}\\
Нехай $G$ -- підпростір $E$ та $G \neq E$. Зафіксуємо $y \notin G$ та розглянемо підпростір $F = \linspan\{G \cup \{y\}\}$. Тобто кожний елемент $x \in F$ записується як $x = g + \lambda y$ при $g \in G, \lambda \in \mathbb{R}$. Визначимо оператор $\tilde{l}(x) = l(g) + \lambda c$, де $c = \tilde{l}(y)$. За побудовою, такий оператор -- лінійний.\\
Тепер залишилося підібрати таке $c \in \mathbb{R}$, щоб виконувалося $\|\tilde{l}\| = \|l\|$ -- тим самим ми й обмеженість доведемо автоматично. Але згідно зі зауваження, нам треба підібрати $c \in \mathbb{R}$, щоб $\|\tilde{l}\| \leq \|l\|$.\\
Обмежимося поки що $\lambda > 0$. Нехай зафіксовано $h_1,h_2 \in G$ та зауважимо, що справедлива нерівність:\\
$l(h_2) - l(h_1) = l(h_2-h_1) \leq |l(h_2-h_1)| \leq \|l\| \|h_2 - h_1\| = \|h \| \|(h_2+y) - (y+h_1)\| \leq \|l\| \|h_1+y\| + \|l\| \|h_2+y\|$.\\
Звідси випливає, що $-\|l\| \|h_1+y\| - l(h_1) \leq \|l\| \|h_2+y\| - l(h_2)$.\\
Оскільки це $\forall h_1,h_2 \in G$, то тоді $\displaystyle\sup_{h_1 \in G} (-\|l\| \|h_1+y\| - l(h_1)) \leq \inf_{h_2 \in G} ( \|l\| \|h_2+y\| - l(h_2))$.\\
Для зручності супремум позначу за $a_1$ та інфімум за $a_2$. Оберемо число $c \in \mathbb{R}$ так, щоб $a_1 \leq c \leq a_2$. Звідси справедлива така нерівність:\\
$\forall h \in G: -\|l\| \|h+y\| - l(h) \leq c \leq \|l\| \|h+y\| - l(h)$.\\
Тепер покладемо елемент $h = \lambda^{-1}g$ та домножимо обидві частини нерівності на $\lambda$. Оскільки ми домовилися $\lambda > 0$, то знаки нерівностей зберігаються. Коротше, отримаємо:\\
$-\|l\| \|g+\lambda y \| - l(g) \leq \lambda c \leq \|l\| \|g + \lambda y \| - l(g)$.\\
$-\|l\| \|g+\lambda y\| \leq l(g) + \lambda c \leq \|l\| \|g+\lambda y\|$.\\
$|\tilde{l}(x)| = |l(g) + \lambda c| \leq \|l\| \|g + \lambda y\| = \|l\| \|x\|$.\\
Власне, далі аналогічними міркуваннями (як в попередньому твердженні) отримаємо $\|\tilde{l}\| \leq \|l\|$.\\
Тепер що робити при $\lambda < 0$. Перепишемо $x = -(-g + (-\lambda)y)$. У нас тепер $-\lambda > 0$ та $-x = t = -g + (-\lambda)y$, звідси отримаємо\\
$|\tilde{l}(t)| \leq \|l\| \|t\| \implies |\tilde{l}(x)| \leq \|l\| \|x\|$.
\bigskip \\
II. \textit{Тепер доведемо, що продовежння на нашому конкретному $E$ існує}.\\
Оскільки $E$ -- сепарабельний, то існує (ми оберемо зліченну) множина $A = \{x_1,x_2,\dots\}$, яка є щільною підмножиною $E$. Також ми досі маємо $G \subset E$ -- підпростір.\\
Позначимо $x_{n_1} \in A$ -- перший з елементів, де $x_{n_1} \notin G$. За кроком І, існує $l_1$ -- продовження $l$ на $G_1 = \linspan\{G \cup \{x_{n_1}\}\}$.\\
Позначимо $x_{n_2} \in A$ -- перший з елементів, де $x_{n_2} \notin G_1$. За кроком І, існує $l_2$ -- продовження $l_1$ на $G_2 = \linspan\{G_1 \cup \{x_{n_2}\}\}$.\\
\vdots \\
Отримаємо ланцюг підпросторів $G \subset G_1 \subset G_2 \subset \dots$ та набір функціоналів $l_1,l_2,\dots$, для яких:\\
$\forall n \geq 1:$ \qquad $l_n \colon G_n \to \mathbb{R}$ -- обмежена; \qquad $l_n|_G = l$; \qquad $\|l_n\| = \|l\|$.\\
Покладемо множину $M = \displaystyle\bigcup_{n=1}^\infty G_n$, яка є лінійною. Визначимо функціонал $L_0 \colon M \to \mathbb{R}$ таким чином: $x \in M \implies x \in G_N \implies L_0(x) = l_N(x)$. Зрозуміло цілком, що $L_0$ -- лінійний, а також $\|L_0\| = \|l\|$.\\
Оскільки $M \supset A$ та $A$ всюди щільна, то $M$ -- всюди щільна. Отже, за попереднім твердженням, існує продовження $L \colon E \to \mathbb{R}$, для якого $\|L\| = \|L_0\| = \|l\|$.\\
Висновок: ми довели теорему Гана-Банаха для випадку, коли $E$ -- дійсний сепарабельний.
\bigskip \\
2. Тепер будемо доводити теорему для $E$ -- довільний дійсний нормований простір. Все ще $G \subset E$. Позначимо за $l_p$ -- продовження $l$  зі збереженням норми на множині $P \supset G$. Таке продовження існує див (1. та I.). Позначимо $X$ -- множина всіх таких продовжень. На ній введемо відношення $\preceq$ таким чином:\\
$l_p \preceq l_q \iff P \subset Q$ та $l_Q(x) = l_P(x), \forall x \in P$.\\
Зрозуміло, що $\preceq$ задає відношення порядку, внаслідок чого $X$ -- частково впорядкована. Зафіксуємо $Y = \{l_{P_\alpha} \mid \alpha \in A\}$ -- будь-яку лінійно впорядкувану підмножину $X$. Знайдемо верхню грань.\\
Для цього покладемо $P_* = \displaystyle\bigcup_{\alpha \in A} P_\alpha$ та на множині $P_*$ задамо функціонал $l_*$ таким чином:\\
$x \in P_* \implies x \in P_{\alpha_0} \implies l_*(x) = l_{\alpha_0}(x)$.\\
Зрозуміло, що $l_*$ -- лінійний, причому $\|l_*\| = \|l\|$. На множині $\bar{P_*}$ продовжимо функціонал, як було в твердженні -- отримаємо функціонал $l_{\bar{P_*}}$, причому $\|l_{\bar{P_*}}\| = \|l_*\| = \|l\|$. Даний функціонал $l_{\bar{P}_*}$ на $\bar{P}_*$ буде верхньою гранню $Y$. Отже, за лемою Цорна, існує максимальний елемент $X$. Це буде функціонал $L$, який визначений на $E$ (у протилежному випадку його можна було би ще продовжити та він не був би максимальним елементом).\\
Висновок: ми довели теорему Гана-Банаха для випадку, коли $E$ -- дійсний (не обов'язково сепарабельний) нормований простір.
\end{proof}
\noindent
Насправді, на цьому теорема Гана-Банаха ще не закінчена. Ми можемо її довести на випадок, коли нормований простір $E$ -- комплексний. Спершу кілька деталей.\\
Нехайй $E$ -- комплексний лінійний нормований простір. Розглянемо одночасно $E_{\mathbb{R}}$ -- асоційований з $E$ дійсний нормований простір; тобто під час множення на скаляр ми допускаємо лише дійсні коефіцієнти. Зауважимо, що $E_{\mathbb{R}} = E$ як множини, утім не як простори.\\
Розглянемо довільний функціонал $l \colon E \to \mathbb{C}$. Раз $l(x) \in \mathbb{C}$, то для кожного $x \in E$ можна записати функціонал як $l(x) = m(x) + i n(x)$. У цьому випадку $m(x) = \Re l(x),\ n(x) = \Im l(x)$.

\begin{proposition}
Нехай $l \colon E \to \mathbb{C}$ -- лінійний та обмежений функціонал. Тоді $m,n \colon E_{\mathbb{R}} \to \mathbb{R}$ задають лінійний обмежений функціонал.
\end{proposition}

\begin{proof}
Нехай $\alpha,\beta \in \mathbb{R}$ та $x,y \in E$. Тоді ми отримаємо наступне:\\
$l(\alpha x + \beta y) = m(\alpha x + \beta y) + i n(\alpha x + \beta y)$ (з одного боку)\\
$l(\alpha x + \beta y) = \alpha l(x) + \beta l(y) = \alpha (m(x) + i n(x)) + \beta (m(y) + i n(y)) = (\alpha m(x) + \beta m(y)) + i(\alpha n(x) + \beta n(y))$ (з іншого боку).\\
Знаючи, що комплексне число рівне тоді й лише тоді, коли дійсні та уявні частини збігаються, отримаємо\\
$m(\alpha x + \beta y) = \alpha m(x) + \beta m(y)$ \qquad $n(\alpha x + \beta y) = \alpha n(x) + \beta n(y)$.\\
Отже, $m,n$ -- лінійний функціонали.\\
Обмеженість $m$ (аналогічно з $n$) випливає з такго ланцюга нерівностей:\\
$|m(x)| \leq |m(x) + in(x)| = |l(x)| \leq \|l\| \|x\|$.
\end{proof}

\begin{proposition}
$n(x) = -m(ix)$.\\
Іншими словами, ми можемо функціонал $l$ відновити повністю, знаючи функціонал $m$.
\end{proposition}

\begin{proof}
$m(ix) + in(ix) = l(ix) = i l(x) = i(m(x) + in(x)) = -n(x) + im(x)$.\\
$\implies n(x) = -m(ix)$.\\
$l(x) = m(x) -im(ix)$.
\end{proof}

Повернімось назад до теореми Гана-Банаха. Доб'ємо її на випадок, коли $E$ -- комплексний нормований простір.
\begin{proof}
Маємо $E \supset G$ -- два комплексних простори та $E_{\mathbb{R}}, G_{\mathbb{R}}$ -- асоційовані простори. Маємо функціонал $l \colon G \to \mathbb{C}$, який визначається дійсним функціоналом $m \colon G_{\mathbb{R}} \to \mathbb{R}$. Оскільки це дійсний функціонал, ми можемо продовжити до $M \colon E_{\mathbb{R}} \to \mathbb{R}$ зі збереженням норми.\\
Покладемо $L(x) = M(x) - i M(ix)$. Неважко буде довести, що $L$ -- комплексний лінійний функціонал. Залишилося довести, що $\|L\| = \|l\|$. Знову ж таки, достатньо довести $\|L\| \leq \|l\|$. Запишемо $L(x) = |L(x)|e^{i \varphi}$, де $\varphi = \arg L(x)$. Тоді\\
$|L(x)| = e^{-i \varphi} L(x) = L(e^{-i \varphi} x) = M(e^{-i \varphi} x) = |M(e^{-i \varphi}x)| \leq \|M\| \|e^{-i \varphi} x\| = \|m\| \|x\| \leq \|l\| \|x\|$.\\
Отже, $\|L\|$. Ми тут юзали той факт, що $L(y) = M(y)$ при $L(y) \in \mathbb{R}$.
\end{proof}

\begin{remark}
Зауважимо, що якщо $G$ -- лінійна множина (але не підпростір), то теорема Гана-Банаха все одно виконується.\\
У цьому випадку $\bar{G}$ буде підпростором $E$. Функціонал $l$ продовжується неперервним чином на $\tilde{G}$, а далі застосовується доведена теорема.
\end{remark}


\subsection{Деякі наслідки з теореми Гана-Банаха}
\begin{corollary}
Нехай $E$ -- лінійний нормований простір та $G \subset E$ -- підпростір. Тоді для будь-якого вектора $y \notin G$ існує функціонал $l \in E'$, для якого $\|l\| = 1,\ l(y) = \rho(y,G),\ l|_G = 0$.\\
\textit{Цей наслідок про існування функціоналу, що поводиться як обчислення відстані від елемента $y$ до підпростору $G$. Ми можемо підібрати такий, щоб норма була 'нормальною'.}
\end{corollary}

\begin{proof}
На підпросторі $F = \linspan\{G \cup \{y\}\}$ визначимо функціонал $l_0$ таким чином:\\
$l_0(g + \lambda y) = \lambda \rho(y,G)$.\\
Цілком зрозуміло, що $l_0$ -- лінійний неперервний функціонал на $F$, також $l_0(y) = \rho(y,G)$, нарешті $l_0(g) = l_0(g+0y) = 0$. Обчислимо $\|l_0\|$.\\
$\|l_0\| = \displaystyle\sup \left\{ \dfrac{|l(g+\lambda y)|}{\|g+\lambda y\|} \mid g + \lambda y \in F \right\} = \sup\left\{ \dfrac{|\lambda| \rho(y,G)}{|\lambda| \cdot \|\lambda^{-1}g + y \|} \mid g+\lambda y \in F \right\} = \\ = \rho(y,G) \sup \{ \|g'-y\|^{-1} \mid g' \in G \} = \rho(y,G) \inf_{g' \in G} \|g'-y\| = 1$, де елемент $g' = \lambda^{-1}g \in G$.\\
За теоремою Банаха, існує продовження $l$ до $E$, причому $\|l\| = \|l_0\| = 1$.
\end{proof}

\begin{corollary}
Для кожного $y \in E \setminus \{0\}$ існує функціонал $l \in E'$, що $\|l\| = 1,\ l(y) = \|y\|$.\\
\textit{Цей наслідок про існування функціоналу, що поводиться як знаходження норми.}\\
\textit{Вказівка: $G = \{0\}$ до попереднього наслідку}.
\end{corollary}

\begin{corollary}
Лінійні непреревні функціонали розділяють точки нормованого простора $E$.\\
Іншими словами, $\forall x_1,x_2 \in E: x_1 \neq x_2: \exists l \in E': l(x_1) \neq l(x_2)$.\\
\textit{Вказівка: попередній наслідок, $y = x_1 - x_2 \neq 0$.}
\end{corollary}

\begin{definition}
Задано $E$ -- нормований простір.\\
Підмножина $M \subset E$ називається \textbf{тотальною}, якщо
\begin{align*}
\overline{\linspan{M}} = E
\end{align*}
Іншими словами, лінійна оболонка $\linspan M$ скрізь щільна.
\end{definition}

\begin{theorem}
Нехай $E$ -- нормований простір та $M \subset E$.\\
$M$ -- тотальна в $E \iff \forall l \in E': l|_M \equiv 0 \implies l|_E \equiv 0$.
\end{theorem}

\begin{proof}
\rightproof Дано: $M$ -- тотальна множина. Нехай $l \in E'$ такий, що $l|_M \equiv 0$. Оскільки функціонал лінійний, то $l|_{\linspan M} \equiv 0$. Оскільки $M$ -- тотальна, то $\linspan M \subset E$ буде щільною підмножиною, тож ми можемо неперервно продовжити $l$ до $E$. Отримаємо $l|_E \equiv 0$.
\bigskip \\
\leftproof Дано: $\forall l \in E': l|_M \equiv 0 \implies l|_E \equiv 0$.\\
!Припустимо, що $M$ не є тотальною. Тобто $\overline{\linspan{M}} \overset{\text{позн.}}{=} G \subsetneq E$, тобто існує вектор $y \in E \setminus G$. За першим наслідком, можна взяти функціонал $l \in E'$, що тіпа описує відстань, тобто $\|l\| = 1,\ l|_G = 0$. Але з того, що $l|_G \equiv 0 \implies l|_M \equiv 0$ випливає $l|_E \equiv 0$. Суперечність!
\end{proof}

\begin{proposition}
\label{kernel_as_subspace}
Нехай $E$ -- нормований простір та $l$ -- лінійний неперервний функціонал з $E$. Тоді $\ker l$ -- замкнений підпростір $E$. Навіть більше: $\ker l$ буде гіперпідпростором, тобто це означає, що $E = \linspan\{\ker l, y\}$ при $y \notin \ker l$.
\end{proposition}

\begin{proof}
Те, що $\ker l$ підпростір, тут все зрозуміло. Якщо взяти $\sequence{x_n} \subset \ker l$ таку, що $x_n \to x$, тоді маємо послідовність $\sequence{l(x_n) = 0}$ -- стаціонарна послідовність, при цьому $0 =l(x_n) \to l(x)$, тому $x \in \ker l$.\\
Нехай $y \notin \ker l$. Тоді доведемо, що кожний елемент $x \in E$ записується як $x = g + \lambda y$, де $g \in \ker l, \lambda \in \mathbb{R} (\mathbb{C})$. Покладемо $\lambda = \dfrac{l(x)}{l(y)}$ та розглянемо вектор $g = x - \lambda y$. Оскільки $l(g) = l(x) - \lambda l(y) = 0$, то звідси $g \in \ker l$. Отже, $x = g + \lambda y$ -- шукане представлення.
\end{proof}

\iffalse %don't need this
\begin{proposition}
Зафіксуємо лінійний функціонал $l$ на $E$. Покладемо множину $\Gamma_c = \{x \in E \mid l(x) = c\}$, що називається \textbf{гіперплощиною}. Позначимо $\Gamma_0 = \ker l$. Тоді існує такий вектор $z \in E$, що $\Gamma_c = \Gamma_0 + z \equiv \{g + z \mid g \in \Gamma_0 \}$.
\end{proposition}

\begin{proof}
Дійсно, зафіксуємо $z \in \Gamma_c$. Тоді для кожного $x \in \Gamma_c$ маємо $l(x-z) = l(x) - l(z) = 0$, тобто $g = x-z \in \Gamma_0 \implies x = g +z$.
\end{proof}

\begin{definition}
Нехай $E$ -- дійсний нормований простір та $A \subset E$, точка $x_0 \in \partial A$. Також нехай $l$ -- лінійний неперервний функціонал на $E$.\\
Гіперплощина $\Gamma_c$ називається \textbf{опорною гіперплощиною} множини $A$, що проходить через точку $x_0$, якщо
\begin{align*}
x_0 \in \Gamma_c \\
A \text{ лежить по одну сторону від гіперплощини } \Gamma_c \text{ (тобто $l(x) - c$ не міняє знак на $A$)}
\end{align*}
\end{definition}

\begin{theorem}
Зокрема маємо $A = B[r;0]$ -- замкнуту кулю, границя $\partial A = S_r(0)$.\\
Через будь-яку точку $x \in S_r(0)$ проходить опорна гіперплощина шара $B[r;0]$.
\end{theorem}

\begin{proof}
Для кожної точки $x_0 \in S_r(0)$ існує лінійний неперервний функціонал $l$ на $E$, де $\|l\| = 1,\ l(x_0) = \|x_0\| = r$. Тоді гіперплощина $\Gamma_r$ -- наша шукана. Дійсно, $x_0 \in \Gamma_r$, бо $l(x_0) = r$.\\
$\forall x \in B[r;0]: l(x) \leq |l(x)| \leq \|x\| \leq r$, тобто весь шар лежить по одну сторону від $\Gamma_r$.
\end{proof}
\fi

\subsection{Загальний вигляд лінійних неперервних функціоналів у деяких банахових просторах}
\subsubsection{Базис Шаудера}
\begin{definition}
Нехай $E$ -- банахів простір.\\
Послідовність $\{e_1,e_2,\dots\} \subset E$ називається \textbf{базисом Шаудера} простора $E$, якщо
\begin{align*}
\forall x \in E: \exists ! x_k \in \mathbb{R} (\mathbb{C}): x = \displaystyle\sum_{k=1}^\infty x_k e_k
\end{align*}
\end{definition}

\begin{proposition}
Нехай $E$ -- банахів простір, що містить базис Шаудера. Тоді $E$ -- сепарабельний.
\end{proposition}

\begin{proof}
\textit{Випадок дійсного нормованого простору}.\\
Оберемо множину $A = \left\{ \displaystyle\sum_{k=1}^\infty x_k e_k \mid x_k \in \mathbb{Q} \right\}$.\\
Нехай $x \in E$, тоді за умовою, $x = \displaystyle\sum_{k=1}^\infty x_k e_k$ єдиним чином. Нехай задане $\varepsilon > 0$. Тоді на кожному з $\left(x_k-\dfrac{\varepsilon}{\|e_k\| 2^k},x_k + \dfrac{\varepsilon}{\|e_k\| 2^k} \right)$ існує раціональне число $y_k \in \mathbb{Q}$. Оберемо $y \in A$ так, що $y = \displaystyle\sum_{k=1}^\infty y_k e_k$. Позначимо $x^{(n)},y^{(n)}$ за часткову суму ряда (перші $n$ додаються). Тоді\\
$\|x^{(n)} - y^{(n)}\| = \displaystyle\left\| \sum_{k=1}^n (x_k-y_k) e_k \right\| \leq \sum_{k=1}^n \| (x_k-y_k) e_k \| = \sum_{k=1}^n |x_k-y_k| \|e_k\| \leq \sum_{k=1}^n \dfrac{\varepsilon}{2^k}$.\\
Далі спрямовуємо $n \to \infty$. Тоді $x^{(n)} \to x, y^{(n)} \to y$. Після чого отримаємо $\|x-y\| \displaystyle \leq \sum_{k=1}^\infty \dfrac{\varepsilon}{2^k} = \varepsilon$. Отже, $A$ скрізь щільна множина, ну тобто $\bar{A} = E$.
\bigskip \\
\textit{Випадок комплексного нормованого простору}.\\
Оберемо множину $A = \left\{ \displaystyle\sum_{k=1}^\infty x_k e_k \mid x_k = \alpha_k + i \beta_k, \alpha_k, \beta_k \in \mathbb{Q} \right\}$. Далі плюс-мінус аналогічно.
\end{proof}

\begin{remark}
Якщо зробити \textcolor{red}{\href{https://projecteuclid.org/download/pdf_1/euclid.acta/1485889774}{*клік*}} сюди, то тут буде стаття про приклад сепарабельного банахового простору, який не містить базис Шаудера. Доведено П.\ Енфло. Власне, це означає, що зворотне твердження не працює.
\end{remark}

\begin{theorem}
Простір $l_p$ містить базис Шаудера. Причому цей базис матиме вигляд $\{e_1,e_2,e_3,\dots\}$, де кожний $e_i = (0,\dots,0,\underbrace{1}_{\text{на $i$-ій позиції}},0,\dots)$.
\end{theorem}

\begin{proof}
І. \textit{Існування}.\\
Фіксуємо елемент $x \in l_p$, де $x = (x_1,x_2,\dots)$ та $\displaystyle\sum_{k=1}^\infty |x_k|^p < +\infty$. Покладемо елемент (що є частковою сумою) $s_n = x_1 e_1 + \dots + x_n e_n$ та доведемо, що послідовність $\sequence{s_n}$ -- фундаментальна. При $n > m$\\
$\|s_n - s_m \|_p = \left\| (0,\dots,0,x_{m+1},\dots,x_n,0,\dots ) \right\|_p = \displaystyle \left( \sum_{k=m+1}^n |x_k|^p \right)^{\frac{1}{p}}$.\\
Фундаментальність $\sequence{s_n}$ випливає зі збіжності числового ряда $\displaystyle\sum_{k=1}^\infty |x_k|^p$. Оскільки $l_p$ -- банахів, то $\sequence{l_p}$ -- збіжний, тобто $\displaystyle\sum_{k=1}^\infty x_k e_k$ збігається до деякого елемента. Зокрема доведемо, що $\displaystyle\sum_{k=1}^\infty x_k e_k = x$.\\
$\|x-s_n\|_p = \displaystyle \left\| x - \sum_{k=1}^n x_k e_k \right\|_p = \| (0,\dots,0,x_{n+1},x_{n+2},\dots)\|_p = \left( \sum_{k=n+1}^\infty |x_k|^p \right)^{\frac{1}{p}}$.\\
Знову зі збіжності числового ряда $\displaystyle\sum_{k=1}^\infty |x_k|^p$ випливає бажане.
\bigskip \\
II. \textit{Єдиність}.\\
!Припустимо, що $x = \displaystyle\sum_{k=1}^\infty y_k e_k$ -- друге представлення. Тоді отримаємо $\displaystyle\lim_{n \to \infty} \sum_{k=1}^n (x_k-y_k)e_k = 0$. Звідси отримаємо $\displaystyle\lim_{n \to \infty}\left\| \sum_{k=1}^n (x_k-y_k) e_k \right\| = \lim_{n \to \infty} \left(\sum_{k=1}^n |x_k-y_k|^p\right)^{\frac{1}{p}} = 0$. Єдина можливість тут -- це $x_k = y_k$ при всіх $k \in \mathbb{N}$ -- суперечність!
\end{proof}

\subsubsection{Простір, що спряжений до $l_p, 1 < p < \infty$}
\begin{theorem}
Нехай $p,p' > 1$ таким чином, що $\dfrac{1}{p} + \dfrac{1}{p'} = 1$. Тоді $(l_p)' \cong l_{p'}$ ізометричним чином. Часто пишуть просто $(l_p)' = l_{p'}$.
\end{theorem}

\begin{lemma}
Для будь-якого $f \in (l_p)'$ існує елемент $(f_k)_{k=1}^\infty \in l_{p'}$, такий, що $f(x) = \displaystyle\sum_{k=1}^\infty f_k x_k$ для всіх $x \in l_p$.
\end{lemma}

\begin{proof}
Нехай $f \in (l_p)'$ (тобто лінійний неперервний функціонал). Тоді звідси отримаємо:\\
$f(x) = \displaystyle f\left( \sum_{k=1}^\infty x_k e_k \right) = f\left( \lim_{n \to \infty} \sum_{k=1}^n x_k e_k \right) = \lim_{n \to \infty} \sum_{k=1}^n x_k f(e_k) = \sum_{k=1}^\infty x_k f(e_k) \overset{f(e_k) \overset{\text{покл.}}{=} f_k}{=} \sum_{k=1}^\infty f_k x_k$.\\
Доведемо, що $(f_k)_{k=1}^\infty \in l_{p'}$. Для цього підберемо елемент $y \in l_p$ ось таким чином, щоб\\
$f(y) = \displaystyle\sum_{k=1}^\infty y_k f_k \overset{\text{був рівний}}{=} \sum_{k=1}^n |f_k|^{p'}$.\\
Можна для цього взяти елемент $y = \left( |f_1|^{p'-1}e^{-i \arg f_1}, \dots, |f_n|^{p'-1}e^{-i \arg f_n}, 0, 0, \dots \right)$. Оскільки $f$ обмежений, то звідси $\displaystyle |f(y)| \leq \|f\| \|y\| = \|f\| \left( \sum_{k=1}^n  ||f_k|^{p'-1} \cdot e^{-i \arg f_k}|^p \right)^{\frac{1}{p}} = \|f\| \left( \sum_{k=1}^n |f_k|^{p'} \right)^{\frac{1}{p}}$.\\
Маючи щойно отриману нерівність та рівність трошки вище, отримаємо\\
$\displaystyle\sum_{k=1}^n |f_k|^{p'} \leq \| f\| \left( \sum_{k=1}^n |f_k|^{p'} \right)^{\frac{1}{p}} \implies \left( \sum_{k=1}^n |f_k|^{p'} \right)^{\frac{1}{p'}} \leq \|f\|, \forall n \in \mathbb{N}$.\\
Остання оцінка стверджує, що ряд збіжний, внаслідок чого $(f_k)_{k=1}^{\infty} \in l_{p'}$.
\end{proof}

\begin{lemma}
Для кожного $(f_k)_{k=1}^\infty \in l_{p'}$ рівність $f(x) = \displaystyle\sum_{k=1}^\infty f_k x_k$ визначає лінійний та неперервний функціонал на $l_p$.
\end{lemma}

\begin{proof}
Нехай $(f_k)_{k=1}^\infty \in l_{p'}$. Завдяки нерівності Гьольдера, отримаємо:\\
$\displaystyle |f(x)| \leq \left( \sum_{k=1}^\infty |f_k|^{p'} \right)^{\frac{1}{p'}} \left( \sum_{k=1}^\infty |x_k|^p \right)^{\frac{1}{p}} = c \| x \|_p < +\infty$.\\
Отже, $f$ -- обмежений та $\|f\| \leq c$. Як доводиться лінійність, цілком зрозуміло.
\end{proof}
\noindent
Під час доведення першої леми ми отримали нерівність $c \leq \|f\|$. Маючи ще нерівність $c \geq \|f\|$ з другої леми, ми отримаємо $\|f\| = c$, тобто $\|f\| =  \displaystyle\left( \sum_{k=1}^\infty |f_k|^{p'} \right)^{\frac{1}{p'}}$. Ліворуч мається на увазі норма від $f \in (l_p)'$, а праворуч норма $(f_k) \in l_{p'}$.
\bigskip \\
Короче, маємо $A \colon l_{p'} \to (l_p)'$, який задається таким чином: $A (f_k)_{k=1}^\infty \color{red}(x) \color{black} = \displaystyle\sum_{k=1}^\infty x_k f_k$ (оператор $A$ повертає функціонал, тому я тут написав аргумент $\color{red}(x)$). Може, красивіше було б написати $A (f_k)_{k=1}^\infty = \displaystyle\sum_{k=1}^\infty \_ \cdot f_k$, ще не знаю. Він є ізоморфізмом, оскільки існує $A^{-1} \colon (l_p)' \to l_{p'}$, що задається як $A^{-1}l = (l(e_1),l(e_2),\dots)$. Усі два оператори коректно визначені за теоремами вище. Також ми довели, що $\|l\| = \| Af \| = \| f \|$.

\subsubsection{Простір, що спряжений до $l_1$}
\begin{theorem}
$(l_1)' \cong l_\infty$ ізометричним чином. Часто пишуть просто $(l_1)' = l_\infty$.
\end{theorem}

\begin{proof}
Нехай $(f_k)^{\infty}_{k=1} \in l_{\infty}$. Визначимо функціонал $f(x) = \displaystyle\sum_{k=1}^\infty f_k x_k$, який вже ясно, що лінійний. Залишилося довести обмеженість.\\
$|f(x)| = \displaystyle \left| \sum_{k=1}^\infty f_k x_k \right| \leq \sup_{k \in \mathbb{N}} |f_k| \sum_{k=1}^\infty |x_k| = \|f\|_\infty \|x\|_1$.\\
Із цього всього ми встановили $l_\infty \subset (l_1)'$.\\
Нехай $f \in (l_1)'$, тобто лінійний та обмежений функціонал. Аналогічним чином отримаємо, що $f(x) = \displaystyle\sum_{k=1}^\infty f_k x_k$, де $f_k = f(e_k)$. Тепер хочемо $(f_k)^{\infty}_{k=1} \in l_\infty$. Дійсно це спрацює, бо\\
$\|f\|_\infty = \displaystyle\sup_{k \in \mathbb{N}} |f_k| = \sup_{k \in \mathbb{N}} |f(e_k)| \leq \sup_{k \in \mathbb{N}} \|f\| \|e_k\|_1 = \|f\| \sup \{1,1,\dots\} = \|f\| < \infty$.\\
Причому ми також довели, що $\|f\| = \|f\|_\infty$.
\end{proof}

\subsubsection{Простори, що спряжені до $l_\infty$}
\begin{proposition}
$(l_\infty)' \supsetneq l_1$.
\end{proposition}
\noindent
Спочатку доведемо вкладення. Дійсно, нехай $(f_k)_{k=1}^\infty \in l_1$, тоді функціонал $f(x) = \displaystyle\sum_{k=1}^\infty f_k x_k,\ x \in l_\infty$ все одно лінійний, а обмеженість доводиться, завдяки оцінки\\
$|f(x)| \leq \displaystyle\sum_{k=1}^\infty |f_k x_k| \leq \sup_{k \in \mathbb{N}} |x_k| \sum_{k=1}^\infty |f_k| = \|x\|_\infty \sum_{k=1}^\infty |f_k|$.\\
Отже, довели вкладення, при цьому ми ще довели $\|f\| \leq \displaystyle\sum_{k=1}^\infty |f_k|$.\\
Якщо покласти такий $x \in l_\infty$, де $x_k = e^{-i \arg f_k}$, то взагалі отримаємо $f(x) = \displaystyle\sum_{k=1}^\infty |f_k| = \|f\|_1$.
\bigskip \\
Час з'ясувати, чому не допускається рівність. Розглянемо лінійну множину $C \subset l_\infty$, яка містить збіжні послідовності комплексних чисел. Визначимо $f(x) = \displaystyle\lim_{k \to \infty} x_k$ для кожного $x = (x_1,x_2,\dots) \in C$. Цілком ясно, що це лінійний функціонал. Обмеженість вилпиває з оцінки $|f(x)| = \displaystyle \left| \lim_{k \to \infty} x_k \right| \leq \sup_{k \in \mathbb{N}} |x_k| = \|x\|$.\\
Отже, $f \in C'$ (лінійний та неперервний функціонал), причому $\|f\| \leq 1$. Ми можемо продовжити функціонал $f$ до функціонала $F \in (l_\infty)'$ зі збереженням норми, за теоремою Гана-Банаха. Функціонал $F$ не можна записати як $F(x) = \displaystyle\sum_{k=1}^\infty x_k f_k$. Представимо, що можна. Маємо послідовність $x \in C$, ліміт не зміниться при зміні скінченного числа членів, тобто $F(x) = f(x)$ залишиться таким самим. Проте із іншого боку, зміниться $F(x) = \displaystyle\sum_{k=1}^\infty x_k f_k$.

\subsubsection{Простір, що спряжений до $L_p, 1 < p < \infty$.}
\begin{theorem}
Нехай $1 < p < \infty$ та $p' > 1$, причому $\dfrac{1}{p} + \dfrac{1}{p'} = 1$. Також задано $(X,\lambda,\mathcal{F})$ -- вимірний простір, де $\lambda$ -- $\sigma$-скінченна міра. Простір $(L_p)' \cong L_{p'}$ ізометричним чином. Ізоморфізм $l \colon (L_p)' \to L_{p'}$ задається наступним чином:\\
$l(x) = \displaystyle\int_X h(q) x(q)\,d\lambda(q)$.\\
\textit{Доведення див.\ в pdf теорії міри.}
\end{theorem}

\subsubsection{Простір, що спряжений до $C(K)$}
Припустимо, що $K$ -- метричний компакт та $\mathfrak{B}(K)$ -- борельова $\sigma$-алгебра.

\begin{definition}
Заряд $\omega$ на вимірній множині $(K, \mathfrak{B}(K))$ назвемо \textbf{регулярним}, якщо
\begin{align*}
\omega_+, \omega_- \text{ -- обидва регулярні}
\end{align*}
Позначення: $W(K)$ -- множина регулярних зарядів.
\end{definition}

\begin{remark}
$W(K)$ буде векторним простором. Також якщо покласти $\|\omega\| = |\omega|(K)$, де $|\omega|$ -- повна варіація заряда, то тоді ми отримаємо нормований простір. Причому $W(K)$ -- банахів додатково.
\end{remark}

\begin{theorem}[Теорема Маркова]
$(C(K))' \cong W(K)$ ізометричним чином. Ізоморфізм $l \colon (C(K))' \to W(K)$ задається таким чином:\\
$l(x) = \displaystyle\int_K x(q)\,d\omega(q)$.\\
\textit{Без доведення. Наведу частинний випадок даної теореми.}
\end{theorem}

\begin{theorem}[Теорема Ріса]
Для кожного функціонала $l \in (C([0,1]))'$ існує фукнція $g$ обмеженої варіації, для якої $l$ можна представити через інтеграл Рімана-Стілт'єса таким чином:\\
$l(x) = \displaystyle\int_0^1 x(t)\,dg(t)$, причому $V(g;[0,1]) = \|l\|$.
\end{theorem}

\begin{proof}
Нехай $l \in (C([0,1]))'$, визначений на підпросторі $M([0,1])$ -- простір обмежених функцій. За теоремою Гана-Банаха, продовжимо до функціонала $L \in (M([0,1])'$.\\
Нехай $t \in [0,1]$, позначимо $u_t = \mathbbm{1}_{[0,t]}$. Сама функція $u_t \in M([0,1])$, тому можна покласти $g(t) = L(u_t)$. Покажемо, що сама $g \in BV([0,1])$.\\
Оберемо розбиття $\pi$ відрізкі $[0,1]$ та позначимо $\varepsilon_i = \sign(g(t_{i}) - g(t_{i-1}))$. Тоді маємо:\\
$V_\pi(g,[0,1]) = \displaystyle\sum_{i=1}^{n} |g(t_i) - g(t_{i-1})| = \sum_{i=1}^n \varepsilon (Lu_{t_{i}} - Lu_{t_{i-1}}) = L \left(\sum_{i=1}^n \varepsilon_i (u_{t_i} - u_{t_{i-1}})\right) = L \left( \sum_{i=1}^n \varepsilon_i \mathbbm{1}_{(t_i,t_{i+1}]} \right)$.\\
Позначимо тимчасово за $\displaystyle z = \left( \sum_{i=1}^n \varepsilon_i \mathbbm{1}_{(t_i,t_{i+1}]} \right)$. Вона може приймати значення $0,1,-1$, тому звідси $\|z\| = 1$. Оскільки $L$ -- обмежений, то\\
$|V_\pi(g,[0,1])| = |Lz| \leq \|L\| \|z\| = \|l\|$.\\
Ця нерівність свідчить про те, що $g \in BV([0,1])$, причому автоматично $V(g,[0,1]) \leq \|l\|$.
\bigskip \\
Тепер доведемо рівність. Нехай $x \in C([0,1])$. Покладемо такі прості функції:\\
$x_n(t) = \displaystyle\sum_{k=1}^n x\left(\dfrac{k-1}{n}\right) \left( u_{\frac{k}{n}}(t) - u_{\frac{k-1}{n}}(t) \right) = \sum_{k=1}^n x\left( \dfrac{k-1}{n} \right) \mathbbm{1}_{\left(\frac{k-1}{n},\frac{k}{n}\right]}(t)$.\\
$L(x_n) = \displaystyle\sum_{k=1}^n x\left(\dfrac{k-1}{n}\right) \left( g\left(\dfrac{k}{n}\right) - g\left(\dfrac{k-1}{n}\right) \right)$.\\
Оскільки $x \in C([0,1])$, то $x \in \mathcal{RS}([0,1],g)$. Більш того, $(x_n)_{n=1}^\infty$ -- послідовність простих функцій, що збігається рівномірно до $x$. Отже, маючи це все, можна написати рівність:\\
$L(x) = \displaystyle\lim_{n \to \infty} L(x_n) = \int_0^1 x(t)\,dg(t)$.\\
Проте $x \in C([0,1])$, тож звідси $L(x) = l(x) = \displaystyle\int_0^1 x(t)\,dg(t)$.\\
Нарешті, $|l(x)| = \displaystyle \left| \int_0^1 x(t)\,dg(t) \right| \leq \left| \int_0^1 |x(t)|\,dg(t) \right| \leq \|x\| V(g,[0,1])$. Таким чином, $\|l\| \leq V(g,[0,1])$.\\
Знайшли варіацію $V(g,[0,1]) = \|l\|$.
\end{proof}

\begin{remark}
Теорема працює, якщо розглянути довільний відрізок $[a,b]$.
\end{remark}

\subsection{Вкладення нормованих просторів}
\begin{theorem}
Нехай $E$ -- лінійний нормований простір. Тоді $E \subset E''$, під другою множиною мається на увазі друге спряження, тобто $E'' = (E')'$. При цьому $\|x\|_E = \|x\|_{E''}$.
\end{theorem}

\begin{proof}
Для зручності елементи простору $E$ позначимо через $x,y,\dots$; елементи простору $E'$ -- через $l,m,\dots$; елементри простору $E''$ -- через $L,M,\dots$\\
Визначимо відображення $\varphi$ ось так: кожному $x \in E$ поставимо в відповідність $\varphi(x) = L_x \in E''$. При цьому ми покладемо $L_x(l) = l(x)$ при всіх $l \in E'$. \\
Доведемо, що $L_x$ -- лінійний та неперервний функціонал. Нехай $l,m \in E', \lambda,\mu \in \mathbb{K}$, тоді звідси $L_x(\lambda l + \mu m) = (\lambda l + \mu m)(x) = \lambda l(x) + \mu m(x) = \lambda L_x(l) + \mu L_x(m)$. Далі маємо\\
$|L_x(l)| = |l(x)| \leq \|l\| \cdot \|x\|$.\\
Отже, довели бажане, причому ми отримали оцінку $\|L_x\| \leq \|x\|$.\\
Тепер доведемо, що саме $\varphi$ -- лінійне відображення. Нехай $x,y \in E, \lambda, \mu \in \mathbb{K}$, тоді ми хочемо довести рівність $\varphi(\lambda x + \mu y) = \lambda \varphi(x) + \mu \varphi(y)$, або що теж саме $L_{\lambda x + \mu y} = \lambda L_x + \mu L_y$. Така рівність має виконуватися для кожного функціонала $l \in E'$. Дійсно,\\
$L_{\lambda x + \mu y}(l) = l(\lambda x + \mu y) = \lambda l(x) + \mu l(y) = \lambda L_x(l) + \mu L_y(l) = (\lambda L_x + \mu L_y)(l)$.\\
Доведемо, що $\varphi$ -- ін'єктивне відображення. !Припустимо, що $x \in \ker \varphi$ та $x \neq 0$. Тоді за наслідком теореми Гана-Банаха, існує функціонал $l \in E'$, для якого $\|l\| = 1,\ l(x) = \|x\|$. Звідси $L_x(l) = l(x) = \|x\| \neq 0$, тобто $L_x \neq 0$. Це означає лише, що $x \notin \ker \varphi$ -- суперечність!\\
Залишилося довести, що $\|x\| = \|L_x\|$. Точніше, залишилося $\|x\| \leq \|L_x\|$. При $x = 0$ все ясно. При $x \neq 0$, знову за наслідком Гана-Банаха, існує функціонал $l \in E'$, для якого $\|l\| = 1,\ l(x) = \|x\|$. Тоді\\
$\|x\| = l(x) = L_x(l) \leq \|L_x\| \|l\| = \|L_x\|$.\\
Отже, $\varphi \colon E \to E''$ -- лінійне та ін'єктивне відображення, що зберігає норму. Значить, $E$ ізометрично ізоморфний $\Im E \subset E''$. Отже, кожний елемент $x \in E$ можемо ототожнити з його елементом $L_x \in E''$. Звідси отримаємо вкладення $E \subset E''$ та рівність $\|x\|_E = \|x\|_{E''}$.
\end{proof}

\begin{definition}
Задано $E$ -- банахів простір.\\
Простір $E$ називають \textbf{рефлексивним}, якщо
\begin{align*}
E'' = \varphi(E),
\end{align*}
де $\varphi \colon E \to E''$, який задавали під час доведення теореми.
\end{definition}

\begin{example}
Зокрема рефлексивними будуть такі простори: $l_p$ та $L_p$ при $1 < p < \infty$.\\
Також скінченновимірний простір $E$ буде рефлексивним.
\end{example}

\begin{example}
Водночас нерефлексивними будуть такі простори: $l_1,\ l_\infty,\ L_1,\ L_\infty$ (останні два нерефлексивні при $\dim L_1 = \infty$, $\dim L_\infty = \infty$; $C(K)$ (буде нерефелексивним, якщо $K$ нескінченна множина).
\end{example}

\begin{theorem}[Теорема Банаха-Штайнгауза]
Задано $E$ -- банахів простір та $(l_n)_{n=1}^\infty$ -- послідовність функціоналів з $E'$. Припустимо, що $\forall x \in E: (l_n(x))_{n=1}^\infty$ -- обмежена послідовність. Тоді $(\|l_n\|)_{n=1}^\infty$ (послідовність норм) -- обмежена.\\
\textit{Дана теорема носить назву 'принцип рівномірної обмеженості'.}
\end{theorem}

\begin{proof}
Нехай $\forall x \in E: (l_n(x))_{n=1}^\infty$ -- обмежена послідовність. Доведемо, що існує замкнений шар $B[a;r]$, де множина $\{l_n(x), x \in B[a;r]\}_{n=1}^\infty$ обмежена.\\ 
!Припустимо навпаки, що множина $\{l_n(x)\}_{n=1}^\infty$ не обмежена в жодному замкненому кулі (як наслідок, в жодному відкритому кулі). \\
Візьмемо довільну відкриту кулю $B(x_0;r_0)$, де ось ця множина $\{l_n(x), x \in B(x_0;r_0)\}_{n=1}^\infty$ не обмежена. Це, що знайдуться $x_1 \in B(x_0;r_0)$ та $n_1 \in \mathbb{N}$, для яких $|l_{n_1}(x_1)| > 1$. Оскільки $l_{n_1}$ неперервний, то нерівність $|l_{n_1}(x)| > 1$ виконується в деякому околі $B(x_1;r_1)$ (?). За необхідністю, зменшимо радіус $r_1$ таким чином, щоб $B[x_1;r_1] \subset B(x_0;r_0)$, причому сам радіус $r_1 \overset{\text{зобов'язаний}}{\leq} \dfrac{r_0}{2}$ (дійсно, можна підібрати $r_1 = \dfrac{r_0 - \rho(x_0,x_1)}{2}$). \\
Ця множина $\{l_n(x), x \in B(x_1;x_1)\}$ теж не обмежена. Тоді знайдуться $x_2 \in B(x_1;x_1)$ та $n_2 > n_1$, для яких $|l_{n_2}(x_2)| > 2$. Аналогічно нерівність $|l_{n_2}(x)| > 2$ виконуватиметься в деякому замкненому шарі $B[x_2;r_2] \subset B(x_1;r_1)$, причому $r_2 \overset{\text{зобов'язаний}}{\leq} \dfrac{r_0}{2^2}$.\\
\vdots \\
Продовжуючи процес, отримаємо послідовність замкнених шарів $B[x_0;r_0] \supset B[x_1;r_1] \supset \dots$, причому $r_k \to 0$, числа $n_1 < n_2 < \dots$ такі, що $|l_{n_k}(x)| > k$ при $x \in B[x_k;r_k]$. За теоремою Кантора, існує точка $x^* = \displaystyle\lim_{k \to \infty} x_k$. Звідси випливає, що $|l_{n_k}(x^*)| > k$ при всіх $k$ -- суперечність! Бо послідовність $(l_{n_k}(x^*))_{k=1}^\infty$ мала б бути обмеженою за початковими умовами.\\
Висновок: існує шар $B[a;r]$, де множина $\{l_n(x), x \in B[a;r]\}$ обмежена. Тобто $\exists c' > 0: \forall x \in B[a;r], \forall n \in \mathbb{N}: |l_n(x)| \leq c'$. Досить буде довести, що множина $\{l_n(x), x \in B[0;1]\}$ обмежена. Для кожного $x \in B[0;1]$ покладемо $x' = rx + a$, тоді $x = \dfrac{1}{r}(x'-a)$. Оскільки $x' \in B[a;r]$, то $|l_n(x')| < c'$. Звідси\\
$|l_n(x)| = \left| l_n \left( \dfrac{1}{r}(x'-a) \right) \right| = \dfrac{1}{r} |l_n(x') - l_n(a)| \leq \dfrac{1}{r} (|l_n(x')| + |l_n(a)|) \leq \dfrac{c'+c_a}{r} = c$.\\
Висновок: $\exists c > 0: \forall x \in B[0;1], \forall n \geq 1: |l_n(x)| \leq c$. Проте умова $x \in B[0;1]$ означає, що $\|x\| \leq 1$. Тобто нерівність $|l_n(x)| \leq c$ для всіх $\|x\| \leq 1$. Зокрема звідси $\displaystyle\sup_{\|x\| \leq 1} |l_n(x)| = \|l_n\| \leq c$.
\end{proof}

\begin{remark}
Пояснення (?). Якби для кожного околу $B(x_1,r)$ (зокрема при $r = \dfrac{1}{n}$) існувала точка, де нерівність порушується, то ми би побудували послідовність, що прямує до $x_1$, при цьому ми би отримали $|l_{n_1}(x)| \leq 1$.
\end{remark}

\begin{remark}
У теоремі Банаха-Штайнгауза умова того, що $E$ -- банахів, -- суттєва.\\
Зокрема розглянемо простір $c_0$ -- послідовності, що збігаються до нуля. Далі розглянемо підпростір $c_{00} \subset c_0$ -- послідовності, де всі члени нулі, починаючи з деякого номера.
\end{remark}

\begin{theorem}[Теорема Банаха-Штайнгауза (для операторів)]
Задано $X,Y$ -- банахів та просто лінійній нормований простір, а також $(A_n)_{n=1}^\infty \in \mathcal{B}(X,Y)$. Припустимо, що $\forall x \in X: (A_n x)_{n=1}^\infty$ -- обмежена послідовність. Тоді $(\|A_n\|)_{n=1}^\infty$ (послідовність норм) -- обмежена.\\
\textit{Доведеня теореми повністю повторюється.}
\end{theorem}

\subsection{Про види збіжностей}
Ми вже знаємо один тип збіжностей. Переформулюю ще раз означення, але доповню це одним словом в дужках.
\begin{definition}
Задано $E$ -- лінійний нормований простір.\\
Послідовність $(x_n)_{n=1}^\infty$ називається \textbf{(сильно) збіжною} до $x \in E$, якщо
\begin{align*}
\lim_{n \to \infty} \|x - x_n\| = 0
\end{align*}
Позначення: $x_n \to x$.\\
Тобто сильна збіжність -- це збіжність за нормою.
\end{definition}

\begin{definition}
Нехай $E$ -- лінійний нормований простір.\\
Послідовність $(x_n)_{n=1}^\infty$ називається \textbf{слабко збіжною} до $x \in E$, якщо
\begin{align*}
\forall l \in E': l(x_n) \to l(x)
\end{align*}
Позначення: $x_n \toweak x$.
\end{definition}
\noindent

Якщо розглянути спряжений простір $E'$, то крім сильної та слабкої збіжності існує ще один тип.

\begin{definition}
Нехай $E$ -- лінійний нормований простір.\\
Послідовність функціоналів $(l_n)_{n=1}^\infty \subset E'$ називається \textbf{слабко* збіжною} до $l \in E'$, якщо
\begin{align*}
\forall x \in E: l_n(x) \to l(x)
\end{align*}
Позначення: $l_n \toweakstar l$.
\end{definition}

\begin{proposition}
Задано $E$ -- лінійний нормований простір та послідовність $(x_n)_{n=1}^\infty$. Тоді:\\
$x_n \to x \implies x_n \toweak x$.
\end{proposition}

\begin{proof}
Дійсно, нехай $x_n \to x$, тобто звідси $\|x-x_n\| \to 0$ при $n \to \infty$. Маючи це, отримаємо $\forall l \in E'$:\\
$|l(x_n)-l(x)| = |l(x_n-x)| \leq \|l\| \|x_n - x\| \to 0$. Таким чином, $x_n \toweak x$.
\end{proof}

\begin{proposition}
Задано $E$ -- лінійний нормований простір та послідовність $(l_n)_{n=1}^\infty \subset E'$. Тоді:\\
$l_n \to l \implies l_n \toweak l \implies l_n \toweakstar l$.
\end{proposition}

\begin{proof}
Імплікація $l_n \to l \implies l_n \toweak l$ була доведена вище. Залишилося $l_n \toweak l \implies l_n \toweakstar l$.\\
Нехай $l_n \toweak l$, тобто $\forall L \in E'': L(l_n) \to L(l)$. Зафіксуємо елемент $x \in E$. Ми вже доводили, що $E \subset E''$, тобто $x \in E''$, де в цьому випадку $x = L_x$ такий, що $L_x(l) = l(x)$. Звідси\\
$l_n(x) = L_x(l_n) \to L_x(l) = l(x)$. Звідси випливає, що $l_n \toweakstar l$.
\end{proof}

\begin{example}
Зараз покажемо, чому в зворотний бік не працює.
\bigskip \\
$x_n \toweak x \centernot\implies x_n \to x$.\\
Розглянемо простір $l_p$ та зафіксуємо послідовність $(e_n)_{n=1}^\infty$, де кожний $e_j$ -- елемент базиса Шаудера. Спочатку покажемо, що $(e_n)_{n=1}^\infty$ слабко збігається. Зафіксуємо довільний функціонал $l \in (l_p)' = l_{p'}$, тобто $l = (l_1,l_2,\dots)$. Це означає, що $\displaystyle\sum_{j=1}^\infty |l_j|^{p'} < + \infty$, а тому за необіхдною умовою, $|l_j|^{p'} \to 0 \implies l_j \to 0$. Із іншого боку, ми вже знаємо, що $l_j = l(e_j) \to 0 = l(0)$ при $j \to \infty$. Це як раз свідчить про те, що $e_j \toweak 0$.\\
Проте зауважимо, що $\|e_j - 0 \| = \|e_j\| = 1 \centernot\to 0$. Це як раз означає, що $e_j \centernot\to 0$.
\bigskip \\
$f_n \toweakstar f \centernot\implies f_n \toweak f$.\\
Розглянемо простір $l_1$ та зафіксуємо послідовність $(f_n)_{n=1}^\infty$, $f_n = ?$. Спочатку покажемо, що $(f_n)_{n=1}^\infty$ слабко* збігається. Зафіксуємо довільний елемент $x \in c_0$. Значить, $x_j \to 0$ при $j \to \infty$. Оберемо $f_n = (-1)^n$. Тоді звідси $f_n(x) = \displaystyle\sum_{k=1}^\infty x_k f_n^k(e_k)$. (TODO: не можу добити)
\end{example}

\begin{proposition}
Утім якщо $E$ -- рефлексивний лінійний нормований простір та $(l_n)_{n=1}^\infty \subset E'$, тоді\\
$l_n \toweak l \iff l_n \toweakstar l$.
\end{proposition}

\begin{remark}
Границя єдина за слабкою* збіжністю, слабкою збіжністю та сильною збіжністю.
\end{remark}

\begin{proposition}
Задано $E$ -- банахів та послідовність $(l_n)_{n=1}^\infty$, яка слабко* збігається. Тоді $(l_n)_{n=1}^\infty$ -- обмежена.
\end{proposition}

\begin{proof}
Дійсно, маємо $\forall x \in E: l_n(x) \to l(x)$, тобто числова послідовність $(l_n(x))_{n=1}^\infty$ збігається, тоді обмежена. Значить, за теоремою Банаха-Штайнгауза, послідовність $(\|l_n\|)_{n=1}^\infty$ обмежена.
\end{proof}

\begin{theorem}
Задано $E$ -- банахів простір та $(l_n)_{n=1}^\infty \subset E'$ -- така послідовність, що $\forall x \in E: (l_n(x))_{n=1}^\infty$ -- фундаментальна. Тоді $\exists l \in E': l_n \toweakstar l$.
\end{theorem}

\begin{proof}
Оскільки $\forall x \in E: (l_n(x))_{n=1}^\infty$ фундаментальна, то (як числова послідовність) вона збіжна. Визначимо функціонал $l(x) = \displaystyle\lim_{n \to \infty} l_n(x)$. Зважаючи на той факт, що $l_n$ -- лінійний, то $l$ -- лінійний в силу граничного переходу. Залишилося довести обмеженість.\\
При кожному $x \in E$ послідовність $(l_n(x))_{n=1}^\infty$ (вже з'ясували) збіжна, тож обмежена. Але за теоремою Банаха-Штайнгауза, $\exists c > 0: \forall n \geq 1: \|l_n\| \leq c$. Значить, $\forall n \geq 1, \forall x \in E: |l_n(x)| \leq \|l\| \|x\| \leq c \|x\|$. Знову переходячи до границі, отримаємо $|l(x)| \leq c \|x\|$.\\
Отже, $\forall x \in E: l_n(x) \to l(x) \implies l_n \toweakstar l$. 
\end{proof}

\begin{theorem}[Критерій слабкої* збіжності]
Задано $E$ -- банахів та множина $M$ -- скрізь щільна в $E$. Нехай $(l_n)_{n=1}^\infty \subset E'$.\\
$l_n \toweakstar l \iff \begin{cases} \forall x \in M: l_n(x) \to l(x) \\ \exists c > 0: \forall n \geq 1: \|l_n\| \leq c \end{cases}$.
\end{theorem}

\begin{proof}
\rightproof Дано: $l_n \toweakstar l$. Тобто $\forall x \in E: l_n(x) \to l(x)$, зокрема $\forall x \in M$. Обмеженість норм $\|l_n\|$ автоматично виконується.
\bigskip \\
\leftproof Дано: ці дві умови. Ми хочемо $\forall y \in E: l_n(y) \to l(y)$.\\
При $x \in M$ маємо наступне:\\
$|l_n(y) - l(y)| \leq |l_n(y) - l_n(x)| + |l_n(x) - l(x)| + |l(x) - l(y)| \leq \| l_n \| \| y - x \| + |l_n(x) - l(x)| + \|l\| \|x-y\| \leq \\
\leq (c + \|l\|) \|x-y\| + |l_n(x) - l(x)|$.\\
Проте $\Cl(M) = E$, тож звідси $\forall y \in E: \forall \varepsilon > 0: \exists x \in M: \|x-y\| < \dfrac{\varepsilon}{2(c+ \|l\|)}$. В силу першої умови, $\exists N: \forall n > N: |l_n(x) - l(x)| < \dfrac{\varepsilon}{2}$.\\
Значить, $|l_n(y) - l(y)| < \varepsilon$.
\end{proof}

\begin{remark}
Судячи з доведення, в \leftproof не обов'язково вимагати бути $E$ повним. Також в формулюванні теореми досить вимагати, щоб $M$ була тотальною в $E$.
\end{remark}

\iffalse
\begin{example}
На просторі $L_2([-1,1])$ визначимо послідовність $l_n(x) = \displaystyle\int_{-1}^1 x(t) \cos (\pi n t)\,dt$. Це справді лінійні та обмежені функціонали, але обмеженість покажу окремо, бо нам ще знадобиться.\\
$|l_n(x)| = \displaystyle\left| \int_{-1}^1 x(t)\,\cos(\pi n t)\,dt \right| \leq \int_{-1}^1 |x(t)| |\cos (\pi n t)|\,dt \leq \left( \int_{-1}^1 |x(t)|^2\,dt \right)^{\frac{1}{2}} \left( \int_{-1}^1 \cos^2 (\pi n t)\,dt \right)^{\frac{1}{2}} \overset{\substack{\text{другий інтеграл} \\ \text{самостійно}}}{=} \\
= \|x\|_2 \left( 1 + \dfrac{\sin 2\pi n}{2\pi n} \right)^{\frac{1}{2}}$.
\bigskip \\
Доведемо зараз, що $l_n \toweakstar 0$. Для цього ми скористаємося критерієм слабкої* збіжності. Ми знаємо, що простір простих функцій вигляду $\displaystyle\sum_{i=1}^k a_i \mathbbm{1}_{(\alpha_i,\beta_i] \cap [-1,1]}$ -- щільна. Візьмемо таку довільну просту функцію. Звідси\\
$l_n(x) = \displaystyle\int_{-1}^1 x(t)\cos(\pi nt)\,dt = \int_{-1}^1 \sum_{i=1}^k a_i \mathbbm{1}_{(\alpha_i,\beta_i] \cap [-1,1]}(t) \cos (\pi nt)\,dt = \sum_{i=1}^k a_i \int_{\alpha_i}^{\beta_i} \cos (\pi nt)\,dt = \\
= \sum_{i=1}^k a_i \dfrac{\sin (\pi n t)}{\pi n} \Big|_{\alpha_i}^{\beta_i} \to 0$ при $n \to \infty$.\\
Ба більше, ми маємо $\|l_n\| \leq \left( 1 + \dfrac{\sin 2 \pi n}{2\pi n} \right)^{\frac{1}{2}} \leq C$, оскільки центральна штука збіжна, тому обмежена. Внаслідок критерію, ми доведемо, що $l_n \toweakstar 0$.
\bigskip \\
Щоправда, $l_n \not\to 0$ (в сенсі норми), оскільки $\|$
\end{example}
\fi

\begin{example}
На просторі $C([0,1])$ визначимо послідовність $l_n(x) = n \displaystyle\int_0^{\frac{1}{n}} x(t)\,dt$. Доведемо, що $l_n$ слабо* збіжний.\\
Розглянемо спочатку простір многочленів на $[0,1]$ -- щільна підмножина $C([0,1])$. Оберемо $x(t) = a_n t^n + \dots + a_1 t + a_0$. Тоді\\
$l_n(x) = n\displaystyle\int_0^{\frac{1}{n}} a_n t^n + \dots + a_1 t + a_0\,dt = n \left( \dfrac{a_n}{n+1}t^{n+1} + \dfrac{a_{n-1}}{n}t^n + \dots + a_0 t \right)\Big|_{0}^{\frac{1}{n}} \to a_0 = x(0)$.\\
Позначимо новий функціонал $l(x) = x(0)$, який справді лінійний та обмежений. Тоді $l_n(x) \to l(x)$.\\
При цьому самі норми будуть обмеженими. Справді,\\
$|l_n(x)| = n \displaystyle\left|\int_0^{\frac{1}{n}} x(t)\,dt\right| \leq n \int_0^{\frac{1}{n}} |x(t)|\,dt \leq \|x\| \implies \|l_n\| \leq 1$.\\
Отже, за критерієм, $l_n \toweakstar l$.
\end{example}

\subsection{Про види збіжностей в операторах}
\begin{definition}
Задані $X,Y$ -- лінійні нормовані простори.\\
Послідовність операторів $(A_n)_{n=1}^\infty$ називається \textbf{рівномірно збіжною} до $A$, якщо
\begin{align*}
\lim_{n \to \infty}\|A_n - A\| = 0
\end{align*}
Позначення: $A_n \substack{\to \\ \to} A$ при $n \to \infty$.
\end{definition}

\begin{definition}
Задані $X,Y$ -- лінійні нормовані простори.\\
Послідовність операторів $(A_n)_{n=1}^\infty$ називається \textbf{сильно збіжною} до $A$, якщо
\begin{align*}
\forall x \in X: \lim_{n \to \infty} A_n x = A x
\end{align*}
Позначення: $A_n \tostrong A$ при $n \to \infty$.
\end{definition}

\begin{definition}
Задані $X,Y$ -- лінійні нормовані простори.\\
Послідовність операторів $(A_n)_{n=1}^\infty$ називається \textbf{слабко збіжною} до $A$, якщо
\begin{align*}
\forall x \in X: A_n x \toweak A x
\end{align*}
Позначення: $A_n \toweak A$ при $n \to \infty$.
\end{definition}

\begin{proposition}
Задані $X,Y$ -- лінійні нормовані простори та послідовність $(A_n)_{n=1}^\infty \subset \mathcal{B}(X,Y)$. Тоді $A_n \substack{\to \\ \to} A \implies A_n \tostrong A \implies A_n \toweak A$.
\end{proposition}

\begin{proof}
Нехай $A_n \substack{\to \\ \to} A$. Зафіксуємо $x \in X$. Тоді\\
$\| A_n x - Ax \| = \| (A_n-A) x\| \leq \|A_n - A\| \|x\| \to 0$ при $n \to \infty$.\\
Значить, ми отримали $A_n \tostrong A$. Ще раз зафіксуємо $x \in X$. Тоді $(A_n x)_{n=1}^\infty$ збігається сильно до $Ax \in Y$. Із сильної збіжності випливає слабка збіжність (минулий розділ), тож $A_n x\toweak A x$. Отже, $A_n \toweak A$.
\end{proof}

\begin{example}
Зараз покажемо, чому в зворотні боки не працюють.
\bigskip \\
$A_n \toweak A \centernot\implies A_n \tostrong A$.\\
Перш за все ознайомимось з оператором $s \colon l_2 \to l_2$ -- оператор зсуву, який працює таким чином: $(x_1,x_2,x_3,\dots) \overset{s}{\mapsto} (0,x_1,x_2,\dots)$. Оператор $s \in \mathcal{B}(l_2,l_2)$ (детально це доводити не буду, бо в цілому ясно). Тепер розглянемо послідовність операторів $(A_n)_{n=1}^\infty$, що задані як $A_n = s^n$, тобто певна кількість зсуву. Зауважимо, що $A_n \toweak O$. \\
Дійсно, нехай $l \in (l_2)'$. Тоді звідси $l(x) = \displaystyle\sum_{k=1}^\infty l_k x_k$ для деякого елементу $(l_k)_{k=1}^\infty \in l_2$. Тоді зауважимо: \\
$\displaystyle |l(A_n x)| = \left| \sum_{k=1}^\infty l_{n+k} x_k \right| \leq \sum_{k=1}^\infty |l_{n+k} x_k| \leq \left( \sum_{k=1}^\infty |l_{n+k}|^2 \right)^{\frac{1}{2}} \left( \sum_{k=1}^\infty |x_k|^2 \right)^{\frac{1}{2}} = \left( \sum_{m=n+1}^\infty |l_m| \right)^{\frac{1}{2}} \left( \sum_{k=1}^\infty |x_k|^2 \right)^{\frac{1}{2}}$.\\
Тоді звідси $l(A_n x) \to 0 = l(0) = l(O x)$ через хвіст ряду. Отже, звідси $A_n \toweak O$.\\
При цьому зауважимо, що $A_n \centernot\tostrong O$. Дійсно,\\
$\|A_n x\| = \|(\underbrace{0,\dots,0}_{\text{$n$ штук}},x_1,x_2,\dots)\| = \displaystyle\sum_{k=1}^\infty |x_k|^2$, якщо взяти $x = e_{p}, p > n$, то отримаємо $\|A_n x\| = 1$.
\bigskip \\
$A_n \tostrong A \centernot\implies A_n \substack{\to \\ \to} A$.\\
Розглянемо послідовність операторів $(A_n)_{n=1}^\infty \subset \mathcal{B}(l_2,l_2)$, де кожний $A_n x = (x_1,\dots,x_n,0,0,\dots)$. Зауважимо, що $A_n \tostrong I$ (тут $I$ -- одиничний оператор).\\
Дійсно, $\|A_n x - I x\| = \|(0,\dots,0,x_{n+1},x_{n+2},\dots)\| = \displaystyle\sum_{k=1}^\infty |x_{n+k}|^2 \to 0$.\\
При цьому зауважимо, що $A_n \not\substack{\to \\ \to} I$. Дійсно, ми маємо наступне:\\
$\|A_n-I\| = \displaystyle\sup_{\|x\|=1} \|(A_n-I)x\| \geq \|(A_n-I)e_j\| = \|e_j\| = 1$ (при $j > n$).
\end{example}

\begin{theorem}
Задані $X,Y$ -- банахові простори та $(A_n)_{n=1}^\infty \subset \mathcal{B}(X,Y)$ -- така послідовність, що $\forall x \in X: (A_n x)_{n=1}^\infty$ -- фундаментальна. Тоді $\exists A \in \mathcal{B}(X,Y): A_n \tostrong A$.\\
\textit{Було шось схоже раніше. Доведення повторюється.}
\end{theorem}

%TODO розібратися
\iffalse
\begin{theorem}
Розглянемо простір $C([0,1])$.\\
$f_n \toweak f \iff \begin{cases} \forall t \in [0,1]: f_n(t) \to f(t) \\ \|f_n\| \leq M,\ M > 0 \end{cases}$.
\end{theorem}

\begin{proof}
\rightproof Дано: $f_n \toweak f$. Розглянемо будь-яку точку $t_0 \in [0,1]$, а згодом оберемо функцію $g(t) = \begin{cases} 0, & x \leq t_0 \\ 1, & x > t_0 \end{cases}$. Тоді зауважимо, що $\displaystyle\int_0^1 f_n(t) - f(t)\,dg(t) = f_n(t) - f(t_0)$ (обчислення інтеграла Стілтьєса). Такому інтегралу відповідає функціонал $l \in (C([0,1]))'$. За умовою, $l(f_n) \to l(f)$, тобто $|l(f_n) - l(f)| = \displaystyle\left| \int_0^1 f_n(t) - f(t)\,dg(t)  \right| = |f_n(t_0) - f(t_0)| \to 0 \implies f_n(t_0) \to f(t_0)$.ґґ
Оскільки $($
\end{proof}
\fi

\subsection{Обернені оператори}
\begin{definition}
Бієктивний оператор $A \in \mathcal{B}(X,Y)$ називається \textbf{оборотним}, якщо
\begin{align*}
\exists A^{-1} \in \mathcal{B}(Y,X)
\end{align*}
\end{definition}

\begin{theorem}
$A \in \mathcal{B}(X,Y)$, що бієктивний -- оборотний $\iff \exists m > 0: \forall x \in X: \|Ax\| \geq m\|x\|$.
\end{theorem}

\begin{proof}
\rightproof Дано: $A^{-1} \in \mathcal{B}(X,Y)$, тобто $\forall y \in Y: \|A^{-1}y\| \leq \|A^{-1}\| \|y\|$. Оскільки $A$ -- бієкція, то $\exists! x \in X: y = Ax$. Значить, підставивши в нерівність, отримаємо $\|A^{-1} Ax\| = \|x\| \leq \|A^{-1}\| \|Ax\|$. Якщо покласти $m = \|A^{-1}\|^{-1} > 0$, то отримаємо бажану нерівність $\|Ax\| \geq m\|x\|$.
\bigskip \\
\leftproof Дано: $\exists m > 0: \forall x \in X: \|Ax\| \geq m\|x\|$. Якщо взяти елемент $x \in \ker A$, то звідси $\|Ax\| = 0 \geq m \|x\|$, тож автоматично $x = 0$. Отже, існує оборотний оператор $A^{-1} \colon Y \to X$. Залишилося переконатися, що $A^{-1} \in \mathcal{B}(Y,X)$. Ми маємо $x = A^{-1}y$, тоді звідси $\|AA^{-1}y\| = \|y\| \geq m \|A^{-1}y\|$, вналслідок чого $\|A^{-1}y\| \leq m^{-1} \|y\|$.
\end{proof}

\begin{theorem}
\label{invertible_and_geometric_progression}
Задано $X$ -- банахів простір та $A \in \mathcal{B}(X,X)$, причому $\|A\| = q < 1$. Тоді оператор $I-A$ буде оборотним, а також $(I-A)^{-1} = I + A + A^2 + \dots$ -- тут збіжність рівномірна.
\end{theorem}

\begin{proof}
Розглянемо послідовність $(S_n)_{n=1}^\infty$, що задана як $S_n = I + A + \dots + A^n$. Доведемо, що вона фундаментальна в $\mathcal{B}(X,X)$.\\
$\|S_{n+p} - S_n\| = \| A^{n+1} + \dots + A^{n+p}\| \leq \|A^{n+1}\| + \dots + \|A^{n+p}\| \leq \|A\|^{n+1} + \dots + \|A\|^{n+p} < q^{n+1} + \dots + q^{n+p} \\ \leq \dfrac{q^{n+1}}{1-q} \to 0$. Оскільки $q < 1$, то фундаментальність доводиться миттєво.\\
Оскільки простір $X$ банахів, то звідси $\mathcal{B}(X,X)$ теж банахів, тому $S_n \substack{\to \\ \to} S \in \mathcal{B}(X,X)$. Залишилося довести, що $I-A$ -- обернений оператор до $S$, тобто $(I-A)S = S(I-A) = I$. Спочатку зауважимо, що справедлива нерівність:\\
$\|(I-A)S_n - (I-A)S\| \leq \|I-A\| \|S-S_n\| \to 0$.\\
Тому нам буде досить довести, що $(I-A)S_n \substack{\to \\ \to} I$.\\
$\| (I-A)S_n - I\| = \|I - A^{n+1} - I\| = \|A^{n+1}\| \leq q^{n+1} \to 0$.
\end{proof}

\begin{remark}
В умовах теореми, справедлива оцінка $\|(I-A)^{-1}\| \leq (1 - \|A\|)^{-1}$.
\end{remark}

\begin{theorem}
\label{sum_of_operators_and_inverse}
Задані $X,Y$ -- банахові простори та оператори $A,B \in \mathcal{B}(X,Y)$, причому оператор $A$ -- оборотний та $B$ заданий так, що $\|B\| \leq \|A^{-1}\|^{-1}$. Тоді оператор $A+B$ -- оборотний.
\end{theorem}

\begin{proof}
Розглянемо оператор $I + A^{-1}B \colon X \to X$. Зауважимо, що $\|A^{-1}B\| \leq \|A^{-1}\| \|B\| < 1$, тому $I + A^{-1}B$ оборотний за попередньою теоремою.\\
Зауважимо, що $A+B = A (I + A^{-1}B)$ -- добуток двох оборотних операторів, тому сам оператор $A+B$ буде теж оборотним.
\end{proof}

\begin{theorem}
Нехай $X,Y$ -- банахові та $A \in \mathcal{B}(X,Y)$ -- бієктивний. Тоді $A$ -- оборотний.\\
\textit{Тобто коли із банахового в увесь банахів простір йде оператор, то обернений уже точно існує.}\\
\textit{Без доведення поки що.}
\end{theorem}

\subsection{Спряжені оператори}
\begin{definition}
Задані $X,Y$ -- нормовані простори та $A \in \mathcal{B}(X,Y)$.\\
\textbf{Спряженим до} $A$ називають оператор $A^* \colon Y' \to X'$, що визначений так:
\begin{align*}
(A^*l)(x) = l(Ax),\ l \in Y',\ x \in X
\end{align*}
(насправді, можна трохи послабити умови та попросити $A \in \mathcal{L}(X,Y)$, але таке рідко буває.)
\end{definition}

\begin{remark}
Дане означення оператора є коректним.\\
!Дійсно, припустимо, що існує $l \in Y'$, якому ставиться в відповідність два різні функціонали $m_1,m_2 \in X'$. Отже, має існувати $x \in X$, щоб $m_1(x) \neq m_2(x)$. Проте за визначенням, $(A^*l)(x) = m_1(x) = m_2(x)$ -- суперечність!
\end{remark}

\begin{theorem}
Задано $A \in \mathcal{B}(X,Y)$. Тоді $A^* \in \mathcal{B}(Y',X')$, при цьому маємо $\|A^*\| = \|A\|$.
\end{theorem}

\begin{proof}
\textit{$A^*$ -- лінійний}.\\
Нехай $\lambda_1,\lambda_2 \in \mathbb{C}$ та $l_1,l_2 \in E'_2$. Тоді для кожного $x \in E_1$ маємо\\
$(A^*(\lambda_1 l_1 + \lambda_2 l_2))(x) = (\lambda_1 l_1 + \lambda_2 l_2)(Ax) = \lambda_1 l_1 (Ax) + \lambda_2 l_2 (Ax) = \lambda_1 (A^*l_1)(x) + \lambda_2 (A^*l_2)(x)$.\\
$A^*(\lambda_1 l_1 + \lambda_2 l_2) = \lambda_1 A^* l_1 + \lambda_2 A^* l_2$.\\
(ми довели, що якщо $A \in \mathcal{L}(X,Y)$, то тоді й $A^* \in \mathcal{L}(Y',X')$, але знову ж таки.)
\bigskip \\
\textit{$A^*$ -- обмежений}.\\
$|(A^*l)(x)| = |l(Ax)| \leq \|l\| \|Ax\| \leq \|l\| \|A\| \|x\|$. Звідси й випливає нерівність $\|A^*l\| \leq \|A\| \|l\|$.
\bigskip \\
\textit{$\|A^*\| = \|A\|$}.\\
Щойно ми довели, що $\|A^*\| \leq \|A\|$. Для зворотної нерівності зробимо наступне. Оберемо будь-який $x \in E_1$, позначимо $y = Ax$. Тоді за наслідком теореми Гана-Банаха, знайдеться $l \in E_2'$, для якого $\|l\| = 1$ та $l(y) = \|y\|$. Звідси $l(Ax) = \|Ax\|$, а тому $\|Ax\| = |l(Ax)| = |(A^*l)(x)| \leq \|A^*\| \|l\| \|x\| = \|A^*\| \|x\|$. Отже, для всіх $x \in E_1$ справедлива нерівність $\|Ax\| \leq \|A^*\| \|x\|$, зокрема $\|A\| \leq \|A^*\|$.
\end{proof}

\begin{theorem}
Нехай $A \in \mathcal{B}(X,Y)$ та $B \in \mathcal{B}(Y,Z)$. Тоді $(BA)^* = A^*B^*$.
\end{theorem}

\begin{proof}
Для кожного $l \in Z',x \in X$ маємо рівність $((BA)^*l)(x) = l(BAx) = l(B(Ax)) = (B^*l)(Ax)$. Функціонал $B^*l \in Y'$ позначимо за $m$. Тоді отримаємо $m(Ax) = (A^*m)(x)$. Отже, \\ $((BA)^*l)(x) = (A^*B^*l)(x)$.
\end{proof}

\begin{theorem}
Нехай $A \in \mathcal{B}(X,Y)$. Тоді $(A^*)^* = A$ за умоовю, що $X,Y$ -- рефлексивні.
\end{theorem}

\begin{proof}
Маємо $A^* \colon Y' \to X'$, беремо до нього спряжений $(A^*)^* \colon X'' \to Y''$, але в силу рефлексивності отримаємо $(A^*)^* \in \mathcal{B}(X,Y)$. У нас вже була відповідність між $E,E''$, що задається таким чином: $E \ni x \mapsto L_x \in E''$. Тоді для кожного $l \in Y'$ та $x \in X$ маємо\\
$((A^*)^*L_x)(l) = L_x(A^*l) = (A^*l)(x) = l(Ax) = (AL_x)(l)$.
\end{proof}
\newpage

\section{Гілбертові простори}
\subsection{Основні означення}
\begin{definition}
\textbf{Передгілбертовим простором} називають лінійний простір $H$ над $\mathbb{C}$, на якому задано півторалінійний функціонал $(\cdot, \cdot) \colon H \times H \to \mathbb{C}$, для якого виконуються такі властивості:
\begin{align*}
\begin{tabular}{cl}
1) & $\forall x \in H: (x, x) \geq 0$\\
2) & $(x, x) = 0 \iff x = 0$\\
3) & $\forall x,y \in H: (x, y) = \overline{(y, x)}$
\end{tabular}
\end{align*}
Такий функціонал називають \textbf{скалярним добутком}. Якщо прибрати умову $(x,x) = 0 \iff x = 0$, то тоді такий функціонал ще називають \textbf{квазіскалярним добутком}.
\end{definition}

\begin{remark}
Насправді, можна не вимагати, що це півторалінійний функціонал. Ми можемо просто додати четверту умову, що цей функціонал лінійний лише за першим аргументом, тоді випливатиме антилінійність за другим аргументом -- значить, буде півторалінійним.
\end{remark}

\begin{theorem}[Нерівність Коші-Буняковського]
Задано $H$ -- передгілбертів простір. Тоді $\forall x,y \in H: |(x,y)|^2 \leq (x,x)(y,y)$.\\
\textit{Було доведено, див.\ pdf з лінійної алгебри. Щоправда, там в умові теореми вимагалася скінченність векторного простору, але під час доведення це ми не використовували.}
\end{theorem}

\begin{remark}
Нерівність Коші-Буняковсього справедлива й для квазіскалярного добутку.
\end{remark}

\begin{remark}
$\|(x,y)\|^2 = (x,x)(y,y) \iff y = \alpha x$ при $\alpha \in \mathbb{C}$.
\end{remark}

\begin{proposition}
Задано $H$ -- передгілбертів простір. Тоді $H$ --  лінійний нормований простір, причому норма задається як $\|x\| = \sqrt{(x,x)}$.
\end{proposition}

\begin{remark}
Якби був квазіскалярний добуток, то ми би задали вже лише напівнорму.
\end{remark}

\begin{proof}
1) $\|x\| = \sqrt{(x,x)} \geq 0$ -- зрозуміло. Також  $\|x\| = \sqrt{(x,x)} = 0 \iff (x,x) = 0 \iff x = 0$.\\
2) $\| \lambda x \| = \sqrt{(\lambda x, \lambda x)} = \sqrt{\lambda \bar{\lambda} (x,x)} = \sqrt{\lambda^2 (x,x)} = |\lambda| \|x\|$.\\
3) $\|x+y\|^2 = (x+y,x+y) = (x,x) + (x,y) + (y,x) + (y,y) = \|x\|^2 + 2 \Re (x,y) + \|y\|^2 \leq \\ \leq \|x\|^2 + 2 |(x,y)| + \|y\|^2 \overset{\text{нер-ть К-Б}}{\leq} \|x\|^2 + 2 \sqrt{(x,x)} \sqrt{(y,y)} + \|y\|^2 = \|x\|^2 + 2 \|x\| \|y\| + \|y\|^2 = (\|x\| + \|y\|)^2$.\\
$\implies \|x+y\| \leq \|x\| + \|y\|$.
\end{proof}

\begin{definition}
Нехай $H$ -- банахів передгілбертів простір. \\
Тоді даний простір $H$ ще називають \textbf{гілбертовим}.
\end{definition}

\begin{proposition}
Задано $H$ -- передгілбертів простір. Тоді $(\cdot, \cdot) \colon H \times H \to \mathbb{R}$ -- неперервне.
\end{proposition}

\begin{proof}
Дійсно, нехай $x_n \to x_0$ та $y_n \to y_0$. Вони будуть збігатися за нормою (у нас $H$ -- нормований), тобто $(\|x_n\|), (\|y_n\|)$ -- збіжні послідовності, тому обмежені. Тоді\\
$|(x_n,y_n) - (x_0,y_0)| \leq |(x_n,y_n) - (x_0,y_n)| + |(x_0,y_n) - (x_0,y_0)| = |(x_n-x_0,y_n)| + |(x_0,y_n-y_0)| \leq \\
\leq \|x_n-x_0\| \|y_n\| + \|x_0\| \|y_n-y_0\| \leq \|x_n - x_0\| M + \|x_0\| \|y_n-y_0\| \to 0$ (число $M$ обмежує $(\|y_n\|)$).\\
Отже, $(x,y) \to (x_0,y_0)$ при $x \to x_0, y \to y_0$.
\end{proof}

\subsection{Факторизація квазіскалярного добутку}
Задано $H$ -- векторний простір зі квазіскалярним добутком $(\cdot,\cdot)$. Позначимо $L = \{x \in E: (x,x) = 0\}$.

\begin{lemma}
$\forall x \in L, \forall y \in E: (x,y) = 0$.\\
\textit{Випливає з нерівності Коші-Буняковсього}.
\end{lemma}

\begin{lemma}
$L$ -- підпростір векторного простору $E$.
\end{lemma}
\noindent
Як було в лінійній алгебрі, встановимо відношення еквівалентності $x \sim y \iff x - y \in L$ на векторному просторі $E$. Ми вже знаємо, що $E/_L$ буде векторним простором, де задаються операції так:\\
$(x_1+L) + (x_2+L) = (x_1+x_2) + L$;\\
$\lambda(x+L) = \lambda x + L$.\\
Тепер уведемо білінійний функціонал ось таким чином: $(x_1+L,x_2+L)_{E/_L} \overset{\text{def.}}{=} (x_1,x_2)_E$. Доведемо, що це буде задавати скалярний добуток на $E/_L$.\\
Спочатку доведемо коректність означення. Дійсно, нехай $x_1+L = y_1+L$ та $x_2+L = y_2+L$. Тоді звідси $x_1-y_1 \in L$ та $x_2-y_2 \in L$. Зауважимо, що\\
$(x_1,x_2)_E - (y_1,y_2)_E = (x_1,x_2)_E - (y_1,x_2)_E + (y_1,x_2)_E - (y_1,y_2)_E = (x_1-y_1,x_2)_E + (y_1,x_2-y_2)_E = 0$.\\
Отже, $(x_1,x_2)_E = (y_1,y_2)_E \implies (x_1+L,x_2+L)_{E/_L} = (y_1+L,y_2+L)_{E/_L}$.\\
Щодо властивостей скалярного добутку. Це вже точно квазіскалярний. Тобто залишилося довести, що $(x+L,x+L)_{E/_L} = 0 \iff x+L = L$.\\
$(x+L,x+L)_{E/_L} = 0 \implies (x,x)_E = 0 \implies x \in L \implies x+L = L$.
 
\subsection{Ортогональне доповнення}
\begin{definition}
Задано $H$ -- гілбертів простір.\\
Вектори $x,y \in H$ будуть називатися \textbf{ортогональними}, якщо
\begin{align*}
(x,y) = 0
\end{align*}
Позначення: $x \perp y$.
\end{definition}

\noindent Я залишу еквівалентне означення, яке менж розповсюджене, проте корисне буде для деяких локальних міркувань. Одне з локальних міркувань -- це майбутня лема (не теорема) Ріса.

\begin{theorem}
Задано $H$ -- гілбертів простір.\\
$x \perp y \iff \forall \lambda \in \mathbb{C}: \|x\| \leq \|x + \lambda y\|$.
\end{theorem}

\begin{proof}
\rightproof Дано: $x \perp y$, тоді $(x,y) = 0$, а звідси отримаємо\\
$\|x + \lambda y\|^2 = (x+\lambda y, x+\lambda y) = \|x\|^2 + \lambda (y,x) + \overline{\lambda}(x,y) + \lambda^2 \|y\|^2 = \|x\|^2 + \lambda^2 \|y\|^2 \geq \|x\|^2$.
\bigskip \\
\leftproof Дано: $\forall \lambda \in \mathbb{C}: \|x\| \leq \|x + \lambda y\|$. Розпишемо ще раз нерівність:\\
$\|x\|^2 \leq \|x\|^2 + \lambda (y,x) + \overline{\lambda}(x,y) + \lambda^2 \|y\|^2$.\\
$2 \Re \lambda (y,x) + |\lambda|^2 \|y\|^2 \geq 0$.\\
Оскільки це виконується для кожного $\lambda \in \mathbb{C}$, то зокрема й для $\lambda = \dfrac{-(x,y)}{\|y\|^2}$ при $y \neq 0$. Отримаємо:\\
$2 \Re \left( \dfrac{-(x,y)}{\|y\|^2} (y,x) \right) + \dfrac{|(x,y)|^2}{\|y\|^2}  \geq 0$\\
$-3 |(x,y)|^2 \geq 0 \implies (x,y) = 0$.
\end{proof}

\begin{definition}
Задано $H$ -- гілбертів простір та $G \subset H$.\\
\textbf{Ортогональним доповненням} підмножини $G$ називають таку множину:
\begin{align*}
G^\perp = \{x \in H \mid \forall y \in G: (x,y) = 0\}
\end{align*}
\end{definition}

\begin{remark}
Маючи вище теорему, ми можемо переписати ортогональне доповнення інакше:
\begin{align*}
G^\perp = \{x \in H \mid \forall y \in G: \|x+y\| \geq \|x\|\}
\end{align*}
\end{remark}

\begin{proposition}
Задано $H$ -- гілбертів простір та $G \subset H$. Тоді $G^\perp$ -- замкнений підпростір.
\end{proposition}

\begin{proof}
Нехай $x_1,x_2 \in G^\perp$, тобто звідси $\forall y \in G: (x_1,y) = 0,\ (x_2,y) = 0$. Звідси випливає, що\\
$(\lambda_1 x_1 + \lambda_2 x_2, y) = \lambda_1 (x_1, y) + \lambda_2 (x_2,y) = 0$, а тому отримали $\lambda_1 x_1 + \lambda x_2 \in G^\perp$.\\
Тепер нехай $\sequence{x_n} \subset G^\perp$, де $x_n \to x$. Тепер маємо $(x_n,y) = 0$, але через неперервність, то при $n \to \infty$ отримаємо $(x,y) = 0$, тому $x \in G^\perp$.
\end{proof}

\iffalse %TODO обробити властивості
\begin{proposition}
Задано $H$ -- гілбертів простір та $G \subset H$. Тоді $(G^\perp)^\perp = \overline{\linspan(G)}$.
\end{proposition}

\begin{remark}
Якщо $G \subset H$ та підпростір, то $(G^\perp)^\perp = G$. Якщо брати довільну $G \subset H$, то тоді $(G^\perp)^\perp = \overline{\linspan(G)}$.
\end{remark}
\fi
\noindent
Уже якось доводилося, що в унітарних просторах $E$ та підпросторі $L \subset E$ кожний вектор розбивається на суму проєктивного та ортогонального вектора. Спробуємо показати, що це можливо також в довільних гілбертових просторах. Це робиться окремо, бо там ми доводили для скінченно вимірного випадку.

\begin{theorem}[Про існування проєкції на підпростір]
Нехай $H$ -- гілбертів простір та $G \subset H$ -- підпростір. Тоді для кожного $x \in H$ існує єдиний $y \in G$ такий, що $x-y \in G^{\perp}$.\\
\textit{По суті, ця теорема каже, що $G^\perp$ точно містить ненульовий вектор, коли $H \neq \{0\}$.}
\end{theorem}

\begin{proof}
I. \textit{Існування}.\\
Якщо $x \in G$, то кладемо вектор $y = x$.\\
Нехай $x \in H \setminus G$. Визначимо відстань $\displaystyle d = \rho(x,G) \overset{\text{def.}}{=} \inf_{y \in G} \|x-y\|$. Оскільки $x \notin G$, то звідси $d > 0$. Відокремимо послідовність $(y_n)_{n=1}^\infty$ таку, що $d_n \overset{\text{позн}}{=} \|x-y_n\| \to d$. Доведемо, що $\sequence{y_n}$ буде збіжною до деякого елемента $y$, який буде нашим шуканим. Для цього треба довести фундаментальність, зокрема довести $\|y_m - y_n\|^2 = (y_m - y_n, y_m - y_n) \to 0$. Натомість будемо оцінювати $|(y_m - y_n, h)|,\ h \in G$.\\
Для кожного вектора $h \in G$ та скаляра $\lambda \in \mathbb{C}$ ми розглянемо вектор $y_n + \lambda h \in G$. Зрозуміло, що $\|x - (y_n + \lambda h)\| \geq d$, але спробуємо ще оцінити дану норму.\\
$\|x - (y_n + \lambda h)\|^2 = (x-y_n-\lambda h, x-y_n-\lambda h) = (x-y_n,x-y_n) + (x-y_n, -\lambda h) + (-\lambda h, x-y_n) + (-\lambda h, -\lambda h) = \|x-y_n\|^2 + |\lambda|^2 \|h\|^2 - \lambda(h,x-y_n) - \bar{\lambda}(x-y_n,h)$.\\
Ми оберемо $\lambda = \dfrac{(x-y_n,h)}{\|h\|^2}$. Тоді отримаємо:\\
$\|x-(y_n+\lambda h)\|^2 = \dfrac{ (x-y_n,h)^2 }{\|h\|^2} - \dfrac{(h,x-y_n)(x-y_n,h)}{\|h\|^2} - \dfrac{(h,x-y_n)(x-y_n,h)}{\|h\|^2} + \|x-y_n\|^2 = \\
= d_n^2 - \dfrac{|(x-y_n,h)|^2}{\|h\|^2}$.\\
Таким чином, отримали $d_n^2 - \dfrac{|(x-y_n,h)|^2}{\|h\|^2} \geq d^2$. Внаслідок чого $|(x-y_n,h)|^2 \leq \|h\|^2 (d_n^2-d^2)$.\\
Далі, $|(y_m-y_n,h)| = |(x-y_n,h) - (x-y_m,h)| \leq |(x-y_n,h)| + |(x-y_m,h)| \leq \|h\| \left(\sqrt{d_n^2-d^2} + \sqrt{d_m^2 - d^2}\right)$.\\
Оберемо вектор $h = y_m - y_n$, тоді отримаємо наступне:\\
$\|y_m-y_n\|^2 \leq \|y_m-y_n\| \left(\sqrt{d_n^2-d^2} + \sqrt{d_m^2 -d^2}\right)$.\\
$\|y_m-y_n\| \leq \sqrt{d_n^2-d^2} + \sqrt{d_m^2 -d^2} \to 0$.\\
Таким чином, послідовність $(y_n)_{n=1}^\infty$ фундаментальна, а в силу повноти збіжна. Тобто $y_n \to y \in G$. Маючи нерівність $|(x-y_n,h)|^2 \leq \|h\|^2 (d_n^2-d^2)$, при $n \to \infty$ отримаємо $(x-y,h) = 0$ для всіх $h \in G$. Це означає, що $x-y \in G^\perp$.
\bigskip \\
II. \textit{Єдиність}.\\
!Припустимо, що існує ще один вектор $y' \in G$ так, щоб $x - y' \in G^\perp$. Тоді\\
$(y-y',h) = (x-y',h) - (x-y,h) = 0, \forall h \in G$. Зокрема якщо обрати $h = y - y'$, то тоді швидко отримаємо $y = y'$ -- суперечність!
\end{proof}
\noindent
Отже, нехай $x \in H$. За щойно доведеною теоремою, $\exists ! x_1 \in H$ такий, що $x_2 \overset{\text{позн.}}{=} x - x_1 \in G^\perp$. Власне, ми отримали однозначний розклад $x = x_1 + x_2$, де вектори $x_1 \in G, x_2 \in G^\perp$.\\
Перший вектор називається \textbf{ортогональною проєкцією}, позначають $x_1 = \pr_G x$.\\
Другий вектор називається \textbf{ортогональним складником}, позначають $x_2 = \ort_G x$.\\
Отже, кожний вектор $x \in H$ має єдиний розклад в $x = \pr_G x + \ort_G x$.
\bigskip \\
Власне, якщо $H$ -- гілбертів простір та $G \subset H$ -- підпростір, то $H = G \oplus G^\perp$.

\begin{proposition}[Теорема Піфагора]
$\|x\|^2 = \|\pr_G x\|^2 + \|\ort_G x\|^2$.\\
\textit{Вказівка: розписати $\|x\|^2$ та зауважити, що $(\pr_G x, \ort_G x) = 0$.}
\end{proposition}

\begin{remark}
\label{equivalent_existence_of_orthogonal_unit_vector}
Буде корисним залишити таке зауваження. Якщо $H$ -- гілбертів простір та $G \subset H$ -- замкнений підпростір, то тоді існує $G^\perp \neq \{0\}$. Внаслідок чого ми знайдемо вектор $y \notin G$, причому $\|y\|=  1$, для якого $\forall x \in G: \|x+y\| \geq 1$.
\end{remark}

\subsection{Простір, спряжений до гілбертового}
\begin{theorem}[Теорема Ріса]
Нехай $H$ -- гілбертів. Тоді для будь-якого $l \in H'$ існує єдиний вектор $u \in H$ такий, що $\forall x \in H: l(x) = (x,u)$.
\end{theorem}

\begin{proof}
I. \textit{Існування.}\\
Якщо $l \equiv 0$, то існує вектор $u = 0$ -- все ясно.\\
Нехай $l \in H'$, де функціонал ненулевий. Розглянемо $G = \ker l$. Існує елемент $y \notin G$ такий, що $\forall x \in H: x = g + \lambda y$ (це виконано за \prpref{kernel_as_subspace}). Можна переписати елемент $x = g' + \lambda(y- \pr_G y)$.\\
Візьмемо вектор $e = y - \pr_Gy$. Звідси $e \in G^\perp$. Отже, маємо $x = g' + \lambda e$, де можна вважати $\|e\| = 1$. Тоді розпишемо функціонал та скалярний добуток:\\
$l(x) = l(g'+\lambda e) = 0 + \lambda l(e)$ (у нас дійсно $g' \in G = \ker l$, бо $g' = g + \lambda \pr_G y$).\\
$(x,e) = (g'+\lambda e,e) = \lambda \|e\|^2 = \lambda$.\\
$l(x) = l(e)(x,e) = (x,\overline{l(e)}e) \implies u = \overline{l(e)}e$.\\
Отриманий елемент $u = \overline{l(e)}e$ -- шуканий вектор, що задовольняє рівності $l(x) = (x,u)$.
\bigskip \\
II. \textit{Єдиність.}\\
!Припустимо, що існує ще один вектор $u' \in H$, для якого $l(x) = (x,u')$. Звідси випливає, що $\forall x \in H: (x,u-u') = 0 \implies u = u'$ -- суперечність!
\end{proof}

\begin{corollary}
$H' = H$.\\
При цьому коли $H \ni u \leftrightarrow l \in H'$ ми маємо $\|u\| = \|l\|$.
\end{corollary}

\begin{proof}
Якщо $l \in H'$, то за теоремою Ріса йому ставиться в відповідність єдиний $u \in H$, де $l(x) = (x,u)$.\\
Для кожного $u \in H$ розглянемо $l(x) = (x,u)$. Зрозуміло, що це лінійний функціонал, а обмеженість випливає з нерівності Коші-Буняковського $|l(x)| = |(x,u)| \leq \|x\| \|u\|$. У силу обмеженості ми маємо $\|l\| \leq \|u\|$. При $x = u$ маємо $l(u) = (u,u) = \|u\|^2$, тобто $\|l\| = \|u\|$.
\end{proof}

\begin{corollary}
Нехай $H$ -- гілбертів простір та $G \subset H$.\\
$G$ -- тотальна в $H \iff \forall h \in H: h \perp G \implies h = 0$.
\end{corollary}

\begin{proof}
\rightproof Дано: $G$ -- тотально в $H$. Нехай $h \in H$ такий, що $h \perp G$. Звідси випливає, що $\forall y \in G: (h,y) = 0$. Зафіксуємо функціонал $l(y) = \overline{(h,y)}$. Тоді маємо $\forall y \in G: l(y) = 0 \implies \forall y \in H: l(y) = 0$, тобто звідси $\forall y \in H: (h,y) = 0$. Значить, обов'язково $h = 0$.
\bigskip \\
\leftproof Дано: $\forall h \in H: h \perp G \implies h = 0$.\\
Нехай $l$ -- лінійний та неперервний функціонал такий, що $\forall y \in G: l(y) = 0$. За теоремою Ріса, даний функціонал $l(y) = (y,u)$ при деякому $u \in H$. Але $\forall y \in G: (y,u) = 0$, тобто звідси $u \perp G \implies u = 0$. Значить, $\forall y \in H: l(y) = (y,0) = 0$. Отже, $G$ -- тотальна в $H$.
\end{proof}

\begin{proposition}
Задано $H$ -- гілбертів простір та $G \subset H$. Тоді $(G^\perp)^\perp = \overline{\linspan(G)}$.\\
Зокрема якщо $G$ -- підпростір $H$, то $(G^\perp)^\perp = G$.
\end{proposition}

\begin{proof}
Спочатку треба довести, що $G \subset (G^\perp)^{\perp}$, але тут (в принципі) ясно.\\
Нехай $h \in (G^\perp)^\perp$ такий, що $h \perp G$. Тобто $h \in G^\perp$. Але тоді звідси $(h,h) = 0 \implies h = 0$. Отже, $G$ -- тотальна в $(G^\perp)^\perp$.
\end{proof}

\subsection{Ортонормовані системи та базиси}
\begin{definition}
Система векторів $\{e_i\}_{i=1}^\infty$ (поки не більш, ніж зліченні системи розглядатимемо) називається \textbf{ортонормованою}, якщо
\begin{align*}
(e_i,e_j) = \delta_{ij} = \begin{cases} 1,& i = j \\ 0, & i \neq j \end{cases},
\end{align*}
де $\delta$ -- дельта-символ Кронекера.
\end{definition}

\begin{example}
У просторі $l_2$ система, яка є базисом Шаудера, -- ортонормована.
\end{example}

\begin{lemma}
Нехай $\{e_j\}_{j=1}^\infty$ -- ортонормована система векторів.\\
$\displaystyle\sum_{i=1}^\infty c_i e_i$ -- збіжний ряд $\iff \displaystyle\sum_{i=1}^\infty |c_i|^2 < \infty$.
\end{lemma}

\begin{proof}
Щоб довести в обидві сторони, треба зауважити, що справедлива рівність:\\
$\displaystyle\left\| \sum_{i=1}^n c_i e_i - \sum_{i=1}^m c_i e_i \right\| = \left\| \sum_{i=m+1}^n c_i e_i \right\| = \left( \sum_{i=m+1}^n c_i e_i, \sum_{k=m+1}^n c_k e_k \right) = \sum_{i=m+1}^n \sum_{k=m+1}^n c_i \overline{c_k} (e_i,e_k) = \sum_{i=m+1}^n |c_i|^2$.\\
Це ми припускали всюди, що $n > m$.
\end{proof}

\begin{theorem}[Нерівність Бесселя]
Нехай $\{e_j\}_{j=1}^\infty$ -- ортонормована система. Тоді $\displaystyle\sum_{i=1}^\infty (x,e_i)e_i$ збігається, причому $\|x\|^2 \geq \displaystyle\sum_{i=1}^\infty |(x,e_i)|^2$.
\end{theorem}

\begin{remark}
До речі, коефіцієнти $(x,e_i)$ називаються \textbf{коефіцієнтами Фур'є}, а сам ряд -- \textbf{рядом Фур'є}. По суті, ми зробили розклад Фур'є в загального випадку та отримали додатково нерівність Бесселя.
\end{remark}

\begin{proof}
$0 \leq \displaystyle\left\| x - \sum_{i=1}^n (x,e_i)e_i \right\|^2 = \left(x - \sum_{i=1}^n (x,e_i)e_i,\ x - \sum_{i=1}^n (x,e_i)e_i \right) = \\ = (x,x) - \left(x, \sum_{i=1}^n (x,e_i)e_i \right)- \left(\sum_{i=1}^n (x,e_i)e_i ,x\right) + \left(\sum_{i=1}^n (x,e_i)e_i ,\sum_{i=1}^n (x,e_i)e_i \right) = \\
= \|x\|^2 - \sum_{i=1}^n \overline{(x,e_i)}(x,e_i) - \sum_{i=1}^n (x,e_i)(e_i,x) + \sum_{i=1}^n |(x,e_i)|^2 = \|x\|^2 - \sum_{i=1}^n |(x,e_i)|^2$.\\
Отже, $\|x\|^2 \geq \displaystyle\sum_{i=1}^\infty |(x,e_i)|^2$. До того, ж за лемою, оскільки цей ряд збіжний за цією нерівністю, то ряд $\displaystyle\sum_{i=1}^\infty (x,e_i)e_i$ збіжний при всіх $x \in H$.
\end{proof}

\begin{remark}
Нехай $\{e_i\}_{i=1}^\infty$ -- ортонормована система. Тоді вже відомо, що $\displaystyle\sum_{i=1}^\infty |(x,e_k)|^2$ -- збіжний ряд, тому звідси за необхідною ознакою збіжності $(x,e_k) \to 0$ при $k \to \infty$, причому для всіх $x \in H$. Внаслідок чого ми отримаємо, що $e_n \toweak 0$. При цьому $\|e_n - e_m\| = \sqrt{2}$, тому $e_n \not\to 0$ (не сильно збіжна).
\end{remark}

\begin{lemma}
Нехай $\{e_i\}_{i=1}^\infty$ -- ортонормована система та $G = \overline{\linspan\{e_i\}_{i=1}^\infty}\subset H$. Тоді $\forall x \in H: \pr_G x = \displaystyle\sum_{i=1}^\infty (x,e_i)e_i$.
\end{lemma}

\begin{proof}
Ми хочемо довести, що $\forall x \in H: x - \displaystyle\sum_{i=1}^\infty (x,e_i)e_i \in G^\perp$. Таким чином, у силу єдиності, $\displaystyle\sum_{i=1}^\infty (x,e_i)e_i = \pr_G x$. Але за тим, чому дорівнює простір $G$, нам буде досить довести це лише для всіх $e_k$.\\
$\displaystyle\left(x - \sum_{i=1}^\infty (x,e_i)e_i,\ e_k\right) = (x,e_k) - \sum_{i=1}^\infty (x,e_i)(e_i,e_k) = (x,e_k) - (x,e_k) = 0$.
\end{proof}

\begin{definition}
Задано $H$ -- гілбертів простір.\\
Система $\{e_i\}_{i =1}^\infty$ називається \textbf{ортонормованим базисом}, якщо
\begin{align*}
\{e_i\}_{i =1}^\infty \text{ -- тотальна в $H$ ортонормована система}.
\end{align*}
\end{definition}
\noindent
Оскільки $\pr_H x = x$, то звідси $\forall x \in H$ маємо однозначний розклад $x = \displaystyle\sum_{i=1}^\infty (x,e_i)x_i$. Також для будь-якого $y = \displaystyle\sum_{i=1}^\infty (y,e_i)e_i$ скалярний добуток $(x,y) = \displaystyle\sum_{i=1}^\infty (x,e_i)\overline{(y,e_i)}$. Зокрема окремо запишу це:

\begin{theorem}[Рівність Парсеваля]
Нехай $H$ -- гілбертів простір та $\{e_i\}_{i=1}^\infty$ -- ортонормований базис. Тоді $\|x\|^2 = \displaystyle\sum_{i=1}^\infty |(x,e_i)|^2$.
\end{theorem}

\begin{theorem}
Нехай $\{e_i\}_{i=1}^\infty$ -- ортонормована система.\\
$\{e_i\}_{i=1}^\infty$ -- ортонормований базис $\iff \forall x \in H$ спраедлива рівність Парсеваля.
\end{theorem}

\begin{proof}
\rightproof \textit{Щойно довели}.
\bigskip \\
\leftproof Дано: $\forall x \in H: \|x\|^2 = \displaystyle\sum_{i=1}^\infty |(x,e_i)|^2$. При доведенні нерівності Бесселя ми отримали рівність $\displaystyle\left\| x - \sum_{i=1}^n (x,e_i)e_i \right\|^2 = \|x\|^2 - \sum_{j=1}^n |(x,e_i)|^2$. Отже, при $n \to \infty$ отримаємо $\displaystyle\sum_{i=1}^n (x,e_i)e_i \to x$. Отже, $\{e_i\}_{i=1}^\infty$ -- ортонормований базис.
\end{proof}

\subsection{Ортогоналізація системи векторів}
Задано $H$ -- гілбертів простір та $\{u_n\}_{n=1}^\infty$ -- якась зліченна система ненулевих векторів. Ми хочемо цю систему ортогоналізувати. Тобто ми побудується ортонормовану систему $\{e_n\}_{n=1}^\infty$ так, щоб $\linspan\{e_n\} = \linspan\{u_n\}$.\\
Перед стартом відкинемо з даної системи кожний вектор, що линейно залежний від попередніх. У нас тоді буде система $\{v_n\}_{n=1}^\infty$, причому все одно $\linspan\{v_n\} = \linspan\{u_n\}$, а сама система може бути як і скінченною, так і зліченною.\\
Маємо вектор $e_1' = v_1$. Його одразу нормалізуємо, тобто $e_1 = \dfrac{e_1}{\|e_1\|}$.\\
Маємо вектор $e_2' = v_2 - \lambda_{11} e_1$, де число $\lambda_{11}$ можна знайти з умови $e'_2 \perp e_1$. Ми вже таке робили в ліналі, тому, можливо, не повторятиму. Далі нормалізовуємо, буде $e_2 = \dfrac{e_2'}{\|e'_2\|}$\\
Маємо вектор $e_3 = v_3 - \lambda_{21}e_1 - \lambda_{22}e_2$, де числа $\lambda_{21},\ \lambda_{22}$ можна знайти з умов $e'_3 \perp e_1,\ e'_3 \perp e_2$. Робимо аналогічно, а далі нормалізовуємо, буде $e_3 = \dfrac{e_3'}{\|e'_3\|}$.
\vdots \\
Отримаємо не більш ніж зліченну систему $\{e_n\}_{n=1}^\infty$. Зауважимо, що за побудовою $\linspan\{e_n\} = \linspan\{v_n\}$. Також кожний вектор із системи $\{v_n\}_{n=1}^\infty$ записується в вигляді $v_{n+1} = e_{n+1}' + \lambda_{n1}e_1 + \dots + \lambda_{nn}e_n$, тобто звідси $v_{n+1} \in \linspan\{e_1,\dots,e_{n+1}\}$, внаслідок чого $\linspan\{v_n\} \subset \linspan\{e_n\}$.\\
Остаточно, $\linspan\{u_n\} = \linspan\{e_n\}$ -- ортогоналізували.

\begin{corollary}
Задано $H$ -- сепарабельний гілбертів простір. Тоді існує ортонормований базис.
\end{corollary}

\begin{proof}
Оскільки $H$ -- сепарабельний, оберемо скрізь щільну підмножину $G = \{u_n\}_{n=1}^\infty$. Застосуємо вище описаний процес ортогоналізації -- отримаємо ортонормовану систему $\{e_n\}_{n=1}^\infty$, причому $\overline{\linspan\{e_n\}_{n=1}^\infty} = H$. За означенням, $\{e_n\}_{n=1}^\infty$ -- ортонормований базис.
\end{proof}

\begin{theorem}
Задано $H$ -- сепарабельний нескінченний гілбертів простір. Тоді $H \cong l_2$ ізоморфним чином. (якщо $H$ скінченний, то там $H \cong \mathbb{C}^n$).
\end{theorem}

\begin{proof}
Саме тому я спочатку навів наслідок вище, щоб можна було грамотно розписати цю теорему. У нас $H$ -- сепарабельний, тому є ортонормований базис $\{e_n\}_{n=1}^\infty$, звідси $\forall x \in H: x = \displaystyle\sum_{k=1}^\infty x_k e_k$.\\
Побудуємо оператор $A \colon H \to l_2$ ось таким чином: $Ax = (x_1,x_2,\dots)$, це числа $x_i$ були взяті з розкладу елемента $x \in H$. Зауважимо, що $(x_1,x_2,\dots) \overset{\text{справді}}{\in} l_2$ (умовно можна пояснити це через рівність Парсеваля). Оператор $A$ -- лінійний, це ясно. Також $A$ -- бієкція, тому що кожному $(x_1,x_2,\dots) \in l_2$ існуватиме (при тому єдиний) елемент $\displaystyle\sum_{k=1}^\infty x_k e_k \in H$, а він буде потрапляти в $H$ (за теоремою про нерівність Бесселя). Нарешті,\\
$(x,y)_H = \displaystyle\sum_{i=1}^\infty (x,e_i) \overline{(y,e_i)} = \sum_{i=1}^\infty x_k \overline{y_k} = (\{x_k\},\{y_k\})_{l_2}$.\\
$\|x\|_H = \displaystyle\sum_{k=1}^\infty |(x,e_k)|^2 = \sum_{k=1}^\infty |x_k|^2 = \| \{x_k\}|_{l_2}$.
\end{proof}

\begin{corollary}
Усі сепарабельні нескінченні гілбертвоі простори між собою ізометричним чином ізоморфні.
\end{corollary}

\iffalse
\begin{theorem}
Задано $H$ -- гілбертів простір. Нехай $\{f_j\}_{j=1}^\infty$ -- деяка система векторів та $G = \overline{\linspan\{f_j\}_{j=1}^\infty}$. Тоді існує ортонормована система $\{e_j\}_{j=1}^{n (\infty)}$ (яка може бути скінченною чи зліченною) така, що $\overline{\linspan\{e_j\}_{j=1}^\infty} = G$.
\end{theorem}

\begin{proof}
Візьмемо $e_1 = \dfrac{f_1}{\|f_1\|}$. Він породжує $G_1 = \overline{\linspan\{e_1\}} = \linspan\{e_1\}$. Якщо $G_1 = G$, то теорема доведена.\\
Інакше виберемо перший вектор із послідовності $(f_j)$ (оберемо $f_{j_2}$ такий, що $f_{j_2} \notin G_1$. Візьмемо $e_2 = \dfrac{\ort_{G_1} f_{j_2}}{\| \ort_{G_1} f_{j_2}\|}$, при цьому $e_2 \perp e_1$. (TODO: тверезо обдумати). Беремо $G_2 = \overline{\linspan\{e_1,e_2\}} = \linspan\{e_1,e_2\}$. Якщо $G_2 = G_1$, то теорема доведена.\\
Інакше виберемо перший вектор із послідовності $(f_j)$ (оберемо $f_{j_3}$ такий, що $f_{j_3} \notin G_2$.\\
\vdots
\end{proof}

\begin{corollary}
Нехай $H$ -- сепарабельний. Тоді існує скінченний (або зліченний) ортонормований базис.
\end{corollary}

\begin{proof}
\rightproof Дано: $H$ -- сепарабельний. Нехай $(f_j)$ -- щільна множина в $H$. Застосуємо ортогоналізацію. Тоді існує ортонормований базис $(e_j)$.
\bigskip \\
\leftproof Дано: існує скінченний чи зліченний ортонормований базис. Тоді $H$ автоматично сепарабельний. Треба розглянути сукупність векторів $\displaystyle\sum_{i=1}^n e_i e_i,\ e_i \in \mathbb{Q}$.
\end{proof}

\begin{theorem}
Нехай $H$ -- сепарабельний, нескінченновимірний гілбертовий простір. Тоді $H \cong l_2$. (ну якщо скінченний, то $H \cong \mathbb{C}^n)$)
\end{theorem}

\begin{proof}
Маємо $H$ -- сепарабельний, тобто існує зліченний ортонормований базис $(e_j)$. Задамо ізоморфізм $H \ni x = \displaystyle\sum_{i=1}^\infty x_i e_i \mapsto (x_1,x_2,\dots) \in l_2$. Остання послідовність справді лежить в $l_2$, завдяки рівності Парсеваля. При цьому $(x,y)_H = \displaystyle\sum_{i=1}^\infty x_i \overline{y_i} = (x,y)_{l_2}$. (TODO: деталізувати).
\end{proof}
\fi

\subsection{Коротко про ортонормовану систему векторів довільної потужності}
\begin{definition}
Система векторів $\{e_\alpha\}_{\alpha \in I}$ називається \textbf{ортонормованою}, якщо
\begin{align*}
(e_\alpha,e_\beta) = \delta_{\alpha \beta} = \begin{cases} 1,& \alpha = \beta \\ 0, & \alpha \neq \beta \end{cases},
\end{align*}
де $\delta$ -- дельта-символ Кронекера.
\end{definition}

\begin{remark}
Зрозуміло, що якщо в гілбертовому просторі існує незліченна ортонормована система, то тоді гілбертів простір не буде більше сепарабельним.
\end{remark}

\begin{lemma}
Задано $H$ -- гілбертів простір та $\{e_\alpha\}_{\alpha \in I}$ -- незліченна ортонормована система. Розглянемо коефіцієнти Фур'є $(x,e_\alpha)$, тоді кількість ненулевих коефіцієнтів -- не більш, ніж зліченна.
\end{lemma}

\begin{proof}
Розглянемо множину $I_n = \left\{ \alpha \in I \mid (x,e_\alpha) > \dfrac{1}{n} \right\}$. Такі множини будуть скінченними.\\
!Припустимо, що ні. Тоді можна взяти зліченну кількість $(\alpha_k)_{k=1}^\infty$, для яких $(x,e_{\alpha_k}) > \dfrac{1}{n}$. Тоді з одного боку, $\displaystyle\sum_{k=1}^\infty |(x,e_{\alpha_k})|^2 = \infty$, проте за нерівністю Бесселя, $\displaystyle\sum_{k=1}^\infty |(x,e_{\alpha_k})|^2 \leq \|x\|^2 < \infty$. Суперечність!\\
Тепер зауважимо, що $\{\alpha \in I \mid (x,e_\alpha) \neq 0\} = \displaystyle\bigcup_{n=1}^\infty I_n$, така множина не більш, ніж зліченна.
\end{proof}

\begin{lemma}
Нехай $\{e_\alpha\}_{\alpha \in I}$ -- ортонормована система та $G = \overline{\linspan\{e_\alpha\}_{\alpha \in I}} \subset H$. Тоді $\forall x \in H: \pr_G x = \displaystyle\sum_{\alpha \in J}(x,e_\alpha) e_\alpha$. Сумуємо по множині $J$, де ненулеві коефіцієнти Фур'є.\\
\textit{Міркування аналогічні.}
\end{lemma}

\begin{definition}
Задано $H$ -- гілбертів простір.\\
Система $\{e_\alpha\}_{\alpha \in I}$ називається \textbf{ортонормованим базисом}, якщо
\begin{align*}
\{e_\alpha\}_{\alpha \in I} \text{ -- тотальна в $H$ ортонормована система}.
\end{align*}
\end{definition}
Аналогічним чином тоді $x = \displaystyle\sum_{\alpha \in J} (x,e_\alpha) e_\alpha$. Сумуємо по множині $J$, де ненулеві коефіцієнти Фур'є. Аналогічно працює рівність Парсеваля та скалярний добуток.

\subsection{Про форми в гілбертових просторах}
Із курсу лінійної алгебри вже знаємо, що така півторалінійна форма $b$. Також ввордили таке поняття, як ермітові форми та квадратичні форми. У гілбертововму просторі поляризаційна тотожність все одно працює.

\begin{definition}
Півторалінійна форма $b \colon H \times H \to \mathbb{C}$ називається \textbf{обмеженою}, якщо
\begin{align*}
\exists C > 0: \forall x,y \in H: |b(x,y)| \leq C \cdot \|x\| \cdot \|y\|
\end{align*}
Аналогічним чином ми визначимо \textbf{норму} форми таким чином:
\begin{align*}
\|b\| = \inf\{C > 0 \mid \forall x, y \in H: |b(x,y)| \leq C\|x\| \|y\|\}
\end{align*}
\end{definition}

\begin{example}
Зокрема, по-перше, скалярний добуток $b(x,y) = (x,y)$ сам по собі є півторалінійною ермітовою формою. По-друге, він -- обмежений. Дійсно, це випливає з нерівності Коші-Буняковського:\\
$|b(x,y)| = |(x,y)| \leq \sqrt{(x,x)} \sqrt{(y,y)} = 1 \cdot \|x\| \cdot \|y\|$.
\end{example}

\begin{proposition}
Задано $A \in \mathcal{B}(H)$. Тоді $b_A(x,y) = (Ax,y)$ задаватиме обмежену півторалінійну ермітову форму.\\
\textit{Вправа: довести.}
\end{proposition}

\begin{theorem}
Нехай заданий $b$ -- обмежений півторалінійний функціонал. Тоді існує єдиний оператор $A \in \mathcal{B}(H)$, для якого $b(x,y) = (Ax,y)$.\\
\textit{Щось таке подібне було в лінійній алгебрі, проте я передоведу.}
\end{theorem}

\begin{proof}
Зафіксуємо $x \in H$, тоді зауважимо, що $f(y) = \overline{b(x,y)}$ буде обмеженим лінійним функціоналом на $H$. За теоремою Ріса, ми можемо знайти єдиний вектор $a_x \in H$, для якого $\overline{b(x,y)} = (y,a_x)$, або $b(x,y) = (a_x,y)$.\\
Таким чином, побудували відображення $A \colon H \to H$, що діє як $Ax = a_x$. Доведемо, що це той самий $A \in \mathcal{B}(H)$.\\
Нехай $x_1,x_2 \in H$ та $\alpha_1,\alpha_2 \in \mathbb{C}$. Маємо наступне:\\
$(A(\alpha_1 x_1 + \alpha_2 x_2), y) = b(\alpha_1 x_1 + \alpha_2 x_2,y) = \alpha_1 b(x_1,y) + \alpha_2 b(x_2,y) = \alpha_1 (Ax_1,y) + \alpha_2 (Ax_2,y) = (Ax_1 + Ax_2, y)$.\\
Оскільки ця рівність виконується для всіх $y \in H$, то звідси $A(\alpha_1 x_1 + \alpha_2 x_2) = \alpha_1 Ax_1 + \alpha_2 Ax_2$.\\
Ми вже знаємо, що $|(Ax,y)| = |(a_x,y)| = |b(x,y)| \leq c \|x\| \|y\|$. Якщо підставити $y = Ax$, то звідси випливатиме оцінка $\|Ax\| \leq c \|x\|$. Довели обмеженість.\\
Єдиність цілком зрозуміло.
\end{proof}

\begin{theorem}
Нехай заданий $b$ -- обмежений півторалінійний функціонал. Уже відомо, що йому асоціюється єдиний оператор $A \in \mathcal{B}(H)$, де $b(x,y) = (Ax,y)$. Тоді $\|b\| = \|A\|$.
\end{theorem}

\begin{proof}
Маємо $|b(x,y)| = |(Ax,y)| \leq \|Ax\| \|y\| \leq \|A\| \|x\| \|y\|$. Тобто $\|A\|$ -- константа, що обмежує форму, тому за означенням норми форми, $\|b\| \leq \|A\|$.\\
Із іншого боку, $\|Ax\|^2 = |b(x,Ax)| \leq \|b\| \|x\| \|Ax\|$, внаслідок чого $\|Ax\| \leq \|b\| \|x\|$. Тобто  $\|b\|$ -- константа, що обмежує оператор, тому за означення норми оператора, $\|A\| \leq \|b\|$.\\
Отже, ці два міркування дають сказати нам $\|b\| = \|A\|$.
\end{proof}

\subsection{Про деякі типи операторів}
\subsubsection{Спряжений оператор (ще раз)}
У гілбертовому просторі $H'$ можна ототожнити з множиною $H$ за теоремою Ріса. Також $l(x) = (x,l)$ звідти ж. Тепер розглянемо оператор $A \in \mathcal{B}(H_1,H_2)$. Ми вже визначали спряжений оператор $A^* \in \mathcal{B}(H_2',H_1') = \mathcal{B}(H_2,H_1)$, що задається як $(A^*l)(x) = l(Ax)$.\\
Відомо, що $(A^*l)(x) = l(Ax) = (Ax,l)$. Із іншого боку, $(A^*l)(x) = (x,A^*l)$.\\
Таким чином, ми маємо еквівалентне означення в гілбертовому просторі:

\begin{theorem}
Оператор $A^* \in \mathcal{B}(H_1,H_2)$ -- спряжений в $A \iff (Ax,l) = (x,A^*l), \forall x \in H_1, \forall l \in H_2$.
\end{theorem}

\subsubsection{Самоспряжений оператор}
\begin{definition}
Оператор $A \in \mathcal{B}(H)$ називається \textbf{самоспряженим}, якщо
\begin{align*}
A = A^*
\end{align*}
\end{definition}
\noindent
За отриманим результатом вище маємо чергове означення:

\begin{theorem}
Оператор $A \in \mathcal{B}(H)$ -- самоспряжений $\iff \forall (Ax,l) = (x,Al), \forall x,l \in H$.
\end{theorem}

\begin{theorem}
Будь-який оператор $A \in \mathcal{B}(H)$ єдиним чином можна подати в вигляді $A = \Re A + i \Im A$, де $\Re A, \Im A$ -- самоспряжені оператори.\\
\textit{Уже в курсі ліналу доводили.}
\end{theorem}

\begin{theorem}
$A \in \mathcal{B}(H)$ -- самоспряжений $\iff b_A$ -- ермітова форма $\iff q_A(x) = b_A(x,x) \in \mathbb{R}$.
\end{theorem}

\begin{proof}
Почну з першої еквівалетності.\\
\rightproof Дано: $A \in \mathcal{B}(H)$ -- самоспряжений. Тоді $b_A(x,y) = (Ax,y) = (x,Ay) = \overline{(Ay,x)} = \overline{b_A(y,x)}$.\\
\leftproof Дано: $b_A$ -- ермітова форма. Тоді $(Ax,y) = b_A(x,y) = \overline{b_A(y,x)} = \overline{(Ay,x)} = (x,Ay)$.
\bigskip \\
Тепер доведемо другу еквівалентність.\\
\rightproof Дано: $b_A$ -- ермітова форма. Тоді $b_A(x,x) = \overline{b_A(x,x)} \implies b_A(x,x) = q_A(x) \in \mathbb{R}$.\\
\leftproof Дано: $q_A \in \mathbb{R}$. Використаємо поляризаційну тотожність:\\
$4b_A(x,y) = q_A(x+y) - q_A(x-y) + i[q_A(x+iy) - q_A(x-iy)]$.\\
$\overline{4b_A(y,x)} = \overline{q_A(y+x) - q_A(y-x) + i[q_A(y+ix) - q_A(y-ix)]} = \\ = \overline{q_A(y+x)} - \overline{q_A(y-x)} + \overline{i q_A(y+ix)} - \overline{i q_A(y-ix)} = q_A(y+x) - q_A(y-x) + \dfrac{1}{i} q_A(y+ix) - \dfrac{1}{i} q_A(y-ix) \\
= q_A(x+y) - q_A(x-y) + i[q_A(x+iy) - q_A(x-iy)]$.\\
Таким чином, $b_A(x,y) = \overline{b_A(y,x)}$, тобто $b_A$ -- ермітова.
\end{proof}

\subsubsection{Невід'ємні та напівобмежені оператори}
\begin{definition}
Оператор $A \in \mathcal{B}(H)$ називається \textbf{невід'ємним}, якщо
\begin{align*}
\forall x \in H: b_A(x,x) \geq 0
\end{align*}
Позначення: $A \geq 0$.\\
Будемо казати, що $A \geq B$, якщо $A-B \geq 0$. Це задає відношення часткового порядку на $\mathcal{B}(H)$.
\end{definition}

\begin{definition}
Оператор $A \in \mathcal{B}(H)$ називається \textbf{напівобмеженим знизу} числом, якщо
\begin{align*}
\exists c \in \mathbb{R}: \forall x \in H: b_A(x,x) \geq c \|x\|^2
\end{align*}
Оператор $A \in \mathcal{B}(H)$ називається \textbf{напівобмеженим зверху} числом, якщо
\begin{align*}
\exists d \in \mathbb{R}: \forall x \in H: b_A(x,x) \leq d \|x\|^2
\end{align*}
\end{definition}
\noindent
Якщо переписати нерівність по-інакшому, то ми отримаємо, що\\
$A$ -- напівобмежений знизу числом $c \in \mathbb{R} \iff A \geq c I$;\\
$A$ -- напівобмежений зверху числом $d \in \mathbb{R} \iff A \leq d I$.

\begin{remark}
Нехай $A \in \mathcal{B}(H)$ -- напівобмежений. Тоді $A$ -- самоспряжений.\\
Це випливає безпосередньо з означення напівобмежених операторів.
\end{remark}

\begin{remark}
Нехай $A \in \mathcal{B}(H)$ -- самоспряжений. Тоді $A$ напівобмежений зверху числом $\|A\|$ та знизу числом $-\|A\|$. Просто тому що справедлива нерівність $|(Ax,x)| \leq \|A\| \|x\|^2$.
\end{remark}

\begin{proposition}
Задано $A \in \mathcal{B}(H)$ -- напівобмежений. Тоді $A$ -- оборотний.\\
\textit{TODO: доробити.}
\end{proposition}

\subsubsection{Проєктор}
\begin{definition}
Задано $H$ -- гілбертів простір та $G \subset H$ -- замкнений підпростір.\\
\textbf{Проєктором в $H$ на $G$} назвемо оператор $P_G$, який діє таким чином:
\begin{align*}
P_g x = \pr_g x
\end{align*}
Якщо $\{e_i\}_{i=1}^\infty$ -- ортнормований базис, то $P_G x = \displaystyle\sum_{i=1}^\infty (x,e_i)e_i$.
\end{definition}

\begin{theorem}
Проєктор на $G$ має такі властивості:
\begin{enumerate}[nosep,wide=0pt,label={\arabic*)}]
\item $P_G \in \mathcal{B}(H)$, причому $\|P_G\| = 1$ за умовою, що $G \neq \{0\}$;
\item $P_G$ -- ідемпотентний оператор;
\item $P_G$ -- самоспряжений оператор;
\item $P_G$ -- невід'ємний оператор.
\end{enumerate}
\end{theorem}

\begin{proof}
Доведемо кожну властивість окремо.
\begin{enumerate}[wide=0pt,label={\arabic*)}]
\item Маємо $x,y \in G$, треба довести $P_G(x+y) = P_Gx + P_Gy$, або теж саме $\pr_G(x+y) = \pr_G x + \pr_G y$.\\
Оскільки $x,y \in H$, то для них існують єдині $\pr_G x,\pr_G y \in G$. Ми доведемо, що $(x+y) - (\pr_G x + \pr_G y) \in G^\perp$, тим самим за єдиністю проєкції отримаємо бажану рівність.\\
$(x+y - \pr_G x - \pr_G y, g) = (x - \pr_G x, g) + (y - \pr_G y, g) = 0 + 0 = 0$.\\
Отже, ми довели $P_G(x+y) = P_G(x) + P_G(y)$.\\
Десь аналогічним чином доводиться, що $P_G(\lambda x) = \lambda P_G(x)$.\\
Далі зауважимо, що $\|P_G x\|^2 = \| \pr_G x\|^2 = \|x\|^2 - \|\ort_G x\|^2 \leq 1 \cdot \|x\|^2$. Останнє повністю доводить той факт, що $P_G \in \mathcal{B}(H)$, причому тут $\|P_G\| \leq 1$. Якби був $G = \{0\}$, то ми би отримали $P_G = 0$. Інакше $\forall g \in G: P_G g = g$, внаслідок чого $\|P_G\| = 1$.

\item Треба довести, що $P_G^2 = P_G$. Маємо $x \in H$, тоді звідси \\
$P_G^2 x = P_G (P_G x) = P_G ( \pr_G x) = \pr_G x = P_G x$.

\item $(P_Gx, y) = (\pr_G x, \pr_G y  + \ort_G y) = (\pr_G x, \pr_G y) = (\pr_G x, \ort_G y) = (\pr_G x, \pr_G y) = (\pr_G x, \pr_G y) = (\ort_G x, \pr_G y) = (\pr_G x + \pr_G x, \pr_G y) = (x, P_G y)$.

\item $(P_G x, x) = (\pr_G x, \pr_G x + \ort_G x) = \|P_Gx\|^2 \geq 0$.
\end{enumerate}
Всі властивості довели.
\end{proof}

\begin{theorem}
Задано $A \in \mathcal{B}(H)$ -- самоспряжений, ідемпотентний оператор. Тоді існує замкнений підпростір $G \subset H$, в якому оператор $A$ поводить себе як проєктор на $G$, тобто $A = P_G$.
\end{theorem}

\begin{proof}
Розглянемо множину $G = \{g \in H \mid Ag = g\}$, іншими словами множина $G = \ker (A - I)$. Оскільки $A$ -- ідемпотентний, то $\forall x \in H: Ax \in G$. Також зазначимо, що $P_G x \in G$. Тому нам досить довести, що $(Ax,g) = (P_Gx,g), \forall g \in G$, а це гарантуватиме $Ax = P_Gx$.\\
$(Ax,g) = (x,Ag) = (x,g) = (x, P_G g) = (P_G x,g)$.
\end{proof}

\subsubsection{Нормальні оператори}
\begin{definition}
Оператор $A \in \mathcal{B}(H)$ називається \textbf{нормальним}, якщо
\begin{align*}
A^* A = AA^*
\end{align*}
Існує таке поняття як комутатор, в цьому випадку $[A,B] = AB - BA$. Значить, означення нормального оператора можна переписати ось так:
\begin{align*}
[A,A^*] = O
\end{align*}
\end{definition}

\begin{theorem}
$A \in \mathcal{B}(H)$ -- нормальний $\iff [\Re A, \Im A] = O$.
\end{theorem}

\begin{proof}
Нам досить буде розписати комутатор $[A,A^*]$, а там стане зрозуміло.\\
$[A,A^*] = AA^* - A^*A = (\Re A + i \Im A)(\Re A - i \Im A) - (\Re A - i \Im A) (\Re A + i \Im A) = \\
= \Re^2 A - i \Re A \Im A + i \Im A \Re A + \Im^2 A - \Re^2 A - i \Re A \Im A + i \Im A \Re A - \Im^2 A = \\
= -2i \Re A \Im A + 2i \Im A \Re A = -2i [\Re A, \Im A]$.\\
Отже, ми автоматично довели, що $A$ -- нормований $\iff [\Re A, \Im A] = O$. 
\end{proof}

\subsection{Унітарні та ізометричні оператори}
\begin{definition}
Лінійний оператор $U \colon H \to H$ називають \textbf{унітарним}, якщо
\begin{align*}
\forall x,y \in H: (Ux, Uy) = (x,y); \\
\Im U = H
\end{align*}
\end{definition}

\begin{remark}
Із означення випливає, що $\|Ux\| = \|x\|, \forall x \in H$, а тому звідси $U \in \mathcal{B}(H),\ \|U\| = 1$.
\end{remark}

\begin{proposition}
Якщо лінійний оператор зберігає норму, то він же зберігає скалярний добуток.\\
\textit{Випливає з поляризаційної тотожності.}
\end{proposition}

%proved once again
\iffalse
\begin{proof}
Маємо оператор $A$ такий, що $\|Ax\| = \|x\|$. Покажемо, що $(Ax,Ay) = (x,y)$.\\
Із одного боку, маємо таку рівність:\\
$(A(x+y),A(x+y)) = (Ax + Ay, Ax + Ay) = (Ax,Ax) + (Ax,Ay) + (Ay,Ax) + (Ay,Ay) = \\ = (x,x) + (Ax,Ay) + (Ay,Ax) + (y,y)$.\\
Із іншого боку, маємо\\
$(A(x+y),A(x,y)) = (x+y,x+y) = (x,x) + (x,y) + (y,x) + (y,y)$.\\
$\implies (Ax,Ay) + (Ay,Ax) = (x,y) + (y,x)$, тобто $2 \Re (Ax,Ay) = 2 \Re (x,y) \implies \Re (Ax,Ay) = \Re (x,y)$.\\
Якщо розписати $(A(x+iy), A(x+iy))$ з двох боків, то так само доведемо, що $\Im (Ax,Ay) = \Im (x,y)$.\\
Таким чином, як комплексні числа, $(Ax,Ay) = (x,y)$.
\end{proof}
\fi

\begin{remark}
Якщо $\dim H < \infty$, то тоді в означенні унітарного оператора досить вимагати збереження скалярного добутку (як це було в ліналі).\\
Дійсно, маємо $\{e_1,\dots,e_n\}$ -- ортонормований базис $H$. Тоді $(Ue_j,Ue_k) = (e_j,e_k) = \delta_{jk}$. Отже, $\{Ue_1,\dots,Ue_n\}$ -- ортонормований базис $H$, причому\\
$x = \displaystyle\sum_{j=1}^n x_j Ue_j = U\left( \sum_{j=1}^n x_j e_j\right)$.\\
Внаслідок чого $\Im U = H$.
\end{remark}

\begin{example}
Розглянемо $T(x_1,x_2,\dots) = (0,x_1,x_2,\dots)$ -- оператор в просторі $l_2$. Зауважимо, що $(Tx,Ty) = (x,y)$. Проте $\Im T \neq H$, оскільки вектор $(1,0,0,\dots) \perp \Im T$.
\end{example}

\begin{theorem}
Задано $U$ -- унітарний оператор. Тоді $U$ -- оборотний, причому $U^{-1} = U^*$, який теж є унітарним.
\end{theorem}

\begin{proof}
Оскільки $\Im U = H$ та $\|Ux\| = \|x\|$, то ми доводимо оборотність. Доведемо, що $U^{-1}$ -- унітарний. Покладемо $y = Ux$, тоді\\
$\|y\| = \|Ux\| = \|x\| = \|U^{-1}y\|$.\\
Тобто $U^{-1}$ зберігає норму, а тому й скалярний добуток. Також зрозуміло, що $\Im U^{-1} = H$.\\
Рівінсть $U^{-1} = U^*$ випливає з такого ланцюга: $(x, U^*y) = (Ux,y) = (Ux, UU^{-1}y) = (x,U^{-1}y)$.
\end{proof}

\begin{proposition}
Задано $U$ -- унітарний оператор. Тоді $U$ -- нормальний.
\end{proposition}

\begin{proof}
$[U,U^*] = UU^* - U^*U = UU^{-1} - U^{-1}U = I - I = O$.
\end{proof}

\begin{definition}
Лінійний оператор $V \colon H_1 \to H_2$ називається \textbf{ізометричним}, якщо
\begin{align*}
\forall x,y \in H: (Vx,Vy)_2 = (x,y)_1
\end{align*}
Ця умова, як вже відомо, рівносильна умові $\|Vx\|_2 = \|x\|_1$.
\end{definition}

\subsection{Матричне представлення операторів у гілбертовому просторі}
\subsubsection{Лінійний оператор в сепарабельному просторі}
Нехай $H$ -- нескінченний сепарабельний простір, тому є $\{e_n\}_{n=1}^\infty$ -- ортонормований базис. Розглянемо оператор $A \in \mathcal{B}(H)$, який можна розкласти за базисом. Отримаємо наступне:\\
$(Ax, e_j) = \displaystyle\sum_{k=1}^\infty x_k (Ae_k, e_j) = \sum_{k=1}^\infty a_{jk} x_k$.\\
Ми тут позначили $a_{jk} = (Ae_k, e_j)$. Визначені числа $a_{jk}$ утворюють нескінченну матрицю $(a_{jk})_{j,k=1}^\infty$, елементами $k$-го стовпчика якого будуть координати вектора $Ae_k$.

\begin{theorem}
Будь-який оператор $A \in \mathcal{B}(H)$ допускає матрице представлення в будь-якому ортонормованому базисі $H$.
\end{theorem}

\begin{remark}
Не кожній нескінченній матриці відповідає лінійний обмежений оператор в гілбертовому просторі (на відміну від скінченновимірних просторах).
\end{remark}

\begin{theorem}
Нехай матриця $(a_{jk})_{j,k=1}^\infty$ задовольняє умові $\displaystyle\sum_{j,k=1}^\infty |a_{jk}|^2 < \infty$. Тоді матриця $(a_{jk})_{j,k=1}^\infty$ визначає оператор $A \in \mathcal{B}(H)$.
\end{theorem}

\begin{proof}
За нерівністю Коші-Буняковського, отримаємо\\
$|(Ax,e_j)|^2 = \displaystyle\left| \sum_{k=1}^\infty a_{jk} x_k \right|^2 \leq \sum_{k=1}^\infty |a_{jk}|^2 \|x\|^2, \qquad j \in \mathbb{N}$.\\
Просумовуючи це все за $j$, отримаємо\\
$\|Ax\|^2 = \displaystyle\sum_{j=1}^\infty |(Ax,e_j)|^2 \leq \sum_{j,k=1}^\infty |a_{jk}|^2 \|x\|^2$.\\
Таким чином, $A \in \mathcal{B}(H)$, причому $\|A\| \leq \displaystyle \left( \sum_{j,k=1}^\infty |a_{jk}|^2 \right)^{\frac{1}{2}}$.
\end{proof}

\begin{remark}
Зауважимо, що одинична матриця $(a_{jk})_{j,k=1}^\infty$, де $a_{jk} = \delta_{jk}$, задає лінійний обмежений оператор $A = I \in \mathcal{B}(H)$. При цьому ця матриця не задовольняє умові теореми вище.\\
Тому дана достатня умова є доволі строгою, треба послабити.
\end{remark}

\begin{theorem}
Матриця $(a_{jk})_{j,k=1}^\infty$ задає оператор $A \in \mathcal{B}(H) \iff$ справедливі такі умови:
\begin{enumerate}[nosep,wide=0pt,label={\arabic*)}]
\item $\displaystyle\sum_{k=1}^\infty a_{jk} x_k$ збіжний для всіх $j \in \mathbb{N}$; де числа $x_k$ беруться з розкладу $x = \displaystyle\sum_{k=1}^\infty x_k e_k$;
\item $\displaystyle\sum_{j=1}^\infty \left| \sum_{k=1}^\infty a_{jk} x_k \right|^2 < \infty$;
\item $\exists c > 0: \forall x \in H: \displaystyle\sum_{j=1}^\infty \left| \sum_{k=1}^\infty a_{jk} x_k \right|^2  \leq c \|x\|^2$.
\end{enumerate}
\textit{(TODO: добити)}.
\end{theorem}

\subsection{Оператори Гілберта-Шмідта}
Нехай $H$ -- сепарабельний гілбертів простір та $\{e_n\},\ \{f_n\}$ -- два ортонормовані базиси в $H$. Припустимо, що $A \in \mathcal{B}(H)$ задовольняє умові $\displaystyle\sum_{j,k=1}^\infty |(Af_k,e_j)|^2 < \infty$.\\
Дана величина ніяк не залежить від вибору пари базису. Дійсно, зауважимо, що $\|Ae_j\|^2 = \displaystyle\sum_{j=1}^\infty |(Ae_j,f_k)|$, а тому ця величина вище $\displaystyle\sum_{j,k=1}^\infty |(Af_k,e_j)|^2 = \sum_{j=1}^\infty \|Ae_j\|^2$. Тому неважливо, який базис буде другим.\\
Більш того, знаючи, що $(Ae_j,f_k) = (e_j,A^*f_k)$ можна аналогічно довести, що неважливо, який базис буде першим.\\
Отже, коректним буде ось таке означення:

\begin{definition}
Оператор $A \in \mathcal{B}(H)$ називається \textbf{оператором Гілберта-Шмідта}, якщо для деякого (а тому й для кожного) ортонормованого базиса $\{e_n\}$ збігається ряд:
\begin{align*}
\displaystyle\sum_{j=1}^\infty \|Ae_j\|^2 = \sum_{j,k=1}^\infty |a_{jk}|^2 < \infty
\end{align*}
Позначення: $S_2(H)$ -- набір всіх операторів Гілберта-Шмідта.\\
\textbf{Абсолютною нормою} (чи \textbf{нормою Гілберта-Шмідта}) оператора $A$ називають величину
\begin{align*}
|A| = \left(\sum_{j=1}^\infty \|Ae_j\|^2 \right)^{\frac{1}{2}}
\end{align*}
\end{definition}

\begin{remark}
$\emptyset \subsetneq S_2(H) \subsetneq \mathcal{B}(H)$.\\
Дійсно, розглянемо оператор $A \in \mathcal{B}(H)$, у якого лише скінченне число матричних елементів $a_{jk}$, для яких $a_{jk} \neq 0$. Тоді цілком зрозуміло, що такий $A \in S_2(H)$. Тобто $S_2(H) \neq \emptyset$.\\
Далі, $I \notin S_2(H)$, просто тому що $\displaystyle\sum_{j=1}^\infty \|Ie_j\|^2 = \sum_{j=1}^\infty \|e_j\|^2 = \infty$.
\end{remark}

\begin{proposition}
$S_2(H)$ -- нормований простір із нормою $|A|$ -- нормою Гілберта-Шмідта. Причому про саму норму Гілберта-Шмідта відомо, що $\forall A \in S_2(H): \|A\| \leq |A|$.
\end{proposition}

\begin{proof}
Спочатку доведемо, що $S_2(H)$ буде підпростором. Дійсно, нехай $A,B \in S_2(H)$, тоді\\
$\displaystyle\sum_{j=1}^\infty \|(\lambda A + \mu B)\| e_j \|^2 = \sum_{j=1}^\infty \| \lambda A e_j + \mu B e_j \|^2 \leq |\lambda|^2\sum_{j=1}^\infty \|Ae_j\|^2 + |\mu|^2 \sum_{j=1}^\infty \|Be_j\|^2 < \infty$.\\
Отже, ми довели $\lambda A + \mu B \in S_2(H)$.\\
Тепер покажемо, що $|A| = \displaystyle\left(\sum_{j=1}^\infty \|Ae_j\|^2 \right)^{\frac{1}{2}}$ задаватиме норму. Дійсно,
\begin{enumerate}[nosep,wide=0pt,label={\arabic*)}]
\item $|A| \geq 0$ -- все ясно. Тепер $|A| = 0 \iff \|Ae_j\|^2 = 0, \forall j \geq 1 \iff Ae_j = 0, \forall j \geq 1 \iff A = O$.
\item $|\lambda A| = \displaystyle\left(\sum_{j=1}^\infty \|(\lambda A)e_j\|^2 \right)^{\frac{1}{2}} = \left( |\lambda|^2 \sum_{j=1}^\infty \|Ae_j\|^2 \right)^{\frac{1}{2}} = |\lambda| |A|$.
\item $|A+B| = \displaystyle\left(\sum_{j=1}^\infty \|(A+B)e_j\|^2 \right)^{\frac{1}{2}} \leq \left(\sum_{j=1}^\infty \|Ae_j\|^2 + \sum_{j=1}^\infty \|Be_j\|^2 \right)^{\frac{1}{2}} \leq \left(\sum_{j=1}^\infty \|Ae_j\|^2 \right)^{\frac{1}{2}} + \left(\sum_{j=1}^\infty \|(Be_j\|^2 \right)^{\frac{1}{2}} = |A| + |B|$.
\end{enumerate}
Всі властивості норми доведені.\\
Нехай $A \in S_2(H)$, тоді оцінимо даний оператор:\\
$\|Ax\| = \displaystyle\| A \sum_{k=1}^\infty (x,e_k) e_k \| = \| \sum_{k=1}^\infty (x,e_k) Ae_k \| \leq \sum_{k=1}^\infty |(x,e_k)| \|Ae_k\| \overset{\text{н-ть Гьольдера}}{\leq} \left( \sum_{k=1}^\infty |(x,e_k)|^2 \right)^{\frac{1}{2}} \left( \sum_{k=1}^\infty \|Ae_k\|^2 \right)^{\frac{1}{2}} = \\ \overset{\text{р-ть Парсеваля}}{=} \|x\| |A|$.\\
Отже, ми довели, що $\|A\| \leq |A|$. До речі, рівність виконується $\iff \rank A = 1$ (TODO: довести)
\end{proof}

\begin{proposition}
$A \in S_2(H) \iff A^* \in S_2(H)$.\\
При цьому $|A| = |A^*|$.\\
\textit{Вправа: довести.}
\end{proposition}

Все це можна повторити в випадку операторів $A \in \mathcal{B}(H_1,H_2)$, мені зараз лінь.
\newpage

\section{Компактні оператори}
\subsection{Спектр та резольвента оператора}
Цей розділ про узагальнення поняття власних чисел та власних векторів операторів на нескінченний простір. На жаль, ці означення нам не підійдуть.

\begin{example}
Зокрема розглянемо оператор $A \colon L_2([0,1],\lambda) \to L_2([0,1],\lambda)$ таким чином: $(Af)(t) = t \cdot f(t)$. Якщо припустити, що $\mu$ -- власне число, тоді існує ненульова функція $f \in L_2([0,1],\lambda)$, для якої $Af = \lambda f$, тобто звідси $t f(t) = \mu f(t) \pmod \lambda$. Але тоді ми можемо отримати в результаті $f = 0 \pmod \lambda$, що суперечить.\\
Отже, такий оператор не містить власні числа.
\end{example}

\begin{definition}
Нехай $E$ -- нормований простір та $A \in \mathcal{B}(E)$.\\
Число $\lambda \in \mathbb{C}$ називатимемо \textbf{регулярною точкою} оператор $A$, якщо
\begin{align*}
\exists (A-\lambda I)^{-1} \overset{\text{позн.}}{=} R_\lambda(A)
\end{align*}
Існуючий оператор називають \textbf{резольвентою оператора $A$ в точці $\lambda$}.\\ 
Позначення: $\rho(A)$ -- множина всіх регулярних точок оператора $A$.
\end{definition}

\begin{definition}
Нехай $E$ -- нормований простір та $A \in \mathcal{B}(E)$.\\
\textbf{Спектром} оператора $A$ називають множину
\begin{align*}
\sigma(A) = \mathbb{C} \setminus \rho(A)
\end{align*}
По суті, спектр -- це просто доповнення до множини всіх регулярних точок.
\end{definition}

\begin{example}
Повертаючись до $A \colon L_2([0,1],\lambda) \to L_2([0,1],\lambda)$, що заданий як $(Af)(t) = t \cdot f(t)$, ми можемо довести, що $\sigma(A) = [0,1]$.
\end{example}

\begin{remark}
Можна трошки детально про це говорити. От нехай $A \colon E \to E$ -- лінійний оператор (не обов'язково навіть обмежений), оберемо $\lambda \in \mathbb{C}$. Будемо розглядати оператор $A - \lambda I$, який розбиває на кілька випадків:
\begin{enumerate}[wide=0pt,label={\arabic*)}]
\item $\ker (A-\lambda I) \neq \{0\}$, тоді існує ненульовий розв'язок $x$, для якого $(A-\lambda I)x = 0$. У цьому випадку справді $\lambda$ називають власним числом. Множину всіх власних чисел позначають за $\sigma_{\text{т}}(A)$ -- так званий \textbf{точковий спектр}.
\bigskip \\
Далі йде $\ker(A-\lambda I) = \{0\}$, який розбиває на ще три випадки:
\item $\Im(A-\lambda I) \neq X$, але при цьому $\overline{\Im(A-\lambda I)} = X$. Тоді число $\lambda$ потрапляє в сукупність $\sigma_{\text{н}}(A)$ -- так званий \textbf{неперервний спектр}.
\item $\overline{\Im(A-\lambda I)} \neq X$. Тоді число $\lambda$ потрапляє в сукупність $\sigma_{\text{з}}(A)$ -- так званий \textbf{залишковий спектр}.
\item $\Im(A-\lambda I) = X$. Тоді оскільки одночасно $\ker(A-\lambda I) = \{0\}$, то в нас буде існувати $(A-\lambda I)^{-1}$ -- та сама резольвента оператора $A$ в точці $\lambda$. Саме число $\lambda$ потрапляє в сукупність $\rho(A)$ -- множина \textbf{регулярних точок}.
\end{enumerate}
Таким чином, ми розбили $\mathbb{C} = \sigma_{\text{т}}(A) \sqcup \sigma_{\text{н}}(A) \sqcup \sigma_{\text{з}}(A) \sqcup \rho(A)$. Якщо об'єднати перші три, то ми отримаємо \textbf{спектр} оператора $A$, тобто $\sigma(A) = \sigma_{\text{т}}(A) \sqcup \sigma_{\text{н}}(A) \sqcup \sigma_{\text{з}}(A)$. Таким чином, $\mathbb{C} = \sigma(A) \sqcup \rho(A)$.
\end{remark}

\begin{example}
Зокрема маємо $A \colon l_2 \to l_2$, що задається як $A(x_1,x_2,x_3,\dots) = (0,x_1,x_2,\dots)$.\\
Нехай $(A-\lambda I)x = 0$, тобто $(-\lambda x_1, x_1 - \lambda x_2, x_2 - \lambda x_3, \dots) = (0,0,0,\dots)$. Звідси випливатиме $\lambda x_1 = 0$. Якщо $\lambda = 0$, то отримаємо $x_1 = x_2 = \dots = 0$. Якщо $x_1 = 0$, то звідси $\lambda x_2 = 0$, тут аналогічно не можемо брати $\lambda \neq 0$, тож $x_2 = 0$, \dots\\
Коротше, не існує власних чисел, тобто маємо $\sigma_{\text{т}}(A) = \emptyset$.\\
Тепер розглянемо спряжений оператор $A^* \colon l_2 \to l_2$, що задається як $A^*(y_1,y_2,\dots) = (y_2,y_3,\dots)$. Ми знаємо, що в гілбертовому просторі $l_2 = \ker (A-\lambda I) \oplus \Im(A-\lambda I)^*$.
\end{example}

\begin{theorem}
Нехай $E$ -- нормований простір та $A \in \mathcal{B}(E)$. Тоді $\sigma(A)$ -- замкнена множина та при цьому $\sigma(A) \subset B[0, \|A\|]$.
\end{theorem}

\begin{proof}
Припустимо, що $\lambda \notin B[0, \|A\|]$, тобто це означає, що $|\lambda| < \|A\|$. Тоді звідси маємо:\\
$(A-\lambda I) = -\dfrac{1}{\lambda}\left(I - \dfrac{A}{\lambda}\right)$.\\
Оскільки $\left\| \dfrac{A}{\lambda} \right\| < 1$, то тоді існує $\left(I - \dfrac{A}{\lambda}\right)^{-1}$, внаслідок чого існуватиме $(A-\lambda I)^{-1}$. Тобто звідси $\lambda \in \rho(A) \implies \lambda \notin \sigma(A)$. Отже, довели щойно $\sigma(A) \subset B[0, \|A\|]$.\\
Для доведення замкненості $\sigma(A)$ досить довести відкритість $\rho(A)$. Нехай $\lambda_0 \in \rho(A)$, тоді існує резольвента $(A-\lambda_0 I)^{-1}$. Ми хочемо довести, що $\lambda_0$ буде внутрішньою точкою. Для цього розглянемо оператор $A-\lambda I = (A-\lambda_0 I) - (\lambda - \lambda_0)I$. Оператор $(A-\lambda_0 I)$ оборотний, а на другий оператор хочемо оцінку $\| (\lambda - \lambda_0) I\| \leq \| (A-\lambda_0 I)^{-1} \|^{-1}$. Якщо покласти $r = \|(A-\lambda_0 I)^{-1}\|^{-1}$, то тоді $|\lambda-\lambda_0| < r$. За \thref{sum_of_operators_and_inverse}, оператор $A-\lambda I$ -- оборотний, тобто $\lambda \in \rho(A)$. Довели $B(\lambda_0,r) \subset \rho(A)$.
\end{proof}

\begin{remark}
При $\lambda \in B(\lambda_0,r)$ матимемо рівномірну обмеженість $\|(A-\lambda I)^{-1}\|$.
\end{remark}

\begin{proof}
Дійсно, трошечки розпишемо необхідне:\\
$A -\lambda I = (A - \lambda_0 I) - (\lambda - \lambda_0) I = (A-\lambda_0 I) \left(I - (\lambda - \lambda_0)(A -\lambda_0 I)^{-1}\right)$.\\
Візьмемо оборотність з двох боків:\\
$(A - \lambda I)^{-1} = \left[ I - (\lambda - \lambda_0)(A-\lambda_0 I)^{-1}\right]^{-1} (A-\lambda_0 I)^{-1}$.\\
$\|(A-\lambda I)^{-1}\| \leq \| (A-\lambda_0 I)^{-1} \| \| (I- (\lambda - \lambda_0)(A-\lambda_0 I)^{-1})^{-1}\| \boxed{\leq}$\\
Тимчасово для зручності позначу $(\lambda-\lambda_0)(A - \lambda_0 I)^{-1} = T$. Ми заздалегідь зауважимо, що $\|T\| = |\lambda-\lambda_0| \|(A-\lambda_0I)^{-1}\| < r \|(A-\lambda_0 I)^{-1}\|$. Тепер зробимо оцінку\\
$\|(I-T)^{-1}\| = \| I + T + T^2 + \dots \| \leq \|I\| + \|T\| + \|T\|^2 + \dots = \dfrac{1}{1- \|T\|} < \dfrac{1}{1 - r \|(A-\lambda_0 I)^{-1} \|}$.\\
$\boxed{\leq} \dfrac{\|(A-\lambda_0 I)^{-1}\|}{1 - r \|(A-\lambda_0 I)^{-1}\|}$.
\end{proof}

\begin{theorem}[Тотожність Гілберта]
Нехай $E$ -- нормований простір та $A \in \mathcal{B}(E)$. Також нехай $z_1,z_2 \in \rho(A)$. Тоді \\ $R_{z_1}(A) - R_{z_2}(A) = (z_1-z_2) R_{z_1}(A) R_{z_2}(A)$.
\end{theorem}

\begin{remark}
Щоб легше було запам'ятати. Резольвенту $R_{\lambda}(A) = (A-\lambda I)^{-1}$ можна асоціювати з числом $\dfrac{1}{A-\lambda}$ (тут $A,\lambda$ теж числа). Тоді якщо розписати різницю, то\\
$\dfrac{1}{A - z_1} - \dfrac{1}{A - z_2} = \dfrac{z_1-z_2}{(A-z_1)(A-z_2)} = (z_1-z_2) \dfrac{1}{A-z_1} \dfrac{1}{A-z_2}$.
\end{remark}

\begin{proof}
Розглянемо ось таку різницю:\\
$(A - z_2 I) - (A - z_1I) = (z_1 - z_2)I$.\\
Домножимо ліворуч на $R_{z_1}(A)$ та праворуч на $R_{z_2}(A)$ -- отримаємо:\\
$R_{z_1}(A) (A- z_2I) R_{z_2}(A) - R_{z_1}(A) (A - z_1I) R_{z_2}(A) = (z_1-z_2) R_{z_1}(A) R_{z_2}(A)$.\\
Згадавши, що з себе представляє резольвента, отримаємо відповідну рівність:\\
$R_{z_1}(A) - R_{z_2}(A) = (z_1-z_2) R_{z_1}(A) R_{z_2}(A)$.
\end{proof}

\begin{corollary}
Резольвенти оператора $A$ комутують між собою. Тобто $R_{z_1}(A) R_{z_2}(A) = R_{z_2}(A) R_{z_1}(A)$.
\end{corollary}

\begin{theorem}
Відображення $z \mapsto R_z(A)$ буде неперервним на $\rho(A)$.
\end{theorem}

\begin{proof}
Дійсно, маємо $z_0 \in \rho(A)$. При $z \to z_0$ хочемо довести, що $R_{z}(A) \to R_{z_0}(A)$.\\
$\| R_z(A) - R_{z_0}(A) \| = |z - z_0| \| R_z(A) R_{z_0}(A) \| \leq |z-z_0| C^2 \to 0$.\\
Остання оцінка отрималася в результаті того, що в околі регулярної точки резольвенти рівномірно обмежені за першим зауваження.
\end{proof}

\begin{theorem}
Нехай $E$ -- нормований простір та $A \in \mathcal{B}(E)$. Також нехай $z \in \rho(A)$. Тоді\\
$\exists \displaystyle\lim_{n \to 0} \dfrac{R_{z+h}(A) - R_z(A)}{h} = R_z^2(A)$.\\
Тобто ця теорема каже про диференційованість резольвенти в регулярних точках.
\end{theorem}

\begin{proof}
$\dfrac{R_{z+h}(A) - R_z(A)}{h} = \dfrac{1}{h} (z+h-z) R_{z+h}(A) R_z(A) \overset{R_z \text{ -- неперервна}}{\to} R_z(A) R_z(A) = R_z^2(A)$.
\end{proof}

\begin{corollary}
Нехай $E$ -- нормований простір. Зафіксуємо будь-яку точку $x \in E$ та функціонал $l \in E'$. Розглянемо комплекснозначну функцію $f_{x,l}(z) = l(R_z(A)x)$ на $\rho(A)$. Тоді така функція -- аналітична на $\rho(A)$.
\end{corollary}

\begin{theorem}
Нехай $E$ -- нормований простір та $A \in \mathcal{B}(E)$. Тоді $\sigma(A) \neq \emptyset$.
\end{theorem}

\begin{proof}
!Припустимо, що $\sigma(A) = \emptyset$, тобто звідси $\rho(A) = \mathbb{C}$, а тому функція $l(R_z(A)x)$ -- аналітична на всій комплексній площині. Виникає бажання застосувати теорему Луівілля, але спочатку треба показати обмеженість.\\
Для цього треба рівномірна обмеженість $\|R_z(A)\|$. Маємо два випадки:\\
1) $z \in B[0, 2\|A\|]$.\\
У цьому випадку $\|R_z(A)\|$ -- неперервна функція на $B[0, 2\|A\|]$, тому звідси $\exists c_1: \forall z \in B[0,2\|A\|]: \|R_z(A)\| \leq c_1$.
\bigskip \\
2) $z \notin B[0, 2\|A\|]$.\\
$\|R_z(A)\| = \|(A-zI)^{-1}\| = \dfrac{1}{|z|} \left\|  \left(I - \dfrac{A}{z}\right)^{-1} \right\| \overset{\left\| \frac{A}{z}\right\| < \frac{1}{2}}{=} \left\| I + \dfrac{A}{z} + \dfrac{A^2}{z^2} + \dots \right\| \leq \dfrac{1}{|z|} \dfrac{1}{1 - \dfrac{1}{|z|} \|A\|} \leq \dfrac{2}{|z|} \leq \dfrac{1}{\|A\|}$.\\
Отже, дійсно на $\mathbb{C}$ маємо рівномірну обмеженість $\|R_z(A)\|$. При цьому із того ланцюге нерівностей ми отримали $\|R_z(A)\| \to 0$ при $z \to \infty$.\\
Тепер покажемо обмеженість комплекснозначної функції:\\
$|f_{x,l}(z)| = |l(R_z(A)x)| \leq \|l\| \| R_z(A)x\| \leq \|l\| \|R_z(A) \| \|x\| \leq C \|l\| \|x\|$.\\
Отже, за теоремою Луівілля, $f_{x,l}(z) = C_{x,l}$ -- стала функція. Тільки оскільки $|f_{x,l}(z)| \to 0$ при $z \to \infty$, то звідси $f_{x,l}(z) = 0$. Зокрема звідси отримаємо $R_z(A) = 0$. Тобто ми отримали оборотний оператор, який нульовий -- суперечність!
\end{proof}

\subsection{Компактні оператори}
\begin{definition}
Задано $E_1,E_2$ -- нормовані простори та $A \colon E_1 \to E_2$ -- лінійний оператор (не обов'язково обмежений).\\
Оператор $A$ називається \textbf{компактним} (інколи називають \textbf{цілком неперервним}), якщо
\begin{align*}
\forall M \subset E_1 \text{ -- обмежена}: A(M) \text{ -- передкомпакт.}
\end{align*}
Коротше кажучи, якщо образ довільної обмеженої мноижин $E_1$ -- передкомпакт.\\
Позначення: $\mathcal{K}(E_1,E_2)$ -- множина всіх компактних операторів.
\end{definition}

\begin{proposition}
Задано $E_1,E_2$ -- нормовані простори.\\
$A \in \mathcal{K}(E_1,E_2) \iff A(B[0;1])$ -- передкомпакт.
\end{proposition}

\begin{proof}
\rightproof \textit{Миттєво з означення.}
\bigskip \\
\leftproof Дано: $A(B[0;1])$ -- передкомпакт. Нехай $M \subset E_1$ -- деяка обмежена множина. Тоді існує відкрита куля $B(x;R) \supset M$. Доведемо, що $A(M)$ буде передкомпактом.\\
Нехай $\sequence{y_n} \subset A(M)$ -- обмежена, кожний $y_n = Ax_n,\ x_n \in M$. Розглянемо послідовність $\sequence{\dfrac{y_n - Ax}{R}} \subset A(B[0;1])$, яка буде також обмеженою. Тоді можна виділити збіжну підпослідовність $\left( \dfrac{y_{n_k} - Ax}{R}\right)_{k=1}^\infty$, зокрема збіжною буде підпослідовність $(y_{n_k})_{k=1}^\infty$.
\end{proof}

\begin{proposition}
$\mathcal{K}(E_1,E_2) \subset \mathcal{B}(E_1,E_2)$.\\
Іншими словами, всі компактні оператори -- обмежені автоматично.
\end{proposition}

\begin{proof}
Дійсно, нехай $A \in \mathcal{K}(E_1,E_2)$, тоді $A(B[0;1])$ буде передкомпактом, яка автоматично обмежена. Нехай $x \in E_1$. Тоді зауважимо, що елемент $\dfrac{x}{\|x\|} \in B[0;1]$. Звідси випливає, що $\left| A\left( \dfrac{x}{\|x\|} \right)\right| \leq C \implies \|Ax\| \leq C\|x\|$. Таким чином, $A \in \mathcal{B}(E_1,E_2)$.
\end{proof}

\begin{example}
У зворотному це не працює.\\
Дійсно, розглянемо одиничний оператор $I \in \mathcal{B}(E)$. Стверджується наступне:\\
$I \in \mathcal{K}(E) \iff \dim E < \infty$.
\end{example}
\noindent
Але щоб мені це довести, треба відволіктися трошки та довести кілька тверджень.

\begin{theorem}[Лема Ріса]
Задано $E$ -- нормований простір та $G \subsetneq E$ -- замкнений підпростір. Тоді $\forall \varepsilon > 0: \exists y_\varepsilon \notin G, \|y_\varepsilon\| = 1: \forall x \in G: \|y_\varepsilon - x\| > 1-\varepsilon$.
\end{theorem}

\begin{remark}
Цю теорему ще наизвають теорему про існування майже ортогонального вектора.\\
Ми обговорили цю ситуацію в частинному випадку нормованого простору, саме в гілбертовому просторі $H$ та замкненому підпросторі $G \subset H$ (див. \rmref{equivalent_existence_of_orthogonal_unit_vector}).\\
У загальному нормованому просторі $E$ можуть не існувати такі вектори $y \notin G$, щоб $\|y - x\| \geq 1$, тобто можуть не існувати ортогональні вектори, так би мовити. Однак ми можемо скільки завгодно наблизитися до одиниці, тобто $\exists y_\varepsilon \notin G: \|y_\varepsilon\| = 1: \|y_\varepsilon - x\| > 1-\varepsilon$. Це й пояснює альтернативну назву "майже ортогональний".
\end{remark}

\begin{proof}
Оберемо елемент $z \notin G$. Позначимо $\delta = \rho(z,G)$, який в свою чергу $\delta > 0$. Якби $\delta = 0$, то за критерієм інфімуму ми би знайшли послідовність $\sequence{y_n}$, для якої $\|y_n - z \| < \dfrac{1}{n} \to 0$ -- ми би отримали, що $z$ -- гранична $G$, де водночас $z \notin G$ -- суперечить за рахунок замкненості $G$.\\
Звідси $\forall \eta > 0: \exists x_\eta \in G: \delta \leq \|z - x_\eta\| < \delta + \eta$ за критерієм інфімуму.\\
Нехай $\varepsilon > 0$. Ми підкрутимо таке значення $\eta$, щоб $\dfrac{\eta}{\delta + \eta} = \varepsilon$. Доведемо, що $y_\varepsilon = \dfrac{z-x_\eta}{\|z - x_\eta \|}$ буде шуканим вектором. Цілком зрозуміло, що $y_\varepsilon \notin G,\ \|y_\varepsilon\| = 1$. Доведемо нерівність.\\
$\|y_\varepsilon - x\| = \left\| \dfrac{z-x_\eta}{\|z - x_\eta \|} - x \right\| = \dfrac{1}{\|z- x_\eta\|} \| z - (x_\eta + x \|z-x_\eta\|) \|$.\\
Зазначимо, що вектор $x_\eta + x \|z-x_\eta\| \in G$, тому звідси $\|z - (x_\eta + x \|z-x_\eta\|) \| \geq \rho(z,G) = \delta$. Повертаючись до ланцюга рівностей, отримаємо:\\
$\|y_\varepsilon - x\| \geq \dfrac{\delta}{\|z - x_\eta\|} > \dfrac{\delta}{\delta + \eta} = 1 - \dfrac{\eta}{\delta + \eta} = 1- \varepsilon$.
\end{proof}

\begin{theorem}
Задано $E$ -- нормований простір.\\
$\dim E < \infty \iff$ кожна підмножина $E$ -- передкомпактна.
\end{theorem}

\begin{proof}
\leftproof Дано: $\dim E = \infty$. Ми хочемо довести, що замкнена куля $B[0;1]$ не буде компактом.\\
У якості $x_1$ оберемо будь-який вектор одиничної сфери та покладемо $G_1 = \linspan\{x_1\}$. За лемою Ріса, знайдеться вектор $x_2 = y_{\frac{1}{2}} \notin G_1$ такий, що $\|x_2\| = 1$ та $\forall x \in G_1: \|x_2-x\| > \dfrac{1}{2}$, зокрема звідси $\|x_2 - x_1\| > \dfrac{1}{2}$.\\
Покладемо $G_2 = \linspan\{x_1,x_2\}$. Тоді знову за лемою Ріса, знайдеться вектор $x_3 = y_{\frac{1}{2}}$ такий, що $\|x_3\| = 1$ та $\forall x \in G_2: \|x_3-x\| > \dfrac{1}{2}$. Зокрема $\|x_3 - x_1\| > \dfrac{1}{2},\ \|x_3 - x_2\| > \dfrac{1}{2}$.\\
\vdots \\
У силу того, що $\dim E = \infty$, то цей процес буде продовжуватися. Побудуємо послідовність $\sequence{x_n} \subset B[1;0]$ таку, що $\forall m,n \geq 1: \| x_m - x_n \| > \dfrac{1}{2}$. Така обмежена послідовність не містить збіжної підпослідовності.
\bigskip \\
\rightproof Дано: $\dim E < \infty$, тоді відомо, що $E \cong \mathbb{C}^n$. Будь-яка обмежена підмножина $\mathbb{R}^n$ -- передкомпактна за Больцано-Ваєрштраса.
\end{proof}
\noindent
Повертаємося назад до розмов про компактні простори.

\begin{proposition}
Припустимо, що $\dim E_2 < \infty$. Тоді $\mathcal{K}(E_1,E_2) = \mathcal{B}(E_1,E_2)$.\\
Інакше кажучи, за додатковими умовами обмежений оператор може бути компактним.
\end{proposition}

\begin{proof}
Вкладення $\mathcal{K}(E_1,E_2) \subset \mathcal{B}(E_1,E_2)$ в нас уже є. Треба тепер тільки зворотний бік.\\
Припустимо, що $A \in \mathcal{B}(E_1,E_2)$. Оберемо довільну обмежену множину $M \subset E_1$. Тоді оскільки $\dim E_2 < \infty$, то $A(M) \subset E_2$ буде передкомпактною автоматично. Отже, $A \in \mathcal{K}(E_1,E_2)$.
\end{proof}

\begin{example}
Розглянемо оператор $A \colon C([0,1]) \to C([0,1])$, що задається як $(Af)(t) = \displaystyle\int_0^1 k(t,s) f(x)\,ds$, де функція $k(t,s) = a(t)b(s),\ a,t \in C([0,1])$. Покажемо, що це -- компактний оператор.\\
Дійсно, $(Af)(t) = a(t) \displaystyle\int_0^1 b(s)f(s)\,ds = C \cdot a(t)$. Тобто звідси $\dim \Im A = 1$, що підтвердує компактність оператора.
\end{example}

\subsection{Властивості компактного оператора}
\begin{theorem}
\label{convergence_to_compact}
Задано $E$ -- банахів простір. Розглянемо послідовність $\sequence{A_n} \subset \mathcal{K}(E)$, яка збігається за нормою (із $\mathcal{B}(E)$) до оператора $A$. Тоді $A \in \mathcal{K}(E)$.
\end{theorem}

\begin{proof}
Фіксуємо обмежену послідовність $\sequence{x_n}$ із $E$. Ми хочемо зі послідовності $\sequence{A x_n}$ відокремити збіжну підпослідовність. Будемо діагональним методом доводити.\\
$A_1 \in \mathcal{K}(E)$, тому зі послідовності $\sequence{A_1x_n}$ відокремимо збіжну підпослідовність $\sequence{A_1 x_{n_1}}$.\\
$A_2 \in \mathcal{K}(E)$, тому зі послідовності $\sequence{A_2x_{n_1}}$ відокремимо збіжну підпослідовність $\sequence{A_2 x_{n_2}}$.\\
\vdots \\
Розглянемо діагональну послідовність $\sequence{x_{nn}}$. Нам треба довести, що саме підпослідовність $\sequence{A x_{nn}}$ буде фундаментальною, а внаслідок банаховості $E$ -- збіжною. Почнемо потроху оцінювати норму.\\
Перед цим треба зауважити, що $\sequence{A_k x_{nn}}$ при всіх $k \in \mathbb{N}$ буде збіжною, просто тому що $\sequence{x_{nn}} \subset \sequence{x_{n_k}}$ та $\sequence{A_k x_{n_k}}$ збіжна.\\
$\| A x_{mm} - A x_{nn} \| \leq \|A x_{mm} - A_k x_{mm} \| + \| A_k x_{mm} - A_k x_{nn} \| + \|A_k x_{nn} - A_k x_{mm} \| \overset{\mathcal{K}(E) \subset \mathcal{B}(E)}{\leq} \\
\leq \|A - A_k\| (\|x_{mm}\| + \|x_{nn}\|) + \| A_k x_{nn} - A_k x_{mm} \| \boxed{<}$\\
Оскільки $\sequence{x_n}$ обмежена, то обмеженою буде $\sequence{x_{nn}}$, а там звідси $\|x_{nn}\| \leq c, \forall n \geq 1$.\\
Далі оскільки $\sequence{\|A_k\|}$ -- збіжна, то існує такий номер $k_0$, для якого $\|A - A_{k_0}\| < \dfrac{\varepsilon}{3c}$. Також в силу збіжності $\sequence{A_{k_0} x_{nn}}$ ми отримаємо, що $\exists N: \forall n,m \geq N: \|A_{k_0} x_{nn} - A_{k_0} x_{mm} \| < \dfrac{\varepsilon}{3}$.\\
$\boxed{<} \dfrac{\varepsilon}{3c} \cdot 2c + \dfrac{\varepsilon}{2} = \varepsilon$.
\end{proof}

\begin{theorem}
$\mathcal{K}(E)$ -- підпростір $\mathcal{B}(E)$ за умовою, що $E$ -- банахів.\\
(TODO: провести доведення)
\end{theorem}

\begin{remark}
Ці дві теореми працюють для операторів з $\mathcal{K}(E_1,E_2)$, тільки тут $E_2$ має бути банаховим.
\end{remark}

\begin{theorem}
Нехай $A \in \mathcal{K}(E)$. Тоді $A^* \in \mathcal{K}(E')$.
\end{theorem}

\begin{proof}
Нехай $\sequence{l_n} \subset E'$ -- обмежена послідовність. Хочемо із $\sequence{A^*l_n}$ відокремити збіжну підпослідовність. Зауважимо наступне:\\
$\displaystyle \| A^*l_n \| = \sup_{ \|y\| = 1} |(A^*l_n)(y)| = \sup_{y \in \mathcal{S}(1;0)} |l_n(Ay)| = \sup_{z \in A(S(1;0))} |l_n(z)|$.\\
Для доведення теореми нам досить буде встановити передкомпактність множини $\{l_n\}_{n=1}^\infty \subset C(\overline{AS(1;0)})$. Перевіримо умови виконання теореми Арцела-Асколі.\\
$\forall n \ge 1: |l_n(z)| \leq \|l_n\| \|z\| \leq c \cdot c_1$ -- виконується рівномірна обмеженість.\\
$\forall n \geq 1: |l_n(z_1) - l_n(z_2)| \leq c\|z_1 - z_2\|$ -- виконується умова одностайної неперервності.\\
Отже, існує збіжна підпослідовність $\sequence{l_{n_k}}$, причому $\| l_{n_k} - l_{n_m} \| = \displaystyle\max_{z \in \overline{AS(1;0)}} |l_{n_k}(z) - l_{n_m}(z)| \to 0$. Внаслідок чого $\|A^* l_{n_k} - A^* l_{n_m}\| \to 0$.
\end{proof}

\subsection{Компактні оператори в сепарабельному гілбертовому просторі}
Розглянемо $H$ -- сепарабельний гілбертів простір, оберемо базис $\{e_j\}_{j=1}^\infty$. Розглянемо оператор $A \in S_2(H)$, тобто $\displaystyle\sum_{j,k=1}^\infty |a_{jk}|^2 < \infty$. Позначимо $\mathbb{A}$ за матрицю оператора $A$.

\begin{theorem}
$S_2(H) \subsetneq K(H)$.
\end{theorem}

\begin{proof}
Розглянемо оператор $P_j$ -- проєктор на підпростір, породжений базисом $\{e_1,\dots,e_j\}$. Розглянемо послідовність операторів $A_j = P_j A$, їм відповідають матриці $\mathbb{A}_j$ -- та сама матрица $\mathbb{A}$, тільки, починаючи з $j+1$ рядка, будуть одні нулі. Зауважимо також, що образ $A_j$ буде скінченним, причому $A_j = P_j A \in \mathcal{B}(H)$ (як добуток обмежених), тому $A_j \in \mathcal{K}(H)$. Доведемо, що $\|A - A_j\| \to 0$ при $j \to \infty$.\\
Спочатку доведемо, що $|A-A_j| \to 0$ при $j \to \infty$ за нормою Гілберта-Шмідта.\\
$|A-A_j|^2 = \displaystyle\sum_{k=j+1}^\infty \left( \sum_{l=1}^\infty |a_{kj}|^2 \right) \to 0$ при $j \to \infty$ як залишковий ряд. При цьому ми маємо $\|A\| \leq |A|$ (TODO: чому?), тоді звідси випливає, що $\|A-A_j\| \to 0$. Отже, за \thref{convergence_to_compact}, оператор $A \in \mathcal{K}(H)$.
\bigskip \\
Тепер розглянемо оператор $A$, яка має матрицю $\mathbb{A} = \diag\left(\lambda_1,\lambda_2,\dots\right)$. Зауважимо, що $A \in S_2(H) \iff \displaystyle\sum_{j=1}^\infty |\lambda_j|^2 <\infty$. Також маємо $A \in \mathcal{K}(H) \iff \lambda_j \to 0$.
\end{proof}

\begin{theorem}
Нехай $A \in \mathcal{K}(H)$. Тоді існує послідовність операторів $\sequence{A_n}$, для яких $A_n \to A$, причому $\dim \Im A_n < \infty$.
\end{theorem}

\begin{proof}
Аналогічно розглянемо оператор $P_j$, як було вище, а також оператор $A_j = P_j A$. Доведемо, що $A_j \to A$, тобто ми хочемо $\|A - A_j\| = \|(I-P_j) A\| \to 0$.\\
!Припустимо, що $\|(I-P_j)A \| \not\to 0$, тобто існує $c > 0$, для якої $\|(I - P_j) A\| \geq c$, тобто іншими словами $\displaystyle\sup_{\|x\| = 1} \|(I-P_j)A x \| \geq c$. Тоді можна відокремити послідовність $(x_j)_{j=1}^\infty$ з $\|x_j\| = 1$, для яких $\|(I-P_j) Ax_j\| \geq \dfrac{c}{2}$. Для зручності позначимо $y_j = Ax_j$. Оскільки $A \in \mathcal{K}(H)$, то можна відокремити збіжну підпослідовність $(y_{j_k})_{k=1}^\infty$, де $y_{j_k} \to y$. Власне, звідси $\|(I-P_j) y\| \geq \dfrac{c}{2}$. Проте раніше доводили (TODO: ?), що $I-P_j \tostrong O$ при $j \to \infty$, тобто $(I-P_j)y \to 0$ -- суперечність!
\end{proof}

\subsection{Спектри в компактних операторах}
\begin{theorem}[Альтернатива Фредгольма]
Задано $E$ -- банахів простір, оператор $A \in \mathcal{K}(E)$ та $\lambda \in \mathbb{C} \setminus \{0\}$. Тоді виконується рівно одна з умов:
\begin{enumerate}[nosep,wide=0pt,label={\arabic*)}]
\item $\lambda$ -- власне число $A$;
\item $\lambda$ -- регулярна точка $A$.
\end{enumerate}
\end{theorem}

Перед доведення теореми треба навести корисну лему.
\begin{lemma}
Нехай $A \in \mathcal{K}(E)$ та $\lambda$ -- не власне число. Тоді існує $c > 0$ такий, що $\forall x \in E: \|(A-\lambda I)x\| \geq c\|x\|$.
\end{lemma}

\begin{proof}
!Припустимо, що не виконується нерівність. Тоді для $\dfrac{1}{n} \in \mathbb{N}$ існують точки $z_n \in E$, для яких $\| (A-\lambda I)z_n \| < \dfrac{1}{n}\|z_n\|$. Ми будемо розглядати точки $x_n = \dfrac{z_n}{\|z_n\|}$, тобто з одиничною нормою (там, де раптом $z_n = 0$, можна просто пропустити), звідси $\|(A-\lambda I)x_n\| < \dfrac{1}{n} \to 0$.\\
Позначимо точку $y_n = Ax_n$. Зауважимо, що оскільки $\sequence{x_n}$ обмежена та $A \in \mathcal{K}(E)$, то тоді існує збіжна підпослідовність $y_{n_k} \to y$. Також справедлива така оцінка:\\
$|\lambda| \overset{\text{з одного боку}}{=} \| \lambda x_n\| = \overset{\text{з іншого боку}}{=} \| (\lambda I-A)x_n + Ax_n \| \leq \|y_n\| + \|(A-\lambda I)x_n\|$.\\
$\implies \|y_n\| \geq |\lambda| - \|(A-\lambda I)x_n\|$.\\
Оскільки це виконано $\forall n \in \mathbb{N}$, то зокрема й для $n_k$, а далі при $k \to \infty$ отримаємо $\|y\| \geq |\lambda|$.\\
Утім із іншого боку, зараз доведемо, що $y = 0$. Дійсно, маємо\\
$0 \leq \|(A-\lambda I)y_{n_k} \| = \|(A-\lambda I) A x_{n_k}\| = \|A(A-\lambda I)x_{n_k}\|$.\\
Якщо спрямувати $k \to \infty$, то ми отримаємо $0 \leq \|(A-\lambda I)y\| \leq 0$, що свідчить про $(A-\lambda I)y = 0$. Отже, $y$ -- власне число оператора $A$ -- суперечність!
\end{proof}

Повернімося до доведення альтернативи Фредгольма.
\begin{proof}
Якщо $\lambda$ -- власне число, то закінчили доведення.\\
Тому нехай $\lambda$ таким не є. За щойно доведеною лемою, $\|(A-\lambda I)x\| \geq c \|x\|$ при всіх $x \in E$ для деякого $c > 0$. Тоді звідси (TODO: ?) існує $(A-\lambda I)^{-1} \colon \Im(A-\lambda I) \to E$. За теоремою про замкнений графік, $\Im(A-\lambda I)$ буде замкненою множиною. Нам залишилося показати, що $\Im(A-\lambda I) = E$.\\
!Позначимо $\Im(A-\lambda I) = E_1$ та припустимо, що $E_1 \neq E$.\\
Розглянемо образ $(A-\lambda I)E_1 \overset{\text{позн.}}{=} E_2$. Слід зазначити, що $E_2 \neq E_1$ (адже якби $E_2 = E_1$, то оскільки $(A-\lambda I)^{-1} \colon E \to E_1$ -- бієкція, то $E_1 = (A - \lambda I)E_1 = E$, що не наш випадок).\\
Розглянемо образ $(A-\lambda I)E_2 \overset{\text{позн.}}{=} E_3$. Слід зазначити, що $E_3 \neq E_2$ (аналогічно).\\
\vdots \\
Отримаємо ланцюг вкладень $E \supset E_1 \supset E_2 \supset E_3 \supset \dots$ На кожному з цих вкладень застосуємо лему Ріса. Для $\varepsilon = \dfrac{1}{2}$ будуть існувати точки $x_j \in E_j \setminus E_{j+1}$, причому $\|x_j\| = 1$, для яких $\forall y \in E_{j+1}: \|x_j + y\| > 1 - \dfrac{1}{2}$. Саме ці точки нам дадуть сказати, що $\sequence{A x_n}$ не містить збіжної підпослідовності. Справді, при $m > n$ маємо\\
$\| Ax_n - Ax_m \| = \| \lambda x_n + Ax_n - \lambda x_n - \lambda x_m - Ax_m + \lambda x_m \| = \| \lambda x_n + (A-\lambda I)x_n - \lambda x_m - (A-\lambda I)x_m\| \geq \lambda \cdot \dfrac{1}{2}$.\\
Пояснення до нерівності: $(A-\lambda I)x_n \in E_{n+1}$, далі $\lambda x_m \in E_m \subset E_{n+1}$ та $(A-\lambda I) x_m \in E_{m+1} \subset E_n$. Тобто останні три доданки -- це елемент з $E_{n+1}$, а далі нерівність з леми Ріса.\\
Проте $A \in \mathcal{K}(E)$, тому $\sequence{A x_n}$ зобов'язана мати збіжну підпослідовність -- суперечність!
\end{proof}

\begin{theorem}
Задано $E$ -- банахів та $A \in \mathcal{K}(E)$. Тоді:
\begin{enumerate}[nosep,wide=0pt,label={\arabic*)}]
\item $\sigma(A)$ -- не більш, ніж зліченна множина;
\item Якщо $\lambda \in \sigma(A) \setminus \{0\}$, то $\dim L_\lambda < \infty$ (тобто $\lambda$ -- власне число скінченної кратності);
\item Якщо $\dim E = \infty$, то $0 \in \sigma(A)$ -- єдина гранична точка $\sigma(A)$.
\end{enumerate} 
\end{theorem}

\begin{proof}
Нехай $\lambda \in \sigma(A) \setminus \{0\}$, тоді за альтернативою Фредгольма, $\lambda$ -- власне число $A$. Звузимо оператор $A|_{L_\lambda} \overset{\text{насправді}}{=} \lambda I$. Оскільки $A$ компактний, то тоді обо'язково має бути $\dim L_\lambda < \infty$.
\bigskip \\
Оберемо $r > 0$ та покажемо, що існує скінченна кількість точок $\lambda_j \in \sigma(A)$, які лежать поза межами кола $B(0;r)$.\\
!Припустимо, що там нескінченна кількість точок. Ми можемо взяти різний набір $\lambda_j \in \sigma(A), j \geq 1$, для яких $|\lambda_j| \geq r$. Оскільки це власні числа, то $Ax_j = \lambda_j x_j$ для деяких $(x_j)_{j=1}^\infty$. Зауважимо, що будь-який скінченний набір $\{x_1,\dots,x_n\}$ буде лінійно незалежною. Адже, припустивши, що наступна $\{x_1,\dots,x_n,x_{n+1}\}$ лінійно залежна, тобто $x_{n+1} = \displaystyle\sum_{j=1}^n c_j x_j$, то після дії оператора $A$ отримаємо:\\
$\displaystyle \sum_{j=1}^n c_j \lambda_{n+1} x_j = \lambda_{n+1} x_{n+1} \overset{\text{із одного боку}}{=} Ax_{n+1} \overset{\text{із іншого боку}}{=} \sum_{j=1}^n c_j \lambda_j x_j$.\\
$\displaystyle\sum_{j=1}^n (c_j \lambda_{n+1} - c_j \lambda_j) x_j = 0 \implies c_j \lambda_{n+1} - c_j \lambda_j = 0 \implies c_j = 0 \implies x_{n+1} = 0$ (не наш випадок).\\
Позначимо $E_n = \linspan\{x_1,\dots,x_n\}, n \geq 1$, тоді маємо вкладення $E_1 \subset E_2 \subset E_3 \subset \dots$ За лемою Ріса, існують вектори $y_j \in E_j \setminus E_{j-1}$, причому $\|y_j\|= 1$ та $\forall y \in E_{j-1}: \|y_j + y\| \geq \dfrac{1}{2}$. Аналогічними міркуваннями (як у альтернтиві Фредгольма), отримаємо $\|Ax_n - Ax_m\| \geq \dfrac{r}{2}$, тоді не можна відокремити від $\sequence{Ax_n}$ збіжну підпослідовність -- суперечність, бо $A \in \mathcal{K}(E)$!\\
Отже, 1) довели (TODO: додумати). Якщо $\dim E = \infty$, 
\end{proof}

\subsection{Спектральний радіус оператора}
\subsubsection{Степеневі ряди з операторними коефіцієнтами}
\begin{definition}
Маємо $E$ -- банахів та $\sequence{A_n} \subset \mathcal{B}(E)$,  далі розглянемо \textbf{степеневий ряд з операторними коефіцієнтами}
\begin{align*}
\sum_{k=0}^\infty z^k A_k,\ z \in \mathbb{C}
\end{align*}
Цей ряд буде \textbf{збіжним в точці $z_0 \in \mathbb{C}$}, якщо послідовність часткових сум збігається за нормою в $\mathcal{B}(E)$.
\end{definition}

\begin{lemma}
Нехай для деякого $z_0 \neq 0$ послідовність $\sequence{z_0^n A_n}$ -- обмежена. Тоді при $|z| < |z_0|$ ряд $\displaystyle\sum_{k=0}^\infty z^k A_k$ -- збіжний.
\end{lemma}

\begin{proof}
Маємо нерівність $\displaystyle \left\| \sum_{k=n+1}^{n+p} z^k A_k \right\| \leq \sum_{k=n+1}^{n+p} |z|^k \|A_k\|$. Значить, нам досить буде дослідити збіжність ряда $\displaystyle\sum_{n=0}^\infty |z|^n \|A_n\|$. Ми маємо $|z|^n \|A_n\| = \dfrac{|z|^n}{|z_0|^n} |z_0|^n \|A_n\| = q^n |z_0|^n \|A_n\| \leq c q^n$. Ряд $\displaystyle\sum_{n=0}^\infty cq^n$ збіжний як геометрична прогресія, бо $q < 1$, звідси за ознакою порівняння збіжним буде $\displaystyle\sum_{n=0}^\infty \|z^n A_n\|$.
\end{proof}

\begin{lemma}
Заданий степеневий ряд $\displaystyle\sum_{k=0}^\infty z^k A_k$. Тоді радіус збіжності повністю збігається з радіусом збіжності числового степененвого ряда $\displaystyle\sum_{k=0}^\infty z^n \|A_n\|$.
\end{lemma}

\begin{proof}
Дійсно, розглянемо степеневий ряд $\displaystyle\sum_{k=0}^\infty z^n \|A_n\|$. Його радіус збіжності $r$ можна визначити за ознакою Коші-Адамара.\\ Припустимо, що $0 < r < +\infty$. Тоді ми вже отримували нерівність $\displaystyle\left\|\sum_{k=n+1}^{n+p} z^k A_k\right\| \leq \sum_{k=n+1}^{n+p} |z|^k \|A_k\|$. Тоді якщо $|z| < r$, то отримаємо збіжність $\displaystyle\sum_{k=0}^\infty z^k A_k$.\\
!Тепер припустимо, що $z_0 \in \mathbb{C}$ існує таке, що $|z_0| > r$ та при цьому в даній точці $\displaystyle\sum_{k=0}^\infty z_0^k A_k$ збіжний. Тоді звідси випливатиме обмеженість $\sequence{z_0^n A_n}$, тому за попередньою лемою, збіжним буде  ряд $\displaystyle\sum_{n=0}^\infty \|z^n A_n\|$ при $r < |z| < |z_0|$. При цьому $r$ -- радіус збіжності числового степеневого ряда -- суперечність!
\bigskip \\
Якщо $r = 0$ чи $r = +\infty$, то все цілком зрозуміло з нерівності.
\end{proof}

\begin{lemma}
Нехай $\sequence{\alpha_n}$ -- послідовність невід'ємних чисел, що задовольняє умові $\alpha_{n+m} \leq \alpha_n \alpha_m$ для всіх $n,m \in \mathbb{N}$. Тоді $\exists \displaystyle\lim_{n \to \infty} \sqrt[n]{\alpha_n} < \infty$.
\end{lemma}

\begin{proof}
Спершу зауважимо, що $\sequence{\alpha_n^{\frac{1}{n}}}$ -- обмежена. Дійсно, це випливатиме безпосередньо з ланцюга\\
$\alpha_n \leq \alpha_1 \alpha_{n-1} \leq \alpha_1^2 \alpha_{n-2} \leq \dots \leq \alpha_1^n$.\\
Зафіксуємо тепер $k \in \mathbb{N}$. Поділимо кожне $n$ на $k$ -- отримаємо представлення $n = m_n k + l_n$ при остачі $0 \leq l_n < k$. Звідси випливатиме, що $\alpha_n \leq \alpha_k^{m_n} \alpha_{l_n}$ (при $l_n = 0$ може виникнути $\alpha_0$, але покладемо $\alpha_0 = 1$). Отже, звідси $\alpha_n^{\frac{1}{n}} \leq \alpha_k^{\frac{m_n}{n}} \alpha_{l_n}^{\frac{1}{n}} = \beta_n$.\\
Оскільки $\sequence{\alpha_n^{\frac{1}{n}}}$ обмежена, то існує $\displaystyle\varlimsup_{n \to \infty} \alpha_n^{\frac{1}{n}} = \alpha$. Тобто можна відокремити підпослідовність $\left(\alpha_{n_j}^{\frac{1}{n_j}}\right)_{j=1}^\infty$, для якої $\displaystyle\lim_{j \to \infty} \alpha_{n_j}^{\frac{1}{n_j}} = \alpha$. Дослідімо послідовність $\sequence{\beta_n}$.\\
$\dfrac{m_{n_j}}{n_j} = \dfrac{m_{n_j}}{m_{n_j}k + l_{n_j}} = \dfrac{1}{k + \dfrac{l_{n_j}}{m_{n_j}}} \to \dfrac{1}{k}$ при $j \to \infty$;\\
$\alpha_{l_{n_j}}^{\frac{1}{n_j}} \to 1$ (тут $\alpha_{l_{n_j}} \in \{\alpha_0,\alpha_1,\dots,\alpha_{k-1}\}$).\\
Отже, оскільки $\alpha_{n_j}^{\frac{1}{n_j}} \leq \beta_{n_j}$, то при $j \to \infty$ маємо $\alpha \leq \alpha_k^{\frac{1}{k}}, \forall k \in \mathbb{N}$, а тому відповідно $\alpha \leq \displaystyle\lim_{k \to \infty} \alpha_k^{\frac{1}{k}}$, коротше $\displaystyle\varlimsup_{n \to \infty} \alpha_n^{\frac{1}{n}} \leq \lim_{n \to \infty} \alpha_n^{\frac{1}{n}}$. Єдина така можливість набуття нерівності -- це коли послідовність $\sequence{\alpha_n^{\frac{1}{n}}}$ збігається.
\end{proof}

\subsubsection{Спектральни радіус лінійного неперервного оператора}
\begin{definition}
Маємо $A \in \mathcal{B}(E)$ та $\sigma(A)$ -- спектр.\\
\textbf{Спектральним радіусом} оператора $A$ назвемо число
\begin{align*}
\rho_A = \max_{z \in \sigma(A)} |z|
\end{align*}
\end{definition}

\begin{theorem}
Задано $A \in \mathcal{B}(E)$. Тоді $\rho_A = \displaystyle\lim_{n \to \infty} \sqrt[n]{\|A^n\|}$.
\end{theorem}

\begin{proof}
Розглянемо послідовність $\sequence{\alpha_n}$, де кожний $\alpha_n = \|A^n\|$. Зауважимо, що $\alpha_{n+m} = \|A^{n+m}\| \leq \|A^n\| \|A^m\| = \alpha_n \alpha_m$, а тому за лемою вище існує границя $\displaystyle\lim_{n \to \infty} \sqrt[n]{\|A^n\|}$. Залишилося довести, що саме ця границя буде спектральним радіусом оператора $A$.\\
Розглянемо резольвенту $R_z(A) = (A-zI)^{-1}$. Зауважимо, що $\sigma(A) \subset \overline{B(\rho_A,0)}$ (TODO: ?), тож звідси $R_z$ -- аналітична поза межами $\overline{B(\rho_A,0)}$. Але тоді функція $f(\zeta) = R_{\frac{1}{\zeta}}$ буде аналітичною всередині шара $B\left(\dfrac{1}{\rho_A},0\right)$ (на межах є точки, де аналітичність порушується).\\
Представимо функцію $f$ у вигляді степеневого ряду з операторними коефіцієнтами. При $|\zeta| \leq \|A^{-1}\|$ матимемо $f(\zeta) = (A - \zeta^{-1}I)^{-1} = -\zeta(I - \zeta A)^{-1} = -\zeta \displaystyle\sum_{n=0}^\infty \zeta^n A^n$ -- це було все зроблено за рахунок \thref{invertible_and_geometric_progression}. За другою лемою, радіусом збіжності в правій частині буде $\dfrac{1}{\varlimsup_{n \to \infty} \|A^n\|^{\frac{1}{n}}} = \dfrac{1}{\lim_{n \to \infty} \|A^n\|^{\frac{1}{n}}}$. Ба більше, функція $f(\zeta)$ аналітична всередині кола $B(\rho_A^{-1},0)$, тобто звідси $\rho_A^{-1} = \dfrac{1}{\lim_{n \to \infty} \|A^n\|^{\frac{1}{n}}}$.
\end{proof}

\subsection{Спектральний розклад для компактних самоспряжених операторів}
\begin{lemma}
Нехай $A$ -- самоспряжений оператор. Тоді $\|A^n\| = \|A\|^n$.
\end{lemma}

\begin{proof}
Спочатку доведемо рівність при $n = 2$. У одну сторону все ясно, тобто $\|A^2\| \leq \|A\|^2$. Для іншої сторони матимемо наступне:\\
$\|Ax\|^2 = (Ax,Ax) = (A^2x,x) \overset{\text{н-ть К-Б}}{\leq} \|A^2x\| \|x\|$.\\
Взявши $\sup$ за векторами $x$, для яких $\|x\| = 1$, отримаємо $\|A\|^2 \leq \|A^2\|$.\\
Отже, отримали $\|A^2\| = \|A\|^2$, а внаслідок чого $\left\|A^{2^n}\right\| = \|A\|^{2^n}$ для всіх $n \in \mathbb{N}$.\\
Для кожного числа $m \in \mathbb{N}$ буде $m \leq 2^m$. Якщо припустити, що $\|A^m\| < \|A\|^m$, то отримаємо\\
$\|A\|^{2^m} = \left\|A^{2^m}\right\| \leq \|A^m\| \left\| A^{2^m -m} \right\| < \|A\|^m \|A\|^{2^m - m}$.\\
Така нерівність трошки суперечить, тому автоматично $\|A^m\| = \|A\|^m, \forall m \in \mathbb{N}$.
\end{proof}

\begin{remark}
Зауважимо, що якщо $A \in \mathcal{K}(H)$ -- самоспряжений, то $\rho_A = \|A\|$ (цілком ясно).
\end{remark}

\begin{theorem}
Нехай $A \in \mathcal{K}(H)$ -- самоспряжений. Тоді існує власне число $\lambda$, для якого $|\lambda| = \|A\|$.
\end{theorem}

\begin{proof}
Дійсно, оскільки $A$ -- самоспряжений, то звідси $\|A^n\| = \|A\|^n$, але тоді звідси $\rho_A = \displaystyle\lim_{n \to \infty} \sqrt[n]{\|A^n\|} = \|A\|$.
\end{proof}

Нагадаю твердження, яке було в ліналі, яке копіюється в нашому випадку.
\begin{lemma}
Нехай $A$ -- самоспряжений, тоді:
\begin{enumerate}[nosep,wide=0pt,label={\arabic*)}]
\item Всі власні числа оператора -- дійсні;
\item Власні вектори, що відповідають різним власним числам, ортогональні між собою.
\end{enumerate}
\end{lemma}

\begin{lemma}
Нехай $G$ -- інваріантний для $A \in \mathcal{B}(H)$. Тоді $G^\perp$ -- інваріантний для $A^*$.
\end{lemma}

\begin{proof}
Дійсно, нехай $y \in G^\perp$, для кожного $x \in G$ маємо $Ax \in G$, внаслідок чого $(Ax,y) = 0$. Із іншого боку, $0 = (Ax,y) = (x, A^*y) \implies A^*y \perp x, \forall x$. Це означатиме, що $A^*y \in G^\perp$.
\end{proof}

\begin{theorem}
Нехай $A \in \mathcal{K}(H)$ -- самоспряжений. Тоді $H = \displaystyle\bigoplus_{\substack{\lambda_k \in \sigma(A) \\ \lambda_k \ne 0 }} H_{\lambda_k} \oplus H_0$.\\
У цьому випадку $H_{\lambda_k}$ -- власний підпростір та $H_0$ -- ядро оператора $A$.
\end{theorem}

\begin{proof}
Уже знаємо, що існує $\lambda_1$ таке, що $|\lambda_1| = \|A\|$. Уже відомо ще давно, що власний підпростір $H_{\lambda_1}$ -- інваріантний відносно $A$, звідси $H_{\lambda_1}^\perp \overset{\text{позн.}}{=} H_1$ -- інваріантний відносно $A$.\\
Розглянемо оператор $A_1 = A|_{H_1}$. Якщо раптом $A_1 = O$, то закінчили доведення. У протилежному випадку існує $\lambda_2$ таке, що $|\lambda_2| = \|A_1\|$. Причому зауважимо, що $|\lambda_2| = \|A_1\| \leq \|A\| = |\lambda_1|$, а також $\lambda_2 \neq \lambda_1$. Останнє якби було правдою, $\lambda_2 = \lambda_1$, то ми би мали вектор $x \in H_1$, для якого $Ax = \lambda_2 x = \lambda_1 x$, але тоді $x \in H_{\lambda_1}$, що неможливо. За лемою вище, власні підпростори $H_{\lambda_1}, H_{\lambda_2}$ ортогональні, тому покладемо $H_2 \overset{\text{позн.}}{=} \left(H_{\lambda_1} \oplus H_{\lambda_2}\right)^T$, який досі залишається інваріантним.\\
Розглянемо оператор $A_2 = A|_{H_2}$. Якщо раптом $A_2 = O$, то закінчили доведення. Інакше\\
\vdots \\
Покладемо $H' = \displaystyle\bigoplus_{\substack{\lambda_k \in \sigma(A) \\ \lambda_k \ne 0 }} H_{\lambda_k}$, який інваріантний. Покладемо $H_0 = (H')^\perp$ -- теж інваріантний. Тоді звідси $H_0 = \ker A$.\\
!Якби це не так, то існував би вектор $x \in H_0: Ax \neq 0$, тому $\|A|_{H_0}\| > 0$, внаслідок чого існувало би власне число $\lambda_0 \neq 0$, що суперечить! (бо ми всі перебрали вже).
\end{proof}

\newpage

\section*{Back to лінійна алгебра}
У рамках цього розділу будемо розглядати скінченновимірні простори $L$, тобто $\dim L < \infty$. Нам уже відомо, що $L \cong \mathbb{R}^n$, а в даному просторі задається норма $\|\vec{x}\| = \sqrt{|x_1|^2 + \dots + |x_n|^2}$.

\begin{theorem}
У кожному скінченновимірному просторі всі норми еквівалентні.
\end{theorem}

\begin{proof}
Достатньо довести, що всі норми еквівалентні до $\| \cdot \|_2$.\\
Нехай $\{\vec{e}_1,\dots,\vec{e}_d\}$ -- стандартний базис $\mathbb{R}^d$, тоді звідси $\vec{x} = \displaystyle\sum_{i=1}^d x_i \vec{e}_i$.\\
$\displaystyle\left\| \sum_{i=1}^d x_i \vec{e}_i
 \right\| \leq \sum_{i=1}^d \| x_i e_i \| = \sum_{i=1}^d |x_i| \|e_i\| = \sqrt{\left( \sum_{i=1}^d |x_i| \|\vec{e}_i\| \right)^2} \overset{\text{К-Б}}{\leq} \sqrt{\sum_{i=1}^d \|e_i\|^2} \sqrt{\sum_{j=1}^d |x_j|^2} \leq \sqrt{\sum_{i=1}^d \|e_i\|^2} \sqrt{\sum_{j=1}^\infty |x_j|^2} = \sqrt{\sum_{i=1}^d \|e_i\|^2} \|\vec{x}\|_2 = M \|\vec{x}\|_2$.\\
Зауважимо, що $M \in \mathbb{R}_{\geq 0}$ та не залежить від $\vec{x}$. Отже, $\|\vec{x}\| \leq M \|\vec{x}\|_2$.
\bigskip \\
Розглянемо тепер $S$ -- одинична сфера на $(\mathbb{R}^d, \|\cdot \|_2)$. Відомо, що $S$ -- замкнена множина та обмежена. Тож за лемою Гейне-Бореля, $S$ -- компактна множина. Відомо, що відображення $\| \cdot \| \colon S \to \mathbb{R}_{\geq 0}$ -- неперервне відображення, тож вона досягає найменшого значення $m$ для деякого $\vec{y} \in S$.\\
Припустимо $m = 0$, тоді звідси $\|\vec{y}\| = 0 \implies \vec{y} = \vec{0} \implies \vec{y} \notin S$ -- неможливо. Отже, $m > 0$.\\
Значить, $\forall \vec{y} \in \mathbb{R}^d: \|\vec{y}\|_2 = 1: \|y\| \geq m$. Треба довести те саме для інших векторів.\\
Якщо $\vec{x} = \vec{0}$, то це виконано. Тому $\vec{x} \neq \vec{0}$. Покладемо вектор $\vec{y} = \dfrac{\vec{x}}{\| \vec{x} \|_2}$, причому $\|\vec{y}\|_2 = 1$. Із цього випливає, що $\| \vec{y} \|_2 \leq m \implies m \|\vec{x}\|_2 \leq \|\vec{x}\|$.
\bigskip \\
Всі інші норми будуть еквівалентними в силу транзитивності.
\end{proof}

\begin{definition}
Задано $X,Y$ -- нормовані простори.\\
Вони називаються \textbf{ізоморфними}, якщо існує бієктивний лінійний оператор $A \colon X \to Y$, для якого
\begin{align*}
\forall x \in X: \|Ax\|_Y = \|x\|_X
\end{align*}
Водночас такий оператор $A$ називають \textbf{ізоморфізмом}.\\
Позначення: $X \cong Y$
\end{definition}
\end{document}
