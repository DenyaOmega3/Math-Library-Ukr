\documentclass[a4paper, 10pt]{extarticle}
\usepackage[margin=1in]{geometry}
\usepackage{amsfonts, amsmath, amssymb}
%\usepackage[none]{hyphenat}
\usepackage{fancyhdr} %create a custom header and footer
\usepackage[utf8]{inputenc}
\usepackage[english, main=ukrainian]{babel}
\usepackage{pgfplots}
\usepgfplotslibrary{fillbetween,colormaps}
\usepackage{bm}
\usepackage{float}
\usepackage{physics}
\usepackage{enumitem}
\usepackage[unicode]{hyperref}
\usepackage{scalerel,stackengine}
\usepackage{amsthm}
\usepackage{tikz-cd}
\usepackage{centernot}
\usetikzlibrary{calc,patterns,angles,quotes, matrix}

\usepackage{tikz-3dplot}
\usetikzlibrary{fit,matrix}

\fancyhead{}
\fancyfoot{}
\parindent 0ex
\def\huge{\displaystyle}
\def\defin#1{\textbf{Definition {#1}}}
\def\ex#1{\textbf{Example {#1}}}
\def\rm#1{\textbf{Remark {#1}}}
\def\prp#1{\textbf{Proposition {#1}}}
\def\lm#1{\textbf{Lemma {#1}}}
\def\th#1{\textbf{Theorem {#1}}}
\def\crl#1{\textbf{Corollary {#1}}}
\def\proof{\textbf{Proof.}\\}
\def\proofMI{\textbf{Proof MI.}\\}
\def\bigline{\vspace{5mm}\\}
\def\qed{$\blacksquare$}
\def\dim#1{\textrm{dim} {#1}}
\def\ker#1{\textrm{Ker} {#1}}
\def\contra{\textbf{!}}

\newcommand\thref[1]{\textbf{Th.~\ref{#1}}}
\newcommand\defref[1]{\textbf{Def.~\ref{#1}}}
\newcommand\exref[1]{\textbf{Ex.~\ref{#1}}}
\newcommand\prpref[1]{\textbf{Prp.~\ref{#1}}}
\newcommand\rmref[1]{\textbf{Rm.~\ref{#1}}}
\newcommand\lmref[1]{\textbf{Lm.~\ref{#1}}}
\newcommand\crlref[1]{\textbf{Crl.~\ref{#1}}}


\def\qed{$\blacksquare$}

\def\rightproof{$\boxed{\Rightarrow}$ }
\def\leftproof{$\boxed{\Leftarrow}$ }

\def\noProof{\\ \textit{Без доведення.}}

\newtheoremstyle{theoremdd}% name of the style to be used
  {\topsep}% measure of space to leave above the theorem. E.g.: 3pt
  {\topsep}% measure of space to leave below the theorem. E.g.: 3pt
  {\normalfont}% name of font to use in the body of the theorem
  {0pt}% measure of space to indent
  {\bfseries}% name of head font
  {}% punctuation between head and body
  { }% space after theorem head; " " = normal interword space
  {\thmname{#1}\thmnumber{ #2}\textnormal{\thmnote{ \textbf{#3}\\}}}

\theoremstyle{theoremdd}
\newtheorem{theorem}{Theorem}[subsection]
  
\theoremstyle{theoremdd}
\newtheorem{definition}[theorem]{Definition}

\theoremstyle{theoremdd}
\newtheorem{samedef}[theorem]{Definition}

\theoremstyle{theoremdd}
\newtheorem{example}[theorem]{Example}

\theoremstyle{theoremdd}
\newtheorem{proposition}[theorem]{Proposition}

\theoremstyle{theoremdd}
\newtheorem{remark}[theorem]{Remark}

\theoremstyle{theoremdd}
\newtheorem{lemma}[theorem]{Lemma}

\theoremstyle{theoremdd}
\newtheorem{corollary}[theorem]{Corollary}

\makeatletter
\renewenvironment{proof}[1][Proof.\\]{\par
\pushQED{\hfill \qed}%
\normalfont \topsep6\p@\@plus6\p@\relax
\trivlist
\item\relax
{\bfseries
#1\@addpunct{.}}\hspace\labelsep\ignorespaces
}{%
\popQED\endtrivlist\@endpefalse
}
\makeatother

\newenvironment{pfMI}{\vspace*{-3mm} \textbf{\\ Proof MI. \\}}{\hfill $\blacksquare$}
\newenvironment{pfNoTh}{\textbf{Proof. \\}}{$\blacksquare$}

\DeclareMathOperator{\lcm}{lcm}
\DeclareMathOperator{\pr}{pr}
\DeclareMathOperator{\dir}{dir}


\begin{document}
\tableofcontents
\newpage
	
\section{Вектори}
\subsection{Основні означення}
\begin{definition}
\textbf{Вектором} називають напрямлений відрізок.\\
Позначення: $\vec{a}, \vec{b}, \vec{c}, \dots$.
\begin{figure}[h]
\centering
\begin{tikzpicture}
	\draw[thick, ->] (0,0)--(2,1) node at (1,1) {$\vec{a}$};
	\end{tikzpicture}
\end{figure}
\end{definition}

\begin{definition}
Два вектори $\vec{a},\vec{b}$, що лежать на паралельних прямих (або на одній прямій), називають \textbf{колінеарними}.\\
Позначення: $\vec{a} \parallel \vec{b}$.\\
Інші позначення для колінеарних векторів: \\ 
$\vec{a} \uparrow \uparrow \vec{b}$ -- колінеарні вектори в одному напрямку, вони ще називаються \textbf{співнапрямленими};\\
$\vec{a} \uparrow \downarrow \vec{b}$ -- колінеарні вектори в протилежних напрямках, вони ще називаються \textbf{протилежнонапрямленими}.
\bigskip \\
Позначення: $|\vec{a}|$ -- довжина вектора.
\end{definition}

\begin{definition}
Вектори $\vec{a}$ та $\vec{b}$ називають \textbf{рівними}, якщо:
	\begin{align*}
	\vec{a} \uparrow \uparrow \vec{b};\\
	|\vec{a}| = |\vec{b}|.
	\end{align*}
	Тобто вони співнапрямлені та мають одинакову довжину.
\end{definition}

\begin{definition}
Задамо ілюстративним чином операції над векторами:
	\begin{itemize}[nosep,wide=0pt]
	\item \textbf{додавання;} (є два способи, як додавати);
\begin{figure}[H]
\centering
	\begin{tikzpicture}
	\draw[thick, ->] (0,0)--(2,2) node at (1,1.5) {$\vec{a}$};
	\draw[thick, ->] (2,2)--(4,2) node at (2.5,2.5) {$\vec{b}$};
	\draw[thick, red, ->] (0,0)--(4,2) node at (2,0.5) {$\vec{a}+\vec{b}$};
	\node at (2,-1) {правило трикутника};
	\end{tikzpicture}
	\qquad
	\begin{tikzpicture}
	\draw[thick, ->] (0,0)--(2,2) node at (1,1.5) {$\vec{a}$};
	\draw[thick, ->] (0,0)--(2,0) node at (1,-0.5) {$\vec{b}$};
	
	\draw[thick, dashed] (2,2)--(4,2);
	\draw[thick, dashed] (2,0)--(4,2);
	\draw[thick, red, ->] (0,0)--(4,2) node at (3,1) {$\vec{a}+\vec{b}$};
	\node at (2,-1) {правило паралелограма};
	\end{tikzpicture}
\end{figure}
	\item \textbf{множення на скаляр;}

\begin{figure}[H]
\centering
	\begin{tikzpicture}
	\draw[thick, red, ->] (0,0)--(4,2) node at (3,1) {$\lambda \vec{a}$};
	\draw[thick, ->] (0,0)--(2,1) node at (1,1) {$\vec{a}$};
	\draw[white] (0,-1)--(2,-1);
	\node at (2,-2) {$\text{при } \lambda > 0$};
	\end{tikzpicture}
	\qquad
	\begin{tikzpicture}
	\draw[thick, red, ->] (0,0)--(-4,-2) node at (-3,-1) {$\lambda \vec{a}$};
	\draw[thick, ->] (0,0)--(2,1) node at (1,1) {$\vec{a}$};
	\draw[white] (0,-1)--(2,-1);
	\node at (-1,-2) {$\text{при } \lambda < 0$};
	\end{tikzpicture}
\end{figure}
\end{itemize}
\end{definition}

	Завдяки геометричній інтерпретації, ми можемо довести кілька властивостей, які виконані для всіх векторів $\vec{a}, \vec{b}, \vec{c}$ та для всіх чисел $\forall \lambda, \mu \in \mathbb{R}$:
	\begin{enumerate}[wide=0pt,label={\arabic*)}]
	\item $\vec{a} + \vec{b} = \vec{b} + \vec{a}$;
\begin{figure}[H]
\centering
	\begin{tikzpicture}
	\draw[thick, ->] (0,0)--(2,2) node at (1,1.5) {$\vec{a}$};
	\draw[thick, ->] (2,2)--(4,2) node at (2.5,2.5) {$\vec{b}$};
	\draw[thick, red, ->] (0,0)--(4,2) node at (2,0.5) {$\vec{a}+\vec{b}$};
	\end{tikzpicture}
	\qquad
	\begin{tikzpicture}
	\draw[thick, ->] (2,0)--(4,2) node at (2.5,0) {$\vec{a}$};
	\draw[thick, ->] (0,0)--(2,0) node at (1.5,0.5) {$\vec{b}$};
	\draw[thick, red, ->] (0,0)--(4,2) node at (2,1.6) {$\vec{b}+\vec{a}$};
	\end{tikzpicture}
\end{figure}
	\item $(\vec{a} + \vec{b}) + \vec{c} = \vec{a} + (\vec{b} + \vec{c})$;
\begin{figure}[H]
\centering
	\begin{tikzpicture}
	\draw[thick, ->] (0,0)--(2,2) node at (1,1.5) {$\vec{a}$};
	\draw[thick, ->] (2,2)--(4,2) node at (2.5,2.5) {$\vec{b}$};
	\draw[thick, ->] (4,2)--(6,1) node at (4.5,1.5) {$\vec{c}$};
	\draw[thick, red, ->] (0,0)--(4,2) node at (3,1) {$\vec{a}+\vec{b}$};
	\draw[thick, blue, ->] (0,0)--(6,1) node at (3,-0.1) {$(\vec{a} + \vec{b}) + \vec{c}$};
	\end{tikzpicture}
	\qquad
	\begin{tikzpicture}
	\draw[thick, ->] (0,0)--(2,2) node at (1,1.5) {$\vec{a}$};
	\draw[thick, ->] (2,2)--(4,2) node at (2.5,2.5) {$\vec{b}$};
	\draw[thick, ->] (4,2)--(6,1) node at (4.5,2.3) {$\vec{c}$};
	\draw[thick, red, ->] (2,2)--(6,1) node at (3,1.3) {$\vec{b}+\vec{c}$};
	\draw[thick, blue, ->] (0,0)--(6,1) node at (3,-0.1) {$\vec{a} + (\vec{b} + \vec{c})$};
	\end{tikzpicture}
\end{figure}
	\item $\text{існує $\vec{0}$, так званий нульовий вектор, для якого } \vec{a} + \vec{0} = \vec{a}$; цей нульовий вектор має нульову довжину, тобто $|\vec{0}| = 0$, а геометричним чином представляє собою точку;
	\item $\text{Для кожного $\vec{a}$ існує $\vec{d}$, так званий протилежнй вектор, для якого } \vec{a} + \vec{d} = \vec{0}$; насправді, цей $\vec{d} = -\vec{a}$, тобто це той же самий вектор тільки в протилежному напрямку;\\
	\textit{(для 3), 4), малюнок навряд знадобиться)}
	\item $\lambda(\vec{a}+\vec{b}) = \lambda \vec{a} + \lambda \vec{b}$ \\
	\textit{(випливає з подібності двох трикутників)};
\begin{figure}[H]
\centering
	\begin{tikzpicture}[scale = 0.8]
	\draw[thick, ->] (0,0)--(2,2) node at (1,1.5) {$\vec{a}$};
	\draw[thick, dashed] (2,2)--(4,4);
	\draw[thick, ->] (2,2)--(4,2) node at (3.5,2.5) {$\vec{b}$};
	\draw[thick, dashed] (4,4)--(8,4);
	\draw[thick, red, ->] (0,0)--(4,2) node at (2,0.5) {$\vec{a}+\vec{b}$};
	\draw[thick, red, ->, opacity = 0.2] (0,0)--(8,4) node at (5,2) {$\lambda \vec{a}+ \lambda \vec{b}$};
	\end{tikzpicture}
	\qquad
	\begin{tikzpicture}[scale = 0.8]
	\draw[thick, ->, opacity = 0.2] (0,0)--(2,2) node at (1,1.5) {$\vec{a}$};
	\draw[thick, ->] (0,0)--(4,4) node at (2.5,3) {$\lambda \vec{a}$};
	\draw[thick, ->, opacity = 0.2] (2,2)--(4,2) node at (3.5,2.5) {$\vec{b}$};
	\draw[thick, ->] (4,4)--(8,4) node at (6,4.5) {$\lambda \vec{b}$};
	\draw[thick, red, ->, opacity = 0.2] (0,0)--(4,2) node at (2,0.5) {$\vec{a}+\vec{b}$};
	\draw[thick, red, ->] (0,0)--(8,4) node at (5,2) {$\lambda \vec{a}+ \lambda \vec{b}$};
	\end{tikzpicture}
\end{figure}
	\item $\vec{a} (\lambda + \mu) = \lambda \vec{a} + \mu \vec{a}$;
	\item $(\lambda \mu) \vec{a} = (\lambda \mu) \vec{a}$;
	\item $1 \cdot \vec{a} = \vec{a}$.\\
	\textit{(для 6), 7), 8), малюнок навряд знадобиться)}
	\end{enumerate}

\begin{definition}
	Два вектори, що лежать на перпендикулярних прямих, називають \textbf{ортогональними}.\\
Позначення: $\vec{a} \perp \vec{b}$.
\end{definition}

\subsection{Вступ до поняття базису}
\begin{proposition}
\label{collinear_vectors}
$\vec{a} \parallel \vec{b} \iff \exists \lambda \in \mathbb{R}: \vec{a} = \lambda \vec{b}$.\\
\end{proposition}

\begin{proof}
$\vec{a} \parallel \vec{b} \iff \left[ \begin{gathered} \vec{a} \uparrow \uparrow \vec{b} \\ \vec{a} \uparrow \downarrow \vec{b} \end{gathered} \right. \iff \exists \lambda: \left[ \begin{gathered} \vec{a} = \underset{=\lambda}{\dfrac{|\vec{a}|}{|\vec{b}|}} \vec{b} \\ \vec{a} = \underset{=\lambda}{-\dfrac{|\vec{a}|}{|\vec{b}|}} \vec{b} \end{gathered} \right.$\\
Ці значення $\lambda$ отримали з огляду на відношення двох "відрізків".
\end{proof}

\begin{remark}
$\vec{0} \parallel \vec{b}$ для будь-якого $\vec{b}$.
\end{remark}

\subsubsection{Випадок на площині}
\begin{theorem}
\label{basis_on_plane}
Задані два вектори $\vec{a}$, $\vec{b}$, що не колінеарні. Тоді для кожного вектора $\vec{c}$ на площині існують єдині числа $\alpha, \beta \in \mathbb{R}$, для яких $\vec{c} = \alpha \vec{a} + \beta \vec{b}$.\\
Тобто третій вектор на площині завжди єдиним чином виражається через два неколінеарні вектори.
\end{theorem}

\begin{proof}
	Маємо довільні вектори $\vec{a}$, $\vec{b}$, $\vec{c}$ на площині. Ми перемістимо їх, щоб лежали на одному спільному початку
\begin{figure}[H]
\centering
	\begin{tikzpicture}
	\draw[thick, ->] (0,0)--(1,1) node at (1,1.5) {$\vec{a}$};
	\draw[thick, dashed] (2,2)--(0,0) node [anchor = south] {$A$};
	\draw[thick, ->] (0,0)--(-2,0) node at (-1,-0.5) {$\vec{b}$};
	\draw[thick, dashed] (0,0)--(2,0)  node [anchor = south] {$D$};
	\draw[thick, red, ->] (0,0)--(4,2) node at (2,1.5) {$\vec{c}$};
	\draw[thick, dashed] (2,0)--(4,2) node [anchor = south] {$C$};
	\draw[thick, dashed] (4,2)--(2,2) node [anchor = south] {$B$};
	\end{tikzpicture}
\end{figure}
	Уздовж векторів $\vec{a}$, $\vec{b}$ ми також провели заштриховані лінії для зручності.\\
І. \textit{Існування.}\\
	Із малюнку видно, що $\vec{c} = \overrightarrow{AC} = \overrightarrow{AB} + \overrightarrow{AD}$.\\
	Також з малюнку зауважимо, що $\left. \begin{gathered} \overrightarrow{AD} \parallel \vec{b} \\ \overrightarrow{AB} \parallel \vec{a} \end{gathered} \right. \overset{\textrm{\prpref{collinear_vectors}}}{\implies} \exists \alpha, \beta: \left. \begin{gathered} \overrightarrow{AD} = \beta \vec{b} \\ \overrightarrow{AB} = \alpha \vec{a} \end{gathered} \right.$.\\
	Отже, $\vec{c} = \alpha \vec{a} + \beta \vec{b}$.
\bigskip \\
II. \textit{Єдиність.}\\
!Припустимо, що розклад не є єдиним, тобто $\exists \alpha', \beta': \vec{c} = \alpha' \vec{a} + \beta' \vec{b}$.\\
	Тоді $\vec{0} = \vec{c} - \vec{c} = (\alpha-\alpha') \vec{a} + (\beta - \beta') \vec{b}$.\\
	$\implies (\alpha-\alpha')\vec{a} = (\beta' - \beta)\vec{b} \overset{\textrm{\prpref{collinear_vectors}}}{\implies} \vec{a} \parallel \vec{b}$.\\
	Але за умовою теореми, $\vec{a} \centernot\parallel \vec{b}$. Суперечність!\\ Тому та рівність виконується лише при $\alpha = \alpha'$, $\beta = \beta'$, що й доводить єдиність.
\end{proof}

\begin{definition}
\textbf{Базисом на площині} будемо називати пару неколінеарних векторів $\vec{a},\vec{b}$.\\
Уже довели, що кожний вектор $\vec{c}$ виражається через базисні вектори як $\vec{c} = \alpha \vec{a} + \beta \vec{b}$ -- це \textbf{розклад за базисом}. Коефіцієнти $\alpha, \beta$ задають \textbf{координати} вектора $\vec{c}$ в базисі $\vec{a},\vec{b}$.
\end{definition}
	
\subsubsection{Випадок в просторі}
\begin{definition}
Три вектори в просторі називаються \textbf{компланарними}, якщо вони паралельні одній площині.
\end{definition}

\begin{proposition}
$\vec{a}$, $\vec{b}$, $\vec{c}$ -- компланарні $\iff \exists \alpha, \beta, \gamma \in \mathbb{R}: \alpha \vec{a} + \beta \vec{b} + \gamma \vec{c} = \vec{0}$.
\end{proposition}

\begin{proof}
	\rightproof Дано: $\vec{a}$, $\vec{b}$, $\vec{c}$ -- компланарні. Тоді перемістимо ці вектори так, щоб лежали на одній площині. Тоді один з векторів розкладається за базисом при $\vec{a} \centernot\parallel \vec{b} \implies \vec{c} = \alpha \vec{a} + \beta \vec{b}$.\\
	$\implies \exists \alpha, \beta, \gamma = -1: \alpha \vec{a} + \beta \vec{b} + \gamma \vec{c} = \vec{0}$.\\
	Якщо ж $\vec{a} \parallel \vec{b}$, то $\vec{a} = \lambda \vec{b}\implies \exists \alpha = 1, \beta = -\lambda, \gamma = 0: \alpha \vec{a} + \beta \vec{b} + \gamma \vec{c} = \vec{0}$.
	\bigskip \\
	\leftproof Дано: $\exists \alpha, \beta, \gamma \in \mathbb{R}: \alpha \vec{a} + \beta \vec{b} + \gamma \vec{c} = \vec{0}$.\\
	Не втрачаючи загальності, нехай $\gamma \neq 0$. Тоді звідси отримаємо:\\
	$\vec{c} = \left( -\dfrac{\alpha}{\gamma} \right) \vec{a} + \left( -\dfrac{\beta}{\gamma} \right) \vec{b}$ -- розклад за базисом на площині. Отже, $\vec{c}$ на одній площині з $\vec{a},\vec{b}$, тому вони є компланарними.
\end{proof}

\begin{theorem}
Задані вектори $\vec{a}$, $\vec{b}$, $\vec{c}$, що не компланарні. Тоді для кожного вектора $\vec{d}$ в просторі існують єдині числа $\alpha, \beta, \gamma \in \mathbb{R}$, для яких $\vec{d} = \alpha \vec{a} + \beta \vec{b} + \gamma \vec{c}$.\\
Тобто четвертий вектор в просторі завжди єдиним чином виражається через три некомпланарні вектори.
\end{theorem}

\begin{proof}
	Маємо довільні вектори $\vec{a}$, $\vec{b}$, $\vec{c}$, $\vec{d}$ в просторі. Ми перемістимо їх, щоб лежали на одному спільному початку.
\begin{figure}[H]
\centering
	\begin{tikzpicture}
    % Axes
    %\draw [->] (0,0,0) -- (3,0,0) node [right] {$x$};
    %\draw [->] (0,0,0) -- (0,3,0) node [left] {$y$};
    %\draw [->] (0,0,0) -- (0,0,3) node [left] {$z$};
    \draw[thick, red, ->] (2,0,2)--(0,2,0) node[anchor = south] {$\vec{d}$};
    \draw[thick, ->] (2,0,2)--(3,0,2) node at (3,-0.5,2) {$\vec{a}$};
    \draw[thick, ->] (2,0,2)--(2,0,1) node at (2,0.5,1) {$\vec{b}$};
    \draw[thick, ->] (2,0,2)--(2,1,2) node at (1.8,1.2,2) {$\vec{c}$};
    
    %Construction
    \draw[thick, dashed] (0,0,0)--(2,0,0) node [right] {$B_1$};
    \draw[thick, dashed] (2,0,0)--(2,0,2) node [anchor = north] {$C_1$};
    \draw[thick, dashed] (2,0,2)--(0,0,2) node [left] {$D_1$};
    \draw[thick, dashed] (0,0,2)--(0,0,0) node [left] {$A_1$};
    \draw[thick, dashed] (0,2,0)--(2,2,0) node [right] {$B_2$};
    \draw[thick, dashed] (2,2,0)--(2,2,2) node [right] {$C_2$};
    \draw[thick, dashed] (2,2,2)--(0,2,2) node [left] {$D_2$};
    \draw[thick, dashed] (0,2,2)--(0,2,0) node [left] {$A_2$};
    \draw[thick, dashed] (0,2,0)--(0,0,0);
    \draw[thick, dashed] (2,2,0)--(2,0,0);
    \draw[thick, dashed] (2,2,2)--(2,0,2);
    \draw[thick, dashed] (0,2,2)--(0,0,2);
	\end{tikzpicture}
\end{figure}
	Вздовж векторів $\vec{a}$, $\vec{b}$, $\vec{c}$ ми також провели заштриховані лінії для зручності.\\
	I. \textit{Існування.}\\
	Із малюнку видно, що $\vec{d} = \overrightarrow{C_1D_1} = \overrightarrow{C_1B_1} + \overrightarrow{C_1C_2}$.\\
	А тепер за порядком, що видно ще на малюнку:\\
	$\overrightarrow{C_1D_1},$ $\vec{a}$ -- колінеарні, тому $\overrightarrow{C_1D_1} = \alpha \vec{a}$;\\
	$\overrightarrow{C_1B_1},$ $\vec{b}$ -- колінеарні, тому $\overrightarrow{C_1B_1} = \beta \vec{b}$;\\
	$\overrightarrow{C_1C_2},$ $\vec{c}$ -- колінеарні, тому $\overrightarrow{C_1C_2} = \gamma \vec{c}$.\\
	Тому $\vec{d} = \alpha \vec{a} + \beta \vec{b} + \gamma \vec{c}$.
	\bigskip \\
	II. \textit{Єдиність.} \\
	Доводиться аналогічно, як в \thref{basis_on_plane}. Тільки ми до суперечності прийдемо від того, що ці вектори стануть компланарними після припущення.
\end{proof}

\begin{definition}
\textbf{Базисом в просторі} будемо називати  пару некомпланарних векторів $\vec{a}$, $\vec{b}$, $\vec{c}$.\\
Уже довели, що кожний вектор $\vec{d}$ виражається через базисні вектори як $\vec{d} = \alpha \vec{a} + \beta \vec{b} + \gamma \vec{c}$ -- це \textbf{розклад за базисом}. Коефіцієнти $\alpha,\beta,\gamma$ задають \textbf{координати} вектора $\vec{d}$ в базисі $\vec{a},\vec{b},\vec{c}$.
\end{definition}
\noindent
Можна також позначати вектор $\vec{d}$ таким чином: $\vec{d} = (\alpha, \beta, \gamma)$ -- як набір координат в заданому базисі. У випадку на площині аналогічно можна позначати.

\begin{proposition} 
Заданий базис векторів $\vec{a}$, $\vec{b}$, $\vec{c}$ і вектори $\vec{d}_1 = (\alpha_1, \beta_1, \gamma_1)$, $\vec{d}_2 = (\alpha_2, \beta_2, \gamma_2)$. Тоді:
\begin{enumerate}[nosep,wide=0pt,label={\arabic*)}]
\item $\vec{d}_1 + \vec{d}_2 = (\alpha_1+\alpha_2, \beta_1+\beta_2, \gamma_1+\gamma_2)$;
\item $\lambda \vec{d}_1 = (\lambda \alpha_1, \lambda \beta_1, \lambda \gamma_1)$.
\end{enumerate}
\textit{Вказівка: підставити задані вектори та винести за дужки базисні вектори.}
\bigskip \\
Це твердження працює для випадку векторів на площині.
\end{proposition}

\begin{example}
Нехай $O$ -- точка перетину медіан трикутника $ABC$. Відомо, що $\overrightarrow{AO} = \vec{a}, \overrightarrow{AC} = \vec{b}$. Розкласти вектор $\overrightarrow{AB}$ за базисом $\vec{a}, \vec{b}$.
\begin{figure}[H]
\centering
	\begin{tikzpicture}[scale = 3]
	\draw[thick] (0,0)--(1,1) node[anchor = south] {$A$};
	\draw[thick] (1,1)--(1.5,0) node[anchor = west] {$B$};
	\draw[thick] (1.5,0)--(0,0) node[anchor = east] {$C$};
	\draw[thick] (1,1)--(0.75,0) node[anchor = north] {$D$};
	\draw[fill] ({5/6},{1/3}) circle[radius=0.5 pt] node[anchor = west] {$O$};
	\draw[->, red] (1,1)--({5/6},{1/3}) node at (0.8,0.6) {$\vec{a}$};
	\draw[->, red] (1,1)--(0,0) node at (0.4,0.6) {$\vec{b}$};
	\draw (0.375,-1pt)--(0.375,1pt);
	\draw (1.125,-1pt)--(1.125,1pt);
	\end{tikzpicture}
\end{figure}
	Із малюнку можна сказати, що $\overrightarrow{AB} = \overrightarrow{AC} + \overrightarrow{CB}$.\\
	Перший вектор $\overrightarrow{AC} = \vec{b}$ за умовою задачі. Другий вектор $\overrightarrow{CB} = 2 \overrightarrow{CD}$. Водночас $\overrightarrow{AC} + \overrightarrow{CD} = \overrightarrow{AD}$.\\
	За властивістю медіан трикутників, маємо, що $\dfrac{\abs{\overrightarrow{AO}}}{\abs{\overrightarrow{OD}}} = \dfrac{2}{1}$, тому $\abs{\overrightarrow{OD}} = \dfrac{1}{2} \abs{\overrightarrow{AO}} = \dfrac{1}{2} \abs{\vec{a}}$.\\
	А оскільки вони ще й співнапрямлені, то тоді $\overrightarrow{OD} = \dfrac{1}{2} \vec{a}$.\\
	Звідси $\overrightarrow{AD} = \dfrac{3}{2} \vec{a}$.	Тоді $\overrightarrow{CD} = \overrightarrow{AD} - \overrightarrow{AC} = \dfrac{3}{2} \vec{a} - \vec{b}$.\\
	Остаточно $\overrightarrow{AB} = 3 \vec{a} - \vec{b}$.
\end{example}
	
\subsection{Декартова система координат}
\begin{definition}
Нехай в просторі заданий базис $\vec{a}$, $\vec{b}$, $\vec{c}$. Встановимо фіксовану точку $O$ -- початок координат, туди й прикладемо всі вектори.\\
Початок координат та базисні вектори в сукупності називають \textbf{декартовою системою координат.}
\begin{figure}[H]
\centering
	\begin{tikzpicture}
	\draw[fill] (0,0,0) circle[radius=2 pt] node[anchor = north] {$O$};
	\draw[thick, ->] (0,0,0)--(2,0,1) node[anchor = north] {$\vec{a}$};
	\draw[thick, ->] (0,0,0)--(3,2,0) node[anchor = north] {$\vec{b}$};
	\draw[thick, ->] (0,0,0)--(1,3,0) node[anchor = north west] {$\vec{c}$};
	\end{tikzpicture}
\end{figure}
\end{definition}
Будь-якій точці $M$ в просторі однозначно відповідає вектор $\overrightarrow{OM}$ -- так званий \textbf{радіус-вектор}. Розкладемо цей вектор за базисом:\\
	$\overrightarrow{OM} = \alpha_M \vec{a} + \beta_M \vec{b} + \gamma_M \vec{c} = (\alpha_M, \beta_M, \gamma_M)$.\\
	Через однозначність ми назвемо $(\alpha_M, \beta_M, \gamma_M)$ \textbf{координатами} точки $M$.
	\bigskip \\
	А тепер нехай задані точки $M(\alpha_M, \beta_M, \gamma_M)$, $N(\alpha_N, \beta_N, \gamma_N)$. Знайдемо координати вектора $\overrightarrow{MN}$.\\
	Враховуючи існування початку координат, отримаємо:\\
	$\overrightarrow{MN} = \overrightarrow{ON} - \overrightarrow{OM} = (\alpha_N - \alpha_M, \beta_N - \beta_M, \gamma_N - \gamma_M)$.
	
	
\subsection{Лінійна залежність/незалежність}
\begin{definition}
Система векторів $\{\vec{a}_1, \dots, \vec{a}_n\}$ називається:
\begin{itemize}[nosep,wide=0pt]
	\item \textbf{лінійно незалежною}, якщо з рівності $\alpha_1 \vec{a}_1 + \dots + \alpha_n \vec{a}_n = \vec{0}$, де $\alpha_1, \dots, \alpha_n \in \mathbb{R}$, випливає $\alpha_1 = \dots = \alpha_n = 0$;
	\item \textbf{лінійно залежною}, якщо $\exists \alpha_1, \dots, \alpha_n \in \mathbb{R}: |\alpha_1| + \dots + |\alpha_n| \neq 0: \alpha_1 \vec{a}_1 + \dots + \alpha_n \vec{a}_n = \vec{0}$.
\end{itemize}
\end{definition}

\begin{definition}
Вираз $\alpha_1 \vec{a}_1 + \dots + \alpha_n \vec{a}_n$, де $\alpha_1, \dots, \alpha_n \in \mathbb{R}$, називається \textbf{лінійною комбінацією}.
\end{definition}

\begin{proposition}
$\{\vec{a}$, $\vec{b}\}$ -- лінійно залежна $\iff \vec{a} \parallel \vec{b}$.
\end{proposition}

\begin{proof}
	$\{\vec{a},\vec{b}\}$ -- лінійно залежна $\iff$ $\exists \alpha, \beta: |\alpha| + |\beta| \neq 0: \alpha \vec{a} + \beta \vec{b} = \vec{0} \boxed{\iff}$\\
	Не втрачаючи загальності, ми вважатимемо, що $\alpha \neq 0$.\\
	$\boxed{\iff} \vec{a} = -\dfrac{\beta}{\alpha} \vec{b} \overset{\text{позн.} \lambda = -\frac{\beta}{\alpha}}{=} \lambda \vec{b} \iff \vec{a} \parallel \vec{b}$.
\end{proof}

\begin{corollary}
$\{\vec{a}, \vec{b}\}$ -- лінійно незалежні $\iff \vec{a} \centernot\parallel \vec{b}$.
\end{corollary}

\begin{proposition}
	На площині будь-які три вектори завжди лінійно залежні.
\end{proposition}

\begin{proof}
	Розглянемо вектори $\{\vec{a},\vec{b},\vec{c}\}$. І знову, не втрачаючи загальності, візьмемо перші два вектори та розглянемо два підпункти:\\
	I. $\vec{a} \parallel \vec{b}$.\\
	Тоді $\vec{a} = \lambda \vec{b} \implies 1 \cdot \vec{a} + (-\lambda) \vec{b} + 0 \cdot \vec{c} = \vec{0}$, причому тут $|1|+|-\lambda|+|0| \neq 0$. \\
	Отже, $\{\vec{a}, \vec{b}, \vec{c}\}$ -- лінійно залежна.
	\bigskip \\
	II. $\vec{a} \not\parallel \vec{b}$.\\
	Тоді $\exists \alpha, \beta: \vec{c} = \alpha \vec{a} + \beta \vec{b} \implies \alpha \vec{a} + \beta \vec{b} + (-1) \vec{c} = \vec{0}$, причому тут $|\alpha| + |\beta| + |-1| \neq 0$. Отже, $\{\vec{a}, \vec{b}, \vec{c}\}$ -- лінійно залежна.\\
	Остаточно маємо, що 3 вектори на площині -- лінійно залежні.
\end{proof}

\begin{proposition}
$\{\vec{a},\vec{b},\vec{c} \}$ -- лінійно залежні $\iff$ $\vec{a},\vec{b},\vec{c}$ -- компланарні.
\end{proposition}

\begin{proof}
	$\{\vec{a},\vec{b},\vec{c}\}$ -- лінійно залежні $\iff \exists \alpha, \beta, \gamma: |\alpha| + |\beta| + |\gamma| \neq 0: \alpha \vec{a} + \beta \vec{b} + \gamma \vec{c} = \vec{0} \boxed{\iff}$\\
	Не втрачаючи загальності, нехай $\alpha \neq 0$.\\
$\boxed{\iff} \vec{a} = -\dfrac{\beta}{\alpha} \vec{b} - \dfrac{\gamma}{\alpha} \vec{c} \overset{\textrm{позн.} \lambda = -\frac{\beta}{\alpha}, \mu = -\frac{\gamma}{\alpha}}{=} \lambda \vec{b} + \mu \vec{c} \iff$ $\vec{a},\vec{b},\vec{c}$ -- компланарні.
\end{proof}

\begin{corollary}
	$\{\vec{a},\vec{b},\vec{c}\}$ -- лінійно незалежні $\iff$ $\vec{a},\vec{b},\vec{c}$ -- не компланарні.
\end{corollary}

\begin{proposition}
	В просторі будь-які чотири вектори завжди лінійно залежні.
\end{proposition}

\begin{proof}
	Розглянемо вектори $\{\vec{a},\vec{b},\vec{c},\vec{d}\}$. І знову не втрачаючи загальності, візьмемо перші три вектори та розглянемо два підпункти:\\
	I. $\vec{a}$, $\vec{b}$, $\vec{c}$ -- компланарні.\\
	Тоді $\exists \alpha, \beta: \vec{c} = \alpha \vec{a} + \beta \vec{b} \implies \alpha \vec{a} + \beta \vec{b} + (-1)\vec{c} + 0 \cdot \vec{d} = \vec{0}$, причому тут $|\alpha| + |\beta| + |-1| + |0| \neq 0$. Отже, $\{\vec{a}, \vec{b}, \vec{c}, \vec{d}\}$ -- лінійно залежна.
	\bigskip \\
	II. $\vec{a}$, $\vec{b}$, $\vec{c}$ -- не компланарні.\\
	Тоді $\exists \alpha, \beta, \gamma: \vec{d} = \alpha \vec{a} + \beta \vec{b} + \gamma \vec{c} \implies \alpha \vec{a} + \beta \vec{b} + \gamma \vec{c} + (-1)\vec{d} = \vec{0}$, причому тут $|\alpha| + |\beta| + |\gamma| + |-1| \neq 0$. Отже, $\{\vec{a}, \vec{b}, \vec{c}, \vec{d}\}$ -- лінійно залежні.\\
	Остаточно маємо, що 4 вектори в просторі -- лінійно залежні.
\end{proof}

\begin{corollary}
	У просторі вектори кількістю більше 4 завжди лінійно залежні.
\end{corollary}
	
\subsection{Проєкція на вісь}
\begin{definition}
	\textbf{Проєкцією вектора $\vec{a}$ на вісь $l$} називають вираз:
	\begin{align*}
	\pr_l \vec{a} = \pm |AB|
	\end{align*}
	Якщо $\vec{a}$ утворює гострий кут з віссю $l$, то беремо знак $+$.\\
	Якщо $\vec{a}$ утворює тупий кут з віссю $l$, то беремо знак $-$.
\begin{figure}[H]
\centering
	\begin{tikzpicture}
	\draw[thick, ->] (0,0)--(4,0) node[right] {$l$};
	\draw[thick, ->] (1,1)--(3,2) node[anchor = south east] {$\vec{a}$};
	\draw[thick, dashed] (1,0)--(1,1); \draw[thick, dashed] (3,0)--(3,2);
	\draw (1,-1pt)--(1,1pt) node [anchor = north] {$A$};
	\draw (3,-1pt)--(3,1pt) node [anchor = north] {$B$};
	\draw[very thick, red] (1,0)--(3,0);
	\draw[thick, dashed] (1,1)--(2,1);
	\draw[thick] (1.75,1) arc (0:atan(1/2):0.75) node [anchor = west] {$\alpha$};
	\end{tikzpicture}
\end{figure}
\end{definition}

	Якщо перемістити паралельно вектор $\vec{a}$ так, щоб початок був в точці $A$, то маємо з малюнку:\\
	$|AB| = |\vec{a}| \cos \alpha$.\\
	Ця формула буде також справедливою при розгляданні тупого кута.\\
	Отримали інакшу формулу проєкції:
	\begin{align*}
	\pr_l \vec{a} = |\vec{a}| \cos \alpha
	\end{align*}
	
\begin{proposition}[Властивості проєкції]
Справедливе наступне:
\begin{enumerate}[nosep,wide=0pt,label={\arabic*)}]
	\item $\pr_l (\vec{a} + \vec{b}) = \pr_l \vec{a} + \pr_l \vec{b}$;
	\item $\pr_l (\lambda \vec{a}) = \lambda \cdot \pr_l \vec{a}$.
\end{enumerate}
\end{proposition}

\begin{proof}
	1) Тут є чотири випадки, наведу ілюстративно.
\begin{figure}[H]
\centering
	\begin{tikzpicture}
	%axis
	\draw[thick, ->] (0,0)--(5,0) node[right] {$l$};
	
	%vectors
	\draw[thick, ->] (1,1)--(3,2) node at (2,1.2) {$\vec{a}$};
	\draw[thick, ->] (3,2)--(4,3) node at (4.2,2.5) {$\vec{b}$};
	\draw[thick, red, ->] (1,1)--(4,3) node[anchor = south east] {$\vec{a} + \vec{b}$};
	
	%projection
	\draw[thick, dashed] (1,0)--(1,1); \draw[thick, dashed] (3,0)--(3,2); \draw[thick, dashed] (4,0)--(4,3);
	\draw (1,-1pt)--(1,1pt) node [anchor = north] {$A$};
	\draw (3,-1pt)--(3,1pt) node [anchor = north] {$C$};
	\draw (4,-1pt)--(4,1pt) node [anchor = north] {$B$};
	\end{tikzpicture}
	\qquad
	\begin{tikzpicture}
	%axis
	\draw[thick, ->] (0,0)--(5,0) node[right] {$l$};
	
	%vectors
	\draw[thick, ->] (1,1)--(3,2) node at (2.3,1.2) {$\vec{a}$};
	\draw[thick, ->] (3,2)--(2,3) node at (3.2,2.5) {$\vec{b}$};
	\draw[thick, red, ->] (1,1)--(2,3) node[anchor = south east] {$\vec{a} + \vec{b}$};
	
	%projection
	\draw[thick, dashed] (1,0)--(1,1); \draw[thick, dashed] (3,0)--(3,2); \draw[thick, dashed] (2,0)--(2,3);
	\draw (1,-1pt)--(1,1pt) node [anchor = north] {$A$};
	\draw (3,-1pt)--(3,1pt) node [anchor = north] {$C$};
	\draw (2,-1pt)--(2,1pt) node [anchor = north] {$B$};
	\end{tikzpicture}
\end{figure}

\begin{figure}[H]
\centering
	\begin{tikzpicture}
	%axis
	\draw[thick, ->] (0,0)--(5,0) node[right] {$l$};
	
	%vectors
	\draw[thick, ->] (4,1)--(2,2) node at (2.5,1.5) {$\vec{a}$};
	\draw[thick, ->] (2,2)--(3,3) node at (2,2.5) {$\vec{b}$};
	\draw[thick, red, ->] (4,1)--(3,3) node[anchor = south east] {$\vec{a} + \vec{b}$};
	
	%projection
	\draw[thick, dashed] (4,0)--(4,1); \draw[thick, dashed] (2,0)--(2,2); \draw[thick, dashed] (3,0)--(3,3);
	\draw (4,-1pt)--(4,1pt) node [anchor = north] {$A$};
	\draw (2,-1pt)--(2,1pt) node [anchor = north] {$C$};
	\draw (3,-1pt)--(3,1pt) node [anchor = north] {$B$};

	\end{tikzpicture}
	\qquad
	\begin{tikzpicture}
	%axis
	\draw[thick, ->] (0,0)--(5,0) node[right] {$l$};
	
	%vectors
	\draw[thick, ->] (4,1)--(2,2) node at (2.5,1.5) {$\vec{a}$};
	\draw[thick, ->] (2,2)--(1,3) node at (1.5,2) {$\vec{b}$};
	\draw[thick, red, ->] (4,1)--(1,3) node[anchor = south] {$\vec{a} + \vec{b}$};
	
	%projection
	\draw[thick, dashed] (4,0)--(4,1); \draw[thick, dashed] (2,0)--(2,2); \draw[thick, dashed] (1,0)--(1,3);
	\draw (4,-1pt)--(4,1pt) node [anchor = north] {$A$};
	\draw (2,-1pt)--(2,1pt) node [anchor = north] {$C$};
	\draw (1,-1pt)--(1,1pt) node [anchor = north] {$B$};
	\end{tikzpicture}
\end{figure}
	1.1) $\pr_l (\vec{a} +\vec{b}) = |AB|  = |AC| + |CB| = \pr_l \vec{a} + \pr_l \vec{b}$.\\
	1.2), 1.3), 1.4) аналогічно.
	\bigline
	2) Тут теж чотири випадки, знову ілюстративно.
\begin{figure}[H]
\centering
	\begin{tikzpicture}
%axis
	\draw[thick, ->] (0,0)--(5,0) node[right] {$l$};
	
	%vectors
	\draw[thick, red, ->] (2,1.5)--(4,2.5) node at (3,2.5) {$\lambda \vec{a}$};
	\draw[thick, ->] (2,1.5)--(3,2) node at (2.5,1.5) {$\vec{a}$};
	
	%projection
	\draw[thick, dashed] (2,0)--(2,1.5); \draw[thick, dashed] (3,0)--(3,2); \draw[thick, dashed] (4,0)--(4,2.5);
	\draw (2,-1pt)--(2,1pt) node [anchor = north] {$A$};
	\draw (3,-1pt)--(3,1pt) node [anchor = north] {$C$};
	\draw (4,-1pt)--(4,1pt) node [anchor = north] {$B$};
	\end{tikzpicture}
	\qquad
	\begin{tikzpicture}
%axis
	\draw[thick, ->] (0,0)--(5,0) node[right] {$l$};
	
	%vectors
	\draw[thick, red, ->] (2,1.5)--(1,1) node at (1.2,1.5) {$\lambda \vec{a}$};
	\draw[thick, ->] (2,1.5)--(3,2) node at (2.5,1.5) {$\vec{a}$};
	
	%projection
	\draw[thick, dashed] (2,0)--(2,1.5); \draw[thick, dashed] (3,0)--(3,2); \draw[thick, dashed] (1,0)--(1,1);
	\draw (2,-1pt)--(2,1pt) node [anchor = north] {$A$};
	\draw (3,-1pt)--(3,1pt) node [anchor = north] {$C$};
	\draw (1,-1pt)--(1,1pt) node [anchor = north] {$B$};
	\end{tikzpicture}
\end{figure}

\begin{figure}[H]
\centering
\begin{tikzpicture}
%axis
	\draw[thick, ->] (0,0)--(5,0) node[right] {$l$};
	
	%vectors
	\draw[thick, red, ->] (3,2)--(1,1) node at (1.2,1.5) {$\lambda \vec{a}$};
	\draw[thick, ->] (3,2)--(2,1.5) node at (2.5,1.5) {$\vec{a}$};
	
	%projection
	\draw[thick, dashed] (3,0)--(3,2); \draw[thick, dashed] (2,0)--(2,1.5); \draw[thick, dashed] (1,0)--(1,1);
	\draw (3,-1pt)--(3,1pt) node [anchor = north] {$A$};
	\draw (2,-1pt)--(2,1pt) node [anchor = north] {$C$};
	\draw (1,-1pt)--(1,1pt) node [anchor = north] {$B$};
	\end{tikzpicture}
\qquad
\begin{tikzpicture}
%axis
	\draw[thick, ->] (0,0)--(5,0) node[right] {$l$};
	
	%vectors
	\draw[thick, red, ->] (3,2)--(4,2.5) node at (3,2.5) {$\lambda \vec{a}$};
	\draw[thick, ->] (3,2)--(2,1.5) node at (2.5,1.5) {$\vec{a}$};
	
	%projection
	\draw[thick, dashed] (3,0)--(3,2); \draw[thick, dashed] (2,0)--(2,1.5); \draw[thick, dashed] (4,0)--(4,2.5);
	\draw (3,-1pt)--(3,1pt) node [anchor = north] {$A$};
	\draw (2,-1pt)--(2,1pt) node [anchor = north] {$C$};
	\draw (4,-1pt)--(4,1pt) node [anchor = north] {$B$};
	\end{tikzpicture}
\end{figure}
	2.1) $\pr_l (\lambda \vec{a}) = |AB| = \dfrac{|AB|}{|AC|} |AC| = \lambda \pr_l \vec{a}$.\\
	Дріб дорівнює нашому скаляру, використовуючи подібності трикутників (якщо вісь $l$ перемістити до початку $\vec{a}$).\\
	2.2), 2.3), 2.4) аналогічно.
\end{proof}

\begin{definition}
	\textbf{Проєкцією вектора $\vec{a}$ на вектор $\vec{b}$} називають проєкцію вектора на вісь, напрямок якого задається вектором $\vec{b}$.
\end{definition}
	
\subsection{Скалярний добуток}
\begin{definition}
	\textbf{Скалярним добутком векторів $\vec{a}$, $\vec{b}$} називають величину:
	\begin{align*}
	(\vec{a}, \vec{b}) = |\vec{a}| \cdot |\vec{b}| \cdot \cos \alpha,
	\end{align*}
	де $\alpha$ -- кут між векторами $\vec{a}$ та $\vec{b}$.
\end{definition}

\begin{proposition}[Критерій ортогональності]
	$\vec{a} \perp \vec{b} \iff (\vec{a}, \vec{b}) = 0$.
\end{proposition}

\begin{proof}
Тут ми розгладатимемо ненульові вектори $\vec{a},\vec{b}$. Бо в тому випадку нецікаво.\\
	$(\vec{a}, \vec{b}) = 0 \iff |\vec{a}| |\vec{b}| \cos \alpha = 0 \iff \cos \alpha = 0 \iff \vec{a} \perp \vec{b}$.
\end{proof}

\begin{proposition}[Властивості скалярного добутку]
Справедливе наступне:
\begin{enumerate}[nosep,wide=0pt,label={\arabic*)}]
	\item $(\vec{a}, \vec{b}) = (\vec{b}, \vec{a})$;
	\item $(\vec{a}, \vec{b}) = \pr_{\vec{b}} \vec{a} \cdot |\vec{b}| = \pr_{\vec{a}} \vec{b} \cdot |\vec{a}|$;
	\item $(\vec{a}_1+\vec{a}_2, \vec{b}) = (\vec{a}_1, \vec{b}) + (\vec{a}_2, \vec{b})$;
	\item $(\lambda \vec{a}, \vec{b}) = \lambda (\vec{a}, \vec{b})$;
	\item $(\vec{a}, \vec{a}) = |\vec{a}|^2$;
	\item $\forall \vec{b}: (\vec{a}, \vec{b}) = 0 \implies \vec{a} = \vec{0}$;
	\item $\forall \vec{d}: (\vec{a}, \vec{d}) = (\vec{b}, \vec{d}) \implies \vec{a} = \vec{b}$.
\end{enumerate}
\end{proposition}

\begin{proof}
Доведемо кожну властивість:
\begin{enumerate}[wide=0pt,label={\arabic*)}]
	\item $(\vec{a}, \vec{b}) = |\vec{a}| |\vec{b}| \cos \alpha = |\vec{b}| |\vec{a}| \cos \alpha = (\vec{b}, \vec{a})$;
	\item $(\vec{a}, \vec{b}) = |\vec{a}| |\vec{b}| \cos \alpha = \pr_{\vec{b}} \vec{a} \cdot |\vec{b}| = \pr_{\vec{a}} \vec{b} \cdot |\vec{a}|$;
	\item $(\vec{a}_1+\vec{a}_2, \vec{b}) = \pr_{\vec{b}} (\vec{a}_1 + \vec{a}_2) |\vec{b}| = \pr_{\vec{b}} \vec{a}_1 |\vec{b}| + \pr_{\vec{b}} \vec{a} |\vec{b}| = (\vec{a}_1, \vec{b}) + (\vec{a}_2, \vec{b})$;
	\item $(\lambda \vec{a}, \vec{b}) = \pr_{\vec{b}} (\lambda \vec{a}) |\vec{b}| = \lambda \pr_{\vec{b}} (\vec{a}) |\vec{b}| = \lambda (\vec{a}, \vec{b})$;
	\item $(\vec{a}, \vec{a}) = |\vec{a}| |\vec{a}| \cos 0 = |\vec{a}|^2$;
	\item $\forall \vec{b}: (\vec{a}, \vec{b}) = 0 \overset{\vec{b} = \vec{a}}{\implies} (\vec{a}, \vec{a}) = 0 \implies \vec{a} = \vec{0}$;
	\item $\forall \vec{d}: (\vec{a}, \vec{d}) = (\vec{b}, \vec{d}) \implies (\vec{a}-\vec{b}, \vec{d}) = (\vec{a}, \vec{d}) - (\vec{b}, \vec{d}) = 0 \implies \vec{a} - \vec{b} = 0 \implies \vec{a} = \vec{b}$.
\end{enumerate}
Усі властивості доведені.
\end{proof}

\begin{example}
	Нехай задані вектори $\vec{a}, \vec{b}$, що $|\vec{a}| = 3, |\vec{b}| = 2$, а також кут між ними становить $\dfrac{2 \pi}{3}$. З'ясувати, чи будуть вектори $\vec{a}+2 \vec{b}$ та $3 \vec{a} - \vec{b}$ ортогональними.
	Для цього знайдемо їхній скалярний добуток.
	За властивістю скалярного добутку, маємо:\\
	$(\vec{a} + 2 \vec{b}, 3 \vec{a} - \vec{b}) = 3 (\vec{a}, \vec{a}) - (\vec{a}, \vec{b}) + 6 (\vec{b}, \vec{a}) - 2 (\vec{b}, \vec{b}) = 3 |\vec{a}|^2 + 5 (\vec{a}, \vec{b}) - 2 |\vec{b}|^2 = 27 + 5 |\vec{a}| |\vec{b}| \cos \dfrac{2 \pi}{3} - 8 = 4$.\\
	$(\vec{a} + 2 \vec{b}, 3 \vec{a} - \vec{b}) \neq 0 \Rightarrow \vec{a} + 2 \vec{b} \not\perp 3 \vec{a} - \vec{b}$.
\end{example}

	Розглянемо вектори $\vec{a} = (a_1, a_2, a_3), \vec{b} = (b_1, b_2, b_3)$, які розкладені за деяким базисом $\{\vec{p}, \vec{q}, \vec{r}\}$. Знайдемо скалярний добуток цих векторів $\vec{a},\vec{b}$:\\
	$(\vec{a}, \vec{b}) = (a_1\vec{p} + a_2\vec{q} + a_3\vec{r}$ , $b_1\vec{p} + b_2\vec{q} + b_3\vec{r}) = \\
	a_1 b_1 (\vec{p}, \vec{p}) + a_1 b_2 (\vec{p}, \vec{q}) + a_1 b_3 (\vec{p}, \vec{r}) + 
	+ a_2 b_1 (\vec{q}, \vec{p}) + a_2 b_2 (\vec{q}, \vec{q}) + a_2 b_3 (\vec{q}, \vec{r}) + 
	+ a_3 b_1 (\vec{r}, \vec{p}) + a_3 b_2 (\vec{r}, \vec{q}) + a_3 b_3 (\vec{r}, \vec{r})$.\\
	Поки нічого особливого. Проте якщо ми вимагатимемо, щоб $\vec{p} \perp \vec{q}$, $\vec{q} \perp \vec{r}$, $\vec{r} \perp \vec{p}$, то \\
	$(\vec{p}, \vec{q}) = (\vec{q}, \vec{r}) = (\vec{r}, \vec{p}) = 0$.
	
\begin{definition}
	Базис, в якому всі вектори оротогональні між собою, називається \textbf{ортогональним}.
\end{definition}

	Тоді отримаємо такий вираз:\\
	$(\vec{a}, \vec{b}) = a_1 b_1 |\vec{p}|^2 + a_2 b_2 |\vec{q}|^2 + a_3 b_3 |\vec{r}|^2$.\\
	Уже формула приємніше. Але буде краще, коли додамо умову, що $\vec{p}$, $\vec{q}$, $\vec{r}$ будуть одиничними, тобто $|\vec{p}| = |\vec{q}| = |\vec{r}| = 1$.
	
\begin{definition}
	Ортогональний базис, в якому довжина векторів одинична, називається \textbf{ортонормованим}.\\
	Для такого базису вектори по-особливому позначені: $\{\vec{i}, \vec{j}, \vec{k}\}$.
\end{definition}

\begin{remark}
	Надалі ми будемо мати справу саме з цим ортонормованим базисом.
\end{remark}

	Остаточно отримаємо:
\begin{proposition}
	Для векторів $\vec{a} = (a_1, a_2, a_3), \vec{b} = (b_1, b_2, b_3)$ в ортонормованому базисі скалярний добуток рахується таким чином: $(\vec{a}, \vec{b}) = a_1 b_1 + a_2 b_2 + a_3 b_3$.
\end{proposition}

\begin{corollary}
	Для векторів $\vec{a} = (a_1, a_2, a_3)$ в ортонормованому базисі $|\vec{a}| = \sqrt{a_1^2 + a_2^2 + a_3^2}$.
\end{corollary}

\subsection{Векторний добуток векторів}
\begin{definition}
	Упорядковану трійку некомпланарних векторів $\vec{a}, \vec{b}, \vec{c}$ ми будемо називати \textbf{правою}, якщо сісти на кінець вектора $\vec{c}$, взяти менший з кутів між $\vec{a}$ та $\vec{b}$ та спостерігати поворот від $\vec{a}$ до $\vec{b}$ \textit{проти годинникової} стрілки.

\begin{figure}[H]
\centering
\tdplotsetmaincoords{60}{120}
\begin{tikzpicture}[tdplot_main_coords]
\draw[thick,->] (0,0,0) -- (2,0,0) node[anchor=north east]{$\vec{a}$};
\draw[thick,->] (0,0,0) -- (0,2,0) node[anchor=north west]{$\vec{b}$};
\draw[thick,->] (0,0,0) -- (1,0,2) node[anchor=south]{$\vec{c}$};
\draw[->] (1,0,0) arc (0:90:1);
\end{tikzpicture}
\end{figure}
\end{definition}

\begin{definition}
	Якщо спостерігаємо це явище \textit{за годинниковою} стрілкою, то тоді впорядковану трійку називають \textbf{лівою}.
\begin{figure}[H]
\centering
\tdplotsetmaincoords{60}{120}
\begin{tikzpicture}[tdplot_main_coords]
\draw[thick,->] (0,0,0) -- (2,0,0) node[anchor=north east]{$\vec{a}$};
\draw[thick,->] (0,0,0) -- (0,2,0) node[anchor=north west]{$\vec{b}$};
\draw[thick,->] (0,0,0) -- (-1,0,-2) node[anchor=north]{$\vec{c}$};
\draw[->] (1,0,0) arc (0:90:1);
\end{tikzpicture}
\end{figure}
\end{definition}

\begin{proposition}[Властивості трійки векторів]
\label{vector_trio_properties}
	Задана права трійка $\{\vec{a}, \vec{b}, \vec{c}\}$. Тоді:
	\begin{enumerate}[nosep,wide=0pt,label={\arabic*)}]
	\item $\{\vec{b}, \vec{c}, \vec{a}\}, \{\vec{c}, \vec{a}, \vec{b}\}$ -- права трійка;
	\item $\{\vec{a}, \vec{b}, -\vec{c}\}, \{-\vec{a}, \vec{b}, -\vec{c}\}, \{\vec{a}, -\vec{b}, -\vec{c}\}$ -- ліва трійка;
	\item $\{\vec{b}, \vec{a}, \vec{c}\}, \{\vec{a}, \vec{c}, \vec{b}\}, \{\vec{c}, \vec{b}, \vec{a}\}$ -- ліва трійка.
	\end{enumerate}
	\textit{Випливає з геометричних міркувань.}
\end{proposition}

\begin{definition}
	\textbf{Векторним добутком векторів $\vec{a}, \vec{b}$} називають такий вектор $\vec{c}$, що
	\begin{align*}
	\text{1) } |\vec{c}| = |\vec{a}| |\vec{b}| \sin \alpha \text{, де $\alpha$ -- кут між векторами $\vec{a},\vec{b}$}; \\
	\text{2) }  \vec{c} \perp \vec{a}, \vec{c} \perp \vec{b}; \\
	\text{3) } \{\vec{a}, \vec{b}, \vec{c}\} \text{ -- права трійка}.
	\end{align*}
\begin{figure}[H]
\centering
\tdplotsetmaincoords{60}{120}
\begin{tikzpicture}[tdplot_main_coords]
\draw[thick,->] (0,0,0) -- (2,0,0) node[anchor=north east]{$\vec{a}$};
\draw[thick,->] (0,0,0) -- (0,2,0) node[anchor=north west]{$\vec{b}$};
\draw[thick,->] (0,0,0) -- (0,0,2) node[anchor=south]{$\vec{c}$};
\draw[->] (1,0,0) arc (0:90:1) node at (1,1,0) {$\alpha$};
\draw[thick] (0,0,0.5)--(0,0.5,0.5)--(0,0.5,0);
\draw[thick] (0,0,0.5)--(0.5,0,0.5)--(0.5,0,0);
\end{tikzpicture}
\end{figure}
Позначення: $\vec{c} = [\vec{a}, \vec{b}]$.
\end{definition}

\begin{proposition}[Властивості векторного добутку]
Справедливе наступне:
\begin{enumerate}[nosep,wide=0pt,label={\arabic*)}]
\item $[\vec{a}, \vec{b}] = -[\vec{b}, \vec{a}]$;
\item $[\vec{a}, \vec{b}] = \vec{0} \iff \vec{a} \parallel \vec{b}$;
\item $[\vec{a_1} + \vec{a_2}, \vec{b}] = [\vec{a_1}, \vec{b}] + [\vec{a_2}, \vec{b}]$;
\item $[\lambda \vec{a}, \vec{b}] = \lambda [\vec{a}, \vec{b}]$.
\end{enumerate}
\end{proposition}

\begin{proof}
Доведемо кожну властивість окремо.
\begin{enumerate}[wide=0pt,label={\arabic*)}]
\item $|[\vec{b}, \vec{a}]| = |[\vec{a}, \vec{b}]| = |\vec{a}| |\vec{b}| \sin \alpha$\\
За означенням, $[\vec{a}, \vec{b}]$ та $[\vec{b}, \vec{a}]$ одночасно $\perp \vec{a}$, $\perp \vec{b}$. Звідси випливає (із геометрії), що
$[\vec{a}, \vec{b}] \parallel [\vec{b}, \vec{a}] \implies [\vec{a}, \vec{b}] = \pm [\vec{b}, \vec{a}]$.\\
Коли $\{\vec{b}, \vec{a}, [\vec{b}, \vec{a}]\}$ - права, то $\{\vec{a}, \vec{b}, [\vec{b}, \vec{a}]\}$ - ліва, а тому $\{\vec{a}, \vec{b}, -[\vec{b}, \vec{a}]\}$ - права. Отже, $[\vec{b},\vec{a}] = -[\vec{a}, \vec{b}]$.

\item Тут ми розглядатимемо ненульові вектори $\vec{a},\vec{b}$. Бо в тому випадку нецікаво.\\
$[\vec{a}, \vec{b}] = \vec{0} \iff |[\vec{a}, \vec{b}]| = 0 \iff |\vec{a}| |\vec{b}| \sin \alpha = 0 \iff \alpha = 0,\pi \iff \vec{a} \parallel \vec{b}$.\bigskip \\
3), 4) на потім залишемо. Хоча залишу інший варіант.\\
\item \textit{Альтернативне доведення.}\\
Зробимо ось такий геометричний малюнок. Векторний добуток $[\vec{a_1},\vec{b}]$ відповідатиме блакитній грані. Векторний добуток $[\vec{a_2},\vec{b}]$ відповідатиме зеленій грані. Нарешті, векторний добуток $[\vec{a_1}+\vec{a_2}, \vec{b}]$ відповідатиме останній грані.
\begin{figure}[H]
\centering
\begin{tikzpicture}
\filldraw[draw=black, fill=white] (0,0)--(3,0.5)--(2.5,2)--(-0.5,1.5)--cycle;
\filldraw[draw=black, fill=blue!10] (0,0)--(1,1)--(0.5,2.5)--(-0.5,1.5)--cycle;
\filldraw[draw=black, fill=green!10] (3,0.5)--(1,1)--(0.5,2.5)--(2.5,2)--cycle;
\draw[dashed] (2.5,2)--(-0.5,1.5);

\draw[thick,->] (0,0)--(-0.5,1.5) node[anchor = east] {$\vec{b}$};
\draw[thick,blue,->] (0,0)--(1,1) node[anchor = north] {$\vec{a_1}$};
\draw[thick,green,->] (1,1)--(3,0.5) node[anchor = west] {$\vec{a_2}$};
\draw[thick,red,->] (0,0)--(3,0.5) node[anchor = north east] {$\vec{a_1}+\vec{a_2}$};
\end{tikzpicture}
\end{figure}
Я поки припущу, що $\vec{a_1},\vec{a_2}$ утворюють трикутник при сумуванні.\\
Кожна грань - це паралелограм. Проведемо на кожній грані висоту таким чином, щоб із цих сторін утворити трикутник (цей трикутник буде прямокутним, але щось не можу пояснити). Намалюю якийсь трикутник, щоб було простіше. Одразу позначу довжини сторін.
\begin{figure}[H]
\centering
\begin{tikzpicture}
\draw[red] (0,0)--(3,0);
\draw[green] (3,0)--(2,1);
\draw[blue] (2,1)--(0,0);
\node at (1.5,-0.5) {$|\vec{a_1}+\vec{a_2}| \cos \theta$};
\node at (0,0.7) {$|\vec{a_1}| \cos \varphi_1$};
\node at (3.5,0.7) {$|\vec{a_2}| \cos \varphi_2$};
\end{tikzpicture}
\end{figure}
$\varphi_1$ - кут між $\vec{a_1}, \vec{b}$.\\
$\varphi_2$ - кут між $\vec{a_2}, \vec{b}$.\\
$\theta$ - кут між $\vec{a_1}+\vec{a_2}, \vec{b}$.\\
За теоремою Піфагора, маємо $|\vec{a_1}|^2 \cos^2 \varphi_1 + |\vec{a_2}|^2 \cos^2 \varphi_2 = |\vec{a_1}+\vec{a_2}|^2 \cos^2 \theta$.\\
Помножимо обидві частини рівності на $|\vec{b}|^2$:\\
$|\vec{a_1}|^2 |\vec{b}|^2 \cos^2 \varphi_1 + |\vec{a_2}|^2 |\vec{b}|^2 \cos^2 \varphi_2 = |\vec{a_1}+\vec{a_2}|^2 |\vec{b}|^2 \cos^2 \theta$ (*).\\
Зауважимо тепер, що $[\vec{a_1},\vec{b}]$ перпендикулярний синій стороні, $[\vec{a_2},\vec{b}]$ перпендикулярний зеленій стороні, $[\vec{a_1}+\vec{a_2},\vec{b}]$ перпендикулярний червоній стороні.\\
Якщо на $90^\circ$ повернути малюнок з трикутником за годинниковою стрілкою, то ці векторні добутки будуть лежати на цих сторонах. Їх можна об'єднати, просто тому що виконується рівність (*).
\end{enumerate}
Усі властивості доведені.
\end{proof}

\begin{remark}[Геометричний зміст]
Із означення випливає, що модуль векторного добутку описує площу паралелограма, тобто:\\
$S_{\text{паралелограм}} = |[\vec{a},\vec{b}]|$.
\begin{figure}[H]
\centering
\begin{tikzpicture}
\draw[name path = A, thick,->] (0,0,0) -- (3,0,0) node[anchor=north east]{$\vec{a}$};
\draw[thick,->] (0,0,0) -- (0,3,0) node[anchor=north west]{$\vec{c}$};
\draw[name path = B, thick,->] (0,0,0) -- (0,0,3) node[anchor=south]{$\vec{b}$};
\draw[name path = C, thick, dashed] (0,0,2)--(2,0,2)--(2,0,0);
\fill [blue!50,
    intersection segments={
      of= B and C
    }];
\draw[->] (0.6,0,0) arc [start angle=0,end angle=-90,x radius=0.8,y radius=0.24] node [anchor = north west] {$\alpha$} node[black] at (1,0,0) {$S$};
\draw[thick] (0,0,0.5)--(0,0.5,0.5)--(0,0.5,0);
\draw[thick] (0.5,0,0)--(0.5,0.5,0)--(0,0.5,0);
\end{tikzpicture}
\end{figure}
\end{remark}

\subsection{Мішаний добуток векторів}
\begin{definition}
\textbf{Мішаним добутков трьох векторів $\vec{a}, \vec{b}, \vec{c}$} називають число:
\begin{align*}
(\vec{a}, \vec{b}, \vec{c}) = (\vec{a}, [\vec{b}, \vec{c}])
\end{align*}
Тобто тут скалярний добуток $\vec{a}$ з векторним добутком $\vec{b},\vec{c}$.
\end{definition}

\begin{proposition}[Властивості мішаного добутку]
Справедливе наступне:
\begin{enumerate}[nosep,wide=0pt,label={\arabic*)}]
\item Знак мішаного добутку:\\
+, якщо $\{\vec{a}, \vec{b}, \vec{c}\}$ -- права трійка;\\
-, якщо $\{\vec{a}, \vec{b}, \vec{c}\}$ -- ліва трійка;
\item $(\vec{a}, \vec{b}, \vec{c}) = (\vec{b}, \vec{c}, \vec{a}) = (\vec{c}, \vec{a}, \vec{b}) = -(\vec{b}, \vec{a}, \vec{c}) = -(\vec{a}, \vec{c}, \vec{b}) = -(\vec{c}, \vec{b}, \vec{a})$;
\item $(\vec{a}, \vec{b}, \vec{c}) = 0 \iff \vec{a}, \vec{b}, \vec{c}$ -- компланарні;
\item $(\vec{a}, \vec{b}, \vec{c}) = (\vec{a}, [\vec{b}, \vec{c}]) = ([\vec{a},\vec{b}],\vec{c})$;
\item $(\vec{a}_1+\vec{a}_2, \vec{b}, \vec{c}) = (\vec{a}_1,\vec{b},\vec{c}) + (\vec{a}_2,\vec{b},\vec{c})$;
\item $(\lambda \vec{a}, \vec{b}, \vec{c}) = \lambda (\vec{a},\vec{b},\vec{c})$.
\end{enumerate}
\end{proposition}

\begin{proof}
Доведемо кожну властивість окремо:
\begin{enumerate}[wide=0pt,label={\arabic*)}]
\item $(a,b,c) > 0 \iff pr_{[\vec{b}, \vec{c}]} \vec{a} > 0 \iff$ кут між цими векторами -- гострий $\iff \{\vec{a}, \vec{b}, \vec{c}\}$ -- права.\\
$(a,b,c) < 0 \iff pr_{[\vec{b}, \vec{c}]} \vec{a} < 0 \iff$ кут між цими векторами -- тупий $\iff \{\vec{a}, \vec{b}, \vec{c}\}$ -- ліва.

\item Дивись \prpref{vector_trio_properties}, яка трійка: права та ліва. А далі користуємось властивістю 1), щойно доведеною. Тому й виникають рівності.

\item $(\vec{a}, \vec{b}, \vec{c}) = 0 \iff V = 0$ (див.\ зауваження нижче), тобто ці вектори лежать на одній площині -- компланарні.

\item $(\vec{a}, [\vec{b},\vec{c}]) = (\vec{a}, \vec{b}, \vec{c}) = (\vec{c}, \vec{a}, \vec{b}) = (\vec{c}, [\vec{a}, \vec{b}]) = ([\vec{a},\vec{b}], \vec{c})$.

\item $(\vec{a}_1+\vec{a}_2, \vec{b}, \vec{c}) = (\vec{a}_1+\vec{a}_2, [\vec{b},\vec{c}]) = (\vec{a}_1,[\vec{b},\vec{c}]) + (\vec{a}_2,[\vec{b},\vec{c}]) = (\vec{a}_1,\vec{b},\vec{c}) + (\vec{a}_2,\vec{b},\vec{c})$.

\item $(\lambda \vec{a}, \vec{b}, \vec{c}) = (\lambda \vec{a}, [\vec{b}, \vec{c}]) = \lambda (\vec{a},[\vec{b},\vec{c}]) = \lambda (\vec{a},\vec{b},\vec{c})$.
\end{enumerate}
Усі властивості доведені.
\end{proof}

\begin{remark}[Геометричний зміст]
Із означення та геометричного сенсу векторного добутку випливає, що мішаний добуток описує об'єм паралелепіпеда, тобто:\\
$V_{\text{паралелепіпед}} = |(\vec{a},\vec{b},\vec{c})|$.
\begin{figure}[H]
\centering
\begin{tikzpicture}
\draw[thick,->] (0,0,0) -- (2,0,2) node[anchor=north east]{$\vec{c}$};
\draw[thick,->] (0,0,0) -- (2,0,-2) node[anchor=south]{$\vec{b}$};
\draw[thick,->] (0,0,0) -- (1,2,0) node[anchor=south]{$\vec{a}$};
\draw[thick, dashed] (1,2,0)--(1,0,0) node at (1.2,1,0) {$h$};
\draw[thick,dashed] (2,0,2) -- (4,0,0) -- (2,0,-2) node at (2,0,0) {$S_{\textrm{осн}}$};
\draw[thick, dashed] (2,0,2)--(3,2,2);
\draw[thick, dashed] (4,0,0)--(5,2,0);
\draw[thick, dashed] (2,0,-2)--(3,2,-2);
\draw[thick, dashed] (3,2,2)--(5,2,0)--(3,2,-2)--(1,2,0)--cycle;
\end{tikzpicture}
\end{figure}
Справді, $V = S\cdot h = |[\vec{b}, \vec{c}]| \cdot |\pr_{[\vec{b}, \vec{c}]} \vec{a}| = |[\vec{b}, \vec{b}] \pr_{[\vec{b}, \vec{c}]} \vec{a}| = |(\vec{a}, [\vec{b}, \vec{c}])| = |(\vec{a},\vec{b},\vec{c})|$.
\end{remark}

\textbf{Повернміось до п. 1.7.}\\
Доведемо останні два пункти:\\
3) Візьмемо довільний вектор $\vec{d}$ і знайдемо наступний скалярний добуток:\\
$(\vec{d}, [\vec{a}_1+\vec{a}_2, \vec{b}]) = (\vec{d}, \vec{a}_1+\vec{a}_2, \vec{b}) = (\vec{d}, \vec{a}_1, \vec{b}) + (\vec{d}, \vec{a}_2, \vec{b}) = \\ = (\vec{d}, [\vec{a}_1, \vec{b}]) + (\vec{d}, [\vec{a}_2, \vec{b}]) = (\vec{d}, [\vec{a}_1,\vec{b}]+[\vec{a}_2,\vec{b}])$\\
$\implies [\vec{a}_1+\vec{a}_2, \vec{b}] = [\vec{a}_1,\vec{b}]+[\vec{a}_2,\vec{b}]$.
\bigskip \\
4) Візьмемо довільний вектор $\vec{d}$ і знайдемо наступний скалярний добуток:\\
$(\vec{d}, [\lambda \vec{a}, \vec{b}]) = (\vec{d}, \lambda \vec{a}, \vec{b}) = \lambda (\vec{d}, \vec{a}, \vec{b}) = \lambda (\vec{d}, [\vec{a}, \vec{b}]) = (\vec{d}, \lambda [\vec{a}, \vec{b}])$\\
$\implies [\lambda \vec{a}, \vec{b}] = \lambda [\vec{a}, \vec{b}]$.
\bigskip \\
Перед доведенням іншої формули хочеться повернутися до ортонормованого базису $\{\vec{i},\vec{j},\vec{k}\}$ та зауважити, що між ними векторний добуток буде ось такий:\\
$[\vec{i},\vec{j}] = \vec{k}$, \qquad $[\vec{j},\vec{k}] = \vec{i}$, \qquad $[\vec{k},\vec{i}] = \vec{j}$.\\
Всі ці рівності випливають безспосередьно з означення векторного добутку.
\begin{figure}[H]
\centering
\begin{tikzpicture}
\fill (0,0) circle (1pt) node at (0,-0.3) {$\vec{i}$};
\fill (-1,1) circle (1pt) node at (-1.3,1) {$\vec{j}$};
\fill (1,1) circle (1pt) node at (1.3,1) {$\vec{k}$};
\draw[->] (-0.1,0.1)--(-0.9,0.9);
\draw[->] (-0.9,1)--(0.9,1);
\draw[->] (0.9,0.9)--(0.1,0.1);
\end{tikzpicture}
\caption*{Наприклад, $[\vec{i},\vec{j}] = \vec{k}$. Із точки $\vec{i}$ до точки $\vec{j}$ -- потрапимо до точки $\vec{k}$.}
\end{figure}
Розглянемо вектори $\vec{a} = (a_1,a_2,a_3), \vec{b} = (b_1,b_2,b_3)$, які розкладені за ортонормованим базисом. Знайдемо їхній векторний добуток:\\
$[\vec{a}, \vec{b}] = [a_1\vec{i}+a_2\vec{j}+a_3\vec{k}, b_1\vec{i}+b_2\vec{j}+b_3\vec{k}] = \\ a_1 b_1 [\vec{i}, \vec{i}] + a_1 b_2 [\vec{i}, \vec{j}] + a_1 b_3 [\vec{i}, \vec{k}] + a_2 b_1 [\vec{j}, \vec{i}] + a_2 b_2 [\vec{j}, \vec{j}] + a_2 b_3 [\vec{j}, \vec{k}] + a_3 b_1 [\vec{k}, \vec{i}] + a_3 b_2 [\vec{k}, \vec{j}] + a_3 b_3 [\vec{k}, \vec{k}] = \\
= (a_2 b_3 - a_3 b_2) \vec{i} + (a_1 b_3 - a_3 b_1) \vec{j} + (a_1 b_2 - a_2 b_1) \vec{k} \overset{\text{скорочено}}{=} \begin{vmatrix}
\vec{i} & \vec{j} & \vec{k} \\
a_1 & a_2 & a_3 \\
b_1 & b_2 & b_3
\end{vmatrix}$.\\
Координати вектора $[\vec{a}, \vec{b}] = (a_2b_3-a_3b_2,\ a_1b_3-a_3b_1,\ a_1b_2-a_2b_1) \overset{\text{скорочено}}{=} \left(\begin{vmatrix} a_2 & b_2 \\ a_3 & b_3 \end{vmatrix}, \begin{vmatrix} a_1 & b_1 \\ a_3 & b_3 \end{vmatrix}, \begin{vmatrix} a_1 & b_1 \\ a_2 & b_2 \end{vmatrix} \right)$.\\
Скорочено ми записували через такий математичний об'єкт як \textit{визначник}. Про них поговоримо детально пізніше. Підсумуємо:
\begin{proposition}
Для векторів $\vec{a} = (a_1,a_2,a_3),\ \vec{b} = (b_1,b_2,b_3)$ в ортонормованому базисі векторний добуток рахується таким чином: $[\vec{a}.\vec{b}] = \begin{vmatrix}
\vec{i} & \vec{j} & \vec{k} \\
a_1 & a_2 & a_3 \\
b_1 & b_2 & b_3
\end{vmatrix}$.
\end{proposition}

\textbf{Повернімось до п. 1.8.}\\
Розглянемо вектори $\vec{a} = (a_1,a_2,a_3), \vec{b} = (b_1,b_2,b_3), \vec{c} = (c_1,c_2,c_3)$, які розкладені за ортонормованим базисом. Знайдемо їхній мішаний добуток:\\
$(\vec{a}, \vec{b}, \vec{c}) = ([\vec{a}, \vec{b}], \vec{c}) = c_1 \begin{vmatrix} a_2 & b_2 \\ a_3 & b_3 \end{vmatrix} + c_2 \begin{vmatrix} a_1 & b_1 \\ a_3 & b_3 \end{vmatrix} + c_3 \begin{vmatrix} a_1 & b_1 \\ a_2 & b_2 \end{vmatrix} = \begin{vmatrix}
a_1 & a_2 & a_3 \\
b_1 & b_2 & b_3 \\
c_1 & c_2 & c_3
\end{vmatrix}$.\\
Знову все це ми записали в скороченому вигляді, як це було вище. Підсумуємо:
\begin{proposition}
Для векторів $\vec{a} = (a_1,a_2,a_3),\ \vec{b} = (b_1,b_2,b_3), \vec{c} = (c_1,c_2,c_3)$ в ортонормованому базисі мішаний добуток рахується таким чином: $(\vec{a},\vec{b},\vec{c}) = \begin{vmatrix}
a_1 & a_2 & a_3 \\
b_1 & b_2 & b_3 \\
c_1 & c_2 & c_3
\end{vmatrix}$.
\end{proposition}

\subsection{Подвійний векторний добуток}
\begin{definition}
\textbf{Подвійним векторним добутком векторів $\vec{a}, [\vec{b},\vec{c}]$} називається вектор:
\begin{align*}
\vec{d} = [\vec{a},[\vec{b},\vec{c}]]
\end{align*}
Позначення: $\vec{d} = [\vec{a},\vec{b},\vec{c}]$.
\end{definition}

\begin{proposition}[Формула Лагранжа]
Справедлива рівність $[\vec{a},[\vec{b},\vec{c}]] = \vec{b} (\vec{a}, \vec{c}) - \vec{c} (\vec{a}, \vec{b})$ при ортонормованому базисі.\\
\textit{Розмновна формула: бац мінус цаб :)}
\end{proposition}

\begin{proof}
Розглянемо вектори $\vec{a} = (a_1, a_2, a_3), \vec{b} = (b_1,b_2,b_3), \vec{c} = (c_1,c_2,c_3)$. Оскільки ми в ортонормованому базисі, то $[\vec{b},\vec{c}] = (b_2c_3-b_3c_2,\ b_1c_3-b_3c_1,\ b_1c_2-b_2c_1)$. Далі маємо наступне: $[\vec{a}, [\vec{b},\vec{c}]] = \\ = (a_2 (b_1c_2-b_2c_1) - a_3 (b_1c_3-b_3c_1),\quad a_1(b_1c_2-b_2c_1) - a_3(b_2c_3-b_3c_2), \quad a_1 (b_1c_3 - b_3c_1) - a_2 (b_2c_3 - b_3c_2)) \\
= (a_2b_1c_2-a_2b_2c_1-a_3b_1c_3+a_3b_3c_1,\ a_1b_1c_2-a_1b_2c_1-a_3b_2c_3+a_3b_3c_2,\ a_1b_1c_3-a_1b_3c_1-a_2b_2c_3+a_2b_3c_2) \\
= (a_2b_1c_2-a_2b_2c_1-a_3b_1c_3+a_3b_3c_1 \textcolor{red}{+a_1b_1c_1 - a_1b_1c_1},\\
a_1b_1c_2-a_1b_2c_1-a_3b_2c_3+a_3b_3c_2 \textcolor{red}{+a_2b_2c_2 - a_2b_2c_2},\\
a_1b_1c_3-a_1b_3c_1-a_2b_2c_3+a_2b_3c_2 \textcolor{red}{+a_3b_3c_3 - a_3b_3c_3}) = \\
= (b_1(a_1c_1+a_2c_2$
\end{proof}

\textit{Альтернативне доведення.}
\begin{figure}[H]
\centering
\begin{tikzpicture}
\draw[name path = A, thick,->] (0,0,0) -- (3,0,1) node[anchor=north east]{$\vec{c}$};
\draw[thick,->] (0,0,0) -- (0,3,0) node[anchor=north west]{$[\vec{b}, \vec{c}]$};
\draw[name path = B, thick,->] (0,0,0) -- (0,0,3) node[anchor=south]{$\vec{b}$};
\draw[thick,->] (0,0,0) -- (1,2,1) node[anchor=north west]{$\vec{a}$};
\end{tikzpicture}
\end{figure}
Розглянемо більш детально подвійні добутки. Спочатку створімо вектор $[\vec{b}, \vec{c}]$. За означенням, $[\vec{b}, \vec{c}] \perp \vec{b},\vec{c}$.\\
А потім створімо вже вектор $[\vec{a},\vec{b},\vec{c}]$. За означенням, $[\vec{a},\vec{b},\vec{c}] \perp [\vec{b}, \vec{c}]$.\\
А це означає, що $[\vec{a},\vec{b},\vec{c}]$ лежить на площині, що створена векторами $\vec{b},\vec{c}$, тоді цей вектор розкладемо за базисом: $[\vec{a}, \vec{b}, \vec{c}] = \beta \vec{b} + \gamma \vec{c}$. Залишилось знайти координати $\beta, \gamma$.\\
Нам також відомо, що $[\vec{a},\vec{b},\vec{b}] \perp \vec{a} \implies ([\vec{a}, \vec{b}, \vec{b}], \vec{a}) = 0$.\\
Але з іншого боку, $([\vec{a}, \vec{b}, \vec{b}], \vec{a}) = (\beta \vec{b} + \gamma \vec{c}, \vec{a}) = \beta (\vec{b}, \vec{a}) + \gamma (\vec{c}, \vec{a})$.\\
Таким чином, $(\vec{a}, \vec{b}) \beta + (\vec{a}, \vec{c}) \gamma = 0 \implies \dfrac{\gamma}{(\vec{a}, \vec{b})} = - \dfrac{\beta}{(\vec{a}, \vec{c})} = \lambda$.\\
$\implies [\vec{a}, \vec{b}, \vec{c}] = -\lambda \vec{b} (\vec{a}, \vec{c}) + \lambda \vec{c} (\vec{a}, \vec{b})$\\
А тепер нехай $\vec{a} = \vec{i}, \vec{b} = \vec{j}, \vec{c} = \vec{i}$. Отримаємо, що $\vec{j} = [\vec{i}, \vec{j}, \vec{i}] = -\lambda \vec{j} \implies \lambda = -1$. Підставимо це значення та отримаємо $[\vec{a},[\vec{b},\vec{c}]] = \vec{b} (\vec{a}, \vec{c}) - \vec{c} (\vec{a}, \vec{b})$.

\begin{corollary}[Тотожність Якобі]
$[\vec{a}, [\vec{b}, \vec{c}]] + [\vec{b}, [\vec{c}, \vec{a}]] + [\vec{c}, [\vec{a}, \vec{b}]] = \vec{0}$.
\end{corollary}

\iffalse
\subsection{*Формула ділення відрізка в заданому співвідношенні}
Задані точки $A(x_A,y_A,z_A), B(x_B,y_B,z_B)$. Проведемо відрізок $AB$. Встановимо точку $P(x_P,y_P,z_P) \in AB$ таким чином, що $\dfrac{AP}{PB} = \dfrac{\lambda}{\mu}$
\begin{figure}[H]
\centering
\begin{tikzpicture}
\draw[thick] (0,0)--(4,1);
\draw[fill] (0,0) circle (1pt) node [anchor = north]{$A$};
\draw[fill] (1,0.25) circle (1pt) node [anchor = north]{$P$};
\draw[fill] (4,1) circle (1pt) node [anchor = north]{$B$};
\node at (0.5,0.5) {$\lambda$};
\node at (2.5,0.85) {$\mu$};
\end{tikzpicture}
\end{figure}
Знайдемо координати т. $P$.\\
$\dfrac{AP}{PB} = \dfrac{\lambda}{\mu} \implies \mu AP = \lambda PB$.\\
$\overrightarrow{AP} = (x_P - x_A, y_P - y_A, z_P - z_A)$, $\overrightarrow{PB} = (x_B - x_P, y_B - y_P, z_B - z_P)$.\\
$\mu \overrightarrow{AP} = \lambda \overrightarrow{PB}$.\\
Тому $\mu (x_P - x_A) = \lambda (x_B - x_P) \implies \dots \Rightarrow x_P = \dfrac{\lambda x_B + \mu x_A}{\lambda + \mu}$.\\
Для $y_P, z_P$ аналогічна формула.
\fi
\newpage

\section{Початок аналітичної геометрії. Прямі та площини}
\begin{definition}
\textbf{Рівнянням геометричного місця точок (ГМТ)} називається рівняння
\begin{align*}
F(x,y,z) = 0 \text{ -- випадок в просторі} \\
F(x,y) = 0 \text{ -- випадок на площині},
\end{align*}
якому задовольняє будь-яка точка, що належить даному ГМТ, та не задовольняє жодна точка, що не належить ГМТ.
\end{definition}

\begin{definition}
\textbf{Нормальним вектором прямої (площини)} називають будь-який ненульовий вектор, що перпендикулярний до цієї прямої (площини).
\end{definition}

\begin{definition}
\textbf{Спрямованим вектором прямої (площини)} називають будь-який ненульовий вектор, що паралельний до цієї прямої (площини).
\end{definition}

\begin{remark}
Надалі всюди розглядається ортонормований базис $\{\vec{i},\vec{j},\vec{k}\}$ при роботі з векторами.
\end{remark}

\subsection{Пряма на площині}
Знайдемо рівняння прямої, що проходить через точку $M_0 = (x_0,y_0)$.\\
Задамо пряму $l$, довільну точку $M = (x,y) \in l$ та нормальний вектор $\vec{n} = (A,B)$. Тоді $\overrightarrow{M_0M} = (x-x_0, y-y_0)$. Оскільки $\vec{n} \perp \overrightarrow{M_0 M}$, то це теж саме, що $(\vec{n}, \overrightarrow{M_0 M}) = 0 \iff A(x-x_0)+B(y-y_0)=0$.\\
Отже, отримали \textbf{рівняння прямої, що проходить через точку $M_0$}:
\begin{align*}
A(x-x_0) + B(y-y_0) = 0.
\end{align*}

\begin{figure}[H]
\centering
\begin{tikzpicture}
\draw[thick] (0,0)--(4,2) node[anchor = north] {$l$};
\draw[fill] (2,1) circle (1pt) node [anchor = north]{$M_0$};
\draw[fill] (3,1.5) circle (1pt) node [anchor = north]{$M$};
\draw[thick, ->] (2,1)--(1.5,2) node[anchor = east] {$\vec{n}$};
\draw[thick, ->] (2,1)--(3,1.5);
\end{tikzpicture}
\caption*{Пряма $l$ описується рівнянням вище.}
\end{figure}

Якщо розкрити дужки, отримаємо:\\
$Ax + By \underbrace{-Ax_0 - By_0}_{= C} = Ax+By+C = 0$.\\
Отже, отримали \textbf{загальне рівняння прямої}:
\begin{align*}
Ax + By + C = 0.
\end{align*}

\subsubsection*{Частинні випадки}
\begin{enumerate}[wide=0pt,label={\Roman*.}]
\item $A = 0$.\\
Тоді маємо рівняння $By + C = 0 \implies y = -\dfrac{C}{B}$. Отримали \textbf{пряму, що паралельна осі $OX$}.

\item $B = 0$.\\
Тоді маємо рівняння $Ax + C = 0 \implies x = \dfrac{-C}{A}$. Отримали \textbf{пряму, що паралельна осі $OY$}.

\item $C = 0$.\\
Тоді маємо рівняння $Ax + By = 0$. Отримали \textbf{пряму, що проходить через початок координат}.
\end{enumerate}

\begin{figure}[H]
\centering
\begin{tikzpicture}[scale=0.8]
\draw[thick, ->] (-2,0)--(2,0) node[anchor = north west] {$x$};
\draw[thick, ->] (0,-2)--(0,2) node[anchor = south east] {$y$};
\draw[thick,red] (-2,-1)--(2,-1) node[anchor =north east] {$A=0$};
\end{tikzpicture}
\qquad
\begin{tikzpicture}[scale=0.8]
\draw[thick, ->] (-2,0)--(2,0) node[anchor = north west] {$x$};
\draw[thick, ->] (0,-2)--(0,2) node[anchor = south east] {$y$};
\draw[thick,red] (-1,-2)--(-1,2) node[anchor = east] {$B=0$};
\end{tikzpicture}
\qquad
\begin{tikzpicture}[scale=0.8]
\draw[thick, ->] (-2,0)--(2,0) node[anchor = north west] {$x$};
\draw[thick, ->] (0,-2)--(0,2) node[anchor = south east] {$y$};
\draw[thick, red] (-2,-1)--(2,1) node[anchor = south] {$C=0$};
\end{tikzpicture}
\end{figure}

Розглянемо окремо, коли $A,B,C \neq 0$. Маємо $Ax + By = - C$. Поділимо обидві частини рівності на $C$ -- отримаємо таке:\\
$\dfrac{x}{-\frac{C}{A}} + \dfrac{y}{-\frac{C}{B}} = 1$.\\
Позначмо $-\dfrac{C}{A} = a$, $-\dfrac{C}{B} = b$. Тоді виникне рівняння $\dfrac{x}{a} + \dfrac{y}{b} = 1$.\\
Отже, отримали \textbf{рівняння прямої в відрізках}:
\begin{align*}
\dfrac{x}{a} + \dfrac{y}{b} = 1
\end{align*}
\begin{figure}[H]
\centering
\begin{tikzpicture}
\draw[thick, ->] (-2,0)--(1,0) node[anchor = north] {$x$};
\draw[thick, ->] (0,-2)--(0,3) node[anchor = east] {$y$};
\draw[thick] (-2,-2)--(0.5,3) node[anchor = north] {$l$};
\draw (-1,-2pt)--(-1,2pt) node [anchor = north] {$a$};
\draw (-2pt,2)--(2pt,2) node [anchor = west] {$b$};
\end{tikzpicture}
\end{figure}

Шукаємо тепер рівняння прямої, що проходить через точку $M_0 = (x_0, y_0)$ іншим шляхом.\\
Задано пряму $l$, довільну точку $M = (x,y) \in l$ та спрямований вектор $\vec{a} = (m,n)$. Тоді $\overrightarrow{M_0M} = (x-x_0, y-y_0)$. Оскільки $\vec{a} \parallel \overrightarrow{M_0M}$ то це теж саме, що $\vec{a} = t \overrightarrow{M_0 M} \iff
\dfrac{x-x_0}{m} = \dfrac{y-y_0}{n} = t$.\\
Отже, отримали \textbf{канонічне рівняння прямої, що проходить через точку $M_0$:}
\begin{align*}
\dfrac{x-x_0}{m} = \dfrac{y-y_0}{n}
\end{align*}

\begin{figure}[H]
\centering
\begin{tikzpicture}
\draw[thick] (0,0)--(4,2) node[anchor = north] {$l$};
\draw[fill] (2,1) circle (1pt) node [anchor = north]{$M_0$};
\draw[fill] (3,1.5) circle (1pt) node [anchor = north]{$M$};
\draw[thick, ->] (2,1+1)--(3,1.5+1) node[right] {$\vec{a}$};
\draw[thick, ->] (2,1)--(3,1.5);
\end{tikzpicture}
\caption*{Пряма $l$ описується рівнянням вище}
\end{figure}

Якщо параметризувати, тобто встановити рівність $\dfrac{x-x_0}{m} = \dfrac{y-y_0}{n} = t$, то звідси отримаємо \textbf{параметричне рівняння прямої}:
\begin{align*}
\begin{cases}
x = x_0 + mt\\
y = y_0 + nt
\end{cases}, t \in \mathbb{R}.
\end{align*}

Задані дві точки $M_1 = (x_1,y_1), M_2 = (x_2,y_2)$. Знайдемо рівняння прямої, що проходить через них.\\
Створимо спрямлений вектор $\vec{a} = \overrightarrow{M_2M_1} = (x_2-x_1,y_2-y_1)$. Тоді, використовуючи канонічне рівняння прямої, отримаємо \textbf{рівняння прямої, що проходить через дві точки}:
\begin{align*}
\dfrac{x-x_1}{x_2-x_1} = \dfrac{y-y_1}{y_2-y_1}.
\end{align*}

\subsection{Нормальне рівняння прямої}
У нас вже є загальне рівняння прямої $Ax + By + C = 0$.\\
Поділимо обидві частини рівняння на $\pm \sqrt{A^2+B^2}$, щоб величина $C$ стала недодатною (із плюсюм, коли $C<0$, інакше з мінусом).\\
$\pm \dfrac{A}{\sqrt{A^2+B^2}}x \pm \dfrac{B}{\sqrt{A^2+B^2}}y \pm \dfrac{C}{\sqrt{A^2+B^2}} = 0$.\\
Отримаємо нормальний вектор $\vec{n} = \left(\pm \dfrac{A}{\sqrt{A^2+B^2}}, \pm \dfrac{B}{\sqrt{A^2+B^2}} \right)$, причому $|\vec{n}| = 1$. Більш того, маємо інші деталі:\\
$(\vec{n}, \vec{i}) = \cos \alpha = \pm \dfrac{A}{\sqrt{A^2+B^2}}$;\\
$(\vec{n}, \vec{j}) = \cos \beta = \pm \dfrac{A}{\sqrt{A^2+B^2}}$.\\
Тут $\alpha$ -- кут між нормальним вектором та $OX$, а тут $\beta$ -- кут між нормальним вектором та $OY$.\\
Отже, маємо $\vec{n} = (\cos \alpha, \cos \beta)$.\\
Також позначимо $\pm \dfrac{C}{\sqrt{A^2+B^2}} = -p$, причому $p \geq 0$.\\
Отримаємо \textbf{нормальне рівняння прямої:}
\begin{align*}
x \cos \alpha + y \cos \beta - p = 0\\
p \geq 0
\end{align*}

\begin{figure}[H]
\centering
\begin{tikzpicture}
\draw[thick, ->] (-2,0)--(3,0) node [anchor = north] {$x$};
\draw[thick, ->] (0,-2)--(0,3) node [anchor = east] {$y$};
\node at (0,0) [anchor = north west] {$O$};

\draw[thick] (-2,2.5)--(3,0.5) node [anchor = south] {$l$};
%\draw[fill] (-1,2.1) circle (1pt) node [anchor = south]{$M$};
\node at (-1,2.1) [anchor = south]{$M$};
\node at (-1.5,0.85) [anchor = south]{$P$};
\draw[thick, ->] (-1.5,0.85)--(-1,2.1);
\draw[thick, ->] (0,0)--({2 /sqrt(29)},{5/sqrt(29)}) node[anchor = south]{$\vec{n}$};

\draw[thick, blue] (0.5,0) arc (0:acos(2/sqrt(29)):0.5) node [anchor = west] {$\alpha$};
\draw[thick, red] (0,0.5) arc (90:acos(2/sqrt(29)):0.5) node [anchor = east] {$\beta$};
\end{tikzpicture}
\end{figure}

Час дізнатись, що таке взагалі $p$ і в чому зміст.\\
Нехай задана точка $P = (x_P, y_P) \not \in l$. Візьмемо таку точку $M = (x_M, y_M) \in l$, щоб $\overrightarrow{MP} \parallel \vec{n}$. Тоді розпишемо параметричне рівняння:\\
$\begin{cases}
x_M - x_P = t \cos \alpha\\
y_M - y_P = t \cos \beta
\end{cases}.
$\\
Оскільки $M \in l$, ми можемо її підставити в нормальне рівняння та рівність буде виконуватись:\\
$x_M \cos \alpha + y_M \cos \beta - p = 0$\\
$x_P \cos \alpha + y_P \cos \beta + t(\cos^2 \alpha + \cos^2 \beta) - p = 0$\\
$x_P \cos \alpha + y_P \cos \beta -p = -t$.\\
Маємо таке означення:
\begin{definition}
\textbf{Відхиленням точки $P$ від прямої $l$} називається величина:
\begin{align*}
\delta(P,l) = x_P \cos \alpha + y_P \cos \beta - p
\end{align*}
\end{definition}

Отже, $\delta(P,l)=-t$.\\
Зокрема отримаємо, що відстань від точки $P$ до прямої визначається таким чином:\\
$d(P,l)=|\overrightarrow{PM}| = |t \vec{n}| = |\delta(P,l)|$.\\
Отже, відстань $d(P,l)$ -- це модуль відхилення:
\begin{align*}
d(P,l) = |\delta(P,l)|
\end{align*}
Більш того, $\delta(O,l) = -p \implies d(O,l) = p$.\\
Тобто $p$ відображає відстань від початку координат до прямої:
\begin{align*}
p = d(O,l)
\end{align*}
І наостанок зауважимо інше. Маємо $d(P,l) = |x_P \cos \alpha + y_P \cos \beta - p|$. Її можна переписати іншим шляхом, якщо згадати, що таке $\cos \alpha, \cos \beta, p$ з аналітичної точки зору:
\begin{align*}
d(P,l) = \dfrac{|Ax_P + By_P + C|}{\sqrt{A^2+B^2}}
\end{align*}


\subsection{Площина в просторі}
Знайдемо рівняння площини, що проходить через точку $M_0 = (x_0,y_0,z_0)$.\\
Задамо площину $\pi$, довільну точку $M = (x,y,z) \in \pi$ та нормальний вектор $\vec{n} = (A,B,C)$. Тоді $\overrightarrow{M_0M} = (x-x_0, y-y_0, z-z_0)$. Оскільки $\vec{n} \perp \overrightarrow{M_0 M}$, то це теж саме, що $(\vec{n}, \overrightarrow{M_0 M}) = 0 \iff A(x-x_0)+B(y-y_0)+C(z-z_0)=0$.\\
Отже, отримали \textbf{рівняння площини, що проходить через точку $M_0$}:
\begin{align*}
A(x-x_0) + B(y-y_0) + C(z-z_0) = 0
\end{align*}

\begin{figure}[H]
\centering
\begin{tikzpicture}
\fill[fill = blue!40] (0,0,0)--(0,0,3)--(3,0,3)--(3,0,0)--cycle node [anchor = north] {$\pi$};
\draw[thick, ->] (1.5,0,1.5)--(1.5,1.5,1.5) node[anchor = east] {$\vec{n}$};
\draw[thick, ->] (1.5,0,1.5)--(2.1,0,0.5);
\draw[fill] (1.5,0,1.5) circle (1pt) node [anchor = north]{$M_0$};
\draw[fill] (2.1,0,0.5) circle (1pt) node [anchor = north]{$M$};
\end{tikzpicture}
\caption*{Площина $\pi$ описується рівнянням вище.}
\end{figure}

Якщо розкрити дужки, отримаємо:\\
$Ax + By + Cz \underbrace{-Ax_0 - By_0 - Cz_0}_{= D} = Ax+By+Cz+D = 0$.\\
Отже, отримали \textbf{загальне рівняння площини:}
\begin{align*}
Ax + By + C = 0.
\end{align*}

\subsubsection*{Частинні випадки}
\begin{enumerate}[wide=0pt,label={\Roman*.}]
\item $A = 0$.\\
Оскільки $\vec{n} = (0,B,C)$, то $\vec{n} \perp \vec{i}$ або $\vec{n} \perp OX$. Отже, маємо $By + Cz + D = 0$. Отримали \textbf{площину, що паралельна осі} $OX$.

\item $B = 0$.\\
Оскільки $\vec{n} = (A,0,C)$, то $\vec{n} \perp \vec{j}$ або $\vec{n} \perp OY$. Отже, маємо $Ax + Cz + D = 0$. Отримали \textbf{площину, що паралельна осі} $OY$.

\item $C = 0$.\\
Оскільки $\vec{n} = (A,B,0)$, то $\vec{n} \perp \vec{k}$ або $\vec{n} \perp OZ$. Отже, маємо $Ax + By + D = 0$. Отримали \textbf{площину, що паралельна осі} $OZ$.

\item $A=B=0$.\\
Тоді $\vec{n} = (0,0,C) \perp \vec{i}; \perp \vec{j}$. Отже, маємо $Cz + D = 0$. Отримали \textbf{площину, що паралельна площині} $XOY$.

\item $B=C=0$.\\
Тоді $\vec{n} = (A,0,0) \perp \vec{j}; \perp \vec{k}$. Отже, маємо $Ax + D = 0$. Отримали \textbf{площину, що паралельна площині} $YOZ$.

\item $A=C=0$.
Тоді $\vec{n} = (0,B,0) \perp \vec{i}; \perp \vec{k}$. Отже, маємо $By + D = 0$. Отримали \textbf{площину, що паралельна площині} $XOZ$.

\item $D = 0$.\\
Тоді маємо $Ax + By + Cz = 0$. Отримали \textbf{площину, що проходить через початок координат}.
\end{enumerate}

\begin{figure}[H]
\centering
\begin{tikzpicture}
%plane
\fill[blue!40] (0,1,0)--(2,1,0)--(2,0,1)--(0,0,1) node at (0,-1.5,0) {$B=0$};

\draw[thick, ->, blue] (0,0,0)--(2,0,0) node[anchor = north, black] {$y$};
\draw[thick, ->] (0,0,0)--(0,2,0) node[anchor = east] {$z$};
\draw[thick, ->] (0,0,0)--(0,0,2) node[anchor = north east] {$x$};

\end{tikzpicture}
\qquad
\begin{tikzpicture}
%plane
\fill[blue!40] (0,1,0)--(0,1,1)--(2,1,1)--(2,1,0) node at (0,-1.5,0) {$A=B=0$};

\draw[thick, ->] (0,0,0)--(2,0,0) node[anchor = north] {$y$};
\draw[thick, ->] (0,0,0)--(0,2,0) node[anchor = east] {$z$};
\draw[thick, ->] (0,0,0)--(0,0,2) node[anchor = north east] {$x$};

\end{tikzpicture}
\qquad
\begin{tikzpicture}
%plane
\fill[blue!40] (-1,0,0.5)--(1,0,-0.5)--(1,1,-0.5)--(-1,1,0.5) node at (0,-1.5,0) {$D=0$};

\draw[thick, ->] (0,0,0)--(2,0,0) node[anchor = north] {$y$};
\draw[thick, ->] (0,0,0)--(0,2,0) node[anchor = east] {$z$};
\draw[thick, ->] (0,0,0)--(0,0,2) node[anchor = north east] {$x$};

\end{tikzpicture}
\end{figure}

Розглянемо окремо, коли $A,B,C,D \neq 0$. Маємо $Ax + By + Cz = - D$. Поділимо обидві частини рівності на $D$ -- отримаємо таке:\\
$\dfrac{x}{-\frac{D}{A}} + \dfrac{y}{-\frac{D}{B}} + \dfrac{z}{-\frac{D}{C}} = 1$.\\
Позначмо $-\dfrac{C}{A} = a$, $-\dfrac{B}{A} = b$, $-\dfrac{z}{-\frac{D}{C}} = c$. Тоді виникне рівняння $\dfrac{x}{a} + \dfrac{y}{b} + \dfrac{z}{c} = 1$.\\
Отже, отримали \textbf{рівняння площини в відрізках}:
\begin{align*}
\dfrac{x}{a} + \dfrac{y}{b} + \dfrac{z}{c} = 1
\end{align*}

\begin{figure}[H]
\centering
\begin{tikzpicture}
%plane
\fill[blue!40] (1,0,0)--(0,2,0)--(0,0,3);

%axis
\draw[thick, dashed] (0,0,0)--(1,0,0);
\draw[thick, dashed] (0,0,0)--(0,2,0);
\draw[thick, dashed] (0,0,0)--(0,0,3);
\draw[thick, ->] (1,0,0)--(1.5,0,0) node[anchor = north] {$y$};
\draw[thick, ->] (0,2,0)--(0,2.5,0) node[anchor = east] {$z$};
\draw[thick, ->] (0,0,3)--(0,0,3.5) node[anchor = north east] {$x$};

%points
\node at (1,0,0) [anchor = south] {$b$};
\node at (0,2,0) [anchor = east] {$c$};
\node at (0,0,3) [anchor = south east] {$a$};
\node at (0,0.8,0.5) {$\pi$};
\end{tikzpicture}
\end{figure}

Задані три точки $M_1 = (x_1,y_1,z_1), M_2 = (x_2,y_2,z_2), M_3 = (x_3,y_3,z_3)$. Знайдемо рівняння площини, що проходить через них.\\
Оберему довільну точку $M = (x,y,z)$. Ця точка $M \in (M_1 M_2 M_3) \iff \overrightarrow{M_1M}, \overrightarrow{M_2M_1}, \overrightarrow{M_3M_1}$ -- компланарні $\iff (\overrightarrow{M_1M}, \overrightarrow{M_2M_1}, \overrightarrow{M_3M_1}) = 0$.\\
Отримаємо \textbf{рівняння площини, що проходить через три точки}:
\begin{align*}
\begin{vmatrix}
x-x_1 & y-y_1 & z-z_1 \\
x_2-x_1 & y_2-y_1 & z_2-z_1 \\
x_3-x_1 & y_3-y_1 & z_3-z_1
\end{vmatrix} = 0
\end{align*}

\begin{figure}[H]
\centering
\begin{tikzpicture}
\fill[blue!40] (0,0)--(1,1)--(4,1)--(3,0)--(0,0);
\draw[fill] (1,0.5) circle (1pt) node [anchor = north]{$M_1$};
\draw[fill] (2,0.7) circle (1pt) node [anchor = north]{$M_2$};
\draw[fill] (3,0.3) circle (1pt) node [anchor = north]{$M_3$};
\end{tikzpicture}
\end{figure}

\subsection{Нормальне рівняння площини}
У нас вже є загальне рівняння площини $Ax + By + Cz + D = 0$. Поділимо обидві частини рівняння на $\pm \sqrt{A^2+B^2+C^2}$, щоб величина $D$ стала недодатною (плюс, коли $D<0$, інакше мінус).\\
$\pm \dfrac{A}{\sqrt{A^2+B^2+C^2}}x \pm \dfrac{B}{\sqrt{A^2+B^2+C^2}}y \pm \dfrac{C}{\sqrt{A^2+B^2+C^2}}z \pm \dfrac{D}{\sqrt{A^2+B^2+C^2}} = 0$.\\
Отримаємо нормальний вектор $\vec{n} = \left(\pm \dfrac{A}{\sqrt{A^2+B^2+C^2}}, \pm \dfrac{B}{\sqrt{A^2+B^2+C^2}}, \pm \dfrac{C}{\sqrt{A^2+B^2+C^2}} \right)$, причому $|\vec{n}| = 1$. Більш того, маємо інші деталі:\\
$(\vec{n}, \vec{i}) = \cos \alpha = \pm \dfrac{A}{\sqrt{A^2+B^2+C^2}}$;\\
$(\vec{n}, \vec{j}) = \cos \beta = \pm \dfrac{B}{\sqrt{A^2+B^2+C^2}}$;\\
$(\vec{n}, \vec{k}) = \cos \gamma = \pm \dfrac{C}{\sqrt{A^2+B^2+C^2}}$.\\
Тут $\alpha$ -- кут між нормаллю та $OX$, $\beta$ -- кут між нормальним вектором та $OY$, $\gamma$ -- кут між нормальним вектором та $OZ$.\\
Отже, маємо $\vec{n} = (\cos \alpha, \cos \beta, \cos \gamma)$.\\
Також вважатимемо, що $\pm \dfrac{D}{\sqrt{A^2+B^2+C^2}} = -p$, причому $p \geq 0$.\\
Отримаємо \textbf{нормальне рівняння площини}:
\begin{align*}
x \cos \alpha + y \cos \beta + z \cos \gamma - p = 0 \\
p \geq 0
\end{align*}

\begin{figure}[H]
\centering
\begin{tikzpicture}
%plane
\fill[blue!40] (1,3,1)--(5,1,1)--(5,2,-1)--(1,4,-1)--cycle;
\node at (2.5,5,3) {$\pi$};

%axis
\draw[thick, ->] (0,0,0)--(6,0,0) node[anchor = north] {$y$};
\draw[thick, ->] (0,0,0)--(0,5,0) node[anchor = east] {$z$};
\draw[thick, ->] (0,0,0)--(0,0,4) node[anchor = north east] {$x$};

%normal
\draw[thick, ->] (0,0,0)--(1,2,1) node[anchor = south west] {$\vec{n}$};

%angle
\draw[thick, blue] (0.5,0) arc (0:acos(1/sqrt(6)):0.5) node [anchor = west] {$\beta$};
\draw[thick, red] (0,0.5) arc (90:acos(1/sqrt(6)):0.5) node [anchor = east] {$\gamma$};
%\draw[thick, red] ({1/sqrt(6)},{2/sqrt(6)},{1/sqrt(6)}) arc (0:90:0.5) node [anchor = east] {$\alpha$};

%points
\draw[fill] (4,3,2) circle (1pt) node [anchor = north]{$M$};
\draw[fill] (4+1,3+2,2+1) circle (1pt) node [anchor = south]{$P$};
\draw[thick, ->] (4+1,3+2,2+1)--(4,3,2);
\end{tikzpicture}
\end{figure}

Час дізнатись, що таке взагалі $p$ і в чому зміст (спойлер: зміст аналогічний).\\
Нехай задана точка $P = (x_P, y_P, z_P) \not \in \pi$. Візьмемо таку точку $M = (x_M, y_M, z_M) \in \pi$, щоб $\overrightarrow{MP} \parallel \vec{n}$. Тоді $\overrightarrow{MP} = t \vec{n}$.\\
$\begin{cases}
x_M - x_P = t \cos \alpha\\
y_M - y_P = t \cos \beta\\
z_M - z_P = t \cos \gamma\\
\end{cases}
$\\
Оскільки $M \in \pi$, ми можемо її підставити та рівність буде виконуватись:\\
$x_M \cos \alpha + y_M \cos \beta + z_M \cos \gamma - p = 0$\\
$x_P \cos \alpha + y_P \cos \beta + z_P \cos \gamma + t(\cos^2 \alpha + \cos^2 \beta + \cos^2 \gamma) - p = 0$\\
$x_P \cos \alpha + y_P \cos \beta + z_P \cos \gamma -p = -t$.\\
Маємо таке означення:
\begin{definition}
\textbf{Відхиленням точки $P$ від площини $\pi$} називається величина:
\begin{align*}
\delta(P,\pi) = x_P \cos \alpha + y_P \cos \beta + z_P \cos \gamma - p
\end{align*}
\end{definition}
Отже, $\delta(P,\pi)=-t$.\\
Зокрема отримаємо, що відстань від точки $P$ до площини визначається таким чином:\\
$d(P,\pi)=|\overrightarrow{PM}| = |t \vec{n}| = |\delta(P,\pi)|$.\\
Отже, відстань $d(P,\pi)$ -- модуль відхилення:
\begin{align*}
d(P,\pi) = |\delta(P,\pi)|
\end{align*}
Більш того, $\delta(O,\pi) = -p \implies d(O,\pi) = p$.\\
Тобто $p$ відображає відстань від початку координат до площини:
\begin{align*}
p = d(O,\pi)
\end{align*}
І наостанок зауважимо інше. Маємо $d(P,\pi) = |x_P \cos \alpha + y_P \cos \beta + z_P \cos \gamma - p|$. Її можна переписати іншим шляхом, якщо згадати, що таке $\cos \alpha, \cos \beta, \cos \gamma, p$ з аналітичної точки зору:
\begin{align*}
d(P,\pi) = \dfrac{|Ax_P + By_P + Cz_P + D|}{\sqrt{A^2+B^2+C^2}}.
\end{align*}


\subsection{Пряма в просторі}
Із геометрії відомо, що дві площини перетинаються по прямій. Тому якщо задати два рівняння площини з нормальними векторами $\vec{n_1} = (A_1,B_1,C_1), \vec{n_2} = (A_2,B_2,C_2)$, причому вони не є паралельними, то отримаємо \textbf{рівняння прямої, за якою перетинаються дві площини}:
\begin{align*}
\begin{cases}
A_1x + B_1y + C_1z + D_1 = 0 \\
A_2x + B_2y + C_2z + D_2 = 0
\end{cases}
\end{align*}

Шукаємо тепер рівняння прямої, що проходить через точку $M_0 = (x_0, y_0,z_0)$. Тобто рівняння тепер буде в просторі.\\
Задано пряму $l$, довільну точку $M = (x,y,z) \in l$ та спрямований вектор $\vec{a} = (m,n,p)$. Тоді $\overrightarrow{M_0M} = (x-x_0, y-y_0, z-z_0)$.\\
Оскільки $\vec{a} \parallel \overrightarrow{M_0M}$ то це теж саме, що $\vec{a} = t \overrightarrow{M_0 M} \iff \dfrac{x-x_0}{m} = \dfrac{y-y_0}{n} = \dfrac{z-z_0}{p} = t$.\\
Отже, отримали \textbf{канонічне рівняння прямої, що проходить через точку $M_0$}:
\begin{figure}[H]
\centering
\begin{tikzpicture}
\draw[thick] (0,0)--(4,2) node[anchor = north] {$l$};
\draw[fill] (2,1) circle (1pt) node [anchor = north]{$M_0$};
\draw[fill] (3,1.5) circle (1pt) node [anchor = north]{$M$};
\draw[thick, ->] (2,1+1)--(3,1.5+1) node[right] {$\vec{a}$};
\draw[thick, ->] (2,1)--(3,1.5);
\end{tikzpicture}
\caption*{Пряма $l$ описує рівняння вище.}
\end{figure}

Якщо параметризувати, тобто $\dfrac{x-x_0}{m} = \dfrac{y-y_0}{n} = \dfrac{z-z_0}{p} = t$, то звідси отримаємо \textbf{параметричне рівняння прямої}:
\begin{align*}
\begin{cases}
x = x_0 + mt\\
y = y_0 + nt\\
z = z_0 + pt
\end{cases}, t \in \mathbb{R}
\end{align*}

Задані дві точки $M_1 = (x_1,y_1,z_1), M_2 = (x_2,y_2,z_2)$. Знайдемо рівняння прямої, що проходить через них.\\
Створимо спрямлений вектор $\vec{a} = \overrightarrow{M_2M_1} = (x_2-x_1,y_2-y_1,z_2-z_1)$. Тоді, використовуючи канонічне рівняння прямої, отримаємо \textbf{рівняння прямої, що проходить через дві точки}:
\begin{align*}
\dfrac{x-x_1}{x_2-x_1} = \dfrac{y-y_1}{y_2-y_1} = \dfrac{z-z_1}{z_2-z_1}
\end{align*}

\subsection{Відстані}
\subsubsection{Від точки до прямої в просторі}
Маємо канонічне рівняння прямої, що проходить через точку $M_0 = (x_0,y_0,z_0)$: $\dfrac{x-x_0}{m} = \dfrac{y-y_0}{n} = \dfrac{z-z_0}{p}$. Зафіксуємо точку $M \in l$. Знайдемо відстань від точки $P$ до прямої.
\begin{figure}[H]
\centering
\begin{tikzpicture}
\draw[thick] (0,0)--(4,2) node[anchor = north] {$l$};
\draw[fill] (1,0.5) circle (1pt) node [anchor = north west]{$M_0$};
\draw[fill] (3,1.5) circle (1pt) node [anchor = north west]{$M$};
\draw[thick, ->] (1,0.5)--(3,1.5);
\draw[thick, ->] (1,0.5)--(1,3);
\draw[thick, dashed] (1,3)--(3,4);
\draw[thick, dashed] (3,1.5)--(3,4);
\draw[thick, ->] (2,1-1)--(3,1.5-1) node[right] {$\vec{a}$};
\draw[fill] (1,3) circle (1pt) node [anchor = south]{$P$};
\draw[thick, red] (1,3)--(2,1) node at (2.2,2) [scale=0.8] {$d(P, l)$};
\draw[thick] (1.8,1.4)--(1.3,1.15)--(1.5,1.15-0.4);
\end{tikzpicture}
\end{figure}
Із геометричних міркувань, $S_{\textrm{паралелограм}} = |\overrightarrow{M_0M}| \cdot d(P,l)$.\\
Але з іншого боку, $S_{\textrm{паралелограм}} = |[\overrightarrow{M_0M},\overrightarrow{M_0P}]|$.\\
Разом отримаємо:
\begin{align*}
d(P,l)= \dfrac{\abs{\left[\overrightarrow{M_0M},\overrightarrow{M_0P} \right]}}{|\overrightarrow{M_0 M}|}
\end{align*}

\subsubsection{Від точки до площини}
Насправді, ми вже ці формули знайшли, коли мова йшла про нормальне рівняння.\\
$d(P,\pi) = \dfrac{|Ax_0+By_0+Cz_0+D|}{\sqrt{A^2+B^2+C^2}}$;\\
$d(P,\pi) = |x_P \cos \alpha + y_P \cos \beta +z_P \cos \gamma - p|$.

\subsubsection{Між двома прямими в просторі}
\begin{enumerate}[wide=0pt,label={\Roman*.}]
\item $l_1 \parallel l_2$.\\
Можна обрати довільну точку прямої $l_2$ та рахувати відстань за пунктом 2.6.1.

\item $l_1, l_2$ -- мимобіжні.\\
Необхідно знайти довжину їхнього спільного перпендикуляру.
\begin{figure}[H]
\def\size{3}
\centering
\begin{tikzpicture}
%cube build%
\draw[thick, dashed] (0,0,\size)--(0,0,0)--(\size,0,0);
\draw[thick] (\size,0,0)--(\size,0,\size)--(0,0,\size);
\draw[thick] (0,0,\size)--(0,\size,\size+2)--(\size,\size,\size+2)--(\size,0,\size);
\draw[thick] (\size,0,0)--(\size,\size,2)--(\size,\size,\size+2);
\draw[thick] (0,\size,\size+2)--(0,\size,2)--(\size,\size,2);
\draw[thick, dashed] (0,0,0)--(0,\size,2);

\draw[thick] (-1,0,\size)--(\size+1,0,\size) node [anchor = west] {$l_1$};
\draw[thick] (0,\size,\size+2+1)--(0,\size,2-1) node [anchor = south west] {$l_2$};

\draw[thick, red] (0,\size,\size+2)--(0,0,\size+2) node[anchor = north] {$d(l_1,l_2)$};
\draw[thick, dashed] (0,0,\size+2)--(\size+1,0,\size+2) node[anchor = north] {$l_1$};
\draw[thick, ->] (0,0,\size)--(\size,0,\size) node [anchor = north] {$\vec{a_1}$};
\draw[thick, ->] (0,\size,\size+2)--(0,\size,2) node [anchor = south east] {$\vec{a_2}$};
\draw[thick, ->] (0,0,\size)--(0,\size,\size+2);
\node at (0,0,\size) [anchor = north] {$M_1$};
\node at (0,\size,\size+2) [anchor = south east] {$M_2$};
\end{tikzpicture}
\end{figure}

В цьому випадку, червона лінія відповідає висоті паралелепіпеда:\\
$V_{\textrm{паралелепіпед}} = S_{\textrm{паралелограм}} \cdot h$.\\
Отже,
\begin{align*}
d(l_1,l_2) = \dfrac{\abs{\left(\overrightarrow{M_1M_2}, \vec{a_1}, \vec{a_2}\right)}}{\abs{[\vec{a_1}, \vec{a_2}]}}
\end{align*}
\end{enumerate}

Ми додатково знайдемо рівняння спільного перпендикуляру за цим малюнком:\\
$l_1: \begin{cases}
x = x_1 + tm_1 \\
y = y_1 + tn_1 \\
z = z_1 + tp_1 \\
\end{cases}$ \hspace{1cm}
$l_2: \begin{cases}
x = x_2 + sm_2 \\
y = y_2 + sn_2 \\
z = z_2 + sp_2 \\
\end{cases}$.
\\
Нехай $N_1 \in l_1$ та $N_2 \in l_2$ -- точки спільного перпендикуляру. Тоді $\overrightarrow{N_1 N_2} \perp \vec{a_1}; \overrightarrow{N_1 N_2} \perp \vec{a_2} \iff \begin{cases}
 (\overrightarrow{N_1 N_2}, \vec{a_1}) = 0\\
 (\overrightarrow{N_1 N_2}, \vec{a_2}) = 0\\
 \end{cases}$.\\
Якщо це розписати, то отримаємо систему двох рівнянь з невідомими $t,s$.

\iffalse
%Optional topic
\subsection{*Жмутки}
\defin{2.7.1. Жмутком} прямих на площині називають множину всіх прямих, що проходят через одну фіксовану точку - центр жмутку\\
\begin{tikzpicture}
\draw[thick, ->] (-2,0)--(2.2,0) node[anchor = north west] {$x$};
\draw[thick, ->] (0,-2)--(0,2.2) node[anchor = south east] {$y$};
\draw[fill = black] (1,0.5) circle[radius=2pt];
\draw (-2,2)--(2,0);
\draw (-2,-0.1)--(2,0.7);
\end{tikzpicture}\\
Жмуток однозначно визначається будь-якою парою прямих $l_1$, $l_2$, що не є паралельними\\
$\begin{gathered}
l_1: A_1x+B_1y+C_1 = 0\\
l_2: A_2x + B_2y+C_2 = 0
\end{gathered}
$\\
Позначимо $\begin{gathered}
U(x,y) = A_1x+B_1y+C_1\\
V(x,y) = A_2x + B_2y+C_2
\end{gathered}$\\
\th{2.7.2.} Якщо $l_1$, $l_2$ визначають жмуток, то рівняння жмутка матиме вигляд:\\
$\alpha U(x,y) + \beta V(x,y) = 0$, тут $\alpha^2 + \beta^2 \neq 0$\\
\proof
Зафіксуємо центр жмутку $M_0 = (x_0,y_0)$\\
Оскільки $l_1 \not\parallel l_2$, то відповідно $\vec{n_1} \not\parallel \vec{n_2}$. Візьмемо довільний вектор нормалі $\vec{n}$ на площині та розкладемо за двома неколінеарними векторами:\\
$\vec{n} = \alpha \vec{n_1} + \beta \vec{n_2} = (\alpha A_1 + \beta A_2, \alpha B_1+\beta B_2)$\\
Тут одразу зауважимо, що $\alpha \neq 0$, тому що отримаємо, що $\vec{n} \parallel \vec{n_2}$, а це задає рівняння прямої $l_2$, що не цікаво\\
Аналогічними міркуваннями $\beta \neq 0$\\
Тому і виникає додаткова умова: $\alpha^2 + \beta^2 \neq 0$\\
Тоді ми можемо записати рівняння прямої, що проходить через т. $M_0$:\\
$(\alpha A_1 + \beta A_2)(x-x_0) + (\alpha B_1 + \beta B_2)(y-y_0) = 0$\\
$\alpha(A_1x+B_1y-A_1x_0-B_1y_0) + \beta(A_2x+B_2y-A_2x_0-B_2y_0) = 0$\\
Покладемо $C_1 = -A_1x_0 -B_1y_0$ та $C_2 = -A_2x_0-B_2y_0$. Тоді\\
$\alpha U(x,y) + \beta V(x,y) = 0$ \qed
\bigline
Це все означає, що ми можемо отримати будь-яку пряму, що проходить через т. $M_0 =(x_0,y_0)$
\fi
\newpage

\section{Криві та поверхні другого порядку}
\subsection*{Криві другого порядку}
\begin{definition}
\textbf{Загальне рівняння кривого другого порядку} визначається такою формулою:
\begin{align*}
a_{11}x^2 + 2a_{12}xy + a_{22}y^2 + b_1x+b_2y+c=0
\end{align*}
\end{definition}

\subsection{Еліпс}
\begin{definition}
\textbf{Еліпсом} називають ГМТ, для яких сума відстаней від двох фіксованих точок $F_1,F_2$ -- фокуси еліпса -- є постійною.
\begin{figure}[H]
\centering
\begin{tikzpicture}
\draw (0,0) ellipse (2 and 1);
\draw[fill] (1.2,0) circle[radius=1 pt] node[anchor = north] {$F_1$};
\draw[fill] (-1.2,0) circle[radius=1 pt] node[anchor = north] {$F_2$};
\draw[thick] (-1.2,0)--(0,1)--(1.2,0) node at (0,1) [anchor = south] {$M$};
\end{tikzpicture}
\end{figure}
\end{definition}

Тобто задані дві точки $F_1 = (x_1,y_1)$, $F_2 = (x_2,y_2)$.\\
$M = (x,y)$ -- точка еліпсу $\iff |MF_1|+|MF_2| = 2a$.
\bigskip \\
Нехай $F_1 = (-c,0), F_2 = (c,0), c > 0$. Тоді $|F_1F_2|=2c$.\\
При $2c > 2a$ випадок не є можливим (нерівність трикутника підтвердить).\\
При $2c = 2a$ отримаємо, що точку $M$ належить відрізку $|F_1F_2|$.\\
При $2c < 2a$ знайдемо рівняння:\\
$|MF_1|=\sqrt{(x+c)^2+y^2}, |MF_2|=\sqrt{(x-c)^2+y^2} \implies$\\
$\sqrt{(x+c)^2+y^2} + \sqrt{(x-c)^2+y^2} = 2a$\\
$\sqrt{(x+c)^2+y^2} = 2a - \sqrt{(x-c)^2+y^2}$\\
$(x+c)^2+y^2=4a^2+(x-c)^2+y^2-4a\sqrt{(x-c)^2+y^2}$\\
$4a\sqrt{(x-c)^2+y^2} = 4a^2 -4cx$\\
$a^2(x^2-2cx+c^2+y^2)=a^4-2a^2cx+c^2x^2$\\
$(a^2-c^2)x^2+a^2c^2+a^2y^2=a^4$\\
Позначимо $a^2-c^2=b^2>0$. Отримаємо наступне:\\
$b^2x^2+a^2c^2=a^4-a^2c^2=a^2b^2$.\\
Нарешті, отримаємо \textbf{канонічне рівняння еліпса}:
\begin{align*}
\frac{x^2}{a^2} + \frac{y^2}{b^2} = 1
\end{align*}
із фокусами на $OX$, що симетричні відносно початку координат, а також $b^2 = a^2 - c^2$.
\bigskip \\
Зробимо перевірку для початкового рівняння:\\
$\dfrac{x^2}{a^2} + \dfrac{y^2}{b^2} = 1 \implies y^2 = b^2\left(1 - \dfrac{x^2}{a^2} \right)$\\
$|MF_1| = \sqrt{(x+c)^2+y^2} = \sqrt{x^2+2xc+c^2+b^2 - \dfrac{x^2}{a^2}b^2} = \\ = \sqrt{x^2 \left(1 - \dfrac{b^2}{a^2} \right) + 2xc + a^2} = \sqrt{x^2 \dfrac{c^2}{a^2} + 2xc + a^2} = \sqrt{\left(\dfrac{xc}{a} + a\right)^2} = \abs{\dfrac{xc}{a}+a} = \dfrac{c}{a} \abs{x + \dfrac{a^2}{c}}$\\
Із рівняння $\dfrac{x^2}{a^2} + \dfrac{y^2}{b^2} = 1$ випливає, що $\dfrac{x^2}{a^2} \leq 1 \implies |x| \leq a = a \dfrac{a}{c} = \dfrac{a^2}{c}$.\\
Тому $|MF_1| = \dfrac{c}{a} \left(x + \dfrac{a^2}{c} \right)$. Аналогічно для $|MF_2|$. Тому $|MF_2| = \dfrac{c}{a} \left(\dfrac{a^2}{c} - x\right)$.\\
$\implies |MF_1| + |MF_2| = 2a$.
\bigline
Отже, канонічне $\dfrac{x^2}{a^2} + \dfrac{y^2}{b^2} = 1$ -- нормально задане та описує еліпс.
\begin{figure}[H]
\centering
\begin{tikzpicture}
\draw[thick, ->] (-4.5,0)--(4.7,0) node [anchor = north] {$x$};
\draw[thick, ->] (0,-2.5)--(0,2.7) node [anchor = east] {$y$};

\draw (-4,-2pt)--(-4,2pt) node [anchor = north east] {$-a$};
\draw (4,-2pt)--(4,2pt) node [anchor = north west] {$a$};
\draw (-2pt,-2)--(2pt,-2) node [anchor = north east] {$-b$};
\draw (-2pt,2)--(2pt,2) node [anchor = south east] {$b$};

\draw[dashed] (-4,-2)--(-4,2)--(4,2)--(4,-2)-- cycle;

\draw (0,0) ellipse (4 and 2);
\end{tikzpicture}
\end{figure}

\begin{definition}
Прямі $x = \pm \dfrac{a^2}{c}$ називають \textbf{директрисами еліпса}.\\
Позначення: $\dir_1 = x + \dfrac{a^2}{c}$ \qquad $\dir_2 = x - \dfrac{x^2}{c}$.
\end{definition}

\begin{remark}
Директриси завжди знаходяться за межами еліпсу, оскільки $c < a$, тому $\dfrac{a^2}{c} > a$ і $-\dfrac{a^2}{c} < -a$.
\end{remark}

\begin{definition}
Величину $\dfrac{c}{a}$ називають \textbf{ексцентриситетом}.\\
Позначення: $\varepsilon = \dfrac{c}{a}$.
\end{definition}

\begin{remark}
$\varepsilon < 1$ для випадку еліпса.
\end{remark}

Повернімось до $|MF_1|,|MF_2|$. Там, насправді, "стоїть" \textrm{} відстань від точки $M$, що належить еліпсу, до директрис. Рівності можна записати таким чином:\\
$|MF_1| = \varepsilon \cdot d(\dir_1, M)$ \qquad $|MF_2| = \varepsilon \cdot d(\dir_2, M)$\\
або\\
$\dfrac{|MF_1|}{d(dir_1, M)} = \varepsilon$ \qquad $\dfrac{|MF_2|}{d(dir_2, M)} = \varepsilon$.

\begin{remark}
Директриса еліпса -- прямі, що перпендикулярні до фокальної осі $F_1F_2$ та віддалені від центра еліпса на відстань $\dfrac{a^2}{c}$.
\end{remark}

\begin{theorem}[Геометрична характеристика еліпсу]
Точка $M$ належить деякому еліпсу $\iff \dfrac{d(M,F)}{d(M,\dir)} = \varepsilon$.
\end{theorem}

\begin{proof}
При паралельному переносі та повороті відстані не змінюються. Тому розташуємо еліпс канонічним чином. А для канонічно розташованого еліпсу співвідношення $\dfrac{d(M,F)}{d(M,\dir)} = \varepsilon$ вірне.
\end{proof}

\subsection{Гіпербола}
\begin{definition}
\textbf{Гіперболой} називають ГМТ, для яких модуль різниці відстаней від двох фіксованих точок $F_1,F_2$ -- фокуси гіперболи -- є постійною.
\end{definition}

Тобто задані дві точки $F_1 = (x_1,y_1)$, $F_2 = (x_2,y_2)$.\\
$M = (x,y)$ -- точка гіперболи $\iff ||MF_1|-|MF_2|| = 2a$.
\bigskip \\
Нехай $F_1 = (-c,0), F_2 = (c,0), c>0$. Тоді $|F_1F_2| = 2c$.\\
Випадок $2c<2a$ не є можливим, а при $2c = 2a$ маємо пряму, що містить 'пробіл' між двома точками.\\
При $2c>2a$ знайдемо рівняння:\\
$|MF_1|=\sqrt{(x+c)^2+y^2}, |MF_2|=\sqrt{(x-c)^2+y^2} \Rightarrow$\\
$\abs{\sqrt{(x+c)^2+y^2}-\sqrt{(x-c)^2+y^2}}=2a$\\
$\sqrt{(x+c)^2+y^2} = 2a + \sqrt{(x-c)^2+y^2}$\\
$x^2+2cx+c^2+y^2=4a^2+4a\sqrt{(x-c)^2+y^2}+x^2-2cx+c^2+y^2$\\
$a\sqrt{(x-c)^2+y^2}=-a^2+xc$\\
$a^2(x^2-2cx+c^2+y^2)=a^4-2a^2cx+c^2x^2$\\
$x^2(a^2-c^2)+a^2y^2=a^2(a^2-c^2)$\\
Позначимо $c^2-a^2=b^2>0$. Отримаємо наступне:\\
$-b^2x^2+a^2y^2=-a^2b^2$.\\
Нарешті, отримаємо \textbf{канонічне рівняння гіперболи}:
\begin{align*}
\frac{x^2}{a^2} - \frac{y^2}{b^2} = 1,
\end{align*}
із фокусами на $OX$, що симетричні відносно початку координат, а також $b^2 = c^2 - a^2$.
\bigskip \\
Зробимо перевірку на початкове рівняння:\\
$\dfrac{x^2}{a^2} - \dfrac{y^2}{b^2} = 1 \implies y^2 = b^2 \left(\dfrac{x^2}{a^2} -1 \right)$.\\
$|MF_1| = \sqrt{(x+c)^2+y^2} = \sqrt{x^2+2cx+c^2+b^2\dfrac{x^2}{a^2} - b^2} = \\
= \sqrt{\left(1 + \dfrac{b^2}{a^2} \right)x^2 + 2cx + c^2 - b^2} = \sqrt{\dfrac{c^2}{a^2}x^2 + 2cx + a^2} = \sqrt{\left(\dfrac{c}{a}x + a \right)^2} = \abs{\dfrac{c}{a}x + a} = \dfrac{c}{a} \abs{x + \dfrac{a^2}{c}}$.\\
Із рівняння $\dfrac{x^2}{a^2} - \dfrac{y^2}{b^2} = 1$ випливає, що $\dfrac{x^2}{a^2} \geq 1 \implies |x| \geq a > \dfrac{a^2}{c}$.\\
Тому $|MF_1| = \left[ \begin{gathered} \dfrac{c}{a} \left(x + \dfrac{a^2}{c} \right), x > 0 \\ -\dfrac{c}{a} \left(x + \dfrac{a^2}{c} \right), x < 0 \end{gathered} \right.$.\\
Аналогічно для $|MF_2|$. Тому $|MF_2| = \left[ \begin{gathered} \dfrac{c}{a} \left(x - \dfrac{a^2}{c} \right), x > 0 \\ -\dfrac{c}{a} \left(x - \dfrac{a^2}{c} \right), x < 0 \end{gathered} \right.$\\
$\implies ||MF_1| - |MF_2|| = 2a$.
\bigskip \\
Отже, $\dfrac{x^2}{a^2} - \dfrac{y^2}{b^2} = 1$ -- нормально задане та описує гіперболу.
\begin{figure}[H]
\centering
\begin{tikzpicture}
\draw[thick, ->] (-4.5,0)--(4.5,0) node [anchor = north] {$x$};
\draw[thick, ->] (0,-3.5)--(0,3.5) node [anchor = east] {$y$};

\pgfmathsetmacro{\e}{1.4}   % eccentricity
\pgfmathsetmacro{\a}{1}
\pgfmathsetmacro{\b}{(\a*sqrt((\e)^2-1)}
\draw (-\a,-2pt)--(-\a,2pt) node [anchor = north east] {$-a$};
\draw (\a,-2pt)--(\a,2pt) node [anchor = north west] {$a$};
%\draw (-2pt,-2)--(2pt,-2) node [anchor = north east] {$-b$};
%\draw (-2pt,2)--(2pt,2) node [anchor = south east] {$b$};
\draw plot[domain=-2:2] ({\a*cosh(\x)},{\b*sinh(\x)});
\draw plot[domain=-2:2] ({-\a*cosh(\x)},{\b*sinh(\x)});
\draw[dashed, domain=-3.5:3.5, variable=\x, samples = 1000] plot({\x}, {\b/\a*\x}) node [anchor = south east] {$y = \dfrac{b}{a}x$};
\draw[dashed, domain=-3.5:3.5, variable=\x, samples = 1000] plot({\x}, {-\b/\a*\x}) node [anchor = north east] {$y = -\dfrac{b}{a}x$};
\end{tikzpicture}
\end{figure}

\begin{definition}
Рівняння $y = \pm \dfrac{b}{a}x$ є \textbf{асимптотами гіперболи}.
\end{definition}

\begin{definition}
Прямі $x = \pm \dfrac{a^2}{c}$ називають \textbf{директрисами гіперболи}.\\
Позначення: $\dir_1 = x + \dfrac{a^2}{c}$ \qquad $\dir_2 = x - \dfrac{x^2}{c}$.
\end{definition}

\begin{remark}
Директриси завжди знаходяться між гіперболами, оскільки $c > a$, тому $\dfrac{a^2}{c} < a$ і $-\dfrac{a^2}{c} > -a$
\end{remark}

\begin{definition}
Величину $\dfrac{c}{a}$ називають \textbf{ексцентриситетом}.\\
Позначення: $\varepsilon = \dfrac{c}{a}$.
\end{definition}

\begin{remark}
$\varepsilon > 1$ для впиадку гіперболи.
\end{remark}

Повернімось до $|MF_1|,|MF_2|$. Там, насправді, "стоїть" \textrm{} відстань від точки $M$, що належить еліпсу, до директрис. Рівності можна записати таким чином:\\
$|MF_1| = \varepsilon \cdot d(\dir_1, M)$ \qquad $|MF_2| = \varepsilon \cdot d(\dir_2, M)$\\
або\\
$\dfrac{|MF_1|}{d(\dir_1, M)} = \varepsilon$ \qquad $\dfrac{|MF_2|}{d(\dir_2, M)} = \varepsilon$.

\begin{remark}
Директриса гіперболи -- прямі, що перпендикулярні до фокальної осі $F_1F_2$ та віддалені від початку координат на відстань $\dfrac{a^2}{c}$.
\end{remark}

\begin{theorem}[Геометрична характеристика гіперболи]
Точка $M$ належить деякій гіперболі $\iff \dfrac{d(M,F)}{d(M,\dir)} = \varepsilon$.
\end{theorem}

\begin{proof}
При паралельному переносі та повороті відстані не змінюються. Тому розташуємо гіперболу канонічним чином. А для канонічно розташованої гіперболи співвідношення $\dfrac{d(M,F)}{d(M,\dir)} = \varepsilon$ вірне.
\end{proof}

\subsection{Парабола}
\begin{definition}
\textbf{Параболою} називають ГМТ, для яких відстань до фіксованої точки -- фокуса $F$ -- дорівнює відстані до фіксованої прямої -- директриси $\dir$.
\end{definition}

Тобто $M=(x,y)$ -- точка параболи $\iff |MF|=d(M,\dir)$, або $\dfrac{MF}{d(M,\dir)}=1$.
\bigskip \\
Нехай $F = (p,0)$, а діректриса: $\dir = x + p$, тобто $x = -p$.\\
$d(M,F)=\sqrt{(x-p)^2+y^2}$\\
$d(M,dir)=|x+p|$
$\Rightarrow \sqrt{(x-p)^2+y^2}=|x+p|$\\
$x^2-2px+p^2+y^2=x^2+2px+p^2$.\\
Отримаємо \textbf{канонічне рівняння параболи:}
\begin{align*}
y^2 = 4px
\end{align*}

\begin{figure}[H]
\centering
\begin{tikzpicture}
\draw[thick, ->] (-1.5,0)--(4.5,0) node [anchor = north] {$x$};
\draw[thick, ->] (0,-3)--(0,3) node [anchor = east] {$y$};

\pgfmathsetmacro{\p}{0.5}   % eccentricity

\draw plot[domain=-3:3] ({\x*\x/(4*\p)},{\x});
\draw[thick, dashed] plot[domain=-3:3] ({-\p},{\x}) node[anchor = north east] {$x = -p$};
%\draw plot[domain=-2:2] ({-\a*cosh(\x)},{\b*sinh(\x)});
%\draw[dashed, domain=-3.5:3.5, variable=\x, samples = 1000] plot({\x}, {\b/\a*\x}) node [anchor = south east] {$y = \dfrac{b}{a}x$};
%\draw[dashed, domain=-3.5:3.5, variable=\x, samples = 1000] plot({\x}, {-\b/\a*\x}) node [anchor = north east] {$y = -\dfrac{b}{a}x$};
\end{tikzpicture}
\end{figure}

\subsubsection*{Вироджені криві другого порядку}
\begin{enumerate}[nosep,wide=0pt,label={\Roman*.}]
\item $\dfrac{x^2}{a^2} + \dfrac{y^2}{b^2} = 0$ -- точка.
\item $\dfrac{x^2}{a^2} + \dfrac{y^2}{b^2} = -1$ -- уявний еліпс.
\item $(A_1 x + B_1 y + C_1)(A_2 x + B_2 y + C_2) = 0$ -- пара прямих.
\end{enumerate}

\subsection{Оптичні властивості кривих другого порядку}
\begin{figure}[H]
\centering
\begin{tikzpicture}
\pgfmathsetmacro{\x}{2}
\pgfmathsetmacro{\y}{1}
\draw (0,0) ellipse (\x cm and \y cm);
\draw (-2.5,0)--(2.5,0);

\draw[thick] (1.2,0)--(1.5,{sqrt(1-(1.5)*(1.5)/(\x*\x))})--(-1.2,0);
\draw[thick, dashed] (1.5,{sqrt(1-(1.5)*(1.5)/(\x*\x))})--(2.5, {(2.5+1.2)/2.7 * sqrt(1 - 2.25/4)});

\draw[fill = yellow, draw = yellow] (1.2,0) circle[radius=1 pt] node[anchor = north] {$F_1$};
\draw[fill] (-1.2,0) circle[radius=1 pt] node[anchor = north] {$F_2$};
\end{tikzpicture}
\end{figure}
Через точку $F_1$ проводяться промені. Тоді виявляється, що всі вони будуть збігатись в точці $F_2$.

\begin{figure}[H]
\centering
\begin{tikzpicture}
\draw[thick] (-2.5,0)--(2.5,0);

\pgfmathsetmacro{\e}{1.4}   % eccentricity
\pgfmathsetmacro{\a}{1}
\pgfmathsetmacro{\b}{(\a*sqrt((\e)^2-1)}
\draw plot[domain=-1.5:1.5] ({\a*cosh(\x)},{\b*sinh(\x)});
\draw plot[domain=-1.5:1.5] ({-\a*cosh(\x)},{\b*sinh(\x)});
\draw[fill = yellow, draw = yellow] (1.5,0) circle[radius=1 pt] node[anchor = north] {$F_1$};
\draw[fill] (-1.5,0) circle[radius=1 pt] node[anchor = north] {$F_2$};

\draw[thick] (1.5,0)--(1.2, {\b * sqrt(1.2*1.2/(\a)^2-1)}) -- (2.5,  {\b * sqrt(1.2*1.2/(\a)^2-1)* (2.5+1.5)/2.7});
\draw[thick, dashed] (-1.5,0)--(1.2, {\b * sqrt(1.2*1.2/(\a)^2-1)});
\end{tikzpicture}
\end{figure}

Через точку $F_1$ проводяться промені. Тоді виявляється, що всі вони будуть уявно збігатись в точці $F_2$.

\begin{figure}[H]
\centering
\begin{tikzpicture}
\draw[thick, ->] (-0.5,0)--(2.5,0);

\pgfmathsetmacro{\p}{0.5}   % eccentricity

\draw plot[domain=-2:2] ({\x*\x/(4*\p)},{\x});
\draw[fill = yellow, draw = yellow] (1,0) circle[radius=1 pt] node[anchor = north] {$F_1$};
\draw[thick] (1,0)--(0.6,{sqrt(4*\p*0.6)})--(2.5,{sqrt(4*\p*0.6)});
\end{tikzpicture}\\
Через точку $F_1$ проводяться проміні. Тоді виявляється, що всі вони будуть паралельні вісі абсцис.
\end{figure}

\subsection{Криві другого порядку як конічний перетин}
\begin{figure}[H]
\centering
\includegraphics[width=.5\textwidth]{cone_intersection.jpg}
\end{figure}
Якщо площина паралельна основі конуса, то перетин -- коло.\\
Якщо площина нахилена основі конуса, то перетин -- еліпс.\\
Якщо площина нахилена так, що вона паралельна твірної, то перетин -- парабола.\\
Якщо площина ще сильніше нахилена, то перетин -- гіпербола.


\subsection{Поверхні другого роду}
\begin{definition}
\textbf{Загальне рівняння поверхні другого порядку} визначається такою формулою:
\begin{align*}
a_{11}x^2 + a_{22}y^2 + a_{33}z^2 + 2a_{12}xy + 2a_{13}xz + 2a_{23}yz + b_1x + b_2y + b_3z + c = 0
\end{align*}
\end{definition}

\begin{lemma}
Задано точку $M_0 = (x_0,y_0)$. При обертанні цієї точки навколо вісі $OX$ отримуємо коло, яке задається рівнянням $\begin{cases}
x = x_0 \\
y^2 + z^2 = y_0^2
\end{cases}
$.\\
\textit{Думаю, тут все зрозуміло}.
\begin{figure}[H]
\centering
\begin{tikzpicture}
%plane
\draw[fill] (0.5,0,0.5) circle (1pt) node [anchor = north]{$M_0$};

\draw[thick, ->] (0,0,0)--(2,0,0) node[anchor = north, black] {$y$};
\draw[thick, ->] (0,0,0)--(0,2,0) node[anchor = east] {$z$};
\draw[thick, ->] (0,0,0)--(0,0,2) node[anchor = north east] {$x$};
\draw[dashed] (0,0,0.5) circle (0.5);
\draw[thick, ->] (-2,0.5,-2) arc (0:220:0.5);
\end{tikzpicture}
\end{figure}
\end{lemma}

\begin{lemma}
У площині $XOY$ задана така крива рівнянням $F(x,y)$, що є парною відносно $y$ (тобто крива симетрична відносно $OY$). При обертанні цієї кривої навколо осі $OX$ отримуємо поверхню, яка задається рівнянням $F(x,\sqrt{y^2+z^2})=0$.
\end{lemma}

\begin{proof}
Фіксуємо $M_0 = (x_0,y_0)$ -- точку кривої. Зробимо обертання навколо $OX$, тоді отримаємо коло:\\
$\begin{cases}
x = x_0 \\
y^2 + z^2 = y_0^2
\end{cases}
$.\\
Звідси $|y_0| = \sqrt{y^2+z^2}$. Для точки $x_0,y_0$ виконана рівність: $F(x_0,y_0) = F(x_0, \sqrt{y^2+z^2}) = 0$.\\
Але точка $M_0$ була довільною, тому $F(x,\sqrt{y^2+z^2})=0$, що й є бажаною поверхнею.
\end{proof}

\subsubsection{Еліпсоїд}
У нас вже є еліпс $\dfrac{x^2}{a^2} + \dfrac{y^2}{b^2} = 1$. Обернемо його навколо $OX$, тоді отримаємо $\dfrac{x^2}{a^2} + \dfrac{y^2+z^2}{b^2} = 1$ю\\
Отримаємо еліпсоїд обертання:\\
$\dfrac{x^2}{a^2} + \dfrac{y^2}{b^2} + \dfrac{z^2}{b^2} = 1$\\
Зробимо стискання/розтягнення вздовж осі $OZ$, тобто $z_{old} = \lambda z_{new}$. Тобто $\dfrac{z^2}{b^2} \to \dfrac{\lambda^2 z^2}{b^2} = \dfrac{z^2}{c^2}$.\\
Отримаємо \textbf{канонічне рівняння еліпсоїда:}
\begin{align*}
\dfrac{x^2}{a^2} + \dfrac{y^2}{b^2} + \dfrac{z^2}{c^2} = 1
\end{align*}
\begin{figure}[H]
\centering
\includegraphics[scale=1]{ellipsoid.jpeg}
\end{figure}
%
%\begin{tikzpicture}[scale = 1.5]
%\draw[->] (0,0,0) -- (2.5,0,0)node[below]{$y$};
%\draw[->] (0,0,0) -- (0,1.5,0)node[left]{$z$};
%\draw[->] (0,0,0) -- (0,0,3)node[left]{$x$};
%\draw[pattern = north east lines, pattern color = red, dashed] (-0.8,0,0) ellipse (0.3 and {sqrt(1 - 0.8*0.8/4)});
%\draw[color=red] (0,0,0) ellipse (2cm and 1cm);
%\draw[color = red, dashed](2,0,0) arc (0:180: 2cm and 0.5cm);
%\draw[color = red](-2,0,0) arc (180:360: 2cm and 0.5cm);
%\node at (2,0,0) [anchor = north west] {$b$};
%\node at (0,1,0) [anchor = south west] {$c$};
%\node at (0,0,2.3) [anchor = east] {$a$};
%\end{tikzpicture}

\subsubsection{Гіперболоїд}
\begin{enumerate}[wide=0pt,label={\Roman*.}]
\item Візьмемо гіперболу $\dfrac{z^2}{c^2} - \dfrac{x^2}{a^2} = 1$. Обернемо навколо $OZ$, тоді отримаємо:\\
$\dfrac{z^2}{c^2} - \dfrac{x^2+y^2}{b^2} = 1$.\\
Отримаємо гіперболоїд обертання:\\
$\dfrac{z^2}{c^2} - \dfrac{x^2}{b^2} - \dfrac{y^2}{b^2} = 1$\\
Зробимо стискання/розтягнення вздовж осі $OX$, тобто $x_{old} = \lambda x_{new}$. Тобто $\dfrac{x^2}{b^2} \to \dfrac{\lambda^2 x^2}{b^2} = \dfrac{x^2}{a^2}$.\\
Отримаємо \textbf{канонічне рівняння двопорожнинного гіперболоїда:}
\begin{align*}
\dfrac{z^2}{c^2} - \dfrac{x^2}{a^2} - \dfrac{y^2}{b^2} = 1
\end{align*}
\begin{figure}[H]
\centering
\includegraphics[scale=1]{2-shit-hyperboloid.jpeg}
\end{figure}

\item Візьмемо гіперболу $\dfrac{x^2}{a^2} - \dfrac{z^2}{c^2} = 1$. Обернемо навколо $OZ$, тоді отримаємо:\\
$\dfrac{x^2+y^2}{a^2} - \dfrac{z^2}{c^2} = 1$\\
$\dfrac{x^2}{a^2} + \dfrac{y^2}{a^2} - \dfrac{z^2}{c^2} = 1$\\
Зробимо стискання/розтягнення вздовж осі $OY$, тобто $y_{old} = \lambda y_{new}$. Тобто $\dfrac{y^2}{a^2} \to \dfrac{\lambda^2 y^2}{a^2} = \dfrac{y^2}{b^2}$\\
Отримаємо \textbf{канонічне рівняння однопорожнинного гіперболоїда:}
\begin{align*}
\dfrac{x^2}{a^2} + \dfrac{y^2}{b^2} - \dfrac{z^2}{c^2} = 1
\end{align*}

\begin{figure}[H]
\centering
\includegraphics[scale=1]{1-shit-hyperboloid.jpeg}
\end{figure}
\end{enumerate}

\subsubsection{Параболоїд}
У нас вже є парабола $y^2 = 4px$. Обернемо навколо $OX$, тоді отримаємо:\\
$y^2+z^2 = 4px \implies x = \dfrac{y^2}{4p} + \dfrac{z^2}{4p}$.\\
Знову ті махінації з $OZ$, а також $z \to x$.\\
\begin{enumerate}[wide=0pt,label={\Roman*.}]
\item Отримаємо \textbf{рівняння еліптичного параболоїду:}
\begin{align*}
z = \dfrac{x^2}{a^2} + \dfrac{y^2}{b^2}
\end{align*}
\begin{figure}[H]
\centering
\includegraphics[scale=1]{elliptic-paraboloid.jpeg}
\end{figure}
\item Також отримаємо \textbf{рівняння гіперболічного параболоїду:}
\begin{align*}
z = \dfrac{x^2}{a^2} - \dfrac{y^2}{b^2}
\end{align*}

\begin{figure}[H]
\centering
\includegraphics[scale=1]{hyperbolic-paraboloid.jpeg}
\end{figure}
\end{enumerate}

\subsection{Циліндри}
На площині $XOY$ задана крива $F(x,y) = 0$ -- основа циліндра -- та пряма $l$, що перетинає основу. Через кожну точку основи $M = (x,y,0)$ проведемо пряму, паралельну $l$.\\
Отримана множина точок і є \textbf{циліндром} із основою $F(x,y) = 0$ та прямою $l$.\\

\subsubsection{Еліптичний циліндр}
$\vec{l} = (0,0,1)$\\
$\dfrac{x^2}{a^2} + \dfrac{y^2}{b^2} = 1$\\
Будь-яка точка, що належить еліпсу, буде належати циліндру.

\subsubsection{Гіперболічний циліндр}
$\vec{l} = (0,0,1)$\\
$\dfrac{x^2}{a^2} - \dfrac{y^2}{b^2} = 1$\\
Будь-яка точка, що належить гіперболі, буде належати циліндру.

\subsubsection{Параболічний циліндр}
$\vec{l} = (0,0,1)$\\
$y^2 = 4px$\\
Будь-яка точка, що належить параболі, буде належати циліндру.

\subsection{Конічні поверхні}
На площині $XOY$ задана крива $F(x,y) = 0$ -- основа конуса -- та пряма $l$, що перетинає основу. Через кожну точку основи $M = (x,y,0)$ проведемо пряму, паралельну $l$.\\
Отримана множина точок і є \textbf{циліндром} із основою $F(x,y) = 0$ та точкою $K = (x_k, y_k, \underset{\neq 0}{z_k})$ -- вершина конусу. Проведемо прямі, які проходять через вершину конусу $K$ та через основу\\
Отримана множина точок і є \textbf{конусом} із основою $F(x,y) = 0$ та вершиною $K = (x_k, y_k, \underset{\neq 0}{z_k})$.

\subsubsection*{Вироджені поверхні другого роду}
\begin{enumerate}[nosep,wide=0pt,label={\Roman*.}]
\item $\dfrac{x^2}{a^2} + \dfrac{y^2}{b^2} + \dfrac{z^2}{c^2} = 0$ -- точка.
\item $\dfrac{x^2}{a^2} + \dfrac{y^2}{b^2} + \dfrac{z^2}{c^2} = -1$ -- уявний еліпсоїд.
\item $(A_1 x + B_1 y + C_1 z + D_1)(A_2 x + B_2 y + C_2 z + D_2) = 0$ -- пара площин.
\end{enumerate}
\newpage

\section{Многочлени з дійсними та комплексними коефіцієнтами}
\begin{definition}
\textbf{Многочленом} називають функцію такого вигляду:
\begin{align*}
P_n(x) = a_0 + a_1 x + \dots + a_n x^n,
\end{align*}
де $a_0,a_1,\dots,a_n \in \mathbb{R}$ або $\mathbb{C}$.\\
Для них все зрозуміло з операціями додавання та множення многочлена на скаляр.\\
Позначення: $\mathbb{R}[x]$ або $\mathbb{C}[x]$ -- клас многочленів з дійсними або комплексними коефіцієнтами.
\end{definition}

\begin{definition}
\textbf{Степенем} многочлена $f(x) = a_0 + a_1 x + \dots + a_n x^n$ називають найвищий степінь із всіх одночленів.\\
Позначення: $\deg f$.
\end{definition}

\begin{example}
Зокрема для многочлена $f(x) = x^2 + 1$ маємо $\deg f = 2$.
\end{example}

\subsection{Про подільність многочленів}
\begin{theorem}
Для будь-яких многочленів $f(x)$ та $g(x)$ існують \textit{єдині} многочлени $s(x)$, $r(x)$, для яких $f(x) = s(x) g(x) + r(x)$,\\
де $\deg r < \deg g$ або $r \equiv 0$.
\end{theorem}

\begin{proof}
I. \textit{Існування.}\\
Маємо якісь многочлени $f,g$. Тут є два випадки.
1. $\deg f < \deg g$. Тоді $s(x) = 0$ та $r(x) = f(x)$. Отримали те, що потрібно було.
\bigskip \\
2. $\deg f \geq \deg g$. Проведемо МІ за числом $\deg f -\deg g = k$.\\
База індукції. $k = 0$, тобто $\deg f = \deg g \overset{\text{позн.}}{=} n$.
Тож маємо такі многочлени:\\
$f(x) = a_n x^n + \dots + a_0$;\\
$g(x) = b_n x^n + \dots + b_0$.\\
Перепишемо многочлен $f$ під той лад, що ми хочемо:\\
$f(x) = \underbrace{\dfrac{a_n}{b_n}}_{=s(x)} (b_n x^n + \dots + b_0) + \underbrace{\left(a_{n-1}x^{n-1}+\dots+a_0 - \dfrac{a_n b_{n-1}}{b_n}x^{n-1}- \dots - \dfrac{b_0 a_n}{b_n} \right)}_{= r(x)}$.\\
Важливо зауважити, що тут $\deg r =n-1 < n = \deg g$. Базу індукції довели.
\bigskip \\
Крок індукції. Нехай для $k \geq 0$ теорема виконується, тобто існують $\tilde{s}(x), \tilde{r}(x)$.
\bigskip \\
Доведемо існування $s(x)$ та $r(x)$ для випадку $k+1$. Нехай $\deg f = k+1+m$, тоді $\deg g = m$.\\
$f(x) = a_{m+k+1} x^{m+k+1} + a_{m+k}x^{m+k} + \dots + a_0$;\\
$g(x) = b_m x^m + \dots + b_0$.\\
Приблизно аналогічним чином ми перепишемо $f$ під наш лад:\\
$f(x) = \dfrac{a_{m+k+1}}{b_m}x^{k+1} (b_m x^m + \dots + b_0)
+ \underbrace{\left(a_{m+k}x^{m+k} + \dots + a_0 - \left(\dfrac{b_{m-1}a_{m+k+1}}{b_m}x^{m+k} + \dots + \dfrac{b_0 a_{m+k+1}}{b_m}x^k \right) \right)}_{=p(x)}$\\
$f(x) = \dfrac{a_{m+k+1}}{b_m}x^{k+1}g(x) + p(x)$.\\
Зауважимо у цьому випадку, що $\deg p = m+k < m+k+1 = \deg f$. Також $\deg p - \deg g = k$, тож за припущеннями МІ, для многочленів $p(x),g(x)$ існує $\tilde{s}(x), \tilde{r}(x)$, для яких $p(x) = \tilde{s}(x) g(x) + \tilde{r}(x)$. Звідси\\
$f(x) = \dfrac{a_{m+k+1}}{b_m}x^{k+1}g(x) + \tilde{s}(x)g(x) + \tilde{r}(x) = \underbrace{\left(\dfrac{a_{m+k+1}}{b_m}x^{k+1} + \tilde{s}(x) \right)}_{=s(x)}g(x) + \underbrace{\tilde{r}(x)}_{=r(x)}$.\\
Зауважимо, що $\deg r < \deg g$ (припущення МІ) або $r \equiv 0$.\\
МІ доведено.
\bigskip \\
II. \textit{Єдиність.}\\
!Припустимо, що крім $s,r$ існують ще $s^*,r^*$, такі, що також $f(x) = s^*(x)g(x) + r^*(x)$. Причому $\deg r^* < \deg g$ або $r^* \equiv 0$. Тоді \\ $0 = f(x) - f(x) = (s(x)-s^*(x))g(x) + (r(x)-r^*(x))$.\\
$(s(x)-s^*(x))g(x) = r^*(x)-r(x)$, де $\deg(r^*-r) < \deg g$. Із цієї оцінки випливає, що рівність можлива лише тоді, коли $s(x)-s^*(x) =0 \implies s(x) = s^*(x)$.\\
А звідси й $r(x) = r^*(x)$. Суперечність!
\end{proof}

\begin{example}
Зокрема $x^2-2 = (x-1)(x+1) -1$.\\
Тут ніби ми ділили $x^2-2$ на $x-1$, тоді отримаємо $x+1$ з остачею $-1$.
\end{example}

\subsection{Корені многочленів}
\begin{definition}
Число $a$ називають \textbf{коренем многочлена} $f(x)$, якщо
\begin{align*}
f(a) = 0
\end{align*}
\end{definition}

\begin{theorem}[Теорема Безу]
Остача ділення многочлена $f(x)$ на многочлен $x-a$ дорівнює $f(a)$.
\end{theorem}

\begin{proof}
Дійсно, маємо $f(x) = s(x)(x-a) + r(x)$, причому $\deg r < \deg (x-a) = 1$. Тому єдиний варіант -- це бути $r(x) = c, c \in \mathbb{R}$. Отже, $f(x) = s(x)(x-a) + c$. Нарешті, підставимо $x = a$ -- отримаємо $f(a) = c$. Отже, $f(x) = s(x)(x-a) + f(a)$.
\end{proof}

\begin{corollary}
$a$ -- корінь многочлена $f(x) \iff \exists h: f(x) = (x-a)h(x)$.\\
\textit{Думаю, зрозуміло.}
\end{corollary}

\begin{definition}
Число $a$ називають \textbf{коренем многочлена} $f(x)$ \textbf{кратності} $k$, якщо 
\begin{align*}
f(x) = (x-a)^k g(x),
\end{align*}
причому вимагається $g(a) \neq 0$.\\
Іншими словами, $(x-a)^k \mid f$, але $(x-a)^{k+1} \nmid f$.
\end{definition}

\subsection{Розклад многочлена за степенем лінійного двочлена}
Перед однією теоремою про кратність кореня зауважимо наступне. \\
Маємо многочлен $f(x) = a_0 + a_1 x + \dots + a_n x^n$. Хочеться переписати його в іншому вигляді. Замість $x$ хочеться всюди $x-a$.\\
За теоремою Безу, маємо $f(x) = (x-a)s_1(x) + f(a)$. Тут обов'язково $\deg s_1 = n-1$. Для спрощення позначу $f(a) = c_0$.\\
Знову за теоремою Безу, маємо $f(x) = (x-a)((x-a)s_2(x) + s_1(a)) + c_0 = (x-a)^2 s_2(x) + c_1 (x-a) + c_0$, тут $c_0 = s_1(a)$. Обов'язково $\deg s_2 = n-2$.\\
Знову за теоремою Безу, маємо $f(x) = (x-a)^2 ((x-a)s_3(x) + s_2(a)) + c_1(x-a) + c_0 = (x-a)^3 s_3(x) + c_2 (x-a)^2 + c_1(x-a) + c_0$.\\
\vdots \\
Продовжуємо цей процес, допоки $\deg s_{\text{деякий}} = 1$, отримаємо\\
$f(x) = c_0 + c_1 (x-a) + c_2 (x-a)^2 + \dots + c_{n-2}(x-a)^{n-2} + s_{n-1}(x) (x-a)^n$.\\
Останній раз беремо Безу, отримаємо $s_{n-1}(x) = (x-a) c_n + c_{n-1}$, де $c_{n-1} = s_{n-1}(a)$.\\
Разом отримали таку форму:
\begin{align*}
f(x) = c_0 + c_1(x-a) + c_2(x-a)^2 + \dots + c_n(x-a)^n.
\end{align*}
Тепер питання полягає в тому, як знайти $c_0,c_1,\dots,c_n$. Ми вже знаємо, що многочлен -- диференційована функція (див.\ мат.\ аналіз), тому можемо знайти коефіцієнти через похідні таким чином:\\
$\begin{cases}
f(a) = c_0\\
f'(a) = 1!c_1\\
f''(a) = 2!c_2\\
\vdots \\
f^{(n)}(a) = n!c_n
\end{cases} \implies c_k = \dfrac{f^{(k)}(a)}{k!}, k = \overline{0,n}$.\\
Отже, отримали більш красиву форму:
\begin{align*}
f(x) = f(a) + \dfrac{f'(a)}{1!}(x-a) + \dots + \dfrac{f^{(n)}(a)}{n!}(x-a)^n
\end{align*}
Права частина має особливу назву -- \textbf{многочлен Тейлора}.
\bigskip \\
\textbf{Повернімось до попереднього розділу.}
\begin{theorem}[Критерій кратності кореня]
$a$ -- корінь кратності $k$ многочлена $f(x) \iff f(a)=f'(a)=\dots=f^{(k-1)}(a) =0$, але $f^{(k)}(a) \neq 0$.
\end{theorem}

\begin{proof}
\rightproof Дано: $a$ -- корінь кратності $k$ для $f(x)$, тобто $f(x) = (x-a)^k g(x)$, де $g(a) \neq 0$.\\
Ба більше, якщо $\deg f = n$, то звідси $\deg g = n-k$. Тому розкладемо функцію $g$ за щойно отриманою формулою:\\
$f(x) = (x-a)^k \left(g(a) + \dfrac{g'(a)}{1!}(x-a) + \dots + \dfrac{g^{(n-k)}(a)}{(n-k)!}(x-a)^{n-k} \right) = \\
= (x-a)^k g(a) + (x-a)^{k+1}\dfrac{g'(a)}{1!} + \dots + (x-a)^n\dfrac{g^{(n+k)}(a)}{(n+k)!}$.\\
Але з іншого боку,\\
$f(x) = f(a) + \dfrac{f'(a)}{1!}(x-a) + \dots + \dfrac{f^{(k-1)}(a)}{(k-1)!}(x-a)^{k-1} + \dfrac{f^{(k)}(a)}{k!}(x-a)^{k} + \dots + \dfrac{f^{(n)}(a)}{n!}(x-a)^n$.\\
В першому рівності многочлен починався зі степені $k$, тому й друга рівність має починатись зі степені $k$, звідси й\\
$f(a) = 0 \hspace{0.5 cm} \dfrac{f'(a)}{1!} = 0 \hspace{0.5 cm} \dots \hspace{0.5 cm} \dfrac{f^{(k-1)}(a)}{k!} = 0$.\\
При цьому $\dfrac{f^{(k)}(a)}{k!} \neq 0$. Отже, отримали бажану умову.
\bigskip \\
\leftproof Дано: $f(a)=f'(a)=\dots=f^{(k-1)}(a) =0$, але $f^{(k)}(a) \neq 0$.\\
Розкладемо нашу функцію $f$ за формулою:\\
$f(x) = f(a) + \dfrac{f'(a)}{1!}(x-a) + \dots \dfrac{f^{(k-1)}(a)}{(k-1)!}(x-a)^{k-1} + \dfrac{f^{(k)}(a)}{k!}(x-a)^{k} + \dots + \dfrac{f^{(n)}(a)}{n!}(x-a)^n$\\
Тоді\\
$f(x) = \dfrac{f^{(k)}(a)}{k!}(x-a)^{k} + \dots + \dfrac{f^{(n)}(a)}{n!}(x-a)^n = (x-a)^k \underbrace{\left(\dfrac{f^{(k)}(a)}{k!} + \dots + \dfrac{f^{(n)}(a)}{n!}(x-a)^{n-k} \right)}_{=g(x)}$.\\
Через те, що $f^{(k)}(a) \neq 0$, то звідси $g(a) \neq 0$. Отже, $f(x) = (x-a)^k g(x) \Rightarrow$ $a$ -- корінь кратності $k$.
\end{proof}

\subsection{Комплексні корені многочленів}
\begin{proposition}
Якщо $z \in \mathbb{C}$ є коренем многочлена $f \in \mathbb{R}[x]$, то $\bar{z}$ -- теж корінь многочлена $f$.
\end{proposition}

\begin{proof}
Заданий многочлен $f(x) = a_n x^n + \dots + a_0$, для якого $f(z) = 0$. Тоді маємо:\\
$f(\bar{z}) = a_n \bar{z}^n + \dots + a_0 = \overline{a_n z^n} + \dots + \overline{a_0} = \overline{a_n z^n + \dots + a_0} = \overline{f(z)} = \bar{0} = 0$.
\end{proof}

\begin{proposition}
Корені $z$ та $\bar{z}$ мають однакову кратність.
\end{proposition}

\begin{proof}
Нехай $z$ -- корінь кратності $k$, тобто $f(z) = f'(z) = \dots = f^{(k-1)}(z) = 0, f'^{(k)}(z) \neq 0$.\\
Тоді за міркуваннями попереднього твердження, і$f(\bar{z}) = f'(\bar{z}) = \dots = f^{(k-1)}(\bar{z}) = 0, f'^{(k)}(\bar{z}) \neq 0$. \\ Отже, $\bar{z}$ -- корінь кратності $k$.
\end{proof}

\begin{theorem}[Основна теорема алгебри]
У многочлена $f \in \mathbb{C}[x]$ існує принаймні один корень $z_0 \in \mathbb{C}$.\\
\textit{Без доведення.}\\
\textit{А все тому, що доведення або вимагає знання комплексного аналізу, або треба вчити спеціальні розділи математики. Простого доведення нема.}
\end{theorem}

\begin{corollary}
Заданий многочлен степені $n$ з комплексними коефіцієнтами. Тоді многочлен має рівно $n$ коренів, враховуючи кратність.
\end{corollary}

\begin{proof}
Нехай (за основною теоремою алгебри) $z_1$ -- корінь $f(z) \implies f(z) = (z-z_1)^{k_1} g(z)$, причому $g(z_1) \neq 0$.\\
За основною теоремою алгебри, $g(z)$ має корінь $z_2 \implies g(z) = (z-z_2)^{k_2} g_2(z)$, причому $g_2(z_2) \neq 0$.\\
І знову за основною теоремою алгебри, $g_2(z)$ має корінь $z_3\dots$\\
Продовжуючи до кінця, отримуємо остаточно:\\
$f(z) = A_0(z-z_1)^{k_1}(z-z_2)^{k_2}\dots(z-z_m)^{k_m}$, де \hspace{0.5cm} $k_1 + k_2 + \dots + k_m = n = \deg f$.
\end{proof}

\begin{theorem}
Многочлен $f \in \mathbb{R}[x]$ розкладається на такі прості множники:\\
$f(x) = A_0 (x-a_1)^{k_1} \dots (x-a_m)^{k_m} (x^2+p_1x+q_1)^{s_1} \dots (x^2+p_j+q_j)^{s_j}$, де \\
$k_1 + \dots + k_m + 2s_1 + \dots + 2s_j = n = \deg f$.\\
Тут $A_0, a_1,\dots,a_m, p_1,q_1,\dots,p_j,q_j \in \mathbb{R}$.\\
Ба більше, всі дискримінанти квадратних рівнянь -- від'ємні.
\end{theorem}

\begin{proof}
За попереднім наслідком,\\
$f(x) = A_0(x-a_1)^{k_1} \dots (x-a_m)^{k_m} (x-z_1)^{s_1} (x-\bar{z_1})^{s_1} \dots (x-z_j)^{s_j} (x-\bar{z_j})^{s_j}$\\
Розпишемо такі добутки:\\
$(x-z_1)(x-\bar{z_1}) = x^2 - x(z_1 +\bar{z_1}) + z_1 \bar{z_1} \boxed{=}$\\
$z_1 + \bar{z_1} = 2 \Re z_1 \overset{\textrm{позн.}}{=} -p_1 \in \mathbb{R}$\\
$z_1 \bar{z_1} = |z|^2 \overset{\textrm{позн.}}{=} q_1 \in \mathbb{R}$\\
$\boxed{=} x^2 + p_1x + q_1$, для якого $D < 0$. Бо мали справу з комплексними коренями.\\
І так з рештою. Тому\\
$f(x) = A_0 (x-a_1)^{k_1} \dots (x-a_m)^{k_m} (x^2+p_1x+q_1)^{s_1} \dots (x^2+p_j+q_j)^{s_j}$.
\end{proof}

\subsection{Спільні дільники та кратні двох многочленів}
\begin{definition}
Многочлен $h(x)$ називається \textbf{дільником} многочлена $f(x)$, якщо
\begin{align*}
\exists s(x): f(x) = s(x) h(x).
\end{align*}
Позначення: $h \mid f$.
\end{definition}

\begin{definition}
Многочлен $d(x)$ називається \textbf{спільним дільником} многочленів $f(x)$ та $g(x)$, якщо
\begin{align*}
d \mid f \hspace{1cm} d \mid g
\end{align*}
\end{definition}

\begin{definition}
Многочлен $D(x)$ називається \textbf{найбільшим спільним дільником} многочленів $f(x),g(x)$, якщо
\begin{align*}
\forall d - \text{спільний дільник }f,g: d \mid D
\end{align*}
Позначення: $D = \gcd(f,g)$.
\end{definition}

\begin{corollary}
Нехай $d$ -- довільний дільник многочленів $f,g$. Тоді $\deg d \leq \deg \gcd(f,g)$.\\
\textit{Відносно зрозуміло.}
\end{corollary}

\begin{proposition}
Нехай $D_1, D_2$ -- два найбільші спільні дільники многочленів $f,g$. Тоді $\exists c \in \mathbb{R}: D_1(x) = c D_2(x)$.\\
Тобто найбільші спільні дільники многочленів рівні з точністю до константи.
\end{proposition}

\begin{proof}
Якщо $D_1(x) = \gcd(f,g) (x)$ та $D_2$ -- спільний дільник $f,g$, то $D_2 \mid D_1$, тобто $D_1(x) = s_1(x) D_2(x)$.\\
Якщо $D_2(x) = \gcd(f,g) (x)$ та $D_1$ -- спільний дільник $f,g$, то $D_1 \mid D_2$, тобто $D_2(x) = s_2(x) D_1(x)$.\\
Друге рівняння підставимо в перше -- отримаємо:\\
$D_1(x) = s_1(x) D_2(x) = s_1(x)s_2(x)D_1(x)$\\
$\implies s_1(x)s_2(x) = 1 \implies s_1(x) = c_1, s_2(x) = c_2 = \dfrac{1}{c_1}$.\\
Отже, $D_1(x) = c_1 D_2(x)$.
\end{proof}

\begin{theorem}[Алгоритм Евкліда]
Задані многочлени $f,g$. Запишемо ділення з остачею таким чином:\\
$\begin{cases}
f(x) = s_1(x) g(x) + r_1(x) \\
g(x) = s_2(x) r_1(x) + r_2(x) \\
r_1(x) = s_3(x) r_2(x) + r_3(x) \\
\vdots \\
r_{n-2}(x) = s_n(x)r_{n-1}(x) + r_n(x) \\
r_{n-1}(x) = s_{n+1}(x) r_n(x)
\end{cases}$.\\
Тоді $\gcd(f,g) = r_n(x)$, тобто останній ненульовий остача-многочлен.
\end{theorem}

\begin{remark}
Алгоритм Евкліда має скінченне число ітерацій, просто тому що\\
$\deg f > \deg g > \deg r_1 > \deg r_2 > \dots > \deg r_n$.\\
У силу строгої нерівності рано чи пізно буде остача-многочлен $0$.
\end{remark}

\begin{proof}
Покажемо, що $r_n(x)$ -- спільний дільник $f,g$.\\
Дійсно, якщо $r_{n-1}$ підставити в останнє рівняння, потім $r_{n-2}$ в передостаннє і так до кінця, то отримаємо, що спочатку $r_n \mid g$, а якщо підставити $g$ в найперше рівняння, то $r_n \mid f$.\\
Покажемо, що $r_n(x)$ -- найбільший спільний дільник $f,g$.\\
Дійсно, якщо $d$ -- довільний спільний дільник $f,g$, то звідси $d \mid r_1, d \mid r_2, \dots, d \mid r_n$. Це доводиться, якщо йти згори вниз поступово.\\
Таким чином, $r_n(x) = \gcd(f,g)$.
\end{proof}

\iffalse
\begin{definition}
Многочлен $l(x)$ називається \textbf{спільним кратним} многочленів $f(x)$ та $g(x)$, якщо
\begin{align*}
f | l \hspace{1cm} g | l
\end{align*}
\end{definition}

\begin{definition}
Многочлен $L(x)$ називається \textbf{найменшим спільним кратним} многочленів $f(x),g(x)$, якщо:
\begin{align*}
\forall l - \text{ спільне кратне } f,g: L | l
\end{align*}
Позначення: $L = \lcm(f,g)$.
\end{definition}

\begin{theorem}
$\gcd (f,g) \cdot \lcm (f,g) = f(x)g(x)$.
\end{theorem}

\begin{proof}
Маємо $D(x) = \gcd(f,g)$, тобто звідси $D | f, D | g \implies f(x) = f_1(x)D(x), g(x) = g_1(x)D(x)$.\\
Розглянемо многочлен $L(x) = f_1(x)g_1(x)D(x)$. Це буде спільним кратним $f,g$. Дійсно,\\
$L(x) = f(x)g_1(x) \implies f | L$\\
$L(x) = g(x)f_1(x) \implies g | L$.\\
Оберемо будь-яке спільне кратне $l$ многочленів $f,g$. Тоді звідси $l(x) = f(x) u(x)$ та $l(x) = g(x) v(x)$.\\
\end{proof}

\defin{5.3.5.(1)} Многочлен $m(x)$ називається \textbf{спільним кратним} $f(x)$ та $g(x)$, якщо $m(x)$ ділиться на $f(x)$ та $g(x)$ одночасно
\bigline
\defin{5.3.5.(2)} Многочлен $M(x)$ називається \textbf{\underline{найменшим} спільним кратним} $f(x)$ та $g(x)$, якщо він ділиться без остачі на будь-яке спільне кратне $f(x)$ та $g(x)$\\
Позначення: $M(x) = \textrm{LCM}(f,g)$
\bigline
\textbf{Знаходження НСК}\\
I. Знайти $\textrm{GCD}(f,g) = d(x)$, тобто\\
$f(x) = f_1(x) d(x)$\\
$g(x) = f_2(x) d(x)$\\
II. Тоді $k(x) = f_1(x)f_2(x)d_2(x)$\\
Це є спільним кратним $f(x)$ та $g(x)$\\
Будь-яке спільне кратне повинно ділитись на $f(x)$ та $g(x)$, а отже, й на $f_1(x)f_2(x)d(x) = k(x)$\\
Звідси отримуємо\\
$\textrm{LCM}(f,g) = \dfrac{f(x)g(x)}{\textrm{GCD}(f,g)}$
\bigline
\prp{5.3.6.} Нехай $k_1$ та $k_2$ - НСК $f(x)$ та $g(x)$\\
Тоді $\exists c \in \mathbb{R}: k_1(x) = ck_2(x)$\\
\proof
Скористаємось отриманою щойно формулою\\
Коли $k_1(x) = \textrm{LCM}(f,g)$, то $d_1(x) = \textrm{GCD}(f,g) = \dfrac{f(x)g(x)}{k_1(x)}$\\
Коли $k_2(x) = \textrm{LCM}(f,g)$, то $d_2(x) = \textrm{GCD}(f,g) = \dfrac{f(x)g(x)}{k_2(x)}$\\
Але оскільки $d_1(x) = c^*d_2(x)$, то звідси $\dfrac{1}{k_1(x)} = \dfrac{c^*}{k_2(x)} \Rightarrow k_1(x) = c k_2(x)$ \qed
\bigline
\fi

\subsection{Дробово-раціональні вирази, розклад}
\begin{definition}
\textbf{Дробово-раціональним виразом} називають дріб $\dfrac{P(x)}{Q(x)}$, де $P(x),Q(x)$ -- многочлени.
\end{definition}

\begin{definition}
\textbf{Простими дробами} називають один із дробово-раціональних виразів
\begin{align*}
\dfrac{1}{x-a} \hspace{1cm} \dfrac{1}{(x-a)^k} \hspace{1cm} \dfrac{Ax+B}{x^2+px+q} \hspace{1cm} \dfrac{Ax+B}{(x^2+px+q)^k}
\end{align*}
Останні два дроби -- це в дійсному випадку. Ба більше, дискриминанти знаменників -- від'ємні.
\end{definition}

\subsubsection*{Розклад дробово-раціональних виразів на суму простих дробів}
I. $\deg P \geq \deg Q$.\\
У цьому випадку ми ділимо $P$ на $Q$ (стовпчиком), тоді отримаємо:\\
$P(x) = S(x)Q(x) + P_1(x)$.\\
Звідси $\dfrac{P(x)}{Q(x)} = S(x) + \dfrac{P_1(x)}{Q(x)}$. Але ось тут вже $\deg P_1  < \deg Q$. Зараз ми розглянемо це випадок окремо.
\bigskip \\
II. $\deg P < \deg Q$.\\
У такому разі дріб $\dfrac{P(x)}{Q(x)}$ називають \textbf{правильним}.
Уже відомо, що можна розкласти $Q(x) = E(x-a_1)^{k_1} \dots (x-a_m)^{k_m} (x^2+p_1x+q_1)^{l_1} \dots (x^2+p_s x+q_s)^{l_s}$ саме таким чином.

\begin{lemma}
Заданий $\dfrac{P(x)}{Q(x)}$ -- такий дробово-раціональний вираз, що $\deg P < \deg Q$. Відомо, що $Q(x) = (x-a)^k Q_1(x)$, де $Q_1(a) \neq 0$. Тоді $\exists A \in \mathbb{R}, \exists P_1(x): \deg P_1 < \deg P$, для яких можемо записати\\
$\dfrac{P(x)}{Q(x)} = \dfrac{A}{(x-a)^k} + \dfrac{P_1(x)}{(x-a)^{k-1}Q_1(x)}$.
\end{lemma}

\begin{proof}
Хочемо знайти $A$ та $P_1(x)$, для яких виконується рівність:\\
$\dfrac{P(x)}{(x-a)^kQ_1(x)} = \dfrac{A}{(x-a)^k} + \dfrac{P_1(x)}{(x-a)^{k-1}Q_1(x)}$.\\
$\dfrac{P(x)}{(x-a)^k Q_1(x)} = \dfrac{AQ_1(x)+P_1(x)(x-a)}{(x-a)^kQ_1(x)}$.\\
$P(x) = AQ_1(x) + P_1(x)(x-a)$. Причому дана рівність виконується $\forall x \in \mathbb{R}$.\\
Зокрема для $x=a$ маємо $P(a) = AQ_1(x) \implies A = \dfrac{P(a)}{Q_1(a)}$.\\
Крім того, $P_1(x)(x-a) = P(x) - AQ_1(x)$, а оскліьки $P(a) - AQ_1(a) = 0$, то звідси \\
$P(x) -AQ_1(x) = P_1(x)(x-a)$. Тобто $P_1(x)$ отримується діленням $P(x)-AQ_1(x)$ на $(x-a)$.
\end{proof}

\begin{corollary}
$\dfrac{P(x)}{Q_1(x)(x-a)^k} = \dfrac{A_k}{(x-a)^k} + \dfrac{A_{k-1}}{(x-a)^{k-1}} + \dots + \dfrac{A_2}{(x-a)^2} + \dfrac{A_1}{x-a} + \dfrac{R_1(x)}{Q_1(x)}$.\\
У цьому випадку $Q_1(a) \neq 0$.
\end{corollary}

\begin{lemma}
Заданий $\dfrac{P(x)}{Q(x)}$ -- такий дробово-раціональний вираз, що $\deg P < \deg Q$. Відомо, що $Q(x) = (x^2+px+q)^kQ_1(x)$, де $Q_1(z_0) \neq 0$. Тут в нас $x^2+px+q = (x-z_0)(x+\bar{z_0})$. Тоді $\exists B,C \in \mathbb{R}, \exists P_1(x): \deg P_1 < \deg P$ та $P_1(x)$ не ділиться на $x^2+px+q$, для яких можемо записати\\
$\dfrac{P(x)}{Q(x)} = \dfrac{Bx+C}{(x^2+px+q)^k} + \dfrac{P_1(x)}{(x^2+px+q)^{k-1}Q_1(x)}$.
\end{lemma}

\begin{proof}
Хочемо знайти $B,C$ та $P_1(x)$, для яких виконується рівність:\\
$\dfrac{P(x)}{(x^2+px+q)^k Q_1(x)} = \dfrac{Bx+C}{(x^2+px+q)^k} + \dfrac{P_1(x)}{(x^2+px+q)^{k-1}Q_1(x)}$.\\
$\dfrac{P(x)}{(x^2+px+q)^k Q_1(x)} = \dfrac{(Bx+C)Q_1(x)+P_1(x)(x^2+px+q)}{(x^2+px+q)^k Q_1(x)}$\\
$P(x) = (Bx+C)Q_1(x)+P_1(x)(x^2+px+q)$ $(*)$\\
Нехай $z_0$ та $\bar{z_0}$ -- корені $x^2+px+q$. Вони комплексні та спряжені між собою, бо $D < 0$. Ці корені підставимо в наше рівняння (*) -- отримаємо ось таку систему:
$\begin{cases} 
P(z_0) = (Bz_0+C)Q_1(z_0) \\
P(\bar{z_0}) = (B\bar{z_0}+C) Q_1(\bar{z_0})
\end{cases}$.\\
Перепишемо обережно в вигляді $\begin{cases} 
P(z_0) = (Bz_0+C)Q_1(z_0) \\
\overline{P(z_0)} = (B\bar{z_0}+C) \overline{Q_1(z_0)}
\end{cases}$.\\
Додамо два рівняння, також віднімемо та поділимо на $i$, тоді:\\
$\begin{cases}
2 \Re P(z_0) = 2B \Re (z_0 Q_1(z_0)) + 2C \Re Q(z_0) \\
2 \Im P(z_0) = 2B \Im(z_0 Q_1(z_0)) + 2C \Im Q(z_0)
\end{cases}
$\\
Ну а далі шукаємо $B,C$, які будуть дійсними числами. Повернімось до рівняння:\\
$P(x) = (Bx+C)Q_1(x) + P_1(x)(x^2+px+q)$. Тут $B,C$ вже маємо. Тоді\\
$P_1(x)(x^2+px+q) = P(x) - (Bx+C)Q_1(x)$.\\
Корені $z_0, \bar{z_0}$ многочлена $x^2+px+q$ є коренями $P(x)-(Bx+C)Q_1(x)$. Тому $P(x) - (Bx+C)Q_1(x)$ ділиться на $x^2+px+q$.\\
Остаточно знайдемо $P_1(x)$ шляхом ділення $P(x) - (Bx+C)Q_1(x)$ на $x^2+px+q$.
\end{proof}

\begin{corollary}
$\dfrac{P(x)}{Q_1(x)(x^2+px+q)^k} = \dfrac{B_kx+C_k}{(x^2+px+q)^k} + \dots + \dfrac{B_2x+C_2}{(x^2+px+q)^2} + \dfrac{B_1x+C_1}{x^2+px+q} + \dfrac{R_1(x)}{Q_1(x)}$.\\
У цьому випадку $Q_1(z_0) \neq 0$ (хоча в дійсному випадку це вимагати не обов'язково).
\end{corollary}

\begin{theorem}
Заданий $\dfrac{P(x)}{Q(x)}$ -- такий дробово-раціональний вираз, що $\deg P < \deg Q$. Відомо (із наслідку основної теореми алгебри), що $Q(x) = E(x-a_1)^{k_1} \dots (x-a_m)^{k_m} (x^2+p_1x+q_1)^{l_1} \dots (x^2+p_s x+q_s)^{l_s}$. Тоді\\
$E \cdot \dfrac{P(x)}{Q(x)} = \dfrac{A_{11}}{(x-a_1)} + \dots + \dfrac{A_{1k_1}}{(x-a_1)^{k_1}} + \dfrac{A_{21}}{(x-a_2)} + \dots + \dfrac{A_{2k_2}}{(x-a_2)^{k_2}} + \dots + \dfrac{A_{m1}}{(x-a_m)} + \dots + \dfrac{A_{mk_m}}{(x-a_m)^{k_m}} \\ + \dfrac{B_{11}x + C_{11}}{x^2+p_1x+q_1} + \dots + \dfrac{B_{1l_1}x + C_{1l_1}}{(x^2+p_1x+q_1)^{l_1}} + \dots + \dfrac{B_{s1}x + C_{s1}}{x^2+p_s x+q_s} + \dots + \dfrac{B_{sl_s}x + C_{sl_s}}{(x^2+p_s x+q_s)^{l_s}}$\\
\textit{Ґрунтується на основі двох наслідків з двох лем.}
\end{theorem}
$\scaleobj{5}{\blacksquare}$\\
\end{document}
